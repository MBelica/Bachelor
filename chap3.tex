\chapter{Fundamental domain of periodicity and the Brillouin zone}

Let $\Omega$ be the fundamental domain of periodicity associated with \eqref{the-operator}, for simplicity let $\Omega = \Omega_{0}$ and with that $x_{0} = 0$ being the delta-point contained in $\Omega$. As commonly used by literature the reciprocal lattice for $\Omega$ is equal to $[-\pi, \pi]$, this set is the so called one-dimensional Brillouin zone $B$. For fixed $k \in \overline{B}$, consider now the operator $A_{k}$ on $\Omega$ formally defined by the operation
		\[ -\frac{d^{2}}{dx^{2}} + \rho \delta_{x_{0}} \]
	More precicely, define $A_{k}$ as follows: let us consider the problem to find for $f \in L^{2}(\Omega)$ a $u \in H^{1}_{k}$ such that
	\[ \int_{\Omega} u' \overline{v'} + \rho u(x_{0}) \overline{v(x_{0})} - \mu \int_{\Omega} u \overline{v} = \int_{\Omega} f \overline{v}, \quad \forall v \in H^{1}_{k} \]
	where 
	\begin{eqnarray}
		H^{1}_{k} \coloneqq \Big\{ H^{1}(\Omega): \psi(-\frac{1}{2}) = e^{ik} \psi(-\frac{1}{2}), \psi'(-\frac{1}{2}) = e^{ik} \psi'(-\frac{1}{2}) \notag \\ \text{ and } \psi'(x_{0} - 0) - \psi'(x_{0} + 0) + \rho \psi'(x_{0}) = 0 \Big\} \label{quasi-periodic-condition}	
	\end{eqnarray}

 	Using the fact that $H^{1}_{k}$ is a closed subspace\footnote{I think I will explain this also in more detail} of $H^{1}(\R)$ one can use the same arguments as above for $A$ to show that the resolvent $R_{\mu, k}$ of $A_{k}$ is well defined and analogously as before
		\[ A_{k} \coloneqq R_{\mu, k}^{-1} + \mu \]
	Subsequently, we will now mainly consider the eigenvalue problem
	\begin{equation}
		A_{k} \psi = \lambda \psi \text{ on } \Omega, \label{eigv-problem}
	\end{equation}

	In writing the boundary condition in the form \eqref{quasi-periodic-condition}, we understand $\psi$ extended to the whole of $\R$. In fact, \eqref{quasi-periodic-condition} forms boundary conditions on $\partial \Omega$, so-called semi-periodic boundary conditions. Furthermore we know that \eqref{eigv-problem}, \eqref{quasi-periodic-condition} is a symmetric eigenvalue problem\footnote{explain this in more detail} in $L^{2}(\Omega)$ and $\psi$ from \eqref{eigv-problem} extended to the whole of $\R$ by \eqref{quasi-periodic-condition} solves also the eigenvalue problem of $A$ with the same eigenvalue. \\
	
	Since $\Omega$ is bounded, the subsequently shown compactness can be used to prove that \eqref{eigv-problem}, \eqref{quasi-periodic-condition} has a $\langle \cdot , \cdot \rangle$-orthonormal and complete system $(\psi_{s}(\cdot, k))_{s \in \N}$ of eigenfunctions in $H^{2}_{loc}(\R)$, with corresponding eigenvalues satisfying	
	\[ \lambda_{1}(k) \leq \lambda_{2}(k) \leq \dotsc \leq \lambda_{s}(k) \rightarrow \infty \text{ as } s \rightarrow \infty \]
	The eigenfunctions $\psi_{s}(\cdot, k)$ are called Bloch waves. They can be chosen such that they depend on $k$ in a measurable way (see [M. Reed and B. Simon. Methods of modern mathematical physics I–IV]). % Academic Press (Harcourt Brace Jovanovich, Publishers), New York, 1975–1980., XIII.16, Theorem XIII.98]).
	
\begin{theorem}
	The operator $R_{\mu, k}$ is compact.

	\begin{proof}
	For each bounded sequence $(f_{j})_{j \geq 1} \in L^{2}(\Omega)$ there exist $(u_{j})_{j \geq 1} \in H^{1}_{k}$ with
		\[ R_{\mu, k} f_{j} = u_{j} \quad \forall j \geq 1 \]
	Though such $u_{j}$ has to satisfy
		\[ \int_{\Omega} u_{j}' \overline{v'} + \rho u_{j}(x_{0}) \overline{v(x_{0})} - \mu \int_{\Omega} u_{j} \overline{v} = \int f_{j} \overline{v} \quad \forall v \in H^{1}_{k} \]
	Now, choosing here $v = u_{j}$ and \eqref{eq:potentialestim} yields for $\mu$ small enough % todo do this!!!!!!!
		\[  \| u_{j} \|_{H^{1}(\Omega)} \leq \| f_{j} \|_{L^{2}(\Omega)} \| u_{j} \|_{L^{2}(\Omega)} \leq c \sqrt{vol(\Omega)} \]
	Which shows that this $u_{j}$ is bounded in $H^{1}(\Omega)$. As $H^1(\Omega) \subset C(\Omega)$: 
		\begin{equation}
			|f(x) - f(y)| \leq c |x - y|^{\nicefrac{1}{2}} \text{ for some } c > 0 \label{eq:H1estimation}
		\end{equation} 
		From \eqref{eq:H1estimation} for a $f \in B_{H^{1}} \coloneqq \{ f \in H^{1}_{k}(\Omega) : \| f \| \leq 1 \} ~\exists g \in \{ g_{1}, \dotsc, g_{n_{\epsilon}} \} : \quad \| f - g \|\leq \epsilon$ it follows that 
		\[ |f(x)|^{2} \leq 2 \| f \|^{2}_{L^{2}} + 2 \leq 4 \quad \forall x \in \Omega\]
		And with that we can approximate $f$ by simple functions through partitioning $\Omega$ into $n_{\epsilon}$ equidistant intervals. As our simple function is constant on each subinterval, we chose this constant $c_{k}$ such that $|f(\frac{k}{n}) - c_{k + 1}| < \frac{1}{n}$ which leads to
		\begin{align*}
			\| f - g \|^{2}_{L^{2}} & = \sum_{k = 0}^{n-1} \int_{\frac{k}{n}}^{\frac{k+1}{n}} | f - c_{k+1} |^{2} dx \\
				& =  2 \sum_{k = 0}^{n-1} \int_{\frac{k}{n}}^{\frac{k+1}{n}} | f - f(\frac{k}{n}) |^{2} dx +  2 \sum_{k = 0}^{n-1} \int_{\frac{k}{n}}^{\frac{k+1}{n}} | f(\frac{k}{n}) - c_{k+1} |^{2} dx \\
				& \leq 2 \sum_{n = 0}^{n-1} \frac{1}{n^{2}} + 2 \sum_{n=0}^{n-1} \frac{1}{n^{3}} = \frac{2}{n} + \frac{2}{n^{2}} < \epsilon^{2} \text{ for } n \text{ small enough.}
		\end{align*}		
		Which means $\forall f \in B_{H^{1}_{k}}$. Together with the closure of $H^{1}_{k}$ this yields the compact embedding of $H^{1}_{k}$ in $L^{2}(\Omega)$ such that $R_{\mu, k}$ is compact.
	\end{proof}	
\end{theorem}

Now, we want to transform the eigenvalue problem \eqref{eigv-problem} such that the boundary condition is independent from $k$. Define therefore
	\[ \varphi_{s}(x, k) \coloneqq e^{-ikx} \psi_{s}(x, k) \]
Then,
	\begin{align*}
		A_{k} \psi_{s}(x, k) & = \frac{d^{2}}{dx^{2}} \psi_{s}(x, k)|_{(x_{0} - \frac{1}{2}, x_{0})} \cdot \mathds{1}_{(x_{0} - \frac{1}{2}, x_{0})} + \frac{d^{2}}{dx^{2}} \psi_{s}(x, k)|_{(x_{0}, x_{0}  + \frac{1}{2})} \cdot \mathds{1}_{(x_{0}, x_{0} + \frac{1}{2})} \\
				& = e^{ikx} \left( \frac{d^{2}}{dx^{2}} + ik \right)^{2} \varphi_{s}(x, k)|_{(x_{0} - \frac{1}{2}, x_{0})} \cdot \mathds{1}_{(x_{0} - \frac{1}{2}, x_{0})} \\
				& ~\qquad + e^{ikx} \left( \frac{d^{2}}{dx^{2}} + ik \right)^{2} \varphi_{s}(x, k)|_{(x_{0}, x_{0}  + \frac{1}{2})} \cdot \mathds{1}_{(x_{0}, x_{0} + \frac{1}{2})}
	\end{align*}
Defining the operator $\tilde{A_{k}} \colon D(A_{k}) \rightarrow L^{2}(\R)$, 
	\[ \tilde{A}_{k} \varphi_{s}(x, k) \coloneqq \begin{cases}
 		\left( \frac{d^{2}}{dx^{2}} + ik \right)^{2} \varphi_{s}(x, k)|_{(x_{0} - \frac{1}{2}, x_{0})} & \text{for } x \in (x_{0} - \frac{1}{2}, x_{0}) \\ \left( \frac{d^{2}}{dx^{2}} + ik \right)^{2} \varphi_{s}(x, k)|_{(x_{0}, x_{0}  + \frac{1}{2})} & \text{for } x \in (x_{0}, x_{0} + \frac{1}{2})
 	\end{cases} \] 
Furthermore, using \eqref{eigv-problem} and \eqref{quasi-periodic-condition},
		\[ \varphi_{s}(x - \frac{1}{2}, k) = e^{-ik(x - \frac{1}{2})} \psi_{s}(x - \frac{1}{2}, k) = e^{-ik(x + \frac{1}{2})} \psi_{s}(x + \frac{1}{2}, k) = \varphi_{s}(x + \frac{1}{2}, k) \]
which shows that $(\varphi_{s}(\cdot, k))_{s \in \N}$ is an orthonormal and complete system of eigenfunctions of the periodic eigenvalue problem
	\begin{eqnarray}
		\tilde{A}_{k} \varphi = & \lambda
		 \varphi \text{ on } \Omega, \label{mod-eigv-problem} \\
		 \varphi(x - \frac{1}{2}) = & \varphi(x + \frac{1}{2}) \label{periodic-condition}
	\end{eqnarray}
with the same eigenvalue sequence $(\lambda_{s}(s))_{s \in \N}$ as before. We shall see that the spectrum of the operator $A$ can be constructed from the eigenvalue sequences $(\lambda_{s}(s))_{s \in \N}$ by varying $k$ over the Brillouin zone $B$.\\
	
An important step towards this aim is the Floquet transformation
	\begin{equation}
		(Uf)(x, k) \coloneqq \frac{1}{\sqrt{|B|}} \sum_{n \in \Z} f(x - n) e^{ikn} \quad (x \in \Omega, k \in B) \label{floquet-transformation}
	\end{equation}
		
\begin{theorem} \label{3.4.1}
	$ U \colon L^{2}(\R) \rightarrow L^{2}(\Omega \times B)$ is an isometric isomorphism, with inverse
		\begin{equation}
			(U^{-1}g)(x - n) = \frac{1}{\sqrt{|B|}} \int_{B} g(x, k) e^{-ikn} dk \quad (x \in \Omega, n \in \Z) \label{3.16}
		\end{equation} 
	If $g(\cdot, k)$ is extended to the whole of $\R$ by the semi-periodicity condition \eqref{quasi-periodic-condition}, we have
		\begin{equation}
			U^{-1} g = \frac{1}{\sqrt{|B|}} \int_{B} g(\cdot, k) dk. \label{3.17}
		\end{equation}
		
	\begin{proof}
		For $f \in L^{2}(\R)$,
		\begin{equation}
			\int_{\R} |f(x)|^{2} dx = \sum_{n \in \Z} \int_{\Omega} |f(x - n)|^{2} dx. \label{functionoverperiodicity}
		\end{equation} 
		Here, we can exchange summation and integration by Beppo Levi's Theorem. Therefore, 
		\[ \sum_{n \in \Z} |f(x - n)|^{2} < \infty \text{ for a.e. } x \in \Omega.\]
		Thus, $(Uf)(x, k)$ is well-defined by \eqref{floquet-transformation} (as a Fourier series with variable $k$) for a.e. $x \in \Omega$, and Parseval's equality gives, for these $x$,
		\[ \int_{B}|(Uf)(x,k)|^{2} dk = \sum_{n \in \Z} |f(x - n)|^{2}. \]
		By \eqref{functionoverperiodicity}, this expression is in $L^{2}(\Omega)$, and
		\[ \| Uf \|_{L^{2}(\Omega \times B)} = \|f\|_{L^{2}(\R)}. \]
		We are left to show that $U$ is onto, and that $U^{-1}$ is given by \eqref{3.16} or \eqref{3.17}. Let $g \in L^{2}(\Omega \times B)$, and define
		\begin{equation}
			f(x - n) \coloneqq \frac{1}{\sqrt{|B|}} \int_{B} g(x, k) e^{-ikn} dk \quad (x \in \Omega, n \in\Z).\label{3.19}
		\end{equation}
		For fixed $x \in \Omega$, Parseval's Theorem gives
		\[ \sum_{n \in \Z} |f(x - n)|^{2} = \int_{B} |g(x, k)|^{2} dk, \]
		whence, by integration over $\Omega$,
		\begin{eqnarray}
			\int_{\Omega \times B} |g(x, k)|^{2} dx dk & = \int_{\Omega} \sum_{n \in \Z} |f(x - n)|^{2} dx \\
				& = \sum_{n \in\Z} \int_{\Omega} |f(x-n)|^{2} dx \\
				& = \int_{\R} |f(x)|^{2} dx,	
		\end{eqnarray}
		i.e. $f \in L^{2}(\R)$. Now \eqref{floquet-transformation} gives, for a.e. $x \in\Omega$,
		\[ f(x - n) = \frac{1}{\sqrt{|B|}} \int_{B} (Uf)(x,k) e^{-ikn} dk \quad (n \in \Z), \]
		whence \eqref{3.19} implies $U f = g$ and \eqref{3.16}. Now \eqref{3.17} follows from \eqref{3.16} using $g(x + n, k) = e^{ikn} g(x, k)$.
	\end{proof}				
\end{theorem}