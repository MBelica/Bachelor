\section{The definition of our operator}

\subsubsection{I:} Now, we want to show that $(1)$ has a unique solution using the Lax-Milgram theorem:

Let $\mathcal{H} = H^{1}(\R)$ and define $a[u, v]$ as LHS and $\langle f , v \rangle$ as RHS of $(1)$:
\begin{align*}
	a[u, v] 			  & \coloneqq \int u' \overline{v}' dx + \alpha u(x_{0}) \overline{v}(x_{0}) + \mu \int u \overline{v} dx \\
	\langle f , v \rangle & \coloneqq \int f \overline{v} dx 
\end{align*}

If $\mu \in \R$ \textit{todo check if that is really necessary since Lax-Milgram needs it, (and usse trace? inequality)} % todo check if that is really nescessairy since Lax-Milgram needs it, (and usse trace? inequality)

since $|u(x_{0})|^{2} = |u(x) + \int_{x}^{x_{0}} u'(\tau) d\tau|^{2}$ we can write with labech formula \textit{todo which formular is this?} % todo which formular is this?

\begin{align*}
	a[u, v]  & = \int |u'|^{2} + \alpha | u(x_{0})|^{2} - \mu \int | u |^{2} \\
			 & \leq 2 | u(x) |^{2} + 2 | \int_{x_{0}}^{x} u'(\tau) d \tau |^{2} \\
			 & \leq 2 | u(x) |^{2} + 2 \int_{x_{0}}^{x} | u'(\tau) |^{2} d\tau (x_{1} - x_{0})^{2} \\
			 & \leq 2 | u(x) |^{2} + 2 \int_{x_{0}}^{x_{1}} | u'(\tau) |^{2} d\tau (x_{1} - x_{0})^{2} \\
\end{align*}

	\[ \xRightarrow[]{Integr.} (x_{1} - x_{0}) |u(x_{0})|^{2} \leq 2 \int_{x_{0}}^{x_{1}} | u(x)|^{2} dx + 2 (x_{1} - x_{0})^{2} \int_{x_{0}}^{x_{1}} u'(\tau) d\tau \] \textit{todo is there something missing? can't read what I've written there} % todo is there something missing? can't read what I've written there

	\[ \Rightarrow | u(x_{0})|^{2} \leq \frac{2}{\underbrace{x_{1} - x_{0}}_{\eqqcolon a}} \int_{\R} | u(x) |^{2} dx + \underbrace{2 (x_{1} - x_{0})}_{\eqqcolon b} \int_{\R} |u'(\tau)| d\tau \]

Now $a$ is not independent from $b$, a small $a$ results in a large $b$ and vice versa.

For $\alpha \geq 0$:
	\begin{align*}
		a[u, u] & \overset{per def}{\geq}  \int |u'|^{2} - \mu  |u|^{2} \\
				& \overset{\mu < -1}{\geq} \int |u'|^{2} + \int |u|^{2} \\
				& = \| u \|^{2}_{\mathcal{H}}
	\end{align*}
For $\alpha < 0$
	\begin{align*}
		a[u, u] & \geq \int |u'|^{2} - \mu \int |u| + \alpha a \int |u(x)|^{2} dx + \alpha b \int_{\R} | u'(\tau)| d\tau \\
				& = (1 + \alpha b) \int |u'|^{2} + (\alpha a - \mu) \int |u|^{2} \\
				& = c \| u \|^{2}_{\mathcal{H}}
	\end{align*}

As we want that both coefficients in front of the integrals to be positive $=>$ we chose $x_{n}$ correspondent \textit{ is it really}$x_n$ % is it really x_n
We therefore generally choose $\mu$ 'relatively' close to $-\infty$.

\subsubsection{II:} Now, we have to proof: \\ 

\begin{align*}
	|a[u, v]| & \overset{Cauchy-Schwarz}{\underset{for d=1}{\leq}} \| u' \|_{L^{2}} \| v' \|_{L^{2}} + |\alpha| \left( a \| u \|^{2}_{L^{2}} + b \| u' \|_{L^{2}}^{2} \right)^{\frac{1}{2}} \left(...\text{same vor v?} \right)^{\frac{1}{2}} + |\mu| \| u \|_{L^{2}} \|v \|_{L^{2}} \\
		& \leq C \| u \|_{\mathcal{H}} \| v \|_{\mathcal{H}}	
\end{align*}

\subsubsection{III:}
\begin{align*}
	|\langle f, v \rangle | \leq c \| v \|_{\mathcal{H}}
\end{align*}
Which follows from $(1)$ RHS , Hölder $=> L^{2}$ Norm of both $\leq H^{1}$ Norm! \textit{todo proof} % todo proof
\newline

Lax-Milgram gives as now the unique solution for the given problem.


\subsection{Now to my homework}
For $f \overset{R_{\mu}}{\mapsto} u$ solution of $(1)$ and since $(1)$ is the weak formulation of 
	\[ \underbrace{- u'' + \alpha \delta_{x_{0}} u}_{\eqqcolon A u} - \mu u = f \quad (2)\]
the mapping $f \mapsto u$ gives us also a weak solution of $(2)$.
\newline

Now we focus on this equation: $A u - \mu = f$: \\

\begin{definition}
	$dom A = range R_{\mu} \subset H^{1}(\R)$
	\[ A u \coloneqq \underbrace{f}_{\in L^{2}} + \mu \underbrace{u}_{\in H^{1}} \subset L^{2}  \]	
\end{definition}

So the next steps would be to 
\begin{itemize}
	\item describe domain of $A$ explicitly
	\item show that $A = A^{*}$
\end{itemize}


\textbf{Homework}: Let $u \in dom A$ and $u \in C^{2}(-\infty, x_{0}]$ and also $u \in C^{2}[x_{0}, \infty)$ \\
\[ \Rightarrow \text{ find conditions on } u \text{ at } x_{0} such that our assumption holds \]
by the way $u \in H^{1}(\R)$. \\

Interesting facts: \\

If we take the closure of$ C_{0}^{\infty}(\R)$ but the Closure in $H^{1}$!!! we get:
	\[ \overline[t]{C_{0}^{\infty}(\R)} = H^{1} \]

%%%%%%%%%%%%%%%%%%%%%%%%%%%%%%%%%
 \newpage  % neuer Abschnitt auf neue Seite, kann auch entfallen
%%%%%%%%%%%%%%%%%%%%%%%%%%%%%%%%%