\section{Theoretical Basics}

We start with the set of test functions

	\[ \mathcal{D} = C_{0}^{\infty}( \R ) = \{ f \in C^{\infty} : \operatorname{supp}(f) \text{ is compact in } \R \}  \]

and endow it with the topology: a sequence $ (\varphi_j)_{j\in \mathbb{N}}$ with $\varphi_j \in \mathcal{D}$ converges against ${\varphi}$, if there is a 
compact set $K \subset \R$ with $\operatorname{supp}(\varphi_j) \subset K$ for all $j$ and
	\[ \lim_{j \rightarrow \infty} \sup_{x\in K} \left| \frac{\partial^\alpha}{\partial x^\alpha} \left( \varphi_j (x) - \varphi(x) \right) \right| = 0 \]
for all multi-indizies $\alpha \in \N^n$. The set $\mathcal{D}$ is – endowed with this convergence concept – a  complete locally convex topological vector space satisfying the Heine–Borel property (Rudin 1991, §6.4–5).
\newline

The set of linear functionals from $\mathcal{D}$ to $\R$ we call the set of distributions

	\[ \mathcal{D}' = \{ f \colon D \to \R : f \text{ is linear } \} \]
	
That is, a distribution T assigns to each test function $\varphi$ a real (or complex) scalar $T(\varphi)$  such that

 	\[ T(c_{1} \varphi_{1} + c_{2} \varphi_{2}) = c_{1} T(\varphi_{1}) + c_{2} T(\varphi_{2}) \] 
 	
for all test functions $\varphi_{1}$, $\varphi_{2}$ and scalars $c_{1}$, $c_{2}$. Moreover, $T$ is continuous if and only if

	\[ \lim_{k \to \infty} T(\varphi_{k}) = T \left( \lim_{k \to \infty} \varphi_{k} \right) \]
	
for every convergent sequence $\varphi_{k} \in \mathcal{D}$. (Even though the topology of $\mathcal{D}$ is not metrisable, a linear functional on $\mathcal{D}$ is continuous if and only if it is sequentially continuous.) Equivalently, $T$ is continuous if and only if for every compact subset $K$ of $\R$ there exists a positive constant $C_{K}$ and a non-negative integer $N_{K}$ such that
	
	\[ | T(\varphi) | \leq C_{K} \sup_K | \partial^{\alpha} \varphi | \]
for all test functions $\varphi$ with support contained in $K$ and all multi-indices $\alpha$ with $| \alpha | \leq N_{K}$ (Grubb 2009, p. 14).
\newline
	
The duality pairing between a distribution $T \in \mathcal{D}'$ and a test function $\varphi \in \mathcal{D}$ is denoted using angle brackets by

	\[ \begin{cases} \mathcal{D}' \times \mathcal{D} \to \R \\ (T, \varphi) \mapsto \langle T, \varphi \rangle \end{cases} \]
	
so that $\langle T , \varphi \rangle = T(\varphi)$. One interprets this notation as the distribution $T$ acting on the test function $\varphi$ to give a scalar, or symmetrically as the test function $\varphi$ acting on the distribution $T$.	
\newline

Now taking at distributions a closer look, we distinguish between two kinds:
we call a distribution 
\begin{itemize}
	\item \textbf{regular} if there is locally integrable function 
		\[ f \in L_{loc}^{1}(\R) = \{ f \colon \Omega \to \C \text{ measurable} : f|_{K} \in L_{1}(K) ~ \forall K \subset \R, ~K \text{ compact} \} \]
		such that the distribution $T$ can be written as 
		\[ \langle T_{f} , \varphi \rangle = \int_{\R} f(t) \varphi(t) dt \quad \forall \varphi \in D \]
		Sometimes, one abuses notation by identifying $T_{f}$ with $f$, provided no confusion can arise, and thus the pairing between $T_{f}$ and $\varphi$ is often written
		\[ \langle f, \varphi \rangle = \langle T_{f} , \varphi \rangle = \int_{\R} f(t) \varphi(t) dt \quad \forall \varphi \in D \]
	\item \textbf{singular} if the distribution is not regular, e.g. if there is no integral representation with a locally integrable function.
\end{itemize}

\subsection{The dirac delta distribution}

	\[ f = \delta_{x_{0}}, ~ \langle \delta_{x_{0}} , \varphi \rangle = \varphi(x_{0}) \]
	
It is easy to see that $f^{\epsilon}(x) \rightarrow 0$ almost everywhere and
	\[ \int f(y) dy = \alpha \neq 0 \]
but
	\[ \int f^{\epsilon} dx = \frac{1}{\epsilon} \int f(\underbrace{\frac{x}{\epsilon}}_{\eqqcolon y}) dx = \int f(y) dy \]

\begin{definition}
	For $f \in \mathcal{D}'$ we say $f^{\epsilon} \rightharpoonup f \in \mathcal{D}'$ if 
		\[ \langle f^{\epsilon} , \varphi \rangle \rightarrow \langle f , \varphi \rangle \quad \forall \varphi \in \mathcal{D} \]
\end{definition}

\begin{example}
	Let $f^{\epsilon}(x) = \frac{1}{\epsilon} f(\frac{x - x_{0}}{\epsilon}$, $f \in C_{c}^{\infty}(\R)$ with
		\[ \int_{\R} f(y) dy = \alpha < \infty \]
	$f^{\epsilon} \rightharpoonup \alpha \delta_{x_{0}}$	
\end{example}	

\subsection{The Schrödinger Operator}
An Operator $A$ is defined by three properties:
\begin{itemize}
	\item space, i.e. a Hilbertspace $H$
	\item domain, $dom A \subset H$
	\item operation, $A u$ with $u \in dom A$
\end{itemize}	
	
The Schrödinger Operator	 is definied on the space $H = L^{2}(\R)$ by
	\[ A u = - u'' + V u \]
Now the question what is the domain of $A$ arises.

\begin{example}
		\[ i \cdot \frac{\partial \psi}{\partial t} ) A \psi \]
For $\Omega \subset \R$ the probability of an particle being with $\Omega$ in $t$ can be described as
	\[ \int_{\Omega} \left| \psi(x , t) \right|^{2} \]
\end{example}

	
One now often is interested in quantum states of such particle and we therefore formulate the spectral problem
	\[ A u = \lambda u \]
	
Now assume $V$ is rather good, e.g. $V \in L^{\infty}(\R)$ and $dom A = H^{2}(\R)$.

\begin{definition}
	We call an operator $A$ symmetric if
		\[ ( A u , v )_{H} = (u , A v) \quad \forall u, v \in dom H \]
\end{definition}	

\begin{example}
	Let $A = - \frac{d^{2}}{dx^{2}} + V$ for $V \colon \R \to \R$
	\begin{align*}
		(A u , v ) 	& = \int - u '' \overline{v} dx + \int V u \overline{v} dx \\
					& = \int u' \overline{v'} dx + \int V u \overline{v} dx
	\end{align*}
	\begin{align*}
		( u , A v ) & = \int - u \overline{v ''} dx + \int u \overline{V v} dx \\
					& = \int u' \overline{v'} dx + \int V u \overline{v} dx
	\end{align*}
	And therefore $A$ is symmetric.
\end{example}	

\begin{definition}
	For an operator $A$ the adjoint $A^{*}$ is the unique operator such that $\forall v, v^{*} \in H$ where
	\[ (A u , v)_{H} = ( u , v* ) \quad \forall u \in dom A \]
	Then define $dom A^{*} = \{ v \in H \}$ and $A^{*} v = v^{*}$.
\end{definition}

\begin{definition}
	An operator $A$ is self-adjoint if $A = A*$.
\end{definition}
	
If $A$ is a self-adjoint operator then $A$ is already symmetric but not the other way round. But as soon as $A$ is bounded and symmetric then the operator is self-adjoint.


\begin{example}
	Let $H = L^{2}(0, 1)$ and define $A = - \frac{d^{2}}{dx^{2}}$ on $dom A = C_{0}^{\infty}(0,1)$. As $A$ is symmetric $A^{*}$ is an extension of $A$, e.g. $A^{*} \supset A$. \\
	Now for $v, v^{*} \in L^{2}$
		\[ \int - u'' \overline{v} dx = \int u \overline{v^{*}} \forall u \in C_{0}^{\infty}(0, 1) \]
	Let $v \in C^{\infty}(0, 1)$: $v^{*} \coloneqq - v''$
		\[ \int - u'' \overline{v} dx = - \int u' \overline{v'} dx = - \int u \overline{v''} dx = \int u \overline{v^{*}} dx \]
	$\Rightarrow H^{2}(0, 1) \cap H_{0}^{1}(0, 1)$ $\Rightarrow$ $A$ is self-adjoint.
\end{example}
	
Further, we are going to examine small or even singular particles more closely.  The potential $V$ is a vector such that $grad V = F$ whereas $F$ denotes the force acting upon a particle.

	As in this case $V$ has only a small support one could approximate $V$ with a single-point potential.
	
	But the operator itself is harder to understand
		\[ A u = - u'' + \delta_{x_{0}} u \]
	For $f, g$:
		\[ \int \left( f \cdot g \right) \varphi dx = \int f \left( g \cdot \varphi \right) dx \]
	Now suppose $f \in \mathcal{D}$, $g \in C^{\infty}(\R)$:
		\[ \langle g \cdot f , \varphi \rangle \overset{def}{=} \langle f , \underbrace{g \cdot \varphi}_{\in \mathcal{D}} \rangle \]	
	
\subsection{Main Problem}

For a differential equation we distinguish between three different solution concepts
\begin{itemize}
	\item classical $u \in C^{2}$
	\item strong $u \in H^{2}$
	\item weak $u \in H^{1}$
\end{itemize}
With some conditions to the potential those terms can be equivalent.

\subsubsection{I:} Define $A^{\epsilon}$ on $dom A^{\epsilon} = H^{2}(\R)$ with
	\[ A^{\epsilon} u = - u'' + \frac{1}{\epsilon} f(\frac{x}{\epsilon}) u, ~ f \in C_{c}^{\infty}(\R) \]
while $\int_{\R} f(x) dx = \alpha \neq 0$.	Now the question arises: $A^{\epsilon} \xrightarrow[\epsilon \to 0]{}$ $?$

\subsubsection{II:} $A u - \mu u = f \in L^{2}, u \in \C \setminus \R$ with $A = - u'' + V$ the resolvent.
	\[ \left( A - \mu I \right)^{-1} \] \textit{todo check} % todo check
	if $A$ is self-adjoint $\Rightarrow$ has solution for $\C \setminus \R$ defined and bounded arbetrary $f$ \\
Partial integration yields
	\[ \int u' \overline{v'} dx + \int V u \overline{v} dx + \mu \int u \overline{v} dx = \int f \overline{v} dx \quad (*) \]
which holds for an arbitrary $v \in C_{0}^{\infty}(\R)$<	
	
We say, $u$ is a weak solution if $u \in H^{1}(\R)$ and $(*)$ holds. If the potential is even in $L^{\infty}$ then the solution is also a strong solution (i.e. $u \in H^{2}$) \textit{todo proof} % todo proof

For $f \in L^{2}(\R)$, $Au - \mu u = f$	
If we tale a potential $V$, then $\int V u \overline{v} dx$ could be also written as $V(u \overline{v})$. \\
Now suppose $H = L^{2}(\R), A = - \frac{d^{2}}{dx^{2}} + \alpha \delta_{x_{0}}$
	\[ \int u' \overline{v}' dx + \alpha u(x_{0}) \overline{v}(x_{0}) + \mu \int u \overline{v} dx = \int f \overline{v} dx \quad \forall v \in C_{0}^{\infty}(\R) (1) \]
Since this formular only evaluations $v and u$ in $x_{0}$ $\Rightarrow \forall v \in H^{1}(\R), u \in H^{1}(\R)$ is enough.
\newline

For $d = 1$: $H^{1}(\R) \subset C(\R)$ \textit{todo proof} % todo proof

%%%%%%%%%%%%%%%%%%%%%%%%%%%%%%%%%
 \newpage  % neuer Abschnitt auf neue Seite, kann auch entfallen
%%%%%%%%%%%%%%%%%%%%%%%%%%%%%%%%%