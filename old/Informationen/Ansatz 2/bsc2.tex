\documentclass[14pt,a4paper]{scrartcl}

\usepackage[utf8]{inputenc}
\usepackage[T1]{fontenc}
\usepackage[ngerman]{babel}

\usepackage[pdftex]{graphicx}
\usepackage{latexsym}
\usepackage{amsmath,amssymb,amsthm}
\usepackage{mathtools}
\usepackage[bottom=0.8in]{geometry}

\setlength{\topmargin}{-15mm}
\setlength{\parindent}{0pt}
\renewcommand{\baselinestretch}{1.15}
\DeclareUnicodeCharacter{00A0}{ }

\newtheorem{Satz}{Satz}[section]
\newtheorem{Definition}[Satz]{Definition}     
\newtheorem{Lemma}[Satz]{Lemma}	
\newtheorem*{proof*}{Proof}             
                  
\numberwithin{equation}{section}
 
\def\supp{\operatorname{supp}}

\newcommand{\C}{\mathbb{C}}
\newcommand{\K}{\mathbb{K}}
\newcommand{\R}{\mathbb{R}}
\newcommand{\Q}{\mathbb{Q}}
\newcommand{\Z}{\mathbb{Z}}
\newcommand{\N}{\mathbb{N}}

\DeclareUnicodeCharacter{00A0}{ }

\begin{document}


\pagestyle{headings}

\section{Introduction}
Let $A$ denote the one-dimensional Schrödinger operator with a periodic delta potential at $(x_{k})_{k \geq 1}$ and let $\Omega$ be a fundamental domain of periodicity associated with the periodic delta potential. Now assuming w.l.o.g. that $x_{0} = 0 \in \Omega$ and $|\Omega| = 1$ yields
	\[ A_{0} u \coloneqq A|_{\Omega} u = - \Delta u + \delta(x_{0}) u, \quad \text{for } u \in dom A_{0} \]
For fixed $k \in \R$, we are interested in the spectral problem of the weak formulation of our differential equation such that the main focus point is going to be the following equation
\[ \int_{\Omega} u' \varphi' - \mu \int_{\Omega} u \varphi + c u(x_{0}) \varphi(x_{0}) = \int_{\Omega} f \varphi \forall \varphi \in H^{1}_{k} \]
for a $u \in H^{1}_{k} \coloneqq \big\{ u \in H^{1}(\Omega) : u(\frac{1}{2}) = e^{ik} u(-\frac{1}{2}) \big\}$ and for all $\varphi \in H^{1}_{k}$

\begin{align*}
	dom A_{0} = \Big\{ & u \in H^{2}(-\frac{1}{2}, 0) \cap H^{2}(0, \frac{1}{2}), u(-0) = u(+0), \\
		& u'(-0) - u'(+0) + c u(0) = 0, u(\frac{1}{2}) = e^{ik} u(-\frac{1}{2}), u'(\frac{1}{2}) = e^{ik} u'(-\frac{1}{2}) \Big\} 
\end{align*}
~\

The boundary conditions on $\partial \Omega$, so-called semi-periodic boundary conditions, are stated here to later extended eigenfunctions on the whole of $\R$ by those conditions.
~\\

Subsequently it is shown that $A_{k}$ is well defined on its domain, that given the differential equation with a bounded $f \in L^{2}$ there exists a unique solution $u$ in the domain for all $f \in L^{\infty}$ and that there is an orthonormal basis of eigenfunction to the corresponding eigenvalue problem.
\newpage

\section{The operator $A$}

\begin{Lemma} 
	The operator $A$ is well-defined on its domain.
\end{Lemma}
\begin{proof*}
	To show that $A_{0}$ is well-defined on its domain consider $u \in dom A$ and an arbitrary $v \in H^{1}_{k}$ % or C_{0}^{\infty}
	\[  \int_{\Omega} \left( A_{0} u - \mu u \right) \overline{v} = \int_{\Omega} u' \overline{v}' - \mu \int_{\Omega} u \overline{v} + c u(x_{0}) \overline{v(x_{0})} \quad (*) \]	
	Now choosing a specific $v$ such that $\supp v = (-\frac{1}{2}, x_{0})$ alters $(*)$ to
	\[ \int_{-\frac{1}{2}}^{x_{0}} u' v' dx = \int_{-\frac{1}{2}}^{x_{0}} A u \overline{v} \iff \int_{-\frac{1}{2}}^{x_{0}} u \overline{v}'' dx = \int_{-\frac{1}{2}}^{x_{0}} - A u \overline{v} dx \]

	$\Rightarrow u'' = - A u \in L^{2}$ on $(-\frac{1}{2}, x_{0})$ and an analogously on $(x_{0}, \frac{1}{2})$. One can therefore fix
	\[ dom A \subset \{ u \in H^{2}(-\frac{1}{2}, x_{0}), u \in H^{2}(x_{0}, \frac{1}{2}) \} \]

	Next, integrating $(*)$ on both sides of $x_{0}$ by parts yields 
	\[ -\left( \int_{-\frac{1}{2}}^{x_{0}} + \int_{x_{0}}^{\frac{1}{2}}\right) u'' \overline{v} + \big( u'(x_{0}-0) v(x_{0} - 0) - u'(x_{0} + 0)v(x_{0}+0) \big) + c u(x_{0})\overline{v}(x_{0}) \]
	\[  = - \int_{-\frac{1}{2}}^{x_{0}} u'' v - \int_{x_{0}}^{\frac{1}{2}} u'' v \]

	But as $v \in H^{1}_{k} \subseteq C$ this is equivalent
	\[ u'(x_{0}-0) - u'(x_{0}+0) + c u(x_{0}) = 0 \]
	Such that
	\[ dom A \subset \{  u \in H^{2}(-\infty, x_{0}) \cap H^{2}(x_{0}, \infty), u'(x_{0} - 0) - u'(x_{0} + 0) + c u(x_{0}) = 0 \} \eqqcolon B \]	
	
	The opposite inclusion one shows by taking an arbitrary $u \in B$ and proving that $u$ is in the domain of $A_{0}$. However\footnote{notice, $R_{\mu, 0} \coloneqq (A_{0} - \mu I)^{-1}$ denotes the resolvent}, as $range R_{\mu, 0} = dom A_{0}$ one can also show that $u = \mathcal{R}(R_{\mu, 0})$. As $dom R_{\mu, 0} = L^{2}$ define 
	\[ f \coloneqq A_{0} u = \begin{cases}
								-u'', & (-\frac{1}{2}, x_{0}) \\ 
								-u'', & (x_{0}, \frac{1}{2})
							 \end{cases} \] 
	and show that $u = R_{\mu}(f - \mu u)$:
	\[ \int_{\Omega}(f-\mu u)v = \int_{\Omega} u' v' +c u(x_{0}) v(x_{0}) - \mu \int_{\Omega} u v \]
	\[ \iff - \int_{-\frac{1}{2}}^{x_{0}} u'' v - \int_{x_{0}}^{\frac{1}{2}} u'' v = \int_{\Omega} u' v' + c u(x_{0}) v(x_{0}) \]
	Partial integration with $\supp v = (-\frac{1}{2}, \frac{1}{2})$ yields
	\[ \int_{-\frac{1}{2}}^{x_{0}} u'v' + \int_{x_{0}}^{\frac{1}{2}} u'v' -u'(x_{0}-0) v(x_{0})  + u'(x_{0}+0) v(x_{0}) = \int_{\Omega} u' v' + c u(x_{0}) v(x_{0}) \]
	\[ \iff u(x_{0}+0)v(x_{0}) - u(x_{0}-0) v(x_{0}) - c u(x_{0})v(x_{0}) = 0 \]
	Such that $u \in B$.
\end{proof*}

\newpage

\subsection{$A_{k}$ is a self-adjoint operator}
Last but not least, to show that $A_{k}$ is self-adjoint, we focus first on $R_{\mu, k}^{-1}$ which is given by 
	\[ R_{\mu, k}(A)^{-1} = (A - \lambda I) \] 
First one has to notice that $R_{\mu, k}^{-1}$ is symmetric, as $\forall v \in H^{1}_{k}$:
\begin{align*}
	\langle R_{\mu, k}^{-1} u, v \rangle & = \langle (A - \lambda I) u, v \rangle \\
		& = \int (A - \lambda I)(u) v dx \\
		& = \int u'v' dx - \int \lambda u v dx + c u(x_{0}) v(x_{0}) \\
		& = \int u (A - \lambda I)(v) dx \\
		& = \langle u, (A - \lambda I) v \rangle = \langle u,  R_{\mu, k}^{-1} v \rangle 
\end{align*}

Now as $dom R_{\mu, k} = L^{2}(\R)$ and $range R_{\mu, k} = dom R_{\mu, k}^{-1}$, we want to show that for each $f, g \in L^{2}$
\[ \langle R_{\mu, k} f, g \rangle - \langle f, R_{\mu, k} g  \rangle = \gamma \]
$\gamma = 0$. Now there are $u, v \in dom A_{k}$ with $Rf = u, Rg = v$ applying to $A_{k}$ to $u, v$ one gets for all $\varphi, \psi \in H^{1}_{k}$
\begin{align*}
	\int u' \varphi' + c u(0) \varphi(0) - \mu \int u \varphi & = \int f \varphi \\
	\int v' \psi' + c v(0) \psi(0) - \mu \int v \psi & = \int g \psi
\end{align*}
As it has to hold for all $\varphi, \psi \in H^{1}_{k}$ the special choice of $\varphi = v$ and $\psi = u$ yields $\gamma = 0$ and $R_{\mu, k}$ is therefore symmetric. \\
All in all we can use this to show that $\R_{\mu, k}$ is self-adjoint, as we get for an arbitrary $v^{*} \in domain R_{\mu, k}^{-1}$ there exists a $v \in dom R_{\mu, k}$:
\begin{align*}
	\langle u, v^{*} \rangle & = \langle R_{\mu, k}^{-1} R_{\mu, k} u , v^{*} \rangle \\
		& = \langle R_{\mu, k} u, (R_{\mu, k}^{-1}) v^{*} \rangle \\
		& = \langle R_{\mu, k} u, v \rangle  = \langle  u, R_{\mu, k} v \rangle 
\end{align*}
So $v^{*} \in range R_{\mu, k} = dom R_{\mu, k}^{-1}$ with that also $R_{\mu, k}^{-1}$ is self-adjoint and as $A_{k}$ is simply $R_{\mu, k}^{-1}$ shifted by the real constant $\mu$, $R_{\mu, k}^{-1}$ is self-adjoint as well. \\
\hfill \qed

\subsection{A$_{k}$ being compact}
Let $B_{H^{1}_{k}} = \{ f \in H^{1}_{k}(\Omega) : \| f \| \leq 1 \}$. We want to show that $\forall \epsilon > 0 ~\exists g_{1}, \dotsc, g_{n_{\epsilon}}$:
	\[ \forall f \in B ~\exists g \in \{ g_{1}, \dotsc, g_{n_{\epsilon}} \} : \quad \| f - g \| \leq \epsilon \]
Together with the closure of $H^{1}_{k}$ this yields the compact embedding. Now, as $H^1(\Omega) \subset C(\Omega)$: 
	\[ |f(x) - f(y)| \leq c |x - y|^{\frac{1}{2}} \text{ for some } c > 0 \tag*{$(*)$} \]
Now, for a $f \in B_{H^{1}}$ follows from $(*)$ that 
	\[ |f(x)|^{2} \leq 2 \| f \|^{2}_{L^{2}} + 2 \leq 4 \quad \forall x \in \Omega \]
And with that we can approximate a $f \in B$ by simple functions through partitioning $\Omega$ into $n_{\epsilon}$ equidistant intervals. As our simple function is constant on each subinterval, we chose this constant $c_{k}$ such that
	\[ |f(\frac{k}{n}) - c_{k + 1}| < \frac{1}{n}  \]
such that
\begin{align*}
	\| f - g \|^{2}_{L^{2}} & = \sum_{k = 0}^{n-1} \int_{\frac{k}{n}}^{\frac{k+1}{n}} | f - c_{k+1} |^{2} dx \\
		& =  2 \sum_{k = 0}^{n-1} \int_{\frac{k}{n}}^{\frac{k+1}{n}} | f - f(\frac{k}{n}) |^{2} dx +  2 \sum_{k = 0}^{n-1} \int_{\frac{k}{n}}^{\frac{k+1}{n}} | f(\frac{k}{n} - c_{k+1} |^{2} dx \\
		& \leq 2 \sum_{n = 0}^{n-1} \frac{1}{n^{2}} + 2 \sum_{n=0}^{n-1} \frac{1}{n^{3}} = \frac{2}{n} + \frac{2}{n^{2}} < \epsilon^{2} \text{ for } n \text{ small enough.}
\end{align*}

\newpage

\subsection{Existence of a unique solution}
Next, one has to show that $R_{\mu, k}$ is well-defined, which means that for all $u \in dom(A_{k})$ there exists a unique solution. Lets assume that $f$, as the righthand-side of the given differential equation, is a bounded linear functional. Lax-Milgram's theorem\footnote{formulation and prove in appendix A} would indeed guarantee the existence and uniqueness to prove the existence of a unique solution and whereby $R_{\mu, k}$ being well-defined, but one has to show that $H^{1}_{k}$ is a Hilbert space, for
\begin{align*}
	B(u, \varphi) & \coloneqq \langle \nabla u, \nabla \varphi \rangle + c u(x_{0}) \varphi(x_{0}) - \mu \langle u , \varphi \rangle \\ 
	l(\varphi) & \coloneqq \langle f, \varphi \rangle
\end{align*}  
$B$ is bounded and $B[u,u]$ is coercive.

\subsubsection{$H^{1}_{k}$ being a Hilbert space}
$H^{1}_{k}$ is evidently a subspace of the Hilbert space $H^{1}(\Omega)$, but additionally $H^{1}_{k}$ is also closed, as for an arbitrary sequence $(\psi_{j})_{j \geq 1} \in H_{1, k}$ the value on the boundary coincides. Define $f \coloneqq \psi_{j} - \lim \psi_{j}$ and then
	\begin{align*}
		| f(-\frac{1}{2}) |^{2} & = 2 | f(x) |^{2} + 2 \left( \int_{-\frac{1}{2}}^{x} f'(\tau) d\tau \right)^{2} \\
			& \leq 2 |f(x)|^{2} + 2 \int_{- \frac{1}{2}}^{\frac{1}{2}} |f'|^{2} d\tau \\
			& \leq 2 \| f \|^{2}_{H^{1}(-\frac{1}{2}, \frac{1}{2})}
	\end{align*}
	With that $\psi \in H^{1}_{k}$ as
	\[ \psi(-\frac{1}{2}) = \lim_{j \rightarrow \infty} \psi_{j}(-\frac{1}{2}) = \lim_{j \rightarrow \infty} e^{ik} \psi_{j}(\frac{1}{2}) = e^{ik} \psi(\frac{1}{2}) \]

\subsubsection{The bilinear form $B$ is bounded}
\begin{align*} 
	| B(u, \varphi)| & \coloneqq \left| \langle \nabla u, \nabla \varphi \rangle + c u(x_{0}) \varphi(x_{0}) - \mu \langle u , \varphi \rangle  \right|\\
		& \overset{Schwarz's}{\underset{Inequality}{\leq}} \big| \| \nabla u \| \cdot \| \nabla \varphi \| + c u(x_{0}) \varphi(x_{0}) - \mu \| u \| \cdot \| \varphi\| \big|	
\end{align*}

Again we require $H^{1}(\R) \Subset C(\R)$, we can estimate the modulus of $v(x_{0}) \in \{ u(x_{0}), \varphi(x_{0}) \}$ over the periodicity cell $I_{k}$: 
\begin{align*}
	|v(x_{0})|^{2} & = \left| v(x) + \int_{x}^{x_{0}} \nabla v( \tau ) d\tau \right|^{2} \quad \text{for an arbitrary } x \in ( \inf I, x_{0}) \\
	& \overset{convexity}{\leq} 2 |v(x)|^{2} + 2 \left| \int_{x}^{x_{0}} \nabla v(\tau) d\tau \right|^{2} \\
	& \overset{trace}{\leq}	2 |v(x)|^{2} + 2 \int_{I} \left| \nabla v(\tau) \right|^{2} d\tau \cdot(x_{0} - x)
\end{align*}

Integrating both sides over the interval $I$ yields:
	\[	|v(x_{0})|^{2} \cdot |I| = 2 \int_{I} |v(x)|^{2} dx + 2 \int_{I} \left| \nabla  v(\tau) \right|^{2} d\tau \cdot |I| \cdot (x_{0} - x) \]
	\[ \Rightarrow |v(x_{0})|^{2} = \frac{2}{|I|} \int_{I} |v(x)|^{2} dx + 2 \int_{I} \left| \nabla v(\tau) \right|^{2} d\tau \cdot \underbrace{(x_{0} - x)}_{\leq |I|} \] 
	
and results in the following
\begin{align*}
	| B(u, \varphi)| & \leq \big| \| \nabla u \| \cdot \| \nabla \varphi \| + c \cdot u(x_{0}) \varphi(x_{0}) - \mu \| u \| \cdot \| \varphi\| \big|	 \\
		& \leq \big| \| \nabla u \| \cdot \| \nabla \varphi \| + c \left( u(x_{0})^{2} \varphi(x_{0})^{2} \right)^{\frac{1}{2}} - \mu \| u \| \cdot \| \varphi\| \big|	\\
		& =  \big| \| \nabla u \| \cdot \| \nabla \varphi \|+ 2 c \left( \frac{1}{|I|} \|u\|^{2} + \| \nabla u \|^{2} \cdot |I| \right)^{\frac{1}{2}} \\
		& ~\qquad \cdot \left( \frac{1}{|I|} \|\varphi \|^{2} + \| \nabla \varphi \|^{2}  \cdot |I| \right)^{\frac{1}{2}} - \mu \| u \| \cdot \| \varphi\| \big| \\
		& = \big| (1 + 2c\cdot|I|) \cdot \| \nabla u \| \cdot \| \nabla \varphi \| + (\frac{2c}{|I|} - \mu) \cdot \| u \| \cdot \| \varphi\| \\
		& ~\qquad + 2c \left( \|u\| \cdot \| \nabla \varphi \| + \| \nabla u \| \cdot \| \varphi \| \right) \big| \\
		& \leq \alpha \cdot \| u \|_{H^{1}} \cdot \| \varphi \|_{H^{1}} 
		\tag*{$\qed$}
\end{align*}

\subsubsection{$B[u,u]$ is coercive}
Next, the coercivity for $ c \geq 0$ and as assumed at the start $\mu$ is small enough, here $\mu < -1$
\begin{align*}
	B(u, u) & = \langle \nabla u, \nabla u \rangle + c u(x_{0})^{2} - \mu \langle u , u \rangle \\
		& \geq \langle \nabla u, \nabla u \rangle - \mu \langle u , u \rangle \\
		& \geq \langle \nabla u, \nabla u \rangle  + \langle u , u \rangle \\
		& = \| u \|_{H^{1}}^{2} \\
\text{and for $c < 0$} \\
	B(u, u) & = \langle \nabla u, \nabla u \rangle + c |u(x_{0})|^{2} - \mu \langle u , u \rangle \\
		& = \langle \nabla u, \nabla u \rangle + c \left( \frac{2}{I} \int_{I} |u(x)|^{2} dx + 2 I \int_{I} |\nabla u(\tau)|^{2} d\tau \right) - \mu \langle u , u \rangle  \\
		& = (1 + 2 c I) \| \nabla u \|^{2} + (- 1 + c \frac{2}{I}) \| u \|^{2}  \\
		& \geq \beta \| u \|_{H^{1}}^{2}
		\tag*{$\qed$}
\end{align*}

All in all, Lax-Milgram's theorem now guarantees a unique element $u \in H$ such that
	\[ B(u, v) = l(\varphi) \]
for all $\varphi \in H^{1}_{k}$

\newpage


\newpage

\section{Appendix A}


\subsection{The inverse of a self-adjoint operator} % verw. Funktionalanalysis Weiss VL
If $T \in B(X, Y)$ is invertible, where $X, Y$ are Hilbert spaces, then $T^{*}$ has an inverse and $(T^{*})^{-1} = (T^{-1})^{*}$

\begin{proof*}
	Let $T \in B(X, Y)$ be invertible, notice that $\langle Tv , u \rangle = \langle v , T^{*} u \rangle$ for all $v \in X, u \in Y$. Then $\langle T^{*} (T^{-1})^{*}v, u \rangle = \langle (T^{-1})^{*} v, T u \rangle = \langle v , T^{-1} T u \rangle = \langle u, v \rangle$. \\
	Therefore $T^{*} (T^{-1})^{*} = I$, hence $(T^{-1})^{*} = (T^{*})^{-1}$
\end{proof*}

\subsection{Lax-Milgram}
Let $H$ be a real Hilbert space, with norm $\| \cdot \|$ and inner product $\langle \cdot, \cdot \rangle$ as well as the pairing of $H$ with its dual space. Assume that
	\[ B \colon H \times H \rightarrow R  \]

is a bilinear mapping, for which there exist constant $\alpha, \beta > 0$ such that
	\[ |B[u, v]| \leq \alpha \| u \| \| v \| \quad (u, v \in H) \]
and
	\[ \beta \| u \|^{2} \leq B[u, u] \quad ( u \in H) \]
Finally, let $f \colon H \rightarrow \R$ be a bounded linear functional on $H$. \\

Then there exists a unique element $u \in H$ such that
	\[ B[u, v] = \langle f, v \rangle \]
for all $v \in H$.

\begin{proof*} % verw. Evans: Partial Differential Equations % question I used <> for both, the inner product and the pairing between dual spaces, is that okay?
	For each fixed element $u \in H$, the mapping $v \mapsto B[u, v]$ is a bounded linear functional on $H$; whence the Riesz Representation Theorem asserts the existence of a unique element $w \in H$ satisfying
		\[ B[u, v] = \langle w, v \rangle \quad (*) \]
	Let us write $A u = w$ whenever $(*)$ holds; so that
		\[ B[u, v] = \langle Au, v \rangle \quad (u, v \in H) \]
	We first claim $A \colon H \rightarrow H$ is a bounded linear operator. Indeed if $\lambda_{1}, \lambda_{2} \in \R$ and $u_{1}, u_{2} \in H$, we see for each $v \in H$ that
	\begin{align*}
		\langle A (\lambda_{1} u_{1} + \lambda_{2} u_{2}), v \rangle & = B[\lambda_{1} u_{1} + \lambda_{2} u_{2}, v] \quad (\text{by } (*)) \\
			& = \lambda_{1} B[u_{1}, v] + \lambda_{2} Bu_{2}, v] \\
			& = \lambda_{1} \langle A u_{1}, v \rangle + \lambda_{2} \langle A u_{2}, v \rangle  \quad (\text{by } (*) \text{ again}) \\
			& = \langle \lambda_{1} A u_{1} + \langle_{2} A u_{2}, v \rangle.
	\end{align*}
	This equality obtains for each $v \in H$, and so $A$ is linear. Furthermore
	\[ \| A u \|^{2} = \langle A u, A u \rangle = B[u, Au] \leq \alpha \| u \| \| Au \|. \]
	Consequently $\| A u \| \leq \alpha \|u \|$ for all $u \in H$, and so $A$ is bounded. \\
	Next we assert
	\[ \begin{cases} A \text{ is one-to-one, and} \\ R(A), \text{ the range of } A, \text{ is close in } H. \end{cases} \quad \text{(+)} \]
	To prove this, let us compute
		\[ \beta \| u \|^{2} \leq B[u, u] = \langle Au, u \rangle \leq \| Au \| \| u \| \]
	Hence $\beta \| u \| \leq \| Au \|$. This inequality easily implies (+). \\
	We demonstrate now
		\[ R(A) = H \quad \text{(-)} \]
		For if not, then, since $R(A)$ is closed, there would exist a nonzero element $w \in H$ with $w \in R(A)^{\bot}$. But this fact in turn implies the contradiction $\beta \| w \|^{2} \leq B[w, w] = \langle A w , w \rangle = 0$. \\
	Next, we observe once more from the Riesz' Representation Theorem that
		\[ \langle f, v \rangle = \langle w , v \rangle \text{ for all } v \in H \]
	for some element $w \in H$. We then utilise (+) and (-) to find $u \in H$ satisfying $A u  = w$. Then 
		\[ B[u, v] = \langle A u, v \rangle = \langle w, v \rangle = \langle f, v \rangle (v \in H) \]
	and this is the claim. \\
	Finally, we show there is at most one element $u \in H$ verifying the claim. For if both $B[u, v] = \langle f, v \rangle$ and $B[\tilde{u}, v] = \langle f, v \rangle$, then $B[u - \tilde{u}, v] = 0$ $(v \in H)$. We set $v = u - \tilde{u}$ to find $\beta \| u - \tilde{u}\|^{2} \leq B[u - \tilde{u}, u - \tilde{u}] = 0$.
\end{proof*}

\subsection{Sobolev Embedding} % verw. Spektraltheorie
For $s>d/2$, the follow holds
	\[ H^s\subset C_b(\R^d) \]
	
\begin{proof*}
	For $u\in S(\R^d)$. 
	\begin{align*}
		u(x) & = \int_{\R^d} e^{2i\pi x\xi}(1+|\xi|^2)^{-s/2} \hat u(1+|\xi|^2)^{s/2} d\xi\nonumber\\
		 & \leq \left(\int (1+|\xi|^2)^{-s}d\xi \right)^{1/2} \left(\int |\hat u|^2 (1+|\xi|^2)^sd\xi \right)^{1/2}\nonumber 
	\end{align*}
	with $\left(\int (1+|\xi|^2)^{-s}d\xi \right)^{1/2}\leq \infty$ und $\left(\int |\hat u|^2 (1+|\xi|^2)^sd\xi \right)^{1/2} = ||u||_{H^s}$. 	
\end{proof*}

\subsection{dom A$_{k}$ = range R$_{\mu, k}$}

\begin{proof*}
	As we introduced the variational problem
	\[ \forall v \in H^{1}(\R): \quad \int \nabla u \overline{\nabla v}dx - \mu \int u \overline{v} dx + \alpha u(x_{0}) v(x_{0}) = \int f \overline{v} dx \quad (1) \]
$\exists_{1} u \in H^{1}(\R)$ satisfying $(1)$
	\[ L^{2}(\R) \ni f \mapsto u \eqqcolon R_{\mu} f \]
	
For $f_{1} \neq f_{2} \Rightarrow u_{1} \neq u_{2}$, since:
	\[ \text{Suppose } u_{1} = u_{2} \Rightarrow \int (f_{1} - f_{2}) \overline{v} = 0 \quad \forall \underbrace{v \in H^{1}(\R)}_{\underset{\forall v \in L^{2}(\R)}{\text{and therefore }}} \Rightarrow f_{1} = f_{2} \]
Since $H^{1}$  is dense in $L^{2}$ $\Longrightarrow f_{1} = f_{2}$

	\[ \Rightarrow \begin{rcases*} f = R^{-1}_{\mu} u \\ A u - \mu u \end{rcases*} Au = R_{\mu}^{-1} u + \mu u \]
	
$\Rightarrow dom A = range R_{mu}$
\end{proof*}

%\newpage
%
%\section{Appendix B (please don't read this yet)}
%Here, I'm going to sum up a few things to not lose them but there is not much sense in them, yet... Gonna rewrite all of this at a later point
%\newline
%
%
%The duality pairing between a distribution $T \in \mathcal{D}'$ and a test function $\varphi \in \mathcal{D}$ is denoted using angle brackets by
%
%	\[ \begin{cases} \mathcal{D}' \times \mathcal{D} \to \R \\ (T, \varphi) \mapsto \langle T, \varphi \rangle \end{cases} \]
%	
%so that $\langle T , \varphi \rangle = T(\varphi)$. One interprets this notation as the distribution $T$ acting on the test function $\varphi$ to give a scalar, or symmetrically as the test function $\varphi$ acting on the distribution $T$.	
%\newline
%
%Now taking at distributions a closer look, we distinguish between two kinds:
%we call a distribution 
%\begin{itemize}
%	\item \textbf{regular} if there is locally integrable function 
%		\[ f \in L_{loc}^{1}(\R) = \{ f \colon \Omega \to \C \text{ measurable} : f|_{K} \in L_{1}(K) ~ \forall K \subset \R, ~K \text{ compact} \} \]
%		such that the distribution $T$ can be written as 
%		\[ \langle T_{f} , \varphi \rangle = \int_{\R} f(t) \varphi(t) dt \quad \forall \varphi \in D \]
%		Sometimes, one abuses notation by identifying $T_{f}$ with $f$, provided no confusion can arise, and thus the pairing between $T_{f}$ and $\varphi$ is often written
%		\[ \langle f, \varphi \rangle = \langle T_{f} , \varphi \rangle = \int_{\R} f(t) \varphi(t) dt \quad \forall \varphi \in D \]
%	\item \textbf{singular} if the distribution is not regular, e.g. if there is no integral representation with a locally integrable function.
%\end{itemize}
%
%\subsection{The dirac delta distribution}
%
%	\[ f = \delta_{x_{0}}, ~ \langle \delta_{x_{0}} , \varphi \rangle = \varphi(x_{0}) \]
%	
%It is easy to see that $f^{\epsilon}(x) \rightarrow 0$ almost everywhere and
%	\[ \int f(y) dy = \alpha \neq 0 \]
%but
%	\[ \int f^{\epsilon} dx = \frac{1}{\epsilon} \int f(\underbrace{\frac{x}{\epsilon}}_{\eqqcolon y}) dx = \int f(y) dy \]
%
%\textbf{definition}
%	For $f \in \mathcal{D}'$ we say $f^{\epsilon} \rightharpoonup f \in \mathcal{D}'$ if 
%		\[ \langle f^{\epsilon} , \varphi \rangle \rightarrow \langle f , \varphi \rangle \quad \forall \varphi \in \mathcal{D} \]
%
%\textbf{Example}
%	Let $f^{\epsilon}(x) = \frac{1}{\epsilon} f(\frac{x - x_{0}}{\epsilon}$, $f \in C_{c}^{\infty}(\R)$ with
%		\[ \int_{\R} f(y) dy = \alpha < \infty \]
%	$f^{\epsilon} \rightharpoonup \alpha \delta_{x_{0}}$	
%
%\subsection{The Schrödinger Operator}
%An Operator $A$ is defined by three properties:
%\begin{itemize}
%	\item space, i.e. a Hilbertspace $H$
%	\item domain, $dom A \subset H$
%	\item operation, $A u$ with $u \in dom A$
%\end{itemize}	
%	
%The Schrödinger Operator	 is defined on the space $H = L^{2}(\R)$ by
%	\[ A u = - u'' + V u \]
%Now the question arises: what is the domain of $A$?
%
%\textbf{Example}
%		\[ i \cdot \frac{\partial \psi}{\partial t} ) A \psi \]
%For $\Omega \subset \R$ the probability of an particle being with $\Omega$ in $t$ can be described as
%	\[ \int_{\Omega} \left| \psi(x , t) \right|^{2} \]
%
%	
%One now often is interested in quantum states of such particle and we therefore formulate the spectral problem
%	\[ A u = \lambda u \]
%	
%Now assume $V$ is rather good, e.g. $V \in L^{\infty}(\R)$ and $dom A = H^{2}(\R)$.
%
%\textbf{Definition}
%	We call an operator $A$ symmetric if
%		\[ ( A u , v )_{H} = (u , A v) \quad \forall u, v \in dom H \]
%	
%
%\textbf{Example}
%	Let $A = - \frac{d^{2}}{dx^{2}} + V$ for $V \colon \R \to \R$
%	\begin{align*}
%		(A u , v ) 	& = \int - u '' \overline{v} dx + \int V u \overline{v} dx \\
%					& = \int u' \overline{v'} dx + \int V u \overline{v} dx
%	\end{align*}
%	\begin{align*}
%		( u , A v ) & = \int - u \overline{v ''} dx + \int u \overline{V v} dx \\
%					& = \int u' \overline{v'} dx + \int V u \overline{v} dx
%	\end{align*}
%	And therefore $A$ is symmetric.
%
%\textbf{Definition}
%	For an operator $A$ the adjoint $A^{*}$ is the unique operator such that $\forall v, v^{*} \in H$ where
%	\[ (A u , v)_{H} = ( u , v* ) \quad \forall u \in dom A \]
%	Then define $dom A^{*} = \{ v \in H \}$ and $A^{*} v = v^{*}$.
%
%\textbf{Definition}
%	An operator $A$ is self-adjoint if $A = A*$.
%	
%If $A$ is a self-adjoint operator then $A$ is already symmetric but not the other way round. But as soon as $A$ is bounded and symmetric then the operator is self-adjoint.
%
%
%\textbf{Example}
%	Let $H = L^{2}(0, 1)$ and define $A = - \frac{d^{2}}{dx^{2}}$ on $dom A = C_{0}^{\infty}(0,1)$. As $A$ is symmetric $A^{*}$ is an extension of $A$, e.g. $A^{*} \supset A$. \\
%	Now for $v, v^{*} \in L^{2}$
%		\[ \int - u'' \overline{v} dx = \int u \overline{v^{*}} \forall u \in C_{0}^{\infty}(0, 1) \]
%	Let $v \in C^{\infty}(0, 1)$: $v^{*} \coloneqq - v''$
%		\[ \int - u'' \overline{v} dx = - \int u' \overline{v'} dx = - \int u \overline{v''} dx = \int u \overline{v^{*}} dx \]
%	$\Rightarrow H^{2}(0, 1) \cap H_{0}^{1}(0, 1)$ $\Rightarrow$ $A$ is self-adjoint.
%	
%Further, we are going to examine small or even singular particles more closely.  The potential $V$ is a vector such that $grad V = F$ whereas $F$ denotes the force acting upon a particle.
%
%	As in this case $V$ has only a small support one could approximate $V$ with a single-point potential.
%	
%	But the operator itself is harder to understand
%		\[ A u = - u'' + \delta_{x_{0}} u \]
%	For $f, g$:
%		\[ \int \left( f \cdot g \right) \varphi dx = \int f \left( g \cdot \varphi \right) dx \]
%	Now suppose $f \in \mathcal{D}$, $g \in C^{\infty}(\R)$:
%		\[ \langle g \cdot f , \varphi \rangle \overset{def}{=} \langle f , \underbrace{g \cdot \varphi}_{\in \mathcal{D}} \rangle \]	
%	
%\subsection{Main Problem}
%
%For a differential equation we distinguish between three different solution concepts
%\begin{itemize}
%	\item classical $u \in C^{2}$
%	\item strong $u \in H^{2}$
%	\item weak $u \in H^{1}$
%\end{itemize}
%With some conditions to the potential those terms can be equivalent.
%
%\subsubsection{I:} Define $A^{\epsilon}$ on $dom A^{\epsilon} = H^{2}(\R)$ with
%	\[ A^{\epsilon} u = - u'' + \frac{1}{\epsilon} f(\frac{x}{\epsilon}) u, ~ f \in C_{c}^{\infty}(\R) \]
%while $\int_{\R} f(x) dx = \alpha \neq 0$.	Now the question arises: $A^{\epsilon} \xrightarrow[\epsilon \to 0]{}$ $?$
%
%\subsubsection{II:} $A u - \mu u = f \in L^{2}, u \in \C \setminus \R$ with $A = - u'' + V$ the resolvent.
%	\[ \left( A - \mu I \right)^{-1} \] \textit{todo check}
%	if $A$ is self-adjoint $\Rightarrow$ has solution for $\C \setminus \R$ defined and bounded arbetrary $f$ \\
%Partial integration yields
%	\[ \int u' \overline{v'} dx + \int V u \overline{v} dx + \mu \int u \overline{v} dx = \int f \overline{v} dx \quad (*) \]
%which holds for an arbitrary $v \in C_{0}^{\infty}(\R)$<	
%	
%We say, $u$ is a weak solution if $u \in H^{1}(\R)$ and $(*)$ holds. If the potential is even in $L^{\infty}$ then the solution is also a strong solution (i.e. $u \in H^{2}$) \textit{todo proof}
%
%For $f \in L^{2}(\R)$, $Au - \mu u = f$	
%If we tale a potential $V$, then $\int V u \overline{v} dx$ could be also written as $V(u \overline{v})$. \\
%Now suppose $H = L^{2}(\R), A = - \frac{d^{2}}{dx^{2}} + \alpha \delta_{x_{0}}$
%	\[ \int u' \overline{v}' dx + \alpha u(x_{0}) \overline{v}(x_{0}) + \mu \int u \overline{v} dx = \int f \overline{v} dx \quad \forall v \in C_{0}^{\infty}(\R) (1) \]
%Since this formular only evaluations $v and u$ in $x_{0}$ $\Rightarrow \forall v \in H^{1}(\R), u \in H^{1}(\R)$ is enough.

\end{document}
