\chapter{Completeness of the Bloch waves}

Using the Floquet transformation $U$, we are now able to prove a completeness property of the Bloch waves $\psi_{s}(\cdot, k)$ in $L^{2}(\Omega)$ when we vary $k$ over the Brillouin zone $B$.
	
\begin{theorem}
		For each $f \in L^{2}(\R)$ and $l \in \N$, define
			\begin{equation}
				f_{l}(x) \coloneqq \frac{1}{\sqrt{|B|}} \sum_{s=1}^{l} \int_{B} \langle (Uf)(\cdot, k), \psi_{s}(\cdot, k) \rangle_{L^{2}(\Omega)} \psi_{s}(x, K) dk \quad (x \in \R). \label{3.20}
			\end{equation}
		Then, $f_{l} \rightarrow f$ in $L^{2}(\R)$ as $l \rightarrow \infty$.

	\begin{proof}
		Sine $Uf \in L^{2}(\Omega \times B)$, we have $(Uf)(\cdot, k) \in L^{2}(\Omega)$ for a.e. $k \in B$ by Fubini's Theorem. Since $(\psi_{s}(\cdot, k))_{s \in \N}$ is orthonormal and complete in $L^{2}(\Omega)$ for each $k \in B$, we obtain
			\[ \lim_{l \rightarrow \infty} \| (Uf)(\cdot, k) - g_{l}(\cdot, k) \|_{L^{2}(\Omega)} = 0 \text{ for a.e. } k \in B\]
		where 
			\begin{equation}
				g_{l}(x, k) \coloneqq \sum_{s=1}^{l} \langle(Uf)(\cdot, k), \psi_{s}(\cdot,k)\rangle_{L^{2}(\Omega)} \psi_{s}(x,k). \label{3.21}
			\end{equation}
		Thus, for $\chi(k) \coloneqq \| (Uf)(\cdot, k) - g_{l}(\cdot, k) \|^{2}_{L^{2}(\Omega)}$, we get
			\[ \chi_{l}(k) \rightarrow 0 \text{ as } l \rightarrow \infty \text{ for a.e. } k \in B, \]
		and moreover, by Bessel's inequality,
			\[ \chi_{l}(k) \leq \| (Uf)(\cdot, k) \|^{2}_{L^{2}(\Omega)} \text{ for all } l \in \N \text{ and a.e. } k \in B \]
		and $\|(Uf)(\cdot, k)\|^{2}_{L^{2}(\Omega)}$ is in $L^{1}(B)$ as a function of $k$ by Theorem \ref{3.4.1}. Altogether, Lebesgue's Dominated Convergence theorem implies
			\[ \int_{B} \chi_{l}(k) dk \rightarrow 0 \text{ as } l \rightarrow \infty, \]
		i.e., 
			\begin{equation}
				\| U f - g_{l} \|_{L^{2}(\Omega \times B)} \rightarrow 0 \text{ as } l \rightarrow \infty \label{3.22}
			\end{equation} 
		Using \eqref{3.20}, \eqref{3.21} and \eqref{3.17}, we find that $f_{l} = U^{-1}g_{l}$, whence \eqref{3.22} gives
			\[ \| U(f - f_{l}) \|_{L^{2}(\Omega \times B)} \rightarrow 0 \text{ as } l \rightarrow \infty,\]
		and the assertion follows since $U \colon L^{2}(\R) \rightarrow L^{2}(\Omega \times B)$ is isometric by Lemma \eqref{3.4.1}.
	\end{proof}
\end{theorem}