% todo General:
%	Referenzen
%	Create / Replace images
%	Normen

% todo Want to Read:
%  Very Important:
% 	http://www.lti.kit.edu/rd_download/EPhys5Fcompl.pdf
%  Important: 
% 	https://nano.tu-dresden.de/~rgutierrez/2008SS_WPV_NM/SolutionKronigPenneyModel1.pdf	
% 	The likelihood that the particle will pass through the barrier is given by the transmission coefficient, whereas the likelihood that it is reflected is given by the reflection coefficient. Schrödinger's wave-equation allows these coefficients to be calculated.
%  Interesting:
% 	http://www.ece.uc.edu/~mcahay/QuantumSystems/KroenigPenney.pdf
% 	http://ecee.colorado.edu/~bart/book/book/chapter2/pdf/ch2_3_8.pdf
% 	http://www.tf.uni-kiel.de/matwis/amat/mw2_ge/kap_4/advanced/t4_1_2.html
% 	http://www.phys.ufl.edu/~maslov/phz6426/phz6426_dkp.pdf
% 	http://people.physik.hu-berlin.de/~sokolov/QM1/QMall/node46.html
% 	https://nano.tu-dresden.de/~rgutierrez/2008SS_WPV_NM/SolutionKronigPenneyModel1.pdf
%  Optional:
% 	https://en.wikipedia.org/wiki/Spectrum#Energy
% 	https://en.wikipedia.org/wiki/Continuous_spectrum
% 	https://en.wikipedia.org/wiki/Discrete_spectrum

% todo Andrii Questions:
%	(Chapter 4) Prof. Plum wrote in his proof of the continuity (in the line "Thus the min-max-principle gives") 'for |k - \tilde{k}| \leq 1. I didn't need it and haven't seen how it is relevant. Did I miss something?
%	You wrote in the comments "u \in H^{2}(R \setminus \{ x_{i} : i \in Z \}", but up until now I used "u \in \bigcap_{i \in \Z} H^{2}(x_{i}, x_{i + 1})". What would you prefer
%	A_k vs A, L2 () H1 etc

\documentclass[11pt]{mitthesis}

\usepackage[utf8]{inputenc}
\usepackage[english]{babel}

\usepackage{amsmath}
\usepackage{amssymb}
\usepackage{amsthm}
\usepackage{cmap}
\usepackage{csquotes}
\usepackage{dsfont}
\usepackage{enumitem}
\usepackage{hyperref}
\usepackage{lgrind}
\usepackage{mathtools}
\usepackage{nicefrac}
\usepackage{xpatch}
\usepackage[export]{adjustbox}
\usepackage[dvipsnames]{xcolor}

\pagestyle{plain}

\makeatletter
  \xpatchcmd{\proof}{\@addpunct{.}}{\@addpunct{:}}{}{}
  \DeclareUnicodeCharacter{00A0}{ } 
  \def\supp{\operatorname{supp}}
  \setlength{\parindent}{0pt} 
\makeatother

\newcommand{\C}{\mathbb{C}}
\newcommand{\K}{\mathbb{K}}
\newcommand{\R}{\mathbb{R}}
\newcommand{\Q}{\mathbb{Q}}
\newcommand{\Z}{\mathbb{Z}}
\newcommand{\N}{\mathbb{N}}

\newtheorem{theorem}{Theorem}[chapter]
\newtheorem*{remark}{Remark}
\numberwithin{equation}{chapter}

\begin{document}

\begin{titlepage}

    \includegraphics[scale=0.45,left]{kit-logo.jpg} 
    \vspace*{2.5cm} 

 \begin{center} \Large 
    
    Bachelorarbeit
    \vspace*{3cm}

    {\Huge Titel der Bachelorarbeit}
    \vspace*{3.5cm}

    Martin Belica
    \vspace*{2.5cm}

    Datum der Abgabe
    \vspace*{5cm}


    Betreuung: Name der Betreuerin / des Betreuers \\[0.9cm]
    Fakultät für Mathematik \\[0.9cm]
		Karlsruher Institut für Technologie
  \end{center}
  \vspace{2cm}
\end{titlepage}
 
\tableofcontents
\chapter{Introduction} \label{chap:1}

The problem considered in this thesis arises from the Kronig-Penney model, see for example \cite[chapter 3]{heering2002elektrophysik}, which describes an idealised quantum-mechanical system that models a quantum particle behaving as a matter wave moving in one-dimension through an infinite periodic array of rectangular potential barriers, i.e. through a space area in which a potential attains a local maximum. Such an array commonly occurs in models of periodic crystal lattices where the potential is caused by ions in the crystal structure. Those charged molecules create an electromagnetic field around themselves. Hence, any particle moving through such a crystal would be subject to a recurrent electromagnetic potential. Although a solid particle, simplified as a point mass, would be reflected at such a barrier, there is a possibility that the quantum particle, as it behaves like a wave, penetrates the barrier and continues its movement beyond. 
~\\

Assuming the spacing between all ions is equidistant the potential function $V(x)$ in the lattice can be approximated by a rectangular potential as depicted in Figure \ref{figure-kpm}, where $b$ is the width of the support and $\rho$ the magnitude of the potentials.
	\vspace{0.25cm}
	\begin{figure}[!ht] \centering 
	  \resizebox{.65\linewidth}{!}{
		\definecolor{lightgray}{rgb}{0.3765,0.3765,0.3765}
		\begin{tikzpicture}[line cap=round,line join=round,>=triangle 45,x=1.0cm,y=1.0cm]
			\draw[->,color=lightgray] (-5.5,0.) -- (5.5,0.);
			\foreach \x in {-5.,-4.,-3.,-2.,-1.,1.,2.,3.,4.,5.}
			 \draw[shift={(\x,0)},color=lightgray] (0pt,-2pt);
			\draw[->,color=lightgray] (0.,-1.1) -- (0.,4.25);
			\clip(-5.5,-1.1) rectangle (5.5,4.25);
			\draw (-0.3,3.)-- (0.3,3.);
			\draw (0.3,0.)-- (0.3,3.);
			\draw (-0.3,3.)-- (-0.3,0.);
			\draw (-1.7,3.)-- (-1.7,0.);
			\draw (-2.3,3.)-- (-1.7,3.);
			\draw (-2.3,3.)-- (-2.3,0.);
			\draw (-1.7,3.)-- (-1.7,0.);
			\draw (-3.7,0.)-- (-3.7,3.);
			\draw (-3.7,3.)-- (-4.3,3.);
			\draw (-4.3,3.)-- (-4.3,0.);
			\draw (-3.7,3.)-- (-3.7,0.);
			\draw (-3.7,3.)-- (-4.3,3.);
			\draw (1.7,3.)-- (1.7,0.);
			\draw (1.7,3.)-- (2.3,3.);
			\draw (2.3,3.)-- (2.3,0.);
			\draw (3.7,0.)-- (3.7,3.);
			\draw (3.7,3.)-- (4.3,3.);
			\draw (4.3,3.)-- (4.3,0.);
			\draw (-3.7,0.)-- (-2.3,0.);
			\draw (-1.7,0.)-- (-0.3,0.);
			\draw (0.3,0.)-- (1.7,0.);
			\draw (2.3,0.)-- (3.7,0.);
			\draw (4.3,0.)-- (5.7,0.);
			\draw (-4.3,0.)-- (-5.7,0.);
			\draw [->] (0.65,1.5) -- (0.65,3.);
			\draw [->] (0.65,1.5) -- (0.65,0.);
			\draw [->] (0.65,1.5) -- (0.65,3.);
			\draw [dotted] (0.3,3.)-- (0.65,3.);
			\draw [dotted] (1.,3.)-- (0.65,3.);
			\draw [->] (2.75,-0.5) -- (2.3,-0.5);
			\draw [->,line width=0.4pt] (1.25,-0.5) -- (1.7,-0.5);
			\draw [dotted] (1.7,0.)-- (1.7,-1.);
			\draw [dotted] (2.3,0.)-- (2.3,-0.5);
			\draw [dotted] (2.3,-0.5)-- (2.3,-1.);
			\draw (1.8,-0.05) node[anchor=north west] {$b$};
			\draw (0.76,1.9) node[anchor=north west] {$V_{0}$};
			\draw (0.75,1.92) node[anchor=north west] {$V_{0}$};
			\draw (0.1,4.25) node[anchor=north west,color=lightgray] {$V(x)$};
			\draw (5,-0.1) node[anchor=north west,color=lightgray] {$x$};
		  \end{tikzpicture}
	    }
	  \caption{Potential $V(x)$ of the Kronig-Penney model} \label{figure-kpm} 
	\end{figure}
~\\

In this thesis, we are interested in the spectrum of the operator describing the situation in the Kronig-Penney model when the particle moves through periodically distributed, singular potentials. With respect to the above this means taking the limit $b \rightarrow 0$ while $V_{0}$ remains of order $\rho b^{-1}$. Therefore, we will extend the research in \cite{dorfler2011photonic} where the spectrum of periodic differential operators with smooth coefficients was analysed. We will show that an operator modelling the aforementioned situation has, as in the case of smooth coefficients, a spectrum that consists of a union of compact intervals in $\R$ which form a so-called spectral band. 
~\\

In a physical sense, the spectral bands represent energy levels. Only electron with a energy level within the spectral band can exists inside the crystal. Hence, the possible areas between the intervals, if they exist, represent forbidden energy levels. A result of Bragg's law, standing waves form n the boundary between the spectral band and the forbidden energy levels, see for a detailed explanation \cite[section 3.2]{heering2002elektrophysik}. The closer now the electrons accumulate to the nucleus of the relative ion the energetically more favourable are the standing waves which is a desired state by the electrons. Therefore, the Bragg's law can be considered as cause of the forbidden energy levels. Because of this causality, for the knowing possible energetically levels within periodic crystal lattices we need knowledge about the corresponding spectral properties.
~\\

The remainder of this thesis is structured as follows. We begin with some preliminaries in Chapter 2 to review some concepts from functional analyses and spectral theory. In Chapter 3, we will examine the operator and show that its self-adjointness, which is our first step in analysing its spectrum. The main mathematical tool for analysing the spectrum of such an operator is the so-called Floquet–Bloch theory. We will transfer the spectral problem the operator on the whole of $\R$ to a family of eigenvalue problems on the periodicity cell. Hence, we proceed in Chapter 4 by restricting the problem to its fundamental domain of periodicity. There we are able to analyse the spectrum of the restricted operator by showing the compactness of its resolvent while varying quasi-periodic boundary conditions. In Chapter 5 we introduce two concepts to transfer the results from the restricted case into the unrestricted, namely the above mentioned Floquet transformation and the Bloch waves. Based on this methods, in Chapter 6 we are able to show the main result for the one-dimensional case, i.e. such an operator has a spectrum consisting of a union of compact intervals in $\R$, and extrapolate this result to the multi-dimensional case in Chapter 7. We close this thesis in Chapter 8, where we discuss possible gaps between the compact intervals and list current researches.
\chapter{The one-dimensional Schrödinger operator $A$} \label{chap2}

The mathematical representation of the above introduced problem can be done by introducing a one-dimensional Schrödinger operator $A$ where the potential is given by a periodic delta-distribution. In this chapter we are going to examine properties of $A$ such as its domain and show that $A$ is self-adjoint. Later, in chapters \ref{chap3} and \ref{chap4}, we will need these results to deduce characteristics about the spectrum of $A$.
~\\ ~\\
Formally the operation of $A$ is defined by
\begin{equation}
	- \frac{d^{2}}{dx^{2}} + \rho \sum_{i \in \Z} \delta_{x_{i}} \label{the-operator-A-formally}
\end{equation}
on the whole of $\R$, where $\delta_{x_{i}}$ denotes the Dirac delta distribution supported at the point $x_{i}$. $\Omega_{k}$ will hereafter identify the periodicity cell containing point $x_{k}$ and w.l.o.g. let $x_{0} = 0$ and $|\Omega_{i}| = 1$ for all $i \in \Z$.

~\\  
In general, one cannot say, given $f \in L^{2}(\R)$, in which sense a solution to the formal problem 
	\begin{equation}
		Au = f \label{formal-problem}
	\end{equation}
	exists since the potential in $A$ is consists of a singular distribution. If we suppose for a moment that the problem is smooth, more specifically if the potential is given by \eqref{smooth-potential}, then formally multiplying it by a test function and integrating by parts yields the so called weak-formulation to the problem for whose solution less regularity is needed. Motivated by this, by taking the limit of the potential in the weak-formulation, we henceforth consider the problem to find for $\mu \in \R$ a function $u \in H^{1}(\R)$ such that
\begin{equation}
	\int_{\R} u' \overline{v'} + \rho \sum_{i \in \Z} u(x_{i}) \overline{v(x_{i})} - \mu \int_{\R} u \overline{v} = \int_{\R} f \overline{v} \quad \forall v \in C_{0}^{\infty}(\R), \label{weak-formulation-of-A}
\end{equation}	
holds and call it the weak-formulation of \eqref{formal-problem}.

\begin{remark}
	Since $C_{0}^{\infty}(\R)$ is dense in $H^{1}(\R)$ with respect to the norm in $H^{1}(\R)$, \eqref{weak-formulation-of-A} holds also for all $v \in H^{1}(\R)$. % todo INTRODUCTION: C_0înf dence in H^1
\end{remark}

We should first note that the left-hand side of problem \eqref{weak-formulation-of-A} is actually well-defined and finite, as for any $h \in (0, 1]$ we can estimate
\begin{align}
	\sum_{i \in \Z} |u(x_{i})|^{2} & \leq \sum_{i \in \Z} \left( 2 |u( x_{i} + h )|^{2} +  2 h \int_{x_{i}}^{x_{i} + h} \left| u'(\tau) \right|^{2} d\tau \right) \notag \\
		 & \leq 2 \sum_{i \in \Z} \left( \frac{1}{h} \int_{\Omega_{i}} |u( x )|^{2} dx + h \int_{\Omega_{i}} \left| u'(\tau) \right|^{2} d\tau \right). \label{preestimation-for-potential}
\end{align}
The particularly choice of $h = 1$ yields hence the estimation
\begin{align} 
		\sum_{i \in \Z} |u(x_{i})|^{2} & \leq 2 \| u \|^{2}_{H^{1}(\R)}. \label{estimation-for-potential}
\end{align}

\section{The resolvent-mapping of $A$}

To explicitly define our operator $A$ we will first show that for each $f \in L^{2}(\R)$ the equation \eqref{weak-formulation-of-A} has a unique solution $u \in H^{1}(\R)$. 

\begin{definition}
	Given $f \in L^{2}(\R)$, we define a functional $l_{f} \colon H^{1}(\R) \rightarrow \C$ by
	\[ l_{f}(v) \coloneqq \int_{\R} f v \]
and a sesquilinear form $B_{\mu} \colon H^{1}(\R) \times H^{1}(\R) \rightarrow \C$ for $\mu \in \R$ by
	\[ B_{\mu}[u, v] \coloneqq \int_{\R} u' \overline{v'} + \rho \sum_{i \in \Z} u(x_{i}) \overline{v(x_{i})} - \mu \int_{\R} u \overline{v}. \]
\end{definition}

As a result, \eqref{weak-formulation-of-A} is equivalent to finding for $\mu \in \R$ a function $u \in H^{1}(\R)$ such that
	\begin{equation}
		B_{\mu}[u, v] =  l_{f}(v) \quad \forall v \in H^{1}(\R). \label{weak-formulation-of-A-for-LM}
	\end{equation}
	
The existence of a unique $u \in H^{1}(\R)$ satisfying \eqref{weak-formulation-of-A-for-LM} now follows from Lax-Milgram's Theorem if the sesquilinear form $B_{\mu}$ is bounded and coercive and if $l_{f}$ is a bounded linear functional on $H^{1}(\R)$, which we will prove in the next two theorems, but above all note that $B[u, u] \in \R$.
% todo INTRODUCTION & IMPORTANT: lax-Milgram FIND VERSION WITH COMPLEX
\begin{theorem} \label{2.1:thm-LaxMilgram}
	The sesquilinear form $B_{\mu}$ is (for small enough $\mu \in \R$)
	\begin{enumerate}
		\item[i)] bounded, i.e. there exists a constant $\alpha > 0$ such that
			\[ \left| B_{\mu}[u,v] \right| \leq \alpha \|u\| \|v\| \]
			holds for all $u, v \in H^{1}(\R)$.
		\item[ii)] coercive, i.e. there exists a constant $\beta > 0$ such that
			\[ \beta \|u\|^{2} \leq B_{\mu}[u, u] \]
			for all $u \in H^{1}(\R)$.
	\end{enumerate} 

	\begin{proof} ~\\
		i) The boundedness follows from \eqref{estimation-for-potential} as for an arbitrary $\rho \in \R$ there exists $\alpha \in \R$ such that
		\begin{align*}
			|B(u, \varphi)|^{2} & \leq \| u' \| \| v' \| + 2 |\rho |\sum_{i \in \Z} |u(x_{i})|^{2} |v(x_{i})|^{2} - \mu \| u \| \| v \| \\
				& \leq \| u' \| \| v' \| + 8 |\rho| \| u \|^{2}_{H^{1}(\R)} \| v \|^{2}_{H^{1}(\R)}  - \mu \| u \| \| v \| \\
				& = (8|\rho| - \mu) \| u \| \| v \| + 8 |\rho| \left( \| u \| \| v' \| + \| u' \| \| v \| \right) + (8 |\rho| + 1) \| u' \| \| v'\| \\
				& \leq \alpha \| u \|_{H^{1}(\R)} \| \varphi \|_{H^{1}(\R)}
		\end{align*}
		where $\alpha = \max \left\{ 8|\rho| - \mu , 8 |\rho| + 1 \right\}$. \\
		ii) For the coercivity, we first assume $\rho \geq 0$. Now, if $\mu < -1$ we get 
		\begin{align*}
			B[u, u] & = \langle u' , u' \rangle + \rho \sum_{i \in \Z} \left|u(x_{i})\right|^{2} - \mu \langle u , u \rangle \\
					& \geq \langle u' , u' \rangle  + \langle u , u \rangle \\
					& = \| u \|_{H^{1}(\R)}^{2}.
		\intertext{Analogously for $\rho < 0$, using \eqref{preestimation-for-potential} we can choose $h < \frac{1}{2 |\rho|}$ and with that if $\mu < - \frac{2|\rho|}{h}$ there exists $\beta \in \R$ such that}
			B[u, u] & = \langle u' , u' \rangle + \rho \sum_{i \in \Z} |u(x_{i})|^{2} - \mu 	\langle u , u \rangle \\
					& \geq \langle u' , u' \rangle + 2 \rho \sum_{i \in \Z} \left( \frac{1}{h} \int_{\Omega_{i}} |u( x )|^{2} dx + h \int_{\Omega_{i}} \left| u'(\tau) \right|^{2} d\tau \right) - \mu \langle u , u \rangle \\
					& = (2 \rho h + 1) \| u' \|^{2} + (2 \rho \frac{1}{h} - \mu) \| u \|^{2}  \\
					& \geq \beta \| u \|_{H^{1}(\R)}^{2},
		\end{align*}
		where $\beta = \min \left\{ 2 \rho h + 1u , 2 \rho \frac{1}{h} - \mu \right\}$.
	\end{proof}
\end{theorem}

\begin{theorem}
	Given $f \in L^{2}(\R)$ the functional $l_{f}$ is a bounded linear functional on $H^{1}(\R)$.
	% todo INTRODUCTION Cauchy-Schwarz
	\begin{proof}
		It is easily seen that $l_{f}$ is linear, for the boundedness the Cauchy–Schwarz inequality yields
		\[ | l_{f}(v) | \leq \| f \|_{L^{2}(\R)} \| v \|_{H^{1}(\R)} \]
	\end{proof}
\end{theorem}

Therefore, as in theorem \ref{2.1:thm-LaxMilgram} used we will subsequent assume that $\mu \in \R$ is small enough. In return Lax-Migram's Theorem shows that for any fixed $f \in L^{2}(\R)$ a unique solution $u \in H^{1}(\R)$ to the problem \eqref{weak-formulation-of-A-for-LM} exists. This on the other hand allows us to proceed as follows.

\begin{definition}
	Let us define $R_{\mu} \colon L^{2}(\R) \rightarrow L^{2}(\R), f \mapsto u$ with $u$ being the solution of \eqref{weak-formulation-of-A-for-LM}.
\end{definition}
 
Taking in account that $R_{\mu}$ is a linear operator, which is easy to see, there are two more properties of $R_{\mu}$ for us left to show. 

% todo INTRODUCTION H^1 dense in L^2
\begin{theorem} \label{rmuinj}
	The mapping $R_{\mu}$ is bounded and injective.
	
	\begin{proof}
		For $f \in L^{2}(\R)$ there exists $u \in \mathcal{D}(A)$ such that
		\[ \| R_{\mu} f \|_{L^{2}(\R)}^{2} \leq \| u \|_{H^{1}(\R)}^{2}. \] % todo Martin/andrii check
		Now, using \eqref{estimation-for-potential} with a small enough $\mu \in \R$ yields with Cauchy–Schwarz's inequality
		\[ \| R_{\mu} f \|_{L^{2}(\R)}^{2} \leq \left| \int_{\R} |u'|^{2} + \rho \sum_{i \in \Z} |u(x_{i})|^{2} - \mu \int_{\R} |u|^{2} \right|^{2}  \leq \| f \|_{L^{2}(\R)}^{2} \| u \|_{L^{2}(\R)}^{2} \]	
		Taking in mind that $\mathcal{R}(R_{\mu}) \subseteq H^{1}(\R)$, we know that for $u_{1} = u_{2}$
		\begin{equation}
			0 = B_{\mu}[u_{1}, v] - B_{\mu}[u_{2}, v]= \int_{\R} (f_{1} - f_{2}) \overline{v} \quad \forall v \in H^{1}(\R). \label{f1f2almosteverywhere}
		\end{equation} 
		As $H^{1}(\R)$ is dense in $L^{2}(\R)$ this yields that the equality \eqref{f1f2almosteverywhere} holds also for all $v \in L^{2}(\R)$, hence $f_{1} = f_{2}$ almost everywhere. 
	\end{proof}
\end{theorem}



\section{The domain of $A$}

Resulting from theorem \ref{rmuinj}, we know that $R_{\mu}$ is invertible. This allows us to define the aforementioned operator $A$ explicitly.

\begin{definition}
	Let $A \colon \mathcal{D}(A) \subseteq L^{2}(\R) \rightarrow L^{2}(\R)$ be the linear operator defined by
	\[ A \coloneqq R_{\mu}^{-1} + \mu I, \quad \mathcal{D}(A) = \mathcal{R}(R_{\mu}). \]
\end{definition}

Note that this definition makes sense regarding formal definition in \eqref{the-operator-A-formally} and as show below is independent of $\mu$. % todo Andrii: is this enough for 

\begin{remark}
	This allows us to draw the conclusions that $R_{\mu}$ is the resolvent of $A$.
\end{remark}

We will now use the fact that every element $u \in \mathcal{D}(A) = \mathcal{R}(R_{\mu})$ is a solution of \eqref{weak-formulation-of-A-for-LM} to find additional, necessary  characteristics of $u$ or rather $\mathcal{D}(A)$. However, we already know by Lax-Milram's Theorem that $u \in H^{1}(\R)$.
~\\ ~\\
First, let us for the sake of brevity define
\[ H^{2}\Big(\R \setminus \bigcup_{i \in \Z} x_{i} \Big) \coloneqq \Big\{ u \in L^{2}(\R) : u \in \bigcap_{i \in \Z} H^{2}(x_{i}, x_{i+1}), \sum_{i \in \Z} \|u\|^{2}_{H^{2}(x_{i}, x_{i+1})} < \infty \Big\} \]

Then, considering in \eqref{weak-formulation-of-A} any fixed $k \in \Z$ and an arbitrary test function $v \in C^{\infty}(\R)$ with $\supp v \subseteq [x_{k}, x_{k+1}]$ we get 
	\[ \int_{x_{k}}^{x_{k + 1}} u'(x) \overline{v'(x)} dx = \int_{x_{k}}^{x_{k+1}} A u  \overline{v} \iff \int_{x_{k}}^{x_{k+1}} - u(x) \overline{v''(x)} dx = \int_{x_{k}}^{x_{k+1}} A u \overline{v}, \]
whence $- u'' \in L^{2}(x_{k}, x_{k + 1})$ and $A u = - u''$. Since we chose $k \in \Z$  arbitrary we obtain
	$$ \mathcal{D}(A) \subseteq \Big\{ u \in \bigcap_{i \in \Z} H^{2}(x_{i}, x_{i+1}) \Big\}. $$
Next, a test function $v \in C^{\infty}(\R)$ with the property $\supp v = \Omega_{k}$ yields in \eqref{weak-formulation-of-A} for any $k \in \Z$ through integration by parts on both sides of $x_{k}$ that
	\[ -\left( \int_{x_{k}-\frac{1}{2}}^{x_{k}} + \int_{x_{k}}^{x_{k} + \frac{1}{2}}\right) u'' \overline{v} + \left( u'(x_{k}-0) \overline{v(x_{k})} - u'(x_{k} + 0) \overline{v(x_{k})} \right) \\ \]
	\[ +  \rho u(x_{k})\overline{v(x_{k})} = - \int_{x_{k} - \frac{1}{2}}^{x_{k}} u'' \overline{v} - \int_{x_{k}}^{x_{k} + \frac{1}{2}} u'' \overline{v}. \]
Now then choosing in addition $v$ to be non-zero in $x_{k}$ yields 
	\[ u'(x_{k}-0) - u'(x_{k}+0) + \rho u(x_{k}) = 0, \]
and therefore
	\begin{equation}
		\mathcal{D}(A) \subseteq \Big\{ u \in \bigcap_{i \in \Z} H^{2}(x_{i}, x_{i+1}) : u'(x_{i} - 0) - u'(x_{i} + 0) + \rho u(x_{i}) = 0 ~\forall i \in \Z \Big\}.
	\end{equation} 
Last but not least we need one properties. Choosing a function $v \in C_{0}^{\infty}(\R)$ with $\supp v = (x_{-n}, x_{n+1})$ in \eqref{weak-formulation-of-A} yields with partial integration on every interval $(x_{i}, x_{i+1})$ with that
\[ \sum_{i=-n}^{n-1} -\int_{x_{i}}^{x_{i+1}} u'' \overline{v} + \sum_{i=-n}^{n-1} u' v \Big|_{x_{i}}^{x_{i+1}} + \rho \sum_{i=-n}^{n-1} u(x_{i}) \overline{v(x_{j})} - \mu \int_{x_{-n}}^{x_{n}} u \overline{v} = \int_{x_{-n}}^{x_{n}} f \overline{v} \]
\begin{equation} 
	\iff \sum_{i=-n}^{n-1} \int_{x_{i}}^{x_{i+1}} u'' \overline{v} = - \int_{x_{-n}}^{x_{n}} f \overline{v} - \mu \int_{x_{-n}}^{x_{n}} u \overline{v} \label{refwa}
\end{equation} 
By defining $w_{n} \coloneqq \sum_{i=-n}^{n-1} u'' \mathds{1}_{[x_{i}, x_{i+1}]}$ we can estimate the left-hand side of \eqref{refwa} by
\begin{align}
	\left| \langle w_{n}, v \rangle \right| & \leq \left| \mu \int_{x_{-n}}^{x_{n}} u \overline{v} \right| + \left| \int_{x_{-n}}^{x_{n}} f v \right| \notag \\
		& \leq |\mu| \|u\|_{L^{2}(x_{-n}, x_{n})} \|v\|_{L^{2}(x_{-n}, x_{n})} + \|f\|_{L^{2}(x_{-n}, x_{n})} \|v\|_{L^{2}(x_{-n}, x_{n})} \notag \\
		& \leq c \|v\|_{L^{2}(x_{-n}, x_{n})}, \label{refwa2}
\end{align}
for some $c \in \R$. Since $c$ is independent of $n$ we hence know by \eqref{refwa2} that
	\[ \sum_{i \in \Z} \|u''\|^{2}_{L^{2}(x_{i}, x_{i+1})} < \infty. \]
This yields
	\begin{equation}
		\mathcal{D}(A) \subseteq \Big\{ u \in H^{1}(\R): u \in H^{2}\Big(\R \setminus \bigcup_{i \in \Z} x_{i} \Big), u'(x_{j} - 0) - u'(x_{j} + 0) - \rho u(x_{j}) = 0 ~\forall j \Big\}. \label{firstdomaininclusion} 
	\end{equation}
For an arbitrary $u \in \mathcal{D}(A)$ we know hence from \eqref{firstdomaininclusion} that
	\[ A u = \begin{cases}
					- u'' & \text{ on } (x_{k} - \frac{1}{2}, x_{k}) \\
					- u'' & \text{ on } (x_{k}, x_{k} + \frac{1}{2}),
			 \end{cases} \quad \forall k \in \Z. \]

For the reverse inclusion of \eqref{firstdomaininclusion} we use the operator $R_{\mu}$ but first let us, again for brevity, denote with $B$ the right-hand side of \eqref{firstdomaininclusion}. Now, since $\mathcal{R}(R_{\mu}) = \mathcal{D}(A)$, we proceed by proving each $u \in B$ is also in the range of $R_{\mu}$. More specifically, as $\mathcal{D}(R_{\mu}) = L^{2}(\R)$ define $f \coloneqq - u''$ on $(x_{k}, x_{k + 1})$ for all $i \in \Z$; as we know that $u \in H^{2}\Big(\R \setminus \bigcup_{i \in \Z} x_{i} \Big)$ we ensure $f \in L^{2}$. 
Hence, we have to show $u = R_{\mu}(f - \mu u)$ or equivalently
	\begin{align*}
		 \int_{\R} u' \overline{v'} + \rho \sum_{i \in \Z} u(x_{i}) \overline{v(x_{i})} - \mu \int_{\R} u \overline{v}= \int_{\R}(f-\mu u) \overline{v} \\
		\iff \sum_{i \in \Z} \int_{\Omega_{i}} u' \overline{v'} + \rho u(x_{i}) \overline{v(x_{i})} = - \sum_{i \in \Z} \int_{x_{i}}^{x_{i+1}} u'' \overline{v}.
	\end{align*}


	For each $k \in \Z$ partial integration with a function $v \in C_{c}^{\infty}(\R)$ having $\supp v = (x_{k} - \frac{1}{2}, x_{k} + \frac{1}{2})$ yields
	\[ \int_{\Omega_{k}} u' \overline{v'} + \rho u(x_{k}) \overline{v(x_{k})} =\left( \int_{x_{k} + \frac{1}{2}}^{x_{k}} - \int_{x_{k}}^{x_{k} +\frac{1}{2}} \right) u' \overline{v'} - u'(x_{k}-0) \overline{v(x_{k})}  + u'(x_{k}+0) \overline{v(x_{k})}  \]
	\[ \iff u'(x_{k}-0) - u'(x_{k}+0) - \rho u(x_{k}) = 0. \]
	Such that we conclude
	\begin{align*}
		\mathcal{D}(A) & = \Big\{ u \in H^{1}(\R): u \in H^{2}\Big(\R \setminus \bigcup_{i \in \Z} x_{i} \Big), u'(x_{j} - 0) - u'(x_{j} + 0) - \rho u(x_{j}) = 0 ~\forall j \Big\}.
	\end{align*}


\begin{remark}
	The definition of $A$ is independent of $\mu$ since as seen above the domain is independent of $\mu$ and $\mu$-dependent terms cancel each other out.
\end{remark}
 

\section{The operator $A$ is self-adjoint}

In chapter \ref{chap4}, we will use the fact that the operator $A$ self-adjoint, from with follows that $A$ is a closed and symmetric operator. For this purpose we start by showing that $R_{\mu}$ and $R_{\mu}^{-1}$ are symmetric operators.
% INTRODUCTION CLOSE AND SYMMETRIC SELF ADJOINT 
\begin{theorem} \label{2.2:thm-RmuSymmetric}
	$R_{\mu}$ and $R_{\mu}^{-1}$ are symmetric operator.
	
	\begin{proof}
		First, focus on $R_{\mu}^{-1} = (A - \mu I)$. As for all $v \in \mathcal{D}(A)$:
			\begin{align*}
				\langle R_{\mu}^{-1} u, v \rangle & = \langle (A - \mu I) u, v \rangle \\
					& = \int u'\overline{v'} -  \mu \int u \overline{v} + \rho \sum_{i \in \Z} u(x_{i}) \overline{v(x_{i})} \\
					& = \langle u, (A - \mu I) v \rangle = \langle u,  R_{\mu}^{-1} v \rangle,
			\end{align*}

		thus $R_{\mu}^{-1}$ is symmetric. Now, as $\mathcal{D}(R_{\mu}) = L^{2}(\R)$ and $\mathcal{R}(R_{\mu}) = \mathcal{D}(R_{\mu}^{-1})$ for each $f, g \in L^{2}(\R)$ it follows
		
		\[  \langle R_{\mu} f, g \rangle =  \langle R_{\mu} f, R_{\mu}^{-1} R_{\mu} g \rangle = \langle f, R_{\mu} g \rangle \]
		
		such that $R_{\mu}$ is also symmetric.
	\end{proof}
\end{theorem}

Now, using the fact that $R_{\mu}$ and $R_{\mu}^{-1}$ are both symmetric we can  prove that $A$ is self-adjoint. 

\begin{theorem} \label{2.3:thm-ASelfAdjoint}
	$A$ is a self-adjoint operator.
		
	\begin{proof}
		As we already know that $R_{\mu}$ and $R_{\mu}^{-1}$ are symmetric, showing that $R_{\mu}^{-1}$ is self-adjoint is equivalent to showing that if $v \in \mathcal{D}({R_{\mu}^{-1}}^{*})$ and $v^{*} \in L^{2}(\R)$ are such that
		\begin{align}
			\langle R_{\mu}^{-1} u, v \rangle = \langle u, v^{*} \rangle, \quad \forall u \in \mathcal{D}(R_{\mu}^{-1}) \label{*-condition}
		\end{align}
		then $v \in \mathcal{D}(R_{\mu}^{-1})$ and $R_{\mu}^{-1} v = v^{*}$.
		In \eqref{*-condition} we define $u \coloneqq R_{\mu} f$ for any $f \in L^{2}(\R)$ and use the fact that $R_{\mu}$ is symmetric and defined on the whole of $L^{2}(\R)$:
		\[  \langle f, v \rangle = \langle R_{\mu} f, v^{*} \rangle = \langle f, R_{\mu} v^{*} \rangle, \]
		
		which means that $v \in \mathcal{R}(R_{\mu}) = \mathcal{D}(R_{\mu}^{-1})$ and $R_{\mu}^{-1} v = v^{*}$, i.e. $R_{\mu}^{-1}$ is self-adjoint. As the operator $A$ is simply $R_{\mu}^{-1}$ shifted by $\mu \in \R$, $A$ is self-adjoint as well.		
	\end{proof}
\end{theorem}
% todo INTRODUCTION REAL SPECTRUM
Every symmetric operator has an entirely real spectrum, hence theorem \ref{2.3:thm-ASelfAdjoint} yields our first result about the spectrum of $A$.
\chapter{Fundamental domain of periodicity and the Brillouin zone}

Let $\Omega$ be the fundamental domain of periodicity associated with \eqref{the-operator-A-formally}, for simplicity let $\Omega = \Omega_{0}$ and thus $x_{0} = 0$ being the delta-point contained in $\Omega$. As commonly used by literature the reciprocal lattice for $\Omega$ is equal to $[-\pi, \pi]$, the so called one-dimensional Brillouin zone $B$. For fixed $k \in \overline{B}$, consider now the operator $A_{k}$ on $\Omega$ formally defined by the operation
		\[ -\frac{d^{2}}{dx^{2}} + \rho \delta_{x_{0}}. \]
	More precisely, define $A_{k}$ by considering the problem to find for $f \in L^{2}(\Omega)$ a function $u \in H^{1}_{k}$ such that
	\[ \int_{\Omega} u' \overline{v'} + \rho u(x_{0}) \overline{v(x_{0})} - \mu \int_{\Omega} u \overline{v} = \int_{\Omega} f \overline{v} \quad \forall v \in H^{1}_{k}, \]
	where 
	\begin{eqnarray}
		H^{1}_{k} & \coloneqq & \Big\{ \psi \in H^{1}(\Omega): ~ \psi(\frac{1}{2}) = e^{ik} \psi(-\frac{1}{2}) \Big\}. \label{quasi-periodic-condition}	
	\end{eqnarray}

 	Due to the fact that convergence in $H^{1}_{k}$ implies the convergence on the trace of $\Omega$, $H^{1}_{k}$ is a closed subspace of $H^{1}(\R)$ and one can apply the same arguments as above to show that now the operator $R_{\mu, k} \colon L^{2}(\Omega) \rightarrow H^{1}_{k}, f \mapsto u$ is well-defined and define again % todo define operator
		\[ A_{k} \coloneqq R_{\mu, k}^{-1} + \mu, \] 
	such that $R_{\mu, k}$ is the resolvent of $A_{k}$.	
		
\begin{theorem} \label{3.1:thm-R_mu,k.isCompact}
	The operator $R_{\mu, k}$ is compact.

	\begin{proof}
	For each bounded sequence $(f_{j})_{j \geq 1} \in L^{2}(\Omega)$ there exist $(u_{j})_{j \geq 1} \in H^{1}_{k}$ such that
		\[ u_{j} = R_{\mu, k} f_{j} \quad \forall j \geq 1 \]
	and each $u_{j}$ for $j \geq 1$ has to satisfy
		\begin{equation}
			\int_{\Omega} u_{j}' \overline{v'} + \rho u_{j}(x_{0}) \overline{v(x_{0})} - \mu \int_{\Omega} u_{j} \overline{v} = \int f_{j} \overline{v} \quad \forall v \in H^{1}_{k}. \label{ujsatisfy}
		\end{equation} 
	Now, choosing in \eqref{ujsatisfy} $v = u_{j}$ yields with \eqref{estimation-for-potential} for $\mu$ small enough
		\[  \| u_{j} \|_{H^{1}(\Omega)} \leq \| f_{j} \|_{L^{2}(\Omega)} \| u_{j} \|_{L^{2}(\Omega)} \leq c \sqrt{vol(\Omega)} \]
	Which shows that $(u_{j})_{j \geq 1}$ is bounded in $H^{1}(\Omega)$. As $H^1(\Omega) \subset C(\Omega)$ it holds
		\begin{equation}
			|f(x) - f(y)| \leq c |x - y|^{\nicefrac{1}{2}} \text{ for some } c > 0. \label{eq:H1estimation}
		\end{equation}  
		From \eqref{eq:H1estimation} follows for $f \in B_{H^{1}} \coloneqq \{ f \in H^{1}_{k}(\Omega) : \| f \| \leq 1 \}$ that 
		\[ |f(x)|^{2} \leq 2 \| f \|^{2}_{L^{2}} + 2 \leq 4 \quad \forall x \in \Omega. \]
		Now, for $\epsilon > 0$ we partition $\Omega$ into $n_{\epsilon}$ equidistant intervals $I_{k}$, i.e. $\Omega = \bigcup_{j = 1}^{n_{\epsilon}} I_{j}$. As all $f \in B_{H^{1}_{k}}$ are by \eqref{estimation-for-potential} uniformly bounded on $\Omega$, there exist for each subinterval $I_{k}$ a finite number of constants $c_{1}(I_{k}), \dotsc, c_{\nu_{\epsilon}}(I_{k})$ such that 
			$$ \forall f \in B_{H^{1}_{k}} ~\exists j \in \{1, \dotsc, \nu_{\epsilon} \}: \quad |f(\frac{k}{n_{\epsilon}}) - c_{j}(I_{k})| < \frac{1}{n_{\epsilon}} \quad \forall k \in \{ 1 , \dotsc, n_{\epsilon} \}. $$	
		Hence, a simple function $g \in L^{2}(\Omega)$ with function value $c_{k}$ on interval $I_{k}$ would yield
		\begin{align*}
			\| f - g \|^{2}_{L^{2}} & = \sum_{k = 0}^{n-1} \int_{\frac{k}{n}}^{\frac{k+1}{n}} | f(x) - c_{k+1} |^{2} dx \\
				& =  2 \sum_{k = 0}^{n-1} \int_{\frac{k}{n}}^{\frac{k+1}{n}} | f(x) - f(\frac{k}{n}) |^{2} dx +  2 \sum_{k = 0}^{n-1} \int_{\frac{k}{n}}^{\frac{k+1}{n}} | f(\frac{k}{n}) - c_{k+1} |^{2} dx \\
				& \leq 2 \sum_{n = 0}^{n-1} \frac{c}{n^{2}} + 2 \sum_{n=0}^{n-1} \frac{1}{n^{3}} = \frac{2}{n} \left( c + \frac{1}{n} \right) < \epsilon^{2} \text{ for } n \text{ small enough.}
		\end{align*}		 
		This means for all $\epsilon > 0$ there exists a finite set of simple functions $\{ g_{1}, \dotsc, g_{N} \}$ such that for all $f \in B_{H^{1}_{k}}$ there exists a $\nu \in \{1, \dotsc, N\}$ such that $\| f - g_{\nu} \| \leq \epsilon$. Together with the closure of $H^{1}_{k}$ this yields the compact embedding of $H^{1}_{k}$ in $L^{2}(\Omega)$ and thus $R_{\mu, k}$ is compact.
	\end{proof}	
\end{theorem}		

\section{The Spectrum of $A_{k}$}		
As from now, consider the periodic eigenvalue problem
	\begin{equation}
		A_{k} \psi = \lambda \psi \text{ on } \Omega \text{ for } \psi \in H^{1}_{k}. \label{eigv-problem}
	\end{equation}

In writing the boundary condition in \eqref{quasi-periodic-condition}, we understand $\psi$ extended to the whole of $\R$. In fact, \eqref{quasi-periodic-condition} forms boundary conditions on $\partial \Omega$, so-called semi-periodic boundary conditions. \\
	
Since $\Omega$ is bounded, and $R_{\mu, k}$, as resolvent of $A_{k}$, is a compact and symmetric operator, $A_{k}$ has a purely discrete spectrum satisfying	
	\[ \lambda_{1}(k) \leq \lambda_{2}(k) \leq \dotsc \leq \lambda_{s}(k) \rightarrow \infty \text{ as } s \rightarrow \infty. \]
and the corresponding eigenfunction can be chosen such that they depend on $k$ in a measurable way\footnote{see [M. Reed and B. Simon. Methods of modern mathematical physics I–IV]} and that they form a $\langle \cdot , \cdot \rangle$-orthonormal and complete system $(\psi_{s}(\cdot, k))_{s \in \N}$ of eigenfunctions for \eqref{quasi-periodic-condition}.

Now, we want to transform the eigenvalue problem \eqref{eigv-problem} such that the boundary condition is independent from $k$. Define therefore
	\[ \varphi_{s}(x, k) \coloneqq e^{-ikx} \psi_{s}(x, k). \]
Then,
	\begin{align*}
		A_{k} \psi_{s}(x, k) & = \frac{d^{2}}{dx^{2}} \psi_{s}(x, k)|_{(x_{0} - \frac{1}{2}, x_{0})} \cdot \mathds{1}_{(x_{0} - \frac{1}{2}, x_{0})} + \frac{d^{2}}{dx^{2}} \psi_{s}(x, k)|_{(x_{0}, x_{0}  + \frac{1}{2})} \cdot \mathds{1}_{(x_{0}, x_{0} + \frac{1}{2})} \\
				& = e^{ikx} \left( \frac{d^{2}}{dx^{2}} + ik \right)^{2} \varphi_{s}(x, k)|_{(x_{0} - \frac{1}{2}, x_{0})} \cdot \mathds{1}_{(x_{0} - \frac{1}{2}, x_{0})} \\
				& ~\qquad + e^{ikx} \left( \frac{d^{2}}{dx^{2}} + ik \right)^{2} \varphi_{s}(x, k)|_{(x_{0}, x_{0}  + \frac{1}{2})} \cdot \mathds{1}_{(x_{0}, x_{0} + \frac{1}{2})}.
	\end{align*}
Defining the operator $\tilde{A_{k}} \colon D(A_{k}) \rightarrow L^{2}(\R)$ through 
	\[ \tilde{A}_{k} \varphi_{s}(x, k) \coloneqq \begin{cases}
 		\left( \frac{d^{2}}{dx^{2}} + ik \right)^{2} \varphi_{s}(x, k)|_{(x_{0} - \frac{1}{2}, x_{0})} & \text{for } x \in (x_{0} - \frac{1}{2}, x_{0}) \\ \left( \frac{d^{2}}{dx^{2}} + ik \right)^{2} \varphi_{s}(x, k)|_{(x_{0}, x_{0}  + \frac{1}{2})} & \text{for } x \in (x_{0}, x_{0} + \frac{1}{2})
 	\end{cases} \] 
and using \eqref{eigv-problem} and \eqref{quasi-periodic-condition}, gives
		\[ \varphi_{s}(x - \frac{1}{2}, k) = e^{-ik(x - \frac{1}{2})} \psi_{s}(x - \frac{1}{2}, k) = e^{-ik(x + \frac{1}{2})} \psi_{s}(x + \frac{1}{2}, k) = \varphi_{s}(x + \frac{1}{2}, k). \]
Which shows that $(\varphi_{s}(\cdot, k))_{s \in \N}$ is an orthonormal and complete system of eigenfunctions of the periodic eigenvalue problem
	\begin{eqnarray}
		\tilde{A}_{k} \varphi = & \lambda
		 \varphi \text{ on } \Omega, \label{mod-eigv-problem} \\
		 \varphi(x - \frac{1}{2}) = & \varphi(x + \frac{1}{2}). \label{periodic-condition}
	\end{eqnarray}
with the same eigenvalue sequence $(\lambda_{s}(s))_{s \in \N}$ as in \eqref{eigv-problem}. We shall see that the spectrum of the operator $A$ can be constructed from the eigenvalue sequences $(\lambda_{s}(s))_{s \in \N}$ by varying $k$ over the Brillouin zone $B$.\\
	
\section{The Floquet transormation}
An important step towards this aim is the Floquet transformation
	\begin{equation}
		(Uf)(x, k) \coloneqq \frac{1}{\sqrt{|B|}} \sum_{n \in \Z} f(x - n) e^{ikn} \quad (x \in \Omega, k \in B). \label{floquet-transformation}
	\end{equation}
		
\begin{theorem} \label{3.2:thm-UIsometricIsomorphism}
	$ U \colon L^{2}(\R) \rightarrow L^{2}(\Omega \times B)$ is an isometric isomorphism, with inverse
		\begin{equation}
			(U^{-1}g)(x - n) = \frac{1}{\sqrt{|B|}} \int_{B} g(x, k) e^{-ikn} dk \quad (x \in \Omega, n \in \Z). \label{3.8}
		\end{equation} 
	If $g(\cdot, k)$ is extended to the whole of $\R$ by the semi-periodicity condition \eqref{quasi-periodic-condition}, we have
		\begin{equation}
			U^{-1} g = \frac{1}{\sqrt{|B|}} \int_{B} g(\cdot, k) dk. \label{3.9}
		\end{equation}
		
	\begin{proof}
		For $f \in L^{2}(\R)$,
		\begin{equation}
			\int_{\R} |f(x)|^{2} dx = \sum_{n \in \Z} \int_{\Omega} |f(x - n)|^{2} dx. \label{functionoverperiodicity}
		\end{equation} 
		Here, we can exchange summation and integration by Beppo Levi's Theorem. Therefore, 
		\[ \sum_{n \in \Z} |f(x - n)|^{2} < \infty \text{ for a.e. } x \in \Omega.\]
		Thus, $(Uf)(x, k)$ is well-defined by \eqref{floquet-transformation} (as a Fourier series with variable $k$) for a.e. $x \in \Omega$, and Parseval's equality gives, for these $x$,
		\[ \int_{B}|(Uf)(x,k)|^{2} dk = \sum_{n \in \Z} |f(x - n)|^{2}. \]
		By \eqref{functionoverperiodicity}, this expression is in $L^{2}(\Omega)$, and
		\[ \| Uf \|_{L^{2}(\Omega \times B)} = \|f\|_{L^{2}(\R)}. \]
		We are left to show that $U$ is onto, and that $U^{-1}$ is given by \eqref{3.8} or \eqref{3.9}. Let $g \in L^{2}(\Omega \times B)$, and define
		\begin{equation}
			f(x - n) \coloneqq \frac{1}{\sqrt{|B|}} \int_{B} g(x, k) e^{-ikn} dk \quad (x \in \Omega, n \in\Z).\label{3.11}
		\end{equation}
		For fixed $x \in \Omega$, Parseval's Theorem gives
		\[ \sum_{n \in \Z} |f(x - n)|^{2} = \int_{B} |g(x, k)|^{2} dk, \]
		whence, by integration over $\Omega$,
		\begin{eqnarray}
			\int_{\Omega \times B} |g(x, k)|^{2} dx dk & = \int_{\Omega} \sum_{n \in \Z} |f(x - n)|^{2} dx \\
				& = \sum_{n \in\Z} \int_{\Omega} |f(x-n)|^{2} dx \\
				& = \int_{\R} |f(x)|^{2} dx,	
		\end{eqnarray}
		i.e. $f \in L^{2}(\R)$. Now \eqref{floquet-transformation} gives, for a.e. $x \in\Omega$,
		\[ f(x - n) = \frac{1}{\sqrt{|B|}} \int_{B} (Uf)(x,k) e^{-ikn} dk \quad (n \in \Z), \]
		whence \eqref{3.11} implies $U f = g$ and \eqref{3.8}. Now \eqref{3.9} follows from \eqref{3.8} using $g(x + n, k) = e^{ikn} g(x, k)$.
	\end{proof}				
\end{theorem}

\section{Completeness of the Bloch waves}

Using the Floquet transformation $U$, we are now able to prove a completeness property of the Bloch waves $\psi_{s}(\cdot, k)$ in $L^{2}(\Omega)$ when we vary $k$ over the Brillouin zone $B$.
	
\begin{theorem} \label{3.3:thm-flConvergence}
		For each $f \in L^{2}(\R)$ and $l \in \N$, define
			\begin{equation}
				f_{l}(x) \coloneqq \frac{1}{\sqrt{|B|}} \sum_{s=1}^{l} \int_{B} \langle (Uf)(\cdot, k), \psi_{s}(\cdot, k) \rangle_{L^{2}(\Omega)} \psi_{s}(x, K) dk \quad (x \in \R). \label{3.15}
			\end{equation}
		Then, $f_{l} \rightarrow f$ in $L^{2}(\R)$ as $l \rightarrow \infty$.

	\begin{proof}
		Sine $Uf \in L^{2}(\Omega \times B)$, we have $(Uf)(\cdot, k) \in L^{2}(\Omega)$ for a.e. $k \in B$ by Fubini's Theorem. Since $(\psi_{s}(\cdot, k))_{s \in \N}$ is orthonormal and complete in $L^{2}(\Omega)$ for each $k \in B$, we obtain
			\[ \lim_{l \rightarrow \infty} \| (Uf)(\cdot, k) - g_{l}(\cdot, k) \|_{L^{2}(\Omega)} = 0 \text{ for a.e. } k \in B\]
		where 
			\begin{equation}
				g_{l}(x, k) \coloneqq \sum_{s=1}^{l} \langle(Uf)(\cdot, k), \psi_{s}(\cdot,k)\rangle_{L^{2}(\Omega)} \psi_{s}(x,k). \label{3.16}
			\end{equation}
		Thus, for $\chi(k) \coloneqq \| (Uf)(\cdot, k) - g_{l}(\cdot, k) \|^{2}_{L^{2}(\Omega)}$, we get
			\[ \chi_{l}(k) \rightarrow 0 \text{ as } l \rightarrow \infty \text{ for a.e. } k \in B, \]
		and moreover, by Bessel's inequality,
			\[ \chi_{l}(k) \leq \| (Uf)(\cdot, k) \|^{2}_{L^{2}(\Omega)} \text{ for all } l \in \N \text{ and a.e. } k \in B \]
		and $\|(Uf)(\cdot, k)\|^{2}_{L^{2}(\Omega)}$ is in $L^{1}(B)$ as a function of $k$ by Theorem \ref{3.2:thm-UIsometricIsomorphism}. Altogether, Lebesgue's Dominated Convergence theorem implies
			\[ \int_{B} \chi_{l}(k) dk \rightarrow 0 \text{ as } l \rightarrow \infty, \]
		i.e., 
			\begin{equation}
				\| U f - g_{l} \|_{L^{2}(\Omega \times B)} \rightarrow 0 \text{ as } l \rightarrow \infty \label{3.17}
			\end{equation} 
		Using \eqref{3.15}, \eqref{3.16} and \eqref{3.9}, we find that $f_{l} = U^{-1}g_{l}$, whence \eqref{3.17} gives
			\[ \| U(f - f_{l}) \|_{L^{2}(\Omega \times B)} \rightarrow 0 \text{ as } l \rightarrow \infty,\]
		and the assertion follows since $U \colon L^{2}(\R) \rightarrow L^{2}(\Omega \times B)$ is isometric by Lemma \ref{3.2:thm-UIsometricIsomorphism}.
	\end{proof}
\end{theorem}
\chapter{The spectrum of A}	\label{chap4}

In this section, we will prove the main result stating that
	\begin{equation}
		\sigma(A) = \bigcup_{s \in \N} I_{s} \label{MainResult}
	\end{equation}
where $I_{s} \coloneqq \{ \lambda_{s}(k) : k \in \overline{B} \} ~(s \in \N)$. As $B$ is compact and connected, for each of those sets $I_{s}$ holds that
	\begin{equation}
		I_{s} \text{ is a compact real interval for each } s \in \N,\label{Iisacompactrealinterval}
	\end{equation} 
as $\lambda_{s}$ is a continuous function of $k \in \overline{B}$ for all $s \in \N$, which follows by standard arguments from the fact that the coefficients in the transformed eigenvalue problem \eqref{mod-eigv-problem},  \eqref{periodic-condition} depend continuously on $k$.

Moreover, Poincare's min-max principle for eigenvalues implies that $\mu_{s} \leq \lambda_{s}(k) \text{ for all } s \in \N, k \in \overline{B}$ with $(\mu_{s})_{s \in \N}$ denoting the sequence of eigenvalues of problem \eqref{eigv-problem} with Neumann (``free'') boundary conditions. Since $\mu_{s} \rightarrow \infty$ as $s \rightarrow \infty$, we obtain 
	\[ \min I_{s} \rightarrow \infty \text{ as } s \rightarrow \infty, \]
which together with \eqref{Iisacompactrealinterval} implies that
	\begin{equation}
		\bigcup_{s \in \N} I_{s} \text{ is close.} \label{UIclosed}
	\end{equation} 
	
The first part of the statement \eqref{MainResult} is 

\begin{theorem} \label{4.1:thm-MainResult.FirstInclusion}
	$\sigma(A) \supset \bigcup_{s \in \N} I_{s}.$
	
	\begin{proof}
		Let $\lambda \in \bigcup_{s \in \N} I_{s}$, i.e. $\lambda = \lambda_{s}(k)$ for some $s \in \N$ and some $k \in \overline{B}$, and 
		\begin{equation}
			A_{k} \psi_{s}(\cdot, k) = \lambda \psi_{s}(\cdot, k) \label{firstinclusion-firstequation} % todo Is the k in the index is correct?
		\end{equation}
		We regard $\psi_{s}(\cdot, k)$ as extended to the whole of $\R$ by the boundary condition \eqref{quasi-periodic-condition}, whence, due to the periodicity of $A$, \eqref{firstinclusion-firstequation} holds for all $x \in \R$ and $\psi_{s} \in H^{2}_{loc}(\R)$ \\
		We choose a function $\eta \in H^{2}(\R)$ such that
			\[ \eta(x) = 1 \text{ for } |x| \leq \frac{1}{4}, \quad \eta(x) = 0 \text{ for } |x| \geq \frac{1}{2}, \]
		and define, for each $l \in \N$,
			\[ u_{l}(x) \coloneqq \eta\left(\frac{|x|}{l}\right) \psi_{s}(x, k). \]
	 	Then,
		\begin{align}
			(A - \lambda I) u_{l} & = \sum_{j \in \N} \left[ (- \frac{d^{2}}{dx^{2}} - \lambda) u_{l}|_{(x_{j}, x_{j+1})} \cdot \mathds{1}_{(x_{j}, x_{j+1})} \right] \label{eq:sepofspectraleq} \\
				& = \sum_{j \in \N} \left[ \left(- \frac{d^{2}}{dx^{2}} - \lambda \right) \left( \eta\left(\frac{|\cdot|}{l}\right) \psi_{s}(\cdot, k) \right)\Big|_{(x_{j}, x_{j+1})} \cdot \mathds{1}_{(x_{j}, x_{j+1})} \right] \notag \\
				& ~\qquad - \frac{2}{l} \sum_{j \in \N} \left[ \left( \eta'\left(\frac{|\cdot|}{l}\right) \psi_{s}'(\cdot, k) \right)\big|_{(x_{j}, x_{j+1})} \cdot \mathds{1}_{(x_{j}, x_{j+1})}  \right] \notag \\
				& ~\qquad - \frac{1}{l^{2}} \sum_{j \in \N} \left[ \left( \eta''\left(\frac{|\cdot|}{l}\right) \psi_{s}(\cdot, k) \right)\big|_{(x_{j}, x_{j+1})} \cdot \mathds{1}_{(x_{j}, x_{j+1})} \right] \notag \\
				& = \sum_{j \in \N} \left[ \eta\left(\frac{|\cdot|}{l}\right) \left(- \frac{d^{2}}{dx^{2}} - \lambda \right) \psi_{s}(\cdot, k) |_{(x_{j}, x_{j+1})} \cdot \mathds{1}_{(x_{j}, x_{j+1})} \right] + R \notag
		\end{align}
		where $R$ is a sum of products of derivatives (of order $\geq 1$) of $\eta(\frac{|\cdot|}{l})$, and derivatives (of order $\leq 1$) of $\psi_{s}(\cdot, k)$. Thus (note that $\psi_{s}(\cdot, k) \in H^{2}_{loc}(\R)$), and the semi-periodic structure of $\psi_{s}(\cdot, k)$ implies
		\begin{equation}
			 \| R \| \leq \frac{c}{l} \| \psi_{s}(\cdot, k) \|_{H^{1}(K_{l})} \leq c \frac{1}{\sqrt{l}}, \label{eq:estimofR}
		\end{equation}
		with $K_{l}$ denoting the ball in $\R$ with radius $l$ centered at $x_{0}$. Together with \eqref{firstinclusion-firstequation}, \eqref{eq:sepofspectraleq} and \eqref{eq:estimofR}, this gives
		\[ \| (A - \lambda I) u_{l} \| \leq \frac{c}{\sqrt{l}} \]
		Again, by the semiperiodicity of $\psi_{s}(\cdot, k)$,
		\[ \| u_{l} \| \geq c \| \psi_{s}(\cdot, k) \| \geq c \sqrt{l} \]
		with $c > 0$. We obtain therefore
		\[ \frac{1}{\|u_{l}\|}\| (A - \lambda I) u_{l} \| \leq \frac{c}{l} \]
		Because moreover $u_{l} \in D(A)$, this results in
			\[ \frac{1}{\|u_{l} \|} \| (A - \lambda I) u_{l} \| \rightarrow 0 \text{ as } l \rightarrow \infty \]
		Thus, either $\lambda$ is an eigenvalue of $A$, or $(A - \lambda I)^{-1}$ exists but is unbounded. In both cases, $\lambda \in \sigma(A)$.
	\end{proof}
\end{theorem}	

	
\begin{theorem} \label{4.1:thm-MainResult.SecondInclusion}
	$\sigma(A) \subset \bigcup_{s \in \N} I_{s}.$

	\begin{proof}
		Let $\lambda \in \R \setminus \bigcup_{s \in \N} I_{s}$, we have to prove that $\lambda \in \rho(A)$, i.e. that for each $f \in L^{2}(\R)$ some $u \in D(A)$ exists satisfying $(A-\lambda I)u = f$. For given $f \in L^{2}(\R)$, we define, for $l \in \N$, 
			\[ f_{l}(x) \coloneqq \frac{1}{\sqrt{|B|}} \sum_{s=1}^{l} \int_{B} \langle (Uf)(\cdot, k), \psi_{s}(\cdot, k)\rangle_{L^{2}(\Omega)} \psi_{s}(x,k) dk \]
			and
			\begin{equation}
				u_{l} \coloneqq \frac{1}{\sqrt{|B|}} \sum_{s=1}^{l} \int_{B} \frac{1}{\lambda_{s}(k) - \lambda} \langle (Uf)(\cdot, k), \psi_{s}(\cdot, k)\rangle_{L^{2}(\Omega)} \psi_{s}(x, k) dk \label{ul}
			\end{equation} 
		Here, note that, due to \eqref{UIclosed} some $\delta > 0$ exists such that
			\begin{equation}
				|\lambda_{s}(k) - \lambda| \geq \delta \text{ for all } s \in \N, k \in B \label{lambda-distance}
			\end{equation}

		In particular, consider for fixed $k \in B$ and $v \in \mathcal{D}(A_{k})$:
		\begin{equation}
			(A_{k} - \lambda I) v(\cdot, k) = (Uf)(\cdot, k) \text{ on } \Omega, \label{4.9}			
		\end{equation}
		which has a unique solution as $\lambda \in \R \setminus \bigcup_{s \in \N} I_{s}$. Parseval gives
		\begin{align*}
			\| (Uf)(\cdot, k)\|^{2}_{L^{2}(\Omega)} & = \sum_{s=1}^{\infty} |\langle (Uf)(\cdot, k), \psi_{s}(\cdot, k)\rangle|^{2} \\
			& = \sum_{s=1}^{\infty}|\langle (A - \lambda) v(\cdot, k), \psi_{s}(\cdot, k)\rangle_{L^{2}(\Omega)}|^{2}
		\end{align*}

		Since both $v(\cdot, k)$ and $\psi_{s}(\cdot, k)$ satisfy semi-periodic boundary conditions, $A - \lambda I$ can be moved to $\psi_{s}(\cdot, k)$ in the inner product, and hence \eqref{eigv-problem} and \eqref{lambda-distance} give
		\begin{align*}
			\| (Uf)(\cdot,k)\|^{2}_{L^{2}(\Omega)} & = \sum_{s=1}^{\infty} |\lambda_{s}(k) - \lambda|^{2} |\langle v(\cdot, k), \psi_{s}(\cdot, k)\rangle_{L^{2}(\Omega)}|^{2} \\
			& \geq \delta^{2} \| v(\cdot, k)\|^{2}_{L^{2}(\Omega)}
		\end{align*}
		By Theorem \ref{3.2:thm-UIsometricIsomorphism}, this implies $v \in L^{2}(\Omega \times B)$, and we can define $u \coloneqq U^{-1} v \in L^{2}(\R)$. Thus, \eqref{4.9} gives
			\begin{align*}
				\langle (Uf)(\cdot, k), \psi_{s}(\cdot, k) \rangle_{L^{2}(\Omega)} & = \langle (A - \lambda I)(Uu)(\cdot, k), \psi_{s}(\cdot, k) \rangle_{L^{2}(\Omega)} \\
					& = \langle (Uu)(\cdot,k), (A - \lambda I) \psi_{s}(\cdot, k) \rangle_{L^{2}(\Omega)} \\
					& = (\lambda_{s}(k) - \lambda) \langle Uu(\cdot, k), \psi_{s}(\cdot, k) \rangle_{L^{2}(\Omega)}
			\end{align*}
		whence \eqref{ul} implies
			\[ u_{l}(x) = \frac{1}{\sqrt{|B|}} \sum_{s=1}^{l} \int \langle (Uu)(\cdot, k), \psi_{s}(\cdot, k)\rangle_{L^{2}(\Omega)} \psi_{s}(x, k) dk, \]
		and Theorem \ref{3.3:thm-flConvergence} gives
			\begin{equation}
				u_{l} \rightarrow u, \quad f_{l} \rightarrow f \quad \text{ in } L^{2}(\R). \label{ulflconvergence}
			\end{equation}
		We will now prove that in the distributional sense 
		\begin{equation}
				(A - \lambda I) u_{l} = f_{l} \text{ for all } l \in \N \label{lefttoprove}
			\end{equation} 
		which implies that $\langle u_{l}, (A - \lambda I) v \rangle = \langle f_{l}, v\rangle$ for all $v \in D(A)$, whence Theorem \ref{3.16} implies $u_{l} \in D(A)$, and
			\[ (A - \lambda I) u_{l} = f_{l} \quad \forall l \in \N \]
		Since $A$ is closed, \eqref{ulflconvergence} now implies
			\[ u \in D(A), \text{ and } (A - \lambda I) u = f \]
		which is the desired result.
		
		Left to prove is \eqref{lefttoprove}, i.e. that
			\begin{equation}
				\langle u_{l} , (A - \lambda I) \varphi \rangle_{L^{2}(\R)} = \langle f_{l},\varphi \rangle_{L^{2}(\R)} \quad \forall \varphi \in C^{\infty}_{0}(\R). \label{lefttoshow2}
			\end{equation} 
		Let $\varphi \in C_{0}^{\infty}(\R)$ be fixed, and let $K \subseteq \R$ denote an open interval containing $\supp(\varphi)$ in its interior. Both the functions
		\begin{align*}
			r_{s}(x, k) & \coloneqq \frac{1}{\lambda_{s}(k) - \lambda} \langle (Uf)(\cdot, k), \psi_{s}(\cdot, k) \rangle_{L^{2}(\Omega)} \psi_{s}(x, k) \overline{(A - \lambda I) \varphi(x)}, \\
			t_{s}(x, k) & \coloneqq \langle (Uf)(\cdot, k), \psi_{s}(\cdot, k) \rangle_{L^{2}(\Omega)} \psi_{s}(x, k) \overline{\varphi(x)}
		\end{align*}
		are in $L^{2}(K \times B)$ by Fubini's THeorem, since \eqref{lambda-distance} and the fact that $(A_{k} - \lambda I) \varphi \in L^{\infty}(K)$ and $\varphi \in L^{\infty}(K)$, imply both
		\begin{align*}
			\| r_{s}(x, k) \|_{L^{2}(K \times B)} & \leq c \| (Uf)(\cdot, k) \|^{2}_{L^{2}(\Omega)} \| \psi_{s}(\cdot, k) \|^{2}_{L^{2}(K)} 
		\intertext{and}
			\| t_{s}(x, k) \|_{L^{2}(K \times B)} & \leq \tilde{c} \| (Uf)(\cdot, k) \|^{2}_{L^{2}(\Omega)} \| \psi_{s}(\cdot, k) \|^{2}_{L^{2}(K)},.			
		\end{align*}
		the latter factor is bounded as a function of $k$ because $K$ is covered by a finite number of copies of $\Omega$, and the former is in $L^{1}(B)$ by Theorem \ref{3.2:thm-UIsometricIsomorphism}.
		
		Since $K \times B$ is bounded, $r$ and $t$ are also in $L^{1}(K \times B)$. Therefore, Fubini’s Theorem implies that the order of integration with respect to $x$ and $l$ may be exchanged for $r$ and $t$. Thus, by \eqref{ul},
			\begin{align*}
				\int_{K} u_{l}(x) \overline{(A - \lambda I) \varphi(x)} dx & = \frac{1}{\sqrt{|B|}} \sum_{s=1}^{l} \int_{K} \left( \int_{B} r_{s}(x, k) dk \right) dx \\
					& = \frac{1}{\sqrt{|B|}} \sum_{s=1}^{l} \int_{B} \frac{1}{\lambda_{s}(k) - \lambda} \langle (Uf)(\cdot, k), \psi_{s}(\cdot, k) \rangle_{L^{2}(\Omega)} \\
					& ~\qquad ~\qquad ~\qquad ~\qquad \langle \psi_{s}(\cdot, k), (A - \lambda I) \varphi \rangle_{L^{2}(K)} dk. 
			\end{align*}
			Since $\varphi$ has compact support in the interior of $K$, $(A - \lambda I)$ may be moved to $\psi_{s}(\cdot, k)$, and hence \eqref{eigv-problem} gives % todo I don't see how the compact support is needed. Convergence?
			\begin{align*}
				\int_{K} u_{l}(x) \overline{(A - \lambda I) \varphi(x)} dx					& = \frac{1}{\sqrt{|B|}} \sum_{s=1}^{l} \int_{B} \langle (Uf)(\cdot, k), \psi_{s}(\cdot, k) \rangle_{L^{2}(\Omega)} \langle \psi_{s}(\cdot, k), \varphi \rangle_{L^{2}(K)} dk \\
				 	& = \frac{1}{\sqrt{|B|}} \sum_{s=1}^{l} \int_{B} \left( \int_{K} t_{s}(x, k) dx \right) dk \\
					& = \int_{K} \left[ \frac{1}{\sqrt{|B|}} \sum_{s=1}^{l} \int_{B} \langle (Uf)(\cdot, k), \psi_{s}(\cdot, k) \rangle_{L^{2}(\Omega)} \psi_{s}(x, k) dk \right] \overline{\varphi(x)} dx \\
					& = \int_{K} f_{l}(x) \overline{\varphi(x)} dx,
			\end{align*}
			i.e. \eqref{lefttoshow2}.
	\end{proof}
\end{theorem}

\appendix
%\chapter*{Appendix} \addcontentsline{toc}{chapter}{Appendix} 

\begin{atheorem}[Alternative definition of the Delta-Distribution] \label{athem:delta}
	For the sequence of functionals $\delta_{\epsilon}$ for $\epsilon > 0$, $x_{0} \in \R$ and $f \in \mathfrak{D}(\R)$ defined through $\delta_{\epsilon}(f) \coloneqq \frac{1}{\sqrt{2 \pi} \epsilon} \int_{\R} e^{-\frac{(x - x_{0})^{2}}{2 \epsilon^{2}}} f(x) dx$ it holds that  
 		\[ \delta_{x_{0}}(f) = \lim_{\epsilon \rightarrow 0} \delta_{\epsilon}(f). \]
	
	\begin{proof}
		Given $\delta_{\epsilon}$ and $x_{0} \in \R$ we have
			\[ \delta_{\epsilon}(f) = \frac{1}{\sqrt{2 \pi} \epsilon} \int_{-\infty}^{\infty} f(x) e^{-\frac{(x-x_{0})^{2}}{2 \epsilon^{2}} dx} \]
		Substituting $z \coloneqq \frac{x - x_{0}}{\sqrt{2} \epsilon}$ yields
			\[ \frac{1}{\sqrt{2 \pi} \epsilon} \int_{-\infty}^{\infty} f(x) e^{-\frac{(x-x_{0})^{2}}{2 \epsilon^{2}} dx} = \frac{1}{\sqrt{2 \pi} \epsilon} \int_{-\infty}^{\infty} f(\sqrt{2} \epsilon z + x_{0}) e^{-z^{2}} dz \]
		Using the Taylor series around $x_{0}$ and the Gaussian integral we then get
			\[ \lim_{\epsilon \rightarrow 0} \frac{1}{\sqrt{\pi}} \int_{-\infty}^{\infty} e^{-z^{2}} \left( f(x_{0}) + \mathcal{O}(\epsilon) \right) = f(x_{0}) \frac{1}{\sqrt{\pi}} \int_{-\infty}^{\infty} e^{-z^{2}} = f(x_{0}). \]
	\end{proof}
\end{atheorem}

\begin{atheorem}[$L^{p}$-Approximation by test functions]
	For $U \subseteq \R^{n}$ open, $C_{0}^{\infty}(U)$ is dense in $L^{p}(U)$, if $1 \leq p < \infty$.
	
	\begin{proof}
		See \cite[p. 31]{adams2003sobolev}.
	\end{proof}
\end{atheorem}

\begin{atheorem}[$H^{k}$-Approximation by test functions] 
	Let $\Omega \subseteq \R^{n}$ be an open set. Let $u \in H^{k}(\Omega)$, then there exists a sequence of functions $u_{k} \in C^{\infty}(\Omega)$ such that $\| u_{k} - u \|_{H^{k}(\Omega)}$ as $n \rightarrow \infty$.
	
	\begin{proof}
		See \cite[p. 138]{adams2003sobolev}.
	\end{proof}	
\end{atheorem}


\begin{atheorem}[Bessel's inequality]
	Let $H$ be a Hilbert space, and suppose that $e_1, e_2, ...$ is an orthonormal sequence in $H$. Then, for any $x \in H$ one has
	\[ \sum_{k=1}^{\infty}\left\vert\left\langle x,e_k\right\rangle \right\vert^2 \le \left\Vert x\right\Vert^2 \]
	where $\langle \cdot,\cdot \rangle$ denotes the inner product in the Hilbert space $H$.

	\begin{proof}
		See \cite[p. 233]{werner2006funkana}.
	\end{proof}
\end{atheorem}

\begin{atheorem}[Cauchy–Schwarz inequality]
	Fr all vectors $u$ and $v$ of an inner product space it is true that
		\[ \left| \langle u,v \rangle \right|^{2} \leq \langle u,u \rangle \cdot \langle v,v \rangle, \]
	where $\langle \cdot ,\cdot \rangle$ is the inner product. 

	\begin{proof}
		See \cite[p. 20]{werner2006funkana}.
	\end{proof}
\end{atheorem}

\begin{atheorem}[Closed graph theorem]
	Let $X$ be a Banach space. Is $A$ a closed operator and $\mathcal{D}(A) = X$, then $A$ is continuous on $X$.

	\begin{proof}
		See \cite[p. 417]{evans1998partial}.
	\end{proof}
\end{atheorem}

\begin{atheorem}[Closeness of $H^{1}_{k}$ in $H^{1}(\Omega)$] \label{h1kclosed}
	$H^{1}_{k}$ is a closed subset1 of $H^{1}(\Omega)$, and therefore a Hilbert space with respect to the norm of $H^{1}(\Omega)$.
	
	\begin{proof} 
		Let $(f_{n})_{n \in \N}$ be a sequence in $H^{1}_{k}$ converging to $f \in H^{1}(\Omega)$. We already know that convergence with respect to the $H^{1}$-Norm implies convergence of the function and its derivative almost everywhere. Let us therefore define $g \coloneqq f - f_{n}$ then
		\begin{align*}
			\left| g \left(- \frac{1}{2} \right) \right|^{2} & \leq 2 |g(x)|^{2} + 2 \left( \int_{-\frac{1}{2}}^{x} |g'(\tau)| d\tau \right)^{2} \\
			& \leq 2 |g(x)|^{2} + 2 \int_{-\frac{1}{2}}^{\frac{1}{2}} |g'(\tau)|^{2} d\tau \\
			& \leq 2 \int_{-\frac{1}{2}}^{\frac{1}{2}} |g(\tau)|^{2} d\tau + 2 \int_{-\frac{1}{2}}^{\frac{1}{2}} |g'(\tau)|^{2} d\tau \\
			& = 2 \| g \|_{H^{1}(-\frac{1}{2}, \frac{1}{2})}^{2} \longrightarrow 0
		\end{align*}
		for $j \rightarrow \infty$, and analogously on the other boundary.
	\end{proof}
\end{atheorem} % todo proof

\begin{atheorem}[Compact Embedding Theorem for Sobolev spaces] \label{compact-embedding-theorem} ~\
	\begin{enumerate}[label=\alph*\upshape)]
		\item Let $U \subseteq \R^{n}$ be a bounded open set of class $C^{1}$. Then the following compact embeddings hold:
			\begin{itemize}
				\item $H	^{1}(U) \subseteq L^{q}(U)$ for every $q \in [1, p^{*})$, where $n \geq 3$ and $p^{*} = \frac{2n}{n - 2}$.
				\item $H^{1}(U) \subseteq L^{q}(U)$ for every $q \in [1, \infty)$, if $n = 2$.
			\end{itemize}
		
			\begin{proof}
				Follows from Rellich-Kondrachov Compact Embedding Theorem, see \cite[p. 163]{precup2013linear} and \cite[p. 272]{evans1998partial}.
			\end{proof}
		\item Let $U \subseteq \R$ be a bounded, connected and open set. Then the embedding $H^{1}(U) \subseteq L^{2}(U)$ is compact.
			\begin{proof} 
				As $H^{1}(U) \subseteq C^{\frac{1}{2}}(U)$ we can estimate
					\[ |f(x) - f(y)| \leq c |x - y|^{\frac{1}{2}} \]
				for some $c > 0$ and for all $x, y \in U$. Let $B_{H^{1}_{k}} \coloneqq \{ f \in H^{1}_{k}(U) : \|f\|_{H^{1}(U)} \leq 1 \}$, then for $f \in B_{H^{1}_{k}}$ it holds that
				\begin{equation}
					|f(x)|^{2} \leq 2 \| f\|^{2}_{L^{2}(U)} + 2 \leq 4 \quad \forall x \in U. \label{eqbounded}
				\end{equation} 
				For an arbitrary $\epsilon > 0$ we now partition $U$ into $n_{\epsilon}$ equidistant, disjoint intervals $I_{k}$, i.e. $U = \bigcup_{k = 1}^{n_{\epsilon}} I_{k}$. Since all $f \in B_{H^{1}_{k}}$ are uniformly bounded on $U$ by \eqref{eqbounded}, there exist for each subinterval $I_{k}$ a finite number of constants $c_{1, k}, \dotsc, c_{\nu_{\epsilon}, k}$ such that
					\[ \forall f \in B_{H^{1}_{k}} ~\exists f \in \{1, \dotsc, \nu_{\epsilon} : \left| f\left(\frac{k}{n_{\epsilon}}\right) - c_{j, k} \right| < \epsilon ~\forall k \in \{ 1, \dotsc, n_{\epsilon} \}. \]
				Hence, there are finitely many step functions such that for any $f \in L^{2}(U)$ there exists one of those step functions $g \in L^{2}(U)$, with function value $c_{k}$ on subinterval $I_{k}$ for each $k \in \{1, \dotsc, n_{\epsilon}\}$, such that
				\begin{align*}
					\| f - g\|^{2}_{L^{2}(U)} & = \sum_{k=0}^{n-1} \int_{\frac{k}{n}}^{\frac{k+1}{n}} |f(x) - c_{k+1}|^{2} dx \\
					& \leq 2 \sum_{k=0}^{n-1} \int_{\frac{k}{n}}^{\frac{k+1}{n}} \left|f(x) - f\left(\frac{k}{n}\right)\right|^{2} dx +   \sum_{k=0}^{n-1} \int_{\frac{k}{n}}^{\frac{k+1}{n}} 2 \left| f\left(\frac{k}{n}\right) - c_{k+1} \right|^{2} dx \\ 
					& \leq 2 \sum_{n = 0}^{n-1} \frac{c}{n^{2}} + 2 \sum_{n=0}^{n-1} \frac{1}{n^{3}} = \frac{2}{n^{2}} \left( c + \frac{1}{n} \right) < \epsilon^{2}
				\end{align*}
				for $n$ large enough. This means, in conclusion, that $B_{H^{1}_{k}}$ is totally bounded in $L^{2}(U)$ and in return $H^{1}_{k}$ can be compactly embedded in $L^{2}(U)$.
			\end{proof}
	\end{enumerate}
\end{atheorem}

\begin{atheorem}[Dominated Convergence Theorem]
	Let ${f_n}$ be a sequence of real-valued measurable functions on a measure space $(S, \Sigma, \mu)$. Suppose that the sequence converges pointwise to a function $f$ and is dominated by some integrable function $g$ in the sense that
		\[ |f_n(x)| \le g(x) \]
	for all numbers n in the index set of the sequence and all points $x \in S$. Then f is integrable and
		\[ \lim_{n\to\infty} \int_S |f_n-f|\,d\mu = 0 \]
	which also implies $\lim_{n\to\infty} \int_S f_n\,d\mu = \int_S f\,d\mu$.

	\begin{proof}
		See \cite[p. 516]{werner2006funkana}.
	\end{proof}
\end{atheorem}

\begin{atheorem}[Eigenvectors of a compact, symmetric operator]
	Let $H$ be a separable Hilbert space, and suppose $S \colon H \rightarrow H$ is a compact and symmetric operator. Then there exists a countable orthonormal basis of $H$ consisting of eigenvectors of $S$.
	
	\begin{proof}
		See \cite[p. 645]{evans1998partial}.
	\end{proof}
\end{atheorem}

\begin{atheorem}[Embedding of $H^{1}$ in $C^{\frac{1}{2}}$]
	Let $[a, b]$ be a compact interval in $\R$. Then, $H^{1}([a, b])$ is embedded in $C^{\frac{1}{2}}([a, b])$
	
	\begin{proof}
		See \cite[p. 269]{evans1998partial}.
	\end{proof}
\end{atheorem}

\begin{atheorem}[Equivalent definitions of closed operators]
	Let X, Y be two Banach spaces. A linear operator $A \colon X \supset \mathcal{D}(A)  \rightarrow Y$ is closed if for every sequence $(x_n)_{n \in \N}$ in $\mathcal{D}(A)$ from
	\[ x_{n} \rightarrow x \in X \text{ and } Tx_{n} \rightarrow y \in Y \]
	follows that $x \in D$ and $Tx = y$.

	\begin{proof}
		 See \cite[p. 156]{werner2006funkana}.
	\end{proof}
\end{atheorem}	

\begin{atheorem}[Fubini's theorem for integrable functions]
	Suppose $X$ and $Y$ are $\sigma$-finite measure spaces, and suppose that $X \times Y$ is given the product measure (which is unique as $X$ and $Y$ are $\sigma$-finite). Fubini's theorem states that if $f(x,y)$ is $X \times Y$ integrable, meaning that it is measurable and
		\[  \int_{X\times Y} |f(x,y)|\,\text{d}(x,y)<\infty, \]
	then
		\[ \int_X\left(\int_Y f(x,y)\,\text{d}y\right)\,\text{d}x=\int_Y\left(\int_X f(x,y)\,\text{d}x\right)\,\text{d}y=\int_{X\times Y} f(x,y)\,\text{d}(x,y). \]
	The first two integrals are iterated integrals with respect to two measures, respectively, and the third is an integral with respect to the product measure

	\begin{proof}
		See \cite[p. 514]{werner2006funkana}.
	\end{proof}
\end{atheorem}

\begin{atheorem}[Lax-Milgram]
	Let $H$ be a Hilbert space where $\| \cdot \|$ denotes the norm on $H$, and let $B \colon H \times H \rightarrow \C$ be a sesquilinear form. If there exist constants $\alpha, \beta > 0$ such that
	\begin{enumerate}[label=\alph*\upshape)]
		\item $\left| B[u, v] \right| \leq \alpha \| u \| \|v \| \quad (u, v \in H)$ and
		\item $Re(B[u,u]) \geq \beta \|u\|^{2} \quad (u \in H)$,
	\end{enumerate}
	then there exists to each $l \in H^{*}$ a unique $w \in H$ such that
		\[ B[v, w] = l(v) \]
	hold for all $v \in H$.
		
	\begin{proof}
		See \cite[Amd to problem 51]{plum2015dglhr}.
	\end{proof}
\end{atheorem}

\begin{atheorem}[Monotone Convergence Theorem]
	Let $(X, \Sigma, \mu)$ be a measure space. Let $f_1, f_2, \ldots$  be a pointwise non-decreasing sequence of $[0, \infty]$-valued $\Sigma$–measurable functions, i.e. for every $k \geq 1$ and every $x$ in $X$,
		\[ 0 \leq f_k(x) \leq f_{k+1}(x). \] 
	Next, set the pointwise limit of the sequence $(f_{n})$ to be $f$. That is, for every $x$ in $X$,
		\[ f(x):= \lim_{k\to\infty} f_k(x). \]
	Then $f$ is $\Sigma$–measurable and
		\[ \lim_{k\to\infty} \int f_k \, \mathrm{d}\mu = \int f \, \mathrm{d}\mu. \]

	\begin{proof}
		See \cite[p. 516]{werner2006funkana}.
	\end{proof}
\end{atheorem}

\begin{atheorem}[Orthonormality of $\vartheta$]
	The sequence
		\[ \vartheta_{n}(k) \coloneqq \frac{1}{\sqrt{|B|}} e^{ikn} \]
	forms an orthonormal basis of $L^{2}(B)$.

	\begin{proof}
		 For $m, n \in \N$ we see that
		 \[ \langle \vartheta_{n}, \vartheta_{m} \rangle_{L^{2}(B)} = \frac{1}{|B|} \int_{B} e^{ikn} \overline{e^{ikm}} dk = \frac{1}{|B|} \int_{B} e^{ik(n-m)} dk = \begin{cases} 0 & \text{ for } n \neq m \\ 1 & \text{ for } n = m, \end{cases} \]
		 hence the asserted follows.
	\end{proof}
\end{atheorem}

\begin{atheorem}[Parseval's identity]
	Suppose that $H$ is a Hilbert space with inner product $\langle \cdot,\cdot \rangle$. Let $(e_{n})$ be an orthonormal basis of $H$; i.e., the linear span of the $e_n$ is dense in $H$, and the $e_n$ are mutually orthonormal:
		\[ \langle e_{m},e_{n}\rangle ={\begin{cases}1&{\mbox{if}}\ m=n\\0&{\mbox{if}}\ m\not =n.\end{cases}} \]
	Then Parseval's identity asserts that for every $x \in H$,
		\[ \sum _{n}|\langle x,e_{n}\rangle |^{2}=\|x\|^{2}.\]

	\begin{proof}
		See \cite[p. 236]{werner2006funkana}.
	\end{proof}
\end{atheorem}

\begin{atheorem}[Poincare's min-max principle for eigenvalues]
	Let $X$ be a seperable Hilbert space and $\langle \cdot, \cdot \rangle_{X}$ denote the scalar product on $X$. Let $A \colon \mathcal{D}(A) \rightarrow X$ be a self-adjoint operator where $\mathcal{D}(A) \subseteq X$. If the set of eigenvalues $\lambda_{s}$ is at most countable, then
	\begin{equation}
			\lambda_{s} = \underset{\dim U = s}{\min_{U \subseteq \mathcal{D}(A)}} \max_{v \in U \setminus \{ 0 \} } \frac{\langle A v, v \rangle_{X}}{\langle v, v \rangle_{X}}.  \label{poincare} 
	\end{equation} 

	\begin{proof}
		See \cite[p. 119]{teschl2014mathematical}.
	\end{proof}
\end{atheorem}

\begin{atheorem}[Properties of self-adjoint operators] ~\
	\begin{enumerate}[label=\alph*\upshape)]
		\item Every self-adjoint is symmetric and closed.
			\begin{proof}
				 Follows directly from the definitions for self-adjoint and closed operators.
			\end{proof}
		\item For $A$ being a self-adjoint operator, $\lambda \in \rho(A)$, $(A - \lambda I)^{-1}$ is bounded.
			\begin{proof}
				Since every self-adjoint is closed, $(A - \lambda I)$ is as the shift with $\lambda \in \R$ also closed. Furthermore, the graph of $(A - \lambda I)^{-1}$ is simply the graph of $(A - \lambda I)$ rotated and hence $(A - \lambda I)^{-1}$ is closed as well. The closed Graph Theorem now yields the desired result.
			\end{proof}
	\end{enumerate}
\end{atheorem}

\begin{atheorem}[Properties of the set of resolvent values]
	The resolvent set $\rho(A) \subseteq \mathbb{C}$ of a bounded linear operator $A$ is an open set.
	
	\begin{proof}
		See \cite[p. 259]{werner2006funkana}.
	\end{proof}
\end{atheorem}

\begin{atheorem}[Riesz' representation theorem]
	Let $H$ be a Hilbert space, and let $H^{*}$ denote its dual space, consisting of all continuous linear functionals from $H$ into $\R$ or $\C$. If $x$ is an element of $H$, then the function $\varphi_{x}$, for all $y$ in $H$ defined by
	\[ \varphi_{x}(y) = \left\langle y,x\right\rangle_{H}, \]
	where $\langle \cdot ,\cdot \rangle_{H}$ denotes the inner product of the Hilbert space, is an element of $H^{*}$. Hence, every element of $H^{*}$ can be written uniquely in this form.
	
	\begin{proof}
		See \cite[p. 284]{evans1998partial}
	\end{proof}
\end{atheorem}

\begin{atheorem}[The spectrum of self-adjoint operators] \label{spectrul-sa-real}
	The spectrum of a self-adjoint operator $A$ is real. 
	
	\begin{proof}
		Let $\lambda$ be an eigenvalue of $A$, i.e. there exists $x \in X$ such that $A x = \lambda x$. From this it follows that $\langle A x, x \rangle = \langle \lambda x , x \rangle$. Using then the fact that $A$ is self-adjoint we can further deduce
		\[ \lambda \langle x , x \rangle = \langle \lambda x , x \rangle = \langle A x, x \rangle = \langle x, A x \rangle = \langle x , \lambda x \rangle = \overline{\lambda} \langle  x , x \rangle \]
		Hence, $\lambda = \overline{\lambda}$, which shows the desired result.
	\end{proof}
\end{atheorem}

\begin{atheorem}[Trace Theorem]
	Assume $U$ is bounded and $\partial U$ is $C^{1}$. Then there exists a bounded linear operator
		\[ T \colon H^{1}(U) \rightarrow L^{2}(\partial U) \]
	such that
	\begin{enumerate}[label=\alph*\upshape)]
		\item $Tu = u\big|_{\partial U}$ if $u \in H^{1}(U) \cap C(\overline{U})$
		\item $\|Tu\|_{L^{2}(\partial U)} \leq C \|u\|_{H^{1}(U)}$
	\end{enumerate}
	for each $u \in H^{1}(U)$, with the constant $C$ depending only on $U$.
	
	\begin{proof}
		See \cite[p. 258]{evans1998partial}.
	\end{proof}
\end{atheorem}

\begin{atheorem}[Uniqueness of weak derivatives] \label{athem:uniqueness_of_weak_deriv}
	Let $\Omega \subseteq \R$ be open, if it exists, the $\alpha$-th weak derivative of $u$ is uniquely determined up to a set of measure zero.
	
	\begin{proof}
		Assume that $g, \tilde{g} \in L_{loc}^{1}(\Omega)$ satisfy for all $\varphi \in C_{0}^{\infty}(\Omega)$
		\[ (-1)^{\alpha} \int_{\Omega} f \varphi' = \int_{\Omega} g \varphi  = \int_{\Omega} \tilde{g} \varphi. \]
		Then $\int_{\Omega} \left( g - \tilde{g} \right) \varphi = 0$ for all $\varphi \in C_{0}^{\infty}(\Omega)$, whence $g - \tilde{g} = 0$ almost everywhere.	
	\end{proof}
\end{atheorem}  
\begin{thebibliography}{00000}%{Lam00}
	\bibitem{00000} a
  % Literaturbeispiel: Buch
  %\bibitem[Knu85]{Knuth:1985} Donald E.~Knuth: {\it The
  %  TeXbook}. Addison-Wesley, 1985. 
  % Literaturbeispiel: Paper
  %\bibitem[Lam00]{Lamport:2000} Leslie Lamport: {\it How (La)TeX
   % changed the face of Mathematics}. Mitteilungen der Deutschen
    % Mathematiker-Vereinigung, 1/2000.
\end{thebibliography}
 
      
  % ggf. hier Tabelle mit Symbolen 
  % (kann auch auf das Inhaltsverzeichnis folgen)
\thispagestyle{empty}


\vspace*{8cm}


\section*{Decleration}

 I declare that I have developed and written the enclosed thesis completely by myself, have not used sources or means without declaration in the text and designated the included passage from other works, whether in substance or in principle, as such and that I adhered the statute of the Karlsruhe Institute of Technology for good scientific practice in their currently valid version.


~\\[2ex] 
\noindent
Karlsruhe, den 13. September 2016
~\\[5ex]
% Unterschrift (handgeschrieben)


\end{document}