\chapter{Fundamental domain of periodicity and the Brillouin zone}

Let $\Omega$ be the fundamental domain of periodicity associated with \eqref{the-operator-A-formally}, for simplicity let $\Omega \coloneqq \Omega_{0}$ and thus $x_{0} = 0$ being contained in $\Omega$. As commonly used in literature the reciprocal lattice for $\Omega$ is $[-\pi, \pi]$, the so called one-dimensional Brillouin zone $B$. For fixed $k \in \overline{B}$, in this chapter we consider the operator $A_{k}$ on $\Omega$ formally defined by the operation 
	\[ -\frac{d^{2}}{dx^{2}} + \rho \delta_{x_{0}}. \]
As before, we want to define $A_{k}$ by considering the weak formulation to the corresponding problem, i.e. for $f \in L^{2}(\Omega)$ finding $u \in H^{1}_{k}$ such that the equation
	\[ \int_{\Omega} u' \overline{v'} + \rho u(x_{0}) \overline{v(x_{0})} - \mu \int_{\Omega} u \overline{v} = \int_{\Omega} f \overline{v} \]
holds for all $v \in H^{1}_{k}$ where 
	\begin{eqnarray}
		H^{1}_{k} & \coloneqq & \Big\{ \psi \in H^{1}(\Omega): ~ \psi(\frac{1}{2}) = e^{ik} \psi(-\frac{1}{2}) \Big\}. \label{quasi-periodic-condition}
	\end{eqnarray}
Due to the fact that convergence in $H^{1}_{k}$ implies the convergence on the boundary of $\Omega$, $H^{1}_{k}$ is a closed subspace of $H^{1}(\R)$, and one can apply analogous arguments as above to prove that $R_{\mu, k} \colon L^{2}(\Omega) \rightarrow H^{1}_{k}, f \mapsto u$ is well-defined and injective. Hence we define similarly
	\[ A_{k} \coloneqq R_{\mu, k}^{-1} + \mu, \] 
and see therefore that $R_{\mu, k}$ is the resolvent of $A_{k}$.	
~\\ ~\\
This chapter is going to provide additional information about the operator $R_{\mu, k}$. We will see that the eigenfunctions of $A_{k}$ form a complete and orthonormal system in $H^{1}_{k}$. Using this fact we can then deduce additional properties about the spectrum of $A_{k}$ and $A$ in chapter \ref{chap4}.

\begin{theorem} \label{3.1:thm-Rmuk.isCompact}
	The operator $R_{\mu, k}$ is compact.

	\begin{proof}
	For each bounded sequence $(f_{j})_{j} \in L^{2}(\Omega)$ we can define
		\[ u_{j} \coloneqq R_{\mu, k} f_{j} \quad \text{ for all } j \geq 1. \]
	Each such $u_{j}$ is by definition in $H^{1}_{k}$ and has to satisfies 
		\begin{equation}
			\int_{\Omega} u_{j}' \overline{v'} + \rho u_{j}(x_{0}) \overline{v(x_{0})} - \mu \int_{\Omega} u_{j} \overline{v} = \int f_{j} \overline{v} \quad \forall v \in H^{1}_{k}. \label{ujsatisfy}
		\end{equation} 
	Now, choosing in \eqref{ujsatisfy} $v = u_{j}$ yields with \eqref{estimation-for-potential} for $\mu$ small enough
		\[  \| u_{j} \|_{H^{1}(\Omega)} \leq \| f_{j} \|_{L^{2}(\Omega)} \| u_{j} \|_{L^{2}(\Omega)} \leq c \sqrt{vol(\Omega)} \]
	Which means that $(u_{j})_{j}$ is bounded in $H^{1}(\Omega)$. As $H^1(\Omega) \subseteq C^{\frac{1}{2}}(\overline{\Omega})$ we can further estimate 
		\begin{equation}
			|f(x) - f(y)| \leq c |x - y|^{\frac{1}{2}} \text{ for some } c > 0, \label{eq:H1estimation}
		\end{equation}
	from which for $f \in B_{H^{1}_{k}} \coloneqq \{ f \in H^{1}_{k}(\Omega) : \| f \|_{H^{1}(\Omega)} \leq 1 \}$ it follows that 
		\[ |f(x)|^{2} \leq 2 \| f \|_{L^{2}}^{2} + 2 \leq 4 \quad \forall x \in \Omega. \]
	For an arbitrary $\epsilon > 0$ we now partition $\Omega$ into $n_{\epsilon}$ equidistant, disjoint intervals $I_{k}$, i.e. $\Omega = \bigcup_{j = 1}^{n_{\epsilon}} I_{j}$. As all $f \in B_{H^{1}_{k}}$ are uniformly bounded on $\Omega$ by \eqref{estimation-for-potential}, there exist for each subinterval $I_{k}$ a finite number of constants $c_{1, k}, \dotsc, c_{\nu_{\epsilon}, k}$ such that 
			$$ \forall f \in B_{H^{1}_{k}} ~\exists j \in \{1, \dotsc, \nu_{\epsilon} \}: \quad \left|f\left(\frac{k}{n_{\epsilon}}\right) - c_{j, k}\right| < \epsilon \quad \forall k \in \{ 1 , \dotsc, n_{\epsilon} \}. $$	
	Hence, there are finitely many step functions such that for any $f \in L^{2}(\Omega)$ one of those step functions, let's call it $g \in L^{2}(\Omega)$ with function value $c_{k}$ on sub interval $I_{k}$ for each $k \in \{ 1, \dotsc, n_{\epsilon} \}$, such that
		\begin{align*}
			\| f - g \|^{2}_{L^{2}(\Omega)} & = \sum_{k = 0}^{n-1} \int_{\frac{k}{n}}^{\frac{k+1}{n}} | f(x) - c_{k+1} |^{2} dx \\
				& =  2 \sum_{k = 0}^{n-1} \int_{\frac{k}{n}}^{\frac{k+1}{n}} | f(x) - f(\frac{k}{n}) |^{2} dx +  2 \sum_{k = 0}^{n-1} \int_{\frac{k}{n}}^{\frac{k+1}{n}} | f(\frac{k}{n}) - c_{k+1} |^{2} dx \\
				& \leq 2 \sum_{n = 0}^{n-1} \frac{c}{n^{2}} + 2 \sum_{n=0}^{n-1} \frac{1}{n^{3}} = \frac{2}{n} \left( c + \frac{1}{n} \right) < \epsilon^{2} \text{ for } n \text{ large enough.}
		\end{align*}		 
	This means in conclusion that $B_{H^{1}_{k}}$ is totally bounded in $L^{2}(\Omega)$. Furthermore, $H^{1}_{k}$ is a closed subset of $H^{1}(\Omega)$ as convergence in $H^{1}(\R)$ implies convergence on the boundary, and this yields the compact embedding of $H^{1}_{k}$ in $L^{2}(\Omega)$. Thus, the operator $R_{\mu, k}$ is compact.
	\end{proof}	
\end{theorem}		

\section{The spectrum of the operator $A_{k}$}	
As from now, consider the eigenvalue problem to find $\psi \in H^{1}_{k}$ such that
	\begin{equation}
		A_{k} \psi = \lambda \psi \text{ on } \Omega. \label{eigv-problem}
	\end{equation}
	
In writing the boundary condition in \eqref{quasi-periodic-condition}, we understand $\psi$ extended to the whole of $\R$. In fact, \eqref{quasi-periodic-condition} forms boundary conditions on $\partial \Omega$, so-called semi-periodic boundary conditions. 
~\\
Obviously, $A_{k}$ has the same sequence of eigenfunctions as $R_{\mu, k}$, and if $\tilde{\lambda}$ is an eigenvalue for the eigenfunction $\psi$ of $R_{\mu, k}$ then so is
	\[ \lambda = \frac{1}{\tilde{\lambda}} - \mu \]
 an eigenvalue for the same eigenfunction $\psi$ for the operator $A$. Since $\Omega$ is bounded, and $R_{\mu, k}$ is a compact and symmetric operator, $A_{k}$ has also a purely discrete spectrum satisfying	
	\[ \lambda_{1}(k) \leq \lambda_{2}(k) \leq \dotsc \leq \lambda_{s}(k) \rightarrow \infty \text{ as } s \rightarrow \infty. \]
and the corresponding eigenfunction form a $\langle \cdot , \cdot \rangle$-orthonormal and complete system $(\psi_{s}(\cdot, k))_{s \in \N}$ of eigenfunctions for \eqref{quasi-periodic-condition}. Therefore, we transform the eigenvalue problem \eqref{eigv-problem} such that the boundary condition is independent from $k$, define
	\[ \varphi_{s}(x, k) \coloneqq e^{-ikx} \psi_{s}(x, k). \]
Then,
	\begin{align*}
		A_{k} \psi_{s}(x, k) & = \frac{d^{2}}{dx^{2}} \psi_{s}(x, k)|_{(x_{0} - \frac{1}{2}, x_{0})} \cdot \mathds{1}_{(x_{0} - \frac{1}{2}, x_{0})} + \frac{d^{2}}{dx^{2}} \psi_{s}(x, k)|_{(x_{0}, x_{0}  + \frac{1}{2})} \cdot \mathds{1}_{(x_{0}, x_{0} + \frac{1}{2})} \\
				& = e^{ikx} \left( \frac{d}{dx} + ik \right)^{2} \varphi_{s}(x, k)|_{(x_{0} - \frac{1}{2}, x_{0})} \cdot \mathds{1}_{(x_{0} - \frac{1}{2}, x_{0})} \\
				& ~\qquad + e^{ikx} \left( \frac{d}{dx} + ik \right)^{2} \varphi_{s}(x, k)|_{(x_{0}, x_{0}  + \frac{1}{2})} \cdot \mathds{1}_{(x_{0}, x_{0} + \frac{1}{2})}.
	\end{align*}
Defining the operator $\tilde{A_{k}} \colon \mathcal{D}(A_{k}) \rightarrow L^{2}(\R)$ through 
	\[ \tilde{A}_{k} \varphi_{s}(x, k) \coloneqq \begin{cases}
 		\left( \frac{d}{dx} + ik \right)^{2} \varphi_{s}(x, k)|_{(x_{0} - \frac{1}{2}, x_{0})} & \text{for } x \in (x_{0} - \frac{1}{2}, x_{0}) \\ \left( \frac{d}{dx} + ik \right)^{2} \varphi_{s}(x, k)|_{(x_{0}, x_{0}  + \frac{1}{2})} & \text{for } x \in (x_{0}, x_{0} + \frac{1}{2})
 	\end{cases} \] 
and using \eqref{eigv-problem} and \eqref{quasi-periodic-condition}, hence yields
		\[ \varphi_{s}(x - \frac{1}{2}, k) = e^{-ik(x - \frac{1}{2})} \psi_{s}(x - \frac{1}{2}, k) = e^{-ik(x + \frac{1}{2})} \psi_{s}(x + \frac{1}{2}, k) = \varphi_{s}(x + \frac{1}{2}, k). \]
Which shows that $(\varphi_{s}(\cdot, k))_{s \in \N}$ is an orthonormal and complete system of eigenfunctions of the periodic eigenvalue problem
	\begin{equation}
		\tilde{A}_{k} \varphi = \lambda_{s}(k)
		 \varphi \text{ on } \Omega, \label{mod-eigv-problem}
	\end{equation}
	\begin{equation}
		 \varphi(x - \frac{1}{2}) = \varphi(x + \frac{1}{2}). \label{periodic-condition}
	\end{equation}
with the identical eigenvalue sequence $(\lambda_{s}(s))_{s \in \N}$ as in \eqref{eigv-problem}.
~\\ ~\\
In the next chapter we are going to show that the spectrum of the operator $A$ can be constructed through the eigenvalue sequences $(\lambda_{s}(k))_{s \in \N}$ by varying $k$ over the Brillouin zone $B$. For that we need two results involving the Floquet transformation, which carries the from $L^{2}(\R)$ to $L^{2}(\Omega \times B)$ whereas $\Omega \times B$ is by assumption compact. Even though the following two results do not differ in both the statement or the proof from standard theory, as in \cite{Plum10}, I still want to list them here for completeness.
\section{The Floquet transformation} 	
\begin{theorem} \label{3.2:thm-UIsometricIsomorphism}
	The Floquet transformation $U \colon L^{2}(\R) \rightarrow L^{2}(\Omega \times B)$ 
	\begin{equation}
		(Uf)(x, k) \coloneqq \frac{1}{\sqrt{|B|}} \sum_{n \in \Z} f(x - n) e^{ikn} \quad (x \in \Omega, k \in B). \label{floquet-transformation}
	\end{equation}
	is an isometric isomorphism, with inverse given by
		\begin{equation}
			(U^{-1}g)(x - n) = \frac{1}{\sqrt{|B|}} \int_{B} g(x, k) e^{-ikn} dk \quad (x \in \Omega, n \in \Z). \label{3.8}
		\end{equation} 
	If $g(\cdot, k)$ is extended to the whole of $\R$ by the semi-periodicity condition \eqref{quasi-periodic-condition}, the inverse simplifies to
		\begin{equation}
			U^{-1} g = \frac{1}{\sqrt{|B|}} \int_{B} g(\cdot, k) dk. \label{3.9}
		\end{equation}
		
	\begin{proof}
		For $f \in L^{2}(\R)$,
		\begin{equation}
			\int_{\R} |f(x)|^{2} dx = \sum_{n \in \Z} \int_{\Omega} |f(x - n)|^{2} dx,\label{functionoverperiodicity}
		\end{equation} 
		where we used Beppo Levi's Theorem to exchange summation and integration. This shows that
		\[ \sum_{n \in \Z} |f(x - n)|^{2} < \infty \text{ for almost every } x \in \Omega.\]
		Thus, $(Uf)(x, k)$ is well-defined by \eqref{floquet-transformation} (as a Fourier series with variable $k$) for almost every $x \in \Omega$, and Parseval's equality gives for these $x$
		\[ \int_{B}|(Uf)(x,k)|^{2} dk = \sum_{n \in \Z} |f(x - n)|^{2}. \]
	 	This expression is in $L^{2}(\Omega)$ by \eqref{functionoverperiodicity}, and
		\[ \| Uf \|_{L^{2}(\Omega \times B)} = \|f\|_{L^{2}(\R)}. \]
		
		We still haven't shown that $U$ is onto, and that $U^{-1}$ is given by \eqref{3.8} or \eqref{3.9}. Let $g \in L^{2}(\Omega \times B)$, then define
		\begin{equation}
			f(x - n) \coloneqq \frac{1}{\sqrt{|B|}} \int_{B} g(x, k) e^{-ikn} dk \quad (x \in \Omega, n \in\Z).\label{3.11}
		\end{equation}
		Parseval's Theorem gives for fixed $x \in \Omega$ that $\sum_{n \in \Z} |f(x - n)|^{2} = \int_{B} |g(x, k)|^{2} dk$. Integrating over $\Omega$ then yields
		\begin{align*}
			\int_{\Omega \times B} |g(x, k)|^{2} dx dk & = \int_{\Omega} \sum_{n \in \Z} |f(x - n)|^{2} dx  = \sum_{n \in\Z} \int_{\Omega} |f(x-n)|^{2} dx = \int_{\R} |f(x)|^{2} dx,	
		\end{align*}
		which means $f \in L^{2}(\R)$. For almost every $x \in \Omega$ \eqref{floquet-transformation} gives
		\[ f(x - n) = \frac{1}{\sqrt{|B|}} \int_{B} (Uf)(x,k) e^{-ikn} dk \quad (n \in \Z), \]
		whence \eqref{3.11} implies $U f = g$ and \eqref{3.8}. Now \eqref{3.9} follows from \eqref{3.8} and exploiting $g(x + n, k) = e^{ikn} g(x, k)$.
	\end{proof}				
\end{theorem}

\section{Completeness of the Bloch waves}

Using the Floquet transformation $U$, we can now prove the property of completeness of the Bloch waves $\psi_{s}(\cdot, k)$ in $L^{2}(\Omega)$ when we vary $k$ over the Brillouin zone $B$.
	
\begin{theorem} \label{3.3:thm-flConvergence}
		For each $f \in L^{2}(\R)$ and $l \in \N$, define
			\begin{equation}
				f_{l}(x) \coloneqq \frac{1}{\sqrt{|B|}} \sum_{s=1}^{l} \int_{B} \langle (Uf)(\cdot, k), \psi_{s}(\cdot, k) \rangle_{L^{2}(\Omega)} \psi_{s}(x, k) dk \quad (x \in \R). \label{3.15}
			\end{equation}
		Then, $f_{l} \rightarrow f$ in $L^{2}(\R)$ as $l \rightarrow \infty$.

	\begin{proof}
		The last theorem tells us that $Uf \in L^{2}(\Omega \times B)$, which in return means that $(Uf)(\cdot, k) \in L^{2}(\Omega)$ for almost all $k \in B$ by Fubini's Theorem. As $(\psi_{s}(\cdot, k))_{s \in \N}$ is an orthonormal and complete system of eigenfunctions in $L^{2}(\Omega)$ for each $k \in B$, we derive
			\[ \lim_{l \rightarrow \infty} \| (Uf)(\cdot, k) - g_{l}(\cdot, k) \|_{L^{2}(\Omega)} = 0 \text{ for almost every } k \in B \]
		where 
			\begin{equation}
				g_{l}(x, k) \coloneqq \sum_{s=1}^{l} \langle(Uf)(\cdot, k), \psi_{s}(\cdot,k)\rangle_{L^{2}(\Omega)} \psi_{s}(x,k). \label{3.16}
			\end{equation}
		Moreover, we get by Bessel's inequality
			\[ \| (Uf)(\cdot, k) - g_{l}(\cdot, k) \|^{2}_{L^{2}(\Omega)} \leq \| (Uf)(\cdot, k) \|^{2}_{L^{2}(\Omega)}  \]
		for all $l \in \N$ and almost every $k \in B$. Next, $\|(Uf)(\cdot, k)\|^{2}_{L^{2}(\Omega)} \in L^{1}(B)$ as a function of $k$ by Theorem \ref{3.2:thm-UIsometricIsomorphism}, thus by Lebesgue's Dominated Convergence theorem
		\[ \lim_{l \rightarrow \infty} \int_{B} \| (Uf)(\cdot, k) - g_{l}(\cdot, k) \|^{2}_{L^{2}(\Omega)} dk  = \int_{B} \lim_{l \rightarrow \infty}  \| (Uf)(\cdot, k) - g_{l}(\cdot, k) \|^{2}_{L^{2}(\Omega)} dk = 0. \]
		  All in all, this means
			\begin{equation}
				\| U f - g_{l} \|_{L^{2}(\Omega \times B)} \rightarrow 0 \text{ as } l \rightarrow \infty \label{3.17}
			\end{equation} 
		 If $g(\cdot, k)$ is extended to the whole of $\R$ by the semi-periodicity condition \eqref{quasi-periodic-condition}, using \eqref{3.15}, \eqref{3.16} and \eqref{3.9}, we find that $f_{l} = U^{-1}g_{l}$, whence \eqref{3.17} gives
			\[ \| U(f - f_{l}) \|_{L^{2}(\Omega \times B)} \rightarrow 0 \text{ as } l \rightarrow \infty,\]
	\end{proof}
\end{theorem}