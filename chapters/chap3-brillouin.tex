\chapter{Fundamental domain of periodicity and the Brillouin zone}  \label{chap3}

In this chapter we will restrict the Kronig-Penney to one periodicity cell and examine the spectrum of the resulting operator. Solving the eigenvalue problem on the period cell while varying boundary conditions can be used to determine the eigenvalues of the unrestricted problem, which is exactly the approach we will use in chapter \ref{chap4}.
~\\ ~\\
Let $\Omega$ be the fundamental domain of periodicity associated with \eqref{the-operator-A-formally}, for simplicity let $\Omega \coloneqq \Omega_{0}$ and thus $x_{0} = 0$ being contained in $\Omega$. As commonly used in literature the reciprocal lattice for $\Omega$ is $[-\pi, \pi]$, the so called one-dimensional Brillouin zone $B$. For fixed $k \in \overline{B}$, in this chapter we consider the operator $A_{k}$ on $\Omega$ formally defined by the operation
	\[ -\frac{d^{2}}{dx^{2}} + \rho \delta_{x_{0}}. \]
\begin{definition} 
	For every $k$ we define for brevity the set of quasi-periodic functions  
	\begin{equation}
		H^{1}_{k} \coloneqq \Big\{ \psi \in H^{1}(\Omega): ~ \psi(\frac{1}{2}) = e^{ik} \psi(-\frac{1}{2}) \Big\}. \label{quasi-periodic-condition}		
	\end{equation}
\end{definition}	
\begin{remark}
Due to the fact that convergence in $H^{1}_{k}$ implies the convergence on the boundary of $\Omega$, $H^{1}_{k}$ is a closed subspace of $H^{1}(\R)$.	
\end{remark}
Analogous to before, we define $A_{k}$ by considering the problem of finding for $f \in L^{2}(\Omega)$ a function $u \in H^{1}_{k}$ such that the equation
	\[ \int_{\Omega} u' \overline{v'} + \rho u(x_{0}) \overline{v(x_{0})} - \mu \int_{\Omega} u \overline{v} = \int_{\Omega} f \overline{v} \]
holds for all $v \in H^{1}_{k}$. One can apply the same arguments as above to prove that 
	\[ R_{\mu, k} \colon L^{2}(\Omega) \rightarrow H^{1}_{k}, f \mapsto u \]
is well-defined and injective. Consequently, we define now
	\[ A_{k} \coloneqq R_{\mu, k}^{-1} + \mu I. \] 

This chapter is now going to provide additional information about the operator $R_{\mu, k}$. We shall see that $R_{\mu, k}$ and that the eigenfunctions of $A_{k}$ form a complete and orthonormal system in $H^{1}_{k}$. 

\section{The compactness of  $R_{\mu, k}$}

\begin{theorem} \label{3.1:thm-Rmuk.isCompact}
	The operator $R_{\mu, k}$ is compact.

	\begin{proof}
	Let $(f_{j})_{j} \in L^{2}(\Omega)$ be a  bounded sequence. We will show that 
		\[ u_{j} \coloneqq R_{\mu, k} f_{j} \quad \text{ for all } j \geq 1. \]
	is bounded a bounded sequence as well. Each such $u_{j}$ is by definition in $H^{1}_{k}$ and has to satisfies 
		\begin{equation}
			\int_{\Omega} u_{j}' \overline{v'} + \rho u_{j}(x_{0}) \overline{v(x_{0})} - \mu \int_{\Omega} u_{j} \overline{v} = \int f_{j} \overline{v} \quad \forall v \in H^{1}_{k}. \label{ujsatisfy}
		\end{equation} 
	Now, the particular choice of $v = u_{j}$ in \eqref{ujsatisfy} yields with \eqref{estimation-for-potential} for $\mu$ small enough
		\[  \| u_{j} \|_{H^{1}(\Omega)} \leq \| f_{j} \|_{L^{2}(\Omega)} \| u_{j} \|_{L^{2}(\Omega)} \leq c \sqrt{vol(\Omega)}. \]
	The compact embedding follows from the subsequently show estimation. As $H^1(\Omega) \subseteq C^{\frac{1}{2}}(\overline{\Omega})$ we can further estimate 
		\begin{equation}
			|f(x) - f(y)| \leq c |x - y|^{\frac{1}{2}} \text{ for some } c > 0, \label{eq:H1estimation}
		\end{equation}
	from which for $f \in B_{H^{1}_{k}} \coloneqq \{ f \in H^{1}_{k}(\Omega) : \| f \|_{H^{1}(\Omega)} \leq 1 \}$ it follows that 
		\[ |f(x)|^{2} \leq 2 \| f \|_{L^{2}}^{2} + 2 \leq 4 \quad \forall x \in \Omega. \]
	For an arbitrary $\epsilon > 0$ we now partition $\Omega$ into $n_{\epsilon}$ equidistant, disjoint intervals $I_{k}$, i.e. $\Omega = \bigcup_{j = 1}^{n_{\epsilon}} I_{j}$. As all $f \in B_{H^{1}_{k}}$ are uniformly bounded on $\Omega$ by \eqref{estimation-for-potential}, there exist for each subinterval $I_{k}$ a finite number of constants $c_{1, k}, \dotsc, c_{\nu_{\epsilon}, k}$ such that 
			$$ \forall f \in B_{H^{1}_{k}} ~\exists j \in \{1, \dotsc, \nu_{\epsilon} \}: \quad \left|f\left(\frac{k}{n_{\epsilon}}\right) - c_{j, k}\right| < \epsilon \quad \forall k \in \{ 1 , \dotsc, n_{\epsilon} \}. $$	
	Hence, there are finitely many step functions such that for any $f \in L^{2}(\Omega)$ one of those step functions, let's call it $g \in L^{2}(\Omega)$ with function value $c_{k}$ on sub interval $I_{k}$ for each $k \in \{ 1, \dotsc, n_{\epsilon} \}$, such that
		\begin{align*}
			\| f - g \|^{2}_{L^{2}(\Omega)} & = \sum_{k = 0}^{n-1} \int_{\frac{k}{n}}^{\frac{k+1}{n}} | f(x) - c_{k+1} |^{2} dx \\
				& \leq  2 \sum_{k = 0}^{n-1} \int_{\frac{k}{n}}^{\frac{k+1}{n}} | f(x) - f(\frac{k}{n}) |^{2} dx +  2 \sum_{k = 0}^{n-1} \int_{\frac{k}{n}}^{\frac{k+1}{n}} | f(\frac{k}{n}) - c_{k+1} |^{2} dx \\
				& \leq 2 \sum_{n = 0}^{n-1} \frac{c}{n^{2}} + 2 \sum_{n=0}^{n-1} \frac{1}{n^{3}} = \frac{2}{n} \left( c + \frac{1}{n} \right) < \epsilon^{2} \text{ for } n \text{ large enough.}
		\end{align*}		 
	This means in conclusion that $B_{H^{1}_{k}}$ is totally bounded in $L^{2}(\Omega)$ and in return $H^{1}_{k}$ can be compactly embedded in $L^{2}(\Omega)$. Thus, the operator $R_{\mu, k}$ is compact.
	\end{proof}	
\end{theorem}		

\section{The spectrum of the operator $A_{k}$}	
As from now, consider the eigenvalue problem to find $\psi \in H^{1}_{k}$ such that
	\begin{equation}
		A_{k} \psi = \lambda \psi \text{ on } \Omega. \label{eigv-problem}
	\end{equation}
	
In writing the boundary condition in \eqref{quasi-periodic-condition}, we understand $\psi$ extended to the whole of $\R$. In fact, \eqref{quasi-periodic-condition} forms boundary conditions on $\partial \Omega$, so-called semi-periodic boundary conditions. 
~\\
Obviously, $A_{k}$ has the same sequence of eigenfunctions as $R_{\mu, k}$, and if $\tilde{\lambda}$ is an eigenvalue for the eigenfunction $\psi$ of $R_{\mu, k}$ then respectively is
	\[ \lambda = \frac{1}{\tilde{\lambda}} - \mu \]
 an eigenvalue for the same eigenfunction $\psi$ for the operator $A$, for proof see \cite{WeisFA}. Since $\Omega$ is bounded, and $R_{\mu, k}$ is a compact and symmetric operator, $A_{k}$ has moreover a purely discrete spectrum satisfying	
	\[ \lambda_{1}(k) \leq \lambda_{2}(k) \leq \dotsc \leq \lambda_{s}(k) \rightarrow \infty \text{ as } s \rightarrow \infty. \]
and the corresponding eigenfunction form a $\langle \cdot , \cdot \rangle$-orthonormal and complete system $(\psi_{s}(\cdot, k))_{s \in \N}$ of eigenfunctions for \eqref{quasi-periodic-condition}. We transform the eigenvalue problem \eqref{eigv-problem} such that the boundary condition is independent from $k$, define
	\[ \varphi_{s}(x, k) \coloneqq e^{-ikx} \psi_{s}(x, k). \]
Then,
	\begin{align*}
		A_{k} \psi_{s}(x, k) & = \frac{d^{2}}{dx^{2}} \psi_{s}(x, k)|_{(x_{0} - \frac{1}{2}, x_{0})} \cdot \mathds{1}_{(x_{0} - \frac{1}{2}, x_{0})} + \frac{d^{2}}{dx^{2}} \psi_{s}(x, k)|_{(x_{0}, x_{0}  + \frac{1}{2})} \cdot \mathds{1}_{(x_{0}, x_{0} + \frac{1}{2})} \\
				& = e^{ikx} \left( \frac{d}{dx} + ik \right)^{2} \varphi_{s}(x, k)|_{(x_{0} - \frac{1}{2}, x_{0})} \cdot \mathds{1}_{(x_{0} - \frac{1}{2}, x_{0})} \\
				& ~\qquad + e^{ikx} \left( \frac{d}{dx} + ik \right)^{2} \varphi_{s}(x, k)|_{(x_{0}, x_{0}  + \frac{1}{2})} \cdot \mathds{1}_{(x_{0}, x_{0} + \frac{1}{2})}.
	\end{align*}
Defining the operator $\tilde{A_{k}} \colon \mathcal{D}(A_{k}) \rightarrow L^{2}(\R)$ through 
	\[ \tilde{A}_{k} \varphi_{s}(x, k) \coloneqq \begin{cases}
 		\left( \frac{d}{dx} + ik \right)^{2} \varphi_{s}(x, k)|_{(x_{0} - \frac{1}{2}, x_{0})} & \text{for } x \in (x_{0} - \frac{1}{2}, x_{0}) \\ \left( \frac{d}{dx} + ik \right)^{2} \varphi_{s}(x, k)|_{(x_{0}, x_{0}  + \frac{1}{2})} & \text{for } x \in (x_{0}, x_{0} + \frac{1}{2})
 	\end{cases} \] 
and using \eqref{eigv-problem} and \eqref{quasi-periodic-condition}, hence yields
		\[ \varphi_{s}(x - \frac{1}{2}, k) = e^{-ik(x - \frac{1}{2})} \psi_{s}(x - \frac{1}{2}, k) = e^{-ik(x + \frac{1}{2})} \psi_{s}(x + \frac{1}{2}, k) = \varphi_{s}(x + \frac{1}{2}, k). \]
Which shows that $(\varphi_{s}(\cdot, k))_{s \in \N}$ is an orthonormal and complete system of eigenfunctions of the periodic eigenvalue problem
	\begin{align}
		\tilde{A}_{k} \varphi = \lambda_{s}(k) \varphi \text{ on } \Omega, \label{mod-eigv-problem} \\
		 \varphi(x - \frac{1}{2}) = \varphi(x + \frac{1}{2}). \label{periodic-condition}
	\end{align}
with the identical eigenvalue sequence $(\lambda_{s}(s))_{s \in \N}$ as in \eqref{eigv-problem}.