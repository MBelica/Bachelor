\chapter{The spectrum of the Schrödinger operator in $\R$} \label{chap:6}
Finally, we are ready to prove the main result for the one-dimensional case stating that for the operator $A$ it holds that
	\begin{equation}
		\sigma(A) = \bigcup_{s \in \N} I_{s}, \label{MainResult}
	\end{equation}
where $I_{s} \coloneqq \{ \lambda_{s}(k) : k \in \overline{B} \}$ for all $s \in \N$. We will prove that each $I_{s}$ is a compact interval, which means that the spectrum shows a \enquote{band-gap} structure. 
~\\

To prove this equality we first need to show that $\lambda_{s}(k)$ is continuous in $k$, and hence, $I_{s}$ is for each $k$ a compact interval in $\R$.

\begin{theorem} \label{6.1}
	For all $s \in \N$, the function $k \mapsto \lambda_{s}(k)$ is continuous in $k \in \overline{B}$.
	\begin{proof}
		In the transformed eigenvalue problem \eqref{mod-eigv-problem} the boundary conditions \eqref{periodic-condition} are periodic and independent of $k$. By Poincare's min-max principle (Theorem \ref{athm:poincare}) for eigenvalues we have
		\begin{equation}
			\lambda_{s}(k) = \underset{\dim U = s}{\min_{U \subseteq \mathcal{D}(\tilde{A}_{k})}} \max_{v \in U \setminus \{ 0 \} } \frac{\langle \tilde{A}_{k} v, v \rangle_{L^{2}(\Omega)}}{\langle v, v \rangle_{L^{2}(\Omega)}}.  \label{poincare} 
		\end{equation} 
		Now, let $k \in \overline{B}$ be fixed. For all $\tilde{k} \in \overline{B}$ and all $v \in \mathcal{D}(\tilde{A}_{k})$ using triangular inequality yields for $J \in \left\{ (x_{0} - \frac{1}{2}, x_{0}), (x_{0}, x_{0} + \frac{1}{2}) \right\}$:
		\begin{align}
			 \frac{ \langle \left( \frac{d}{dx} + i\tilde{k} \right) v , \left( \frac{d}{dx} + i\tilde{k} \right) v \rangle_{L^{2}(J)}}{\langle v , v \rangle_{L^{2}(J)}} & \left\{\mathrel{\substack{\leq \\[0.1cm] \geq}}\right\} \frac{ \langle \left( \frac{d}{dx} + ik \right) v , \left( \frac{d}{dx} + ik \right) v \rangle_{L^{2}(J)}}{\langle v , v \rangle_{L^{2}(J)}} \notag \\
			& ~\quad \left\{\mathrel{\substack{+ \\[0.1cm] -}}\right\} \frac{2 |k-\tilde{k}|\|v'\|_{L^{2}(J)} \| v \|_{L^{2}(J)}}{\| v \|^{2}_{L^{2}(J)}} \left\{\mathrel{\substack{+ \\[0.1cm] -}}\right\} \left| |k|^{2} - |\tilde{k}|^{2} \right|. \label{**-condition}
		\end{align}
		Moreover, we find
		\begin{align*}
			2 \| v' \|_{L^{2}(J)} \| v \|_{L^{2}(J)} & \leq 2 \| \left( \frac{d}{dx} + ik \right) v\|_{L^{2}(J)} \| v \|_{L^{2}(J)} + 2|k| \|v\|^{2}_{L^{2}(J)} \\
			& \leq \| \left( \frac{d}{dx} + ik \right) v \|^{2}_{L^{2}(J)} + \| v \|^{2}_{L^{2}(J)} + 2 |k| \| v \|^{2}_{L^{2}(J)} \\
			& \leq \langle  \left( \frac{d}{dx} + ik \right) v,  \left( \frac{d}{dx} + ik \right) v \rangle_{L^{2}(J)} + (1 + 2|k|) \|v\|^{2}_{L^{2}(J)}.
		\end{align*}
		Hence, \eqref{**-condition} yields
		\begin{align*}
			 \frac{ \langle \left( \frac{d}{dx} + i\tilde{k} \right) v , \left( \frac{d}{dx} + i\tilde{k} \right) v \rangle_{L^{2}(J)}}{\langle v , v \rangle_{L^{2}(J)}} & \left\{\mathrel{\substack{\leq \\[0.1cm] \geq}}\right\} (1 \left\{\mathrel{\substack{+ \\[0.1cm] -}}\right\} |k - \tilde{k}|) \frac{ \langle \left( \frac{d}{dx} + ik \right) v , \left( \frac{d}{dx} + ik \right) v \rangle_{L^{2}(J)}}{\langle v , v \rangle_{L^{2}(J)}} \\
			& ~\quad \left\{\mathrel{\substack{+ \\[0.1cm] -}}\right\} \left( |k - \tilde{k}| (1 + 2|k|) + \left| |k|^{2} - |\tilde{k}|^{2} \right| \right).
		\end{align*}		
		Thus the min-max-principle gives for $| k - \tilde{k}| < 1$
		\begin{align*}
			\lambda_{s}(\tilde{k}) & \leq \left( 1 + |k - \tilde{k}| \right) \lambda_{s}(k) + \left( |k - \tilde{k}| (1 + 2|k|) + \left| |k|^{2}- |\tilde{k}|^{2} \right| \right)
		\intertext{and}
				\lambda_{s}(\tilde{k}) & \geq \left( 1 - |k - \tilde{k}| \right) \lambda_{s}(k) - \left( |k - \tilde{k}| (1 + 2|k|) + \left| |k|^{2}- |\tilde{k}|^{2} \right| \right),
		\end{align*}
		which, eventually, yields
		\begin{equation}
			|\lambda_{s}(\tilde{k}) - \lambda_{s}(k)| \leq |k - \tilde{k}| \left( \lambda_{s}(k) + 1 + 2|k| + |k| + |\tilde{k}|\right), \label{lambdacont}
		\end{equation} 
		for $| k - \tilde{k}| < 1$. Now, the eigenvalue $\lambda_{s}(k)$ is by construction also an eigenvalue of the problem \eqref{eigv-problem}, where the operator is dependent on $k$ rather than the boundary conditions. However, all eigenvalues of \eqref{eigv-problem} are by Poincare's min-max-principle (Theorem \ref{athm:poincare}) dominated by eigenvalues of the eigenvalue problem of $A_{k}$ with Dirichlet boundary conditions, since the domain with Dirichlet conditions is a superset of the domain with the quasi-periodic boundary conditions. Since the eigenvalues for the Dirichlet boundary condition are independent of $k$, $\lambda_{s}(k)$ is uniformly bounded, and hence by \eqref{lambdacont}, $\lambda_{s}(k)$ is continuous in $k$.
	\end{proof}
\end{theorem}

\begin{remark}
	As $\overline{B}$ is a compact and connected set and $\lambda_{s}(k)$ is a continuous function of $k \in \overline{B}$ we derive for \eqref{MainResult}
	\begin{equation}
		I_{s} \text{ is a compact real interval for each } s \in \N.\label{Iisacompactrealinterval}
	\end{equation} 	
\end{remark}

From \eqref{poincare} and \eqref{Iisacompactrealinterval} also follows that $\mu_{s} \leq \lambda_{s}(k)$ for all $s \in \N$, $k \in \overline{B}$ with $(\mu_{s})_{s \in \N}$ denoting the sequence of eigenvalues of problem \eqref{eigv-problem} with Neumann boundary conditions, since the domain with Neumann conditions is a subset of the domain with the quasi-periodic boundary conditions. By analogous calculations as in Section \ref{sec:4.3}, but now with Neumann boundary conditions on $\mathcal{D}(A_{k})$, we see that $\mu_{s} \rightarrow \infty$ as $s \rightarrow \infty$, hence, we obtain $\min I_{s} \rightarrow \infty \text{ as } s \rightarrow \infty$, which together with \eqref{Iisacompactrealinterval} implies that
	\begin{equation}
		\bigcup_{s \in \N} I_{s} \text{ is closed.} \label{UIclosed}
	\end{equation}

Using this property we are now able to prove the first inclusion of the main statement \eqref{MainResult}.	

\begin{theorem} \label{4.1:thm-MainResult.FirstInclusion}
	$\sigma(A) \supseteq \bigcup_{s \in \N} I_{s}.$
	
	\begin{proof}
		Let $\lambda \in \bigcup_{s \in \N} I_{s}$, i.e. $\lambda = \lambda_{s}(k)$ for some $s \in \N$ and some $k \in \overline{B}$, and 
		\begin{equation}
			A_{k} \psi_{s}(\cdot, k) = \lambda \psi_{s}(\cdot, k) \label{firstinclusion-firstequation} 
		\end{equation} 
		As introduced in Chapter \ref{chap:4}, we regard $\psi_{s}(\cdot, k)$ extended to the whole of $\R$, whence, due to the periodic structure of $A$, $\psi_{s}$ satisfies
		\[ A \psi_{s} = \lambda \psi_{s} \]
		\enquote{locally}, i.e. $\psi_{s} \in \Big\{ \psi \in  H^{1}_{loc}(\R) : \psi \in H^{2}_{loc}\Big(\R \setminus \bigcup_{i \in \Z} x_{i} \Big), \psi'(x_{j} - 0) - \psi'(x_{j} + 0) + \rho  \psi(x_{j}) = 0 ~\forall j \in \Z \Big\}$, and $ -\psi_{s}'' = \lambda \psi_{s}$ on $\R \setminus \bigcup_{i \in \Z} x_{i}$. Now, if we choose a function $\eta \in H^{2}(\R)$ such that 
			\begin{equation}
				\eta(x) = 1 \text{ for } |x| \leq \frac{1}{2}, \quad \eta(x) = 0 \text{ for } |x| \geq 1, \label{eta}
			\end{equation} 
		we are able to define, for each $l \in \N$,
			\[ \kappa_{l}(x) \coloneqq \eta\left(\frac{|x|}{l}\right) \psi_{s}(x, k) \]		
	 	for all $x \in \R$; for an illustration see Figures \ref{fig:6.1}, \ref{fig:6.2} and \ref{fig:6.3} where a schematic of the real part is depicted; the analogous complex part is not depicted here. As $\psi_{s} \in \mathcal{D}(A)$ we know $\kappa_{l} \in \mathcal{D}(A)$, hence, we see that
		\begin{align}
			(A - \lambda I) \kappa_{l} & = \sum_{i \in \N} \left[ (- \frac{d^{2}}{dx^{2}} - \lambda) \kappa_{l}|_{(x_{i}, x_{i+1})} \cdot \mathds{1}_{(x_{i}, x_{i+1})} \right] \label{eq:sepofspectraleq} \\
				& = \sum_{i \in \N} \left[ \eta\left(\frac{|\cdot|}{l}\right) \left(- \frac{d^{2}}{dx^{2}} - \lambda \right) \psi_{s}(\cdot, k) |_{(x_{i}, x_{i+1})} \cdot \mathds{1}_{(x_{i}, x_{i+1})} \right] + R \notag
		\end{align}
		where $R$ is a sum of products of derivatives of order $\geq 1$ of $\eta\left(\frac{|\cdot|}{l}\right)$ and of order $\leq 1$ of $\psi_{s}(\cdot, k)$. 
		Let use denote with $B_{l}$ the ball around $0$ with radius $l$ and let, for simplicity, $c \in \R$ be a generic constant. Thus, since $\psi_{s}(\cdot, k) \in H^{1}_{loc}(\R)$, the quasi-periodic structure of $\psi_{s}(\cdot, k)$ implies
		\begin{equation}
			 \| R \|_{L^{2}(\R)} \leq \frac{c}{l} \| \psi_{s}(\cdot, k) \|_{H^{1}(B_{l})} \leq c \frac{1}{\sqrt{l}}. \label{eq:estimofR}
		\end{equation}
		Now, the quasi-periodic structure allows us additionally to find an upper boundary for $\kappa_{l}$:
		\begin{equation}
			\| \kappa_{l} \|_{L^{2}(\R)} \geq c \| \psi_{s}(\cdot, k) \|_{L^{2}(K_{l})} \geq c \sqrt{l}. \label{eq:6.11}
		\end{equation} 
		Together with \eqref{UIclosed}, \eqref{firstinclusion-firstequation} and \eqref{eq:sepofspectraleq}, \eqref{eq:6.11} yields
		\[ \frac{1}{\|\kappa_{l}\|_{L^{2}(\R)}}\| (A - \lambda I) \kappa_{l} \|_{L^{2}(\R)} \leq \frac{c}{l}, \]
		which eventually results in the property
			\[ \frac{1}{\|\kappa_{l} \|_{L^{2}(\R)}} \| (A - \lambda I) \kappa_{l} \|_{L^{2}(\R)} \rightarrow 0 \text{ as } l \rightarrow \infty. \]
		Thus, either $\lambda$ is an eigenvalue of $A$, or $(A - \lambda I)^{-1}$ exists but is unbounded. In both cases, $\lambda \in \sigma(A)$.
	\end{proof}
\end{theorem}	
		
\definecolor{lightblue}{rgb}{0.00000,0.44700,0.74100}%
\begin{figure}
 \begin{tikzpicture}
  \begin{axis}[ width=5.85in, height=2.245in, at={(1.011in,2.39in)}, scale only axis, xmin=-2.5, xmax=2.5, ymin=-0.5, ymax=1.25, ticks=none, axis x line=middle,
             axis y line=middle, yticklabels={,,}, xticklabels={,,} after end axis/.code={ \draw[red,->] (axis cs:0,0) -- (axis cs:0.5,0); }]
	\addplot [color=black,solid,forget plot] table[row sep=crcr]{ -2 -0.025 \\ -2 0.025 \\ };
	\addplot [color=black,solid,forget plot] table[row sep=crcr]{ -1.5 -0.025 \\ -1.5 0.025 \\ };
	\addplot [color=black,solid,forget plot] table[row sep=crcr]{ -1 -0.025 \\ -1 0.025 \\ };
	\addplot [color=black,solid,forget plot] table[row sep=crcr]{ -0.5 -0.025 \\ -0.5 0.025 \\ };
	\addplot [color=black,solid,forget plot] table[row sep=crcr]{ 0 -0.025 \\ 0 0.025 \\ };
	\addplot [color=black,solid,forget plot] table[row sep=crcr]{ 0.5 -0.025 \\ 0.5 0.025 \\ };
	\addplot [color=black,solid,forget plot] table[row sep=crcr]{ 1 -0.025 \\ 1 0.025 \\ };
	\addplot [color=black,solid,forget plot] table[row sep=crcr]{ 1.5 -0.025 \\ 1.5 0.025 \\ };
	\addplot [color=black,solid,forget plot] table[row sep=crcr]{ 2 -0.025 \\ 2 0.025 \\ };
	\addplot [color=black,solid,forget plot] table[row sep=crcr]{ 0.015625 0.5 \\ -0.015625 0.5 \\ };
	\node[right, align=left, text=black] at (axis cs:-2,-0.075) {-2};
	\node[right, align=left, text=black] at (axis cs:-1.5,-0.075) {-1.5};
	\node[right, align=left, text=black] at (axis cs:-1,-0.075) {-1};
	\node[right, align=left, text=black] at (axis cs:-0.5,-0.075) {-0.5};
	\node[right, align=left, text=black] at (axis cs:0,-0.075) {0};
	\node[right, align=left, text=black] at (axis cs:0.5,-0.075) {0.5};
	\node[right, align=left, text=black] at (axis cs:1,-0.075) {1};
	\node[right, align=left, text=black] at (axis cs:1.5,-0.075) {1.5};
	\node[right, align=left, text=black] at (axis cs:2,-0.075) {2};
	\node[right, align=left, text=black] at (axis cs:0.0125,0.6) {0.5};
	\node[right, align=left, text=black] at (axis cs:0.0125,1.1) {1};
	
	\addplot [color=black,solid,forget plot] table[row sep=crcr]{%
		-1.5 0\\ -1.49 0\\ -1.48 0\\ -1.47 0\\ -1.46 0\\ -1.45 0\\ -1.44 0\\ -1.43 0\\ -1.42 0\\ -1.41 0\\ -1.4 0\\ -1.39 0\\ -1.38 0\\ -1.37 0\\ -1.36 0\\ -1.35 0\\ -1.34 0\\ -1.33 0\\ -1.32 0\\ -1.31 0\\ -1.3 0\\ -1.29 0\\ -1.28 0\\ -1.27 0\\ -1.26 0\\ -1.25 0\\ -1.24 0\\ -1.23 0\\ -1.22 0\\ -1.21 0\\ -1.2 0\\ -1.19 0\\ -1.18 0\\ -1.17 0\\ -1.16 0\\ -1.15 0\\ -1.14 0\\ -1.13 0\\ -1.12 0\\ -1.11 0\\ -1.1 0\\ -1.09 0\\ -1.08 0\\ -1.07 0\\ -1.06 0\\ -1.05 0\\ -1.04 0\\ -1.03 0\\ -1.02 0\\ -1.01 0\\ -1 0\\ -0.99 0.000986635785864221\\ -0.98 0.00394264934276112\\ -0.97 0.00885637463565569\\ -0.96 0.0157084194356845\\ -0.95 0.0244717418524232\\ -0.94 0.0351117570558744\\ -0.93 0.0475864737669903\\ -0.92 0.0618466599780682\\ -0.91 0.0778360372489925\\ -0.9 0.0954915028125263\\ -0.89 0.114743378612105\\ -0.88 0.135515686289294\\ -0.87 0.157726447035656\\ -0.86 0.181288005125655\\ -0.85 0.206107373853763\\ -0.84 0.232086602510502\\ -0.83 0.259123162949143\\ -0.82 0.287110354217464\\ -0.81 0.315937723657661\\ -0.8 0.345491502812526\\ -0.79 0.375655056417572\\ -0.78 0.406309342707138\\ -0.77 0.437333383217848\\ -0.76 0.468604740235343\\ -0.75 0.5\\ -0.74 0.531395259764657\\ -0.73 0.562666616782152\\ -0.72 0.593690657292862\\ -0.71 0.624344943582428\\ -0.7 0.654508497187474\\ -0.69 0.684062276342339\\ -0.68 0.712889645782536\\ -0.67 0.740876837050858\\ -0.66 0.767913397489498\\ -0.65 0.793892626146236\\ -0.64 0.818711994874345\\ -0.63 0.842273552964344\\ -0.62 0.864484313710706\\ -0.61 0.885256621387895\\ -0.6 0.904508497187474\\ -0.59 0.922163962751008\\ -0.58 0.938153340021932\\ -0.57 0.95241352623301\\ -0.56 0.964888242944126\\ -0.55 0.975528258147577\\ -0.54 0.984291580564316\\ -0.53 0.991143625364344\\ -0.52 0.996057350657239\\ -0.51 0.999013364214136\\ -0.5 1\\ -0.49 1\\ -0.48 1\\ -0.47 1\\ -0.46 1\\ -0.45 1\\ -0.44 1\\ -0.43 1\\ -0.42 1\\ -0.41 1\\ -0.4 1\\ -0.39 1\\ -0.38 1\\ -0.37 1\\ -0.36 1\\ -0.35 1\\ -0.34 1\\ -0.33 1\\ -0.32 1\\ -0.31 1\\ -0.3 1\\ -0.29 1\\ -0.28 1\\ -0.27 1\\ -0.26 1\\ -0.25 1\\ -0.24 1\\ -0.23 1\\ -0.22 1\\ -0.21 1\\ -0.2 1\\ -0.19 1\\ -0.18 1\\ -0.17 1\\ -0.16 1\\ -0.15 1\\ -0.14 1\\ -0.13 1\\ -0.12 1\\ -0.11 1\\ -0.0999999999999999 1\\ -0.0900000000000001 1\\ -0.0800000000000001 1\\ -0.0700000000000001 1\\ -0.0600000000000001 1\\ -0.05 1\\ -0.04 1\\ -0.03 1\\ -0.02 1\\ -0.01 1\\ 0 1\\ 0.01 1\\ 0.02 1\\ 0.03 1\\ 0.04 1\\ 0.05 1\\ 0.0600000000000001 1\\ 0.0700000000000001 1\\ 0.0800000000000001 1\\ 0.0900000000000001 1\\ 0.0999999999999999 1\\ 0.11 1\\ 0.12 1\\ 0.13 1\\ 0.14 1\\ 0.15 1\\ 0.16 1\\ 0.17 1\\ 0.18 1\\ 0.19 1\\ 0.2 1\\ 0.21 1\\ 0.22 1\\ 0.23 1\\ 0.24 1\\ 0.25 1\\ 0.26 1\\ 0.27 1\\ 0.28 1\\ 0.29 1\\ 0.3 1\\ 0.31 1\\ 0.32 1\\ 0.33 1\\ 0.34 1\\ 0.35 1\\ 0.36 1\\ 0.37 1\\ 0.38 1\\ 0.39 1\\ 0.4 1\\ 0.41 1\\ 0.42 1\\ 0.43 1\\ 0.44 1\\ 0.45 1\\ 0.46 1\\ 0.47 1\\ 0.48 1\\ 0.49 1\\ 0.5 1\\ 0.51 0.999013364214136\\ 0.52 0.996057350657239\\ 0.53 0.991143625364344\\ 0.54 0.984291580564316\\ 0.55 0.975528258147577\\ 0.56 0.964888242944126\\ 0.57 0.95241352623301\\ 0.58 0.938153340021932\\ 0.59 0.922163962751008\\ 0.6 0.904508497187474\\ 0.61 0.885256621387895\\ 0.62 0.864484313710706\\ 0.63 0.842273552964344\\ 0.64 0.818711994874345\\ 0.65 0.793892626146236\\ 0.66 0.767913397489498\\ 0.67 0.740876837050858\\ 0.68 0.712889645782537\\ 0.69 0.684062276342339\\ 0.7 0.654508497187474\\ 0.71 0.624344943582428\\ 0.72 0.593690657292862\\ 0.73 0.562666616782152\\ 0.74 0.531395259764657\\ 0.75 0.5\\ 0.76 0.468604740235343\\ 0.77 0.437333383217848\\ 0.78 0.406309342707138\\ 0.79 0.375655056417573\\ 0.8 0.345491502812526\\ 0.81 0.315937723657661\\ 0.82 0.287110354217464\\ 0.83 0.259123162949143\\ 0.84 0.232086602510502\\ 0.85 0.206107373853763\\ 0.86 0.181288005125655\\ 0.87 0.157726447035656\\ 0.88 0.135515686289294\\ 0.89 0.114743378612105\\ 0.9 0.0954915028125263\\ 0.91 0.0778360372489925\\ 0.92 0.0618466599780682\\ 0.93 0.0475864737669903\\ 0.94 0.0351117570558744\\ 0.95 0.0244717418524233\\ 0.96 0.0157084194356845\\ 0.97 0.00885637463565569\\ 0.98 0.00394264934276112\\ 0.99 0.000986635785864221\\ 1 0\\ 1.01 0\\ 1.02 0\\ 1.03 0\\ 1.04 0\\ 1.05 0\\ 1.06 0\\ 1.07 0\\ 1.08 0\\ 1.09 0\\ 1.1 0\\ 1.11 0\\ 1.12 0\\ 1.13 0\\ 1.14 0\\ 1.15 0\\ 1.16 0\\ 1.17 0\\ 1.18 0\\ 1.19 0\\ 1.2 0\\ 1.21 0\\ 1.22 0\\ 1.23 0\\ 1.24 0\\ 1.25 0\\ 1.26 0\\ 1.27 0\\ 1.28 0\\ 1.29 0\\ 1.3 0\\ 1.31 0\\ 1.32 0\\ 1.33 0\\ 1.34 0\\ 1.35 0\\ 1.36 0\\ 1.37 0\\ 1.38 0\\ 1.39 0\\ 1.4 0\\ 1.41 0\\ 1.42 0\\ 1.43 0\\ 1.44 0\\ 1.45 0\\ 1.46 0\\ 1.47 0\\ 1.48 0\\ 1.49 0\\ 1.5 0\\
	};
 \end{axis}
 \end{tikzpicture}%	
 \caption{Example for the function $\eta$} \label{fig:6.1}
\end{figure}
	
\begin{figure}
 \begin{tikzpicture}
  \begin{axis}[ width=5.85in, height=2.245in, at={(1.011in,2.39in)}, scale only axis, xmin=-2.5, xmax=2.5, ymin=-150, ymax=175, ticks=none, axis x line=middle, axis y line=middle, yticklabels={,,}, xticklabels={,,} after end axis/.code={ \draw[red,->] (axis cs:0,0) -- (axis cs:0.5,0); }]
	\addplot [color=black,solid,forget plot] table[row sep=crcr]{ -2 -5 \\ -2 5 \\ };
	\addplot [color=black,solid,forget plot] table[row sep=crcr]{ -1.5 -5 \\ -1.5 5 \\ };
	\addplot [color=black,solid,forget plot] table[row sep=crcr]{ -1 -5 \\ -1 5 \\ };
	\addplot [color=black,solid,forget plot] table[row sep=crcr]{ -0.5 -5 \\ -0.5 5 \\ };
	\addplot [color=black,solid,forget plot] table[row sep=crcr]{ 0 -5 \\ 0 5 \\ };
	\addplot [color=black,solid,forget plot] table[row sep=crcr]{ 0.5 -5 \\ 0.5 5 \\ };
	\addplot [color=black,solid,forget plot] table[row sep=crcr]{ 1 -5 \\ 1 5 \\ };
	\addplot [color=black,solid,forget plot] table[row sep=crcr]{ 1.5 -5 \\ 1.5 5 \\ };
	\addplot [color=black,solid,forget plot] table[row sep=crcr]{ 2 -5 \\ 2 5 \\ };

	\node[right, align=left, text=black] at (axis cs:-2,-20) {-2};
	\node[right, align=left, text=black] at (axis cs:-1.5,-20) {-1.5};
	\node[right, align=left, text=black] at (axis cs:-1,-20) {-1};
	\node[right, align=left, text=black] at (axis cs:-0.5,-20) {-0.5};
	\node[right, align=left, text=black] at (axis cs:0,-20) {0};
	\node[right, align=left, text=black] at (axis cs:0.5,-20) {0.5};
	\node[right, align=left, text=black] at (axis cs:1,-20) {1};
	\node[right, align=left, text=black] at (axis cs:1.5,-20) {1.5};
	\node[right, align=left, text=black] at (axis cs:2,-20) {2};

	\addplot [color=black,solid,forget plot] table[row sep=crcr]{%
		-2.026666166 -0.03779437330893\\ -1.989333836 0.49159102889797\\ -1.970667169 1.50038990186643\\ -1.952000503 2.70669779642262\\ -1.933333836 4.23962612174219\\ -1.914667169 6.12251813611504\\ -1.877333836 10.3675961903172\\ -1.858667169 12.1907440773085\\ -1.840000503 13.3490455714993\\ -1.821332832 13.4698474003064\\ -1.802667169 12.2543866238618\\ -1.784000503 9.53651574296562\\ -1.765335843 5.32709919887023\\ -1.746669176 -0.168781746812373\\ -1.709335843 -13.2808335645107\\ -1.690669176 -19.765276163378\\ -1.67200251 -25.4299185136434\\ -1.653335843 -29.7936050902586\\ -1.634669176 -32.54901782864\\ -1.61600251 -33.6066093301263\\ -1.597335843 -33.1064965794859\\ -1.578669176 -31.3999589194946\\ -1.541335843 -26.5345662490583\\ -1.522669176 -24.61527665202\\ -1.50400251 -23.7983805355628\\ -1.485335843 -24.4823893352151\\ -1.466669176 -26.8420863231301\\ -1.44800251 -30.7862033373891\\ -1.429335843 -35.9445325280391\\ -1.39200251 -47.0973837558116\\ -1.373335843 -51.2845088364044\\ -1.354669176 -53.2804321448723\\ -1.33600251 -52.2536701050413\\ -1.317335843 -47.6406421711307\\ -1.298669176 -39.2497804414718\\ -1.28000251 -27.321449507433\\ -1.261335843 -12.5288953075391\\ -1.205335843 37.4413379449413\\ -1.186669176 51.5444660812333\\ -1.168002509 62.4585276838666\\ -1.149335843 69.5452497433433\\ -1.130669176 72.6248560983922\\ -1.112002509 71.9852106548452\\ -1.093335843 68.332332655463\\ -1.074669176 62.6907810718344\\ -1.056002509 56.2695037720095\\ -1.037335843 50.3097958099074\\ -1.018669176 45.9317300594959\\ -1.000002509 43.9941798788733\\ -0.981335843 44.9834733195722\\ -0.962669176 48.9475797287251\\ -0.944002509 55.4821380956234\\ -0.888002509 79.7897330383805\\ -0.869335843 84.7473985472588\\ -0.850669176 85.7648204220532\\ -0.832002509 81.7584242769937\\ -0.813335843 72.1645711116723\\ -0.794669176 57.0435022367787\\ -0.776002509 37.1102584685194\\ -0.757335843 13.6787853581076\\ -0.720002509 -36.3945883130246\\ -0.701335843 -59.0620573659182\\ -0.682669176 -77.7868401670708\\ -0.664002509 -91.3517747984281\\ -0.645335843 -99.1580266807009\\ -0.626669176 -101.274958907136\\ -0.608002509 -98.4072098076061\\ -0.589335843 -91.7928548621499\\ -0.533335843 -65.341050366876\\ -0.514669176 -59.5776353111306\\ -0.496002509 -57.3176100990414\\ -0.477335843 -58.9555414930104\\ -0.458669176 -64.314544573035\\ -0.440002509 -72.64308940162\\ -0.402669176 -92.6286559916554\\ -0.384002509 -100.761711627235\\ -0.365335842 -105.231579253983\\ -0.346669176 -104.471980820453\\ -0.328002509 -97.4194419193707\\ -0.309335842 -83.6857961962663\\ -0.290669176 -63.6495966652594\\ -0.272002509 -38.4482233659866\\ -0.216002509 48.3408406651207\\ -0.197335842 73.431056442991\\ -0.178669176 93.3020196753158\\ -0.160002509 106.81147726531\\ -0.141335842 113.567749039819\\ -0.122669176 113.938985281487\\ -0.104002509 108.969808441092\\ -0.085335842 100.220884807626\\ -0.048002509 78.7837494812196\\ -0.029335842 69.7570199725717\\ -0.010669176 63.8451923318358\\ 0.007997491 61.9036031051377\\ 0.026664158 64.177430015496\\ 0.045330824 70.270579459766\\ 0.063997491 79.1850540393603\\ 0.101330824 99.1007837341403\\ 0.119997491 106.330355871256\\ 0.138664158 109.29742246796\\ 0.157330824 106.559623864272\\ 0.175997491 97.2636645165321\\ 0.194664158 81.2904447925349\\ 0.213330824 59.3127611643347\\ 0.231997491 32.7517476983893\\ 0.287997491 -53.4077763196265\\ 0.306664158 -76.8020184509458\\ 0.325330825 -94.5046195601613\\ 0.343997491 -105.671799699611\\ 0.362664158 -110.201990608652\\ 0.381330825 -108.710425546452\\ 0.399997491 -102.418861280368\\ 0.418664158 -92.975629431573\\ 0.455997491 -71.8442709765875\\ 0.474664158 -63.52289120788\\ 0.493330825 -58.4023497418747\\ 0.511997491 -57.0929416738232\\ 0.530664158 -59.6221549375721\\ 0.549330825 -65.4346240486486\\ 0.567997491 -73.45482413579\\ 0.586664158 -82.2142798111342\\ 0.605330825 -90.0254955770436\\ 0.623997491 -95.1883523737974\\ 0.642664158 -96.2095687454069\\ 0.661330825 -92.0059989361003\\ 0.679997491 -82.0674342447112\\ 0.698664158 -66.5506411683596\\ 0.717330825 -46.2883658566096\\ 0.735997491 -22.7031344044964\\ 0.773330825 26.963056108708\\ 0.791997491 49.1081389910946\\ 0.810664158 67.2066375833726\\ 0.829330825 80.172150400592\\ 0.847997491 87.5330048803376\\ 0.866664158 89.450203611587\\ 0.885330825 86.6521954229314\\ 0.903997491 80.303264462211\\ 0.922664158 71.8308230518052\\ 0.941330825 62.7374837721779\\ 0.959997491 54.4206426372519\\ 0.978664158 48.017221616159\\ 0.997330825 44.2874893410278\\ 1.015997491 43.5460379787017\\ 1.034664158 45.6451332253345\\ 1.053330825 50.0092501943855\\ 1.090664158 61.5424981837587\\ 1.109330825 66.2973020304288\\ 1.127997492 68.7945186065693\\ 1.146664158 68.0812629901674\\ 1.165330825 63.5677050454768\\ 1.183997492 55.1148940439464\\ 1.202664158 43.0673534393098\\ 1.221330825 28.2238392347034\\ 1.277330825 -20.9576247008023\\ 1.295997492 -34.5619094796191\\ 1.314664158 -45.0139699781911\\ 1.333330825 -51.820614015846\\ 1.351997492 -54.9083155380804\\ 1.370664158 -54.6015595907865\\ 1.389330825 -51.5511609735246\\ 1.407997492 -46.6264628329233\\ 1.445330825 -34.8665000698331\\ 1.463997492 -29.7234840183348\\ 1.482664158 -25.9082334540591\\ 1.501330825 -23.7104568162637\\ 1.519997492 -23.1516266999053\\ 1.538664158 -24.0017341344736\\ 1.557330825 -25.8253362842709\\ 1.575997492 -28.0509974415202\\ 1.594664158 -30.0561125323414\\ 1.613330825 -31.2538749055351\\ 1.631997492 -31.1749024603392\\ 1.650664158 -29.5319668967618\\ 1.669330825 -26.2602218728964\\ 1.687997492 -21.5248325263426\\ 1.706664158 -15.6965345470731\\ 1.743997492 -2.8940719930403\\ 1.762664158 2.92375726495266\\ 1.781330825 7.69019118222752\\ 1.799997492 11.094780193203\\ 1.818664158 13.0171559954624\\ 1.837330825 13.524473759406\\ 1.855997492 12.8414020681107\\ 1.874664159 11.2986606932326\\ 1.911997492 7.05667723381621\\ 1.930664159 4.98118491323854\\ 1.949330825 3.20984937798551\\ 1.967997492 1.7919724965559\\ 1.986664159 0.661938336740332\\ 2.023997492 -1.20735852187683\\
	};
  \end{axis}
 \end{tikzpicture}%
 \caption{A schematic of the real part of a Bloch wave in one dimension} \label{fig:6.2}
\end{figure}

\begin{figure}
 \begin{tikzpicture}
  \begin{axis}[ width=5.85in, height=2.245in, at={(1.011in,2.39in)}, scale only axis, xmin=-2.5, xmax=2.5, ymin=-150, ymax=175, ticks=none, axis x line=middle, axis y line=middle, yticklabels={,,}, xticklabels={,,} after end axis/.code={ \draw[red,->] (axis cs:0,0) -- (axis cs:0.5,0); }]
	\addplot [color=black,solid,forget plot] table[row sep=crcr]{ -2 -5 \\ -2 5 \\ };
	\addplot [color=black,solid,forget plot] table[row sep=crcr]{ -1.5 -5 \\ -1.5 5 \\ };
	\addplot [color=black,solid,forget plot] table[row sep=crcr]{ -1 -5 \\ -1 5 \\ };
	\addplot [color=black,solid,forget plot] table[row sep=crcr]{ -0.5 -5 \\ -0.5 5 \\ };
	\addplot [color=black,solid,forget plot] table[row sep=crcr]{ 0 -5 \\ 0 5 \\ };
	\addplot [color=black,solid,forget plot] table[row sep=crcr]{ 0.5 -5 \\ 0.5 5 \\ };
	\addplot [color=black,solid,forget plot] table[row sep=crcr]{ 1 -5 \\ 1 5 \\ };
	\addplot [color=black,solid,forget plot] table[row sep=crcr]{ 1.5 -5 \\ 1.5 5 \\ };
	\addplot [color=black,solid,forget plot] table[row sep=crcr]{ 2 -5 \\ 2 5 \\ };

	\node[right, align=left, text=black] at (axis cs:-2,-20) {-2};
	\node[right, align=left, text=black] at (axis cs:-1.5,-20) {-1.5};
	\node[right, align=left, text=black] at (axis cs:-1,-20) {-1};
	\node[right, align=left, text=black] at (axis cs:-0.5,-20) {-0.5};
	\node[right, align=left, text=black] at (axis cs:0,-20) {0};
	\node[right, align=left, text=black] at (axis cs:0.5,-20) {0.5};
	\node[right, align=left, text=black] at (axis cs:1,-20) {1};
	\node[right, align=left, text=black] at (axis cs:1.5,-20) {1.5};
	\node[right, align=left, text=black] at (axis cs:2,-20) {2};

	\addplot [color=black,solid,forget plot] table[row sep=crcr]{%
		-2.026666166 0\\ -1.989333836 0\\ -1.970667169 0\\ -1.952000503 0\\ -1.933333836 0\\ -1.914667169 0\\ -1.877333836 0\\ -1.858667169 0\\ -1.840000503 0\\ -1.821332832 0\\ -1.802667169 0\\ -1.784000503 0\\ -1.765335843 0\\ -1.746669176 0\\ -1.709335843 0\\ -1.690669176 0\\ -1.67200251 0\\ -1.653335843 0\\ -1.634669176 0\\ -1.61600251 0\\ -1.597335843 0\\ -1.578669176 0\\ -1.541335843 0\\ -1.522669176 0\\ -1.50400251 0\\ -1.485335843 0\\ -1.466669176 0\\ -1.44800251 0\\ -1.429335843 0\\ -1.39200251 0\\ -1.373335843 0\\ -1.354669176 0\\ -1.33600251 0\\ -1.317335843 0\\ -1.298669176 0\\ -1.28000251 0\\ -1.261335843 0\\ -1.205335843 0\\ -1.186669176 0\\ -1.168002509 0\\ -1.149335843 0\\ -1.130669176 0\\ -1.112002509 0\\ -1.093335843 0\\ -1.074669176 0\\ -1.056002509 0\\ -1.037335843 0\\ -1.018669176 0\\ -1.000002509 0\\ -0.981335843 0.153417328419455\\ -0.962669176 0.663405796564656\\ -0.944002509 1.73802972170755\\ -0.888002509 9.56206308959792\\ -0.869335843 13.5442370584999\\ -0.850669176 17.5690279844068\\ -0.832002509 20.7591490219188\\ -0.813335843 22.0984420050327\\ -0.794669176 20.594630323315\\ -0.776002509 15.4794007746817\\ -0.757335843 6.40673009060272\\ -0.720002509 -21.721173157078\\ -0.701335843 -38.4724088856498\\ -0.682669176 -54.8809438813505\\ -0.664002509 -69.1677953413045\\ -0.645335843 -79.8570814422941\\ -0.626669176 -86.0269161833623\\ -0.608002509 -87.460411498115\\ -0.589335843 -84.6781995608259\\ -0.533335843 -64.6303195810211\\ -0.514669176 -59.5155038060921\\ -0.496002509 -57.3729668042809\\ -0.477335843 -58.9760331855737\\ -0.458669176 -64.3142455909003\\ -0.440002509 -72.640674341176\\ -0.402669176 -92.6284504122704\\ -0.384002509 -100.761778449986\\ -0.365335842 -105.231617063446\\ -0.346669176 -104.471983835188\\ -0.328002509 -97.4194388997134\\ -0.309335842 -83.6857951154887\\ -0.290669176 -63.6495966905279\\ -0.272002509 -38.4482235004021\\ -0.216002509 48.340840660431\\ -0.197335842 73.4310564452071\\ -0.178669176 93.3020196763562\\ -0.160002509 106.811477265355\\ -0.141335842 113.567749039726\\ -0.122669176 113.938985281459\\ -0.104002509 108.969808441094\\ -0.085335842 100.22088480763\\ -0.048002509 78.7837494812198\\ -0.029335842 69.7570199725716\\ -0.010669176 63.8451923318357\\ 0.007997491 61.9036031051377\\ 0.026664158 64.1774300154961\\ 0.045330824 70.2705794597661\\ 0.063997491 79.1850540393602\\ 0.101330824 99.1007837341398\\ 0.119997491 106.330355871257\\ 0.138664158 109.297422467967\\ 0.157330824 106.559623864279\\ 0.175997491 97.2636645164798\\ 0.194664158 81.2904447923348\\ 0.213330824 59.3127611643437\\ 0.231997491 32.7517477004058\\ 0.287997491 -53.407776348783\\ 0.306664158 -76.8020185082894\\ 0.325330825 -94.5046194185856\\ 0.343997491 -105.671798766654\\ 0.362664158 -110.201989629414\\ 0.381330825 -108.710432334612\\ 0.399997491 -102.418888429089\\ 0.418664158 -92.9756313395347\\ 0.455997491 -71.8432576575246\\ 0.474664158 -63.5216068041749\\ 0.493330825 -58.4085313857062\\ 0.511997491 -57.121550775509\\ 0.530664158 -59.6339041102971\\ 0.549330825 -65.179343909444\\ 0.567997491 -72.3980208563883\\ 0.586664158 -79.6095481233246\\ 0.605330825 -85.0365296497442\\ 0.623997491 -87.0798470271697\\ 0.642664158 -84.6104333517968\\ 0.661330825 -77.1871536867392\\ 0.679997491 -65.1506349049075\\ 0.698664158 -49.572542462165\\ 0.717330825 -32.0773041996446\\ 0.735997491 -14.5638818329428\\ 0.773330825 13.9317317164673\\ 0.791997491 22.6510961274442\\ 0.810664158 27.1462296469225\\ 0.829330825 27.8614521599061\\ 0.847997491 25.6569695274289\\ 0.866664158 21.5995188965371\\ 0.885330825 16.7500924223234\\ 0.903997491 11.9869581663639\\ 0.922664158 7.89971774095781\\ 0.941330825 4.76517990841293\\ 0.959997491 2.59468844521616\\ 0.978664158 1.22989958922934\\ 0.997330825 0.45644799778314\\ 1.015997491 0\\ 1.034664158 0\\ 1.053330825 0\\ 1.090664158 0\\ 1.109330825 0\\ 1.127997492 0\\ 1.146664158 0\\ 1.165330825 0\\ 1.183997492 0\\ 1.202664158 0\\ 1.221330825 0\\ 1.277330825 0\\ 1.295997492 0\\ 1.314664158 0\\ 1.333330825 0\\ 1.351997492 0\\ 1.370664158 0\\ 1.389330825 0\\ 1.407997492 0\\ 1.445330825 0\\ 1.463997492 0\\ 1.482664158 0\\ 1.501330825 0\\ 1.519997492 0\\ 1.538664158 0\\ 1.557330825 0\\ 1.575997492 0\\ 1.594664158 0\\ 1.613330825 0\\ 1.631997492 0\\ 1.650664158 0\\ 1.669330825 0\\ 1.687997492 0\\ 1.706664158 0\\ 1.743997492 0\\ 1.762664158 0\\ 1.781330825 0\\ 1.799997492 0\\ 1.818664158 0\\ 1.837330825 0\\ 1.855997492 0\\ 1.874664159 0\\ 1.911997492 0\\ 1.930664159 0\\ 1.949330825 0\\ 1.967997492 0\\ 1.986664159 0\\ 2.023997492 0\\
		};
  \end{axis}
 \end{tikzpicture}%
 \caption{Resulting real part of the function $u_{l}$} \label{fig:6.3}
\end{figure}		

The other inclusion can be established by using the properties of the Floquet transformation shown above and the completeness of the Bloch waves. Hence, this proof follows that for a general $m$-th order linear differential operator with periodic coefficients. Again, for the sake of completeness, we include the proof here, as given in \cite[Section 3.6]{dorfler2011photonic}.
~\newpage
 \begin{theorem} \label{4.1:thm-MainResult.SecondInclusion}

	$\sigma(A) \subseteq \bigcup_{s \in \N} I_{s}.$

	\begin{proof}
		Let $\lambda \in \R \setminus \bigcup_{s \in \N} I_{s}$. Hence, due to \eqref{UIclosed}, there exists some $\delta > 0$ such that
			\begin{equation}
				|\lambda_{s}(k) - \lambda| \geq \delta \quad \text{ for all } s \in \N, k \in B \label{lambda-distance}
			\end{equation}
		We are going to prove that $\lambda \in \rho(A)$, i.e. for each $f \in L^{2}(\R)$ there exists some $u \in \mathcal{D}(A)$ satisfying $(A-\lambda I)u = f$. For an arbitrary $f \in L^{2}(\R)$ and $l \in \N$, we define at first
			\[ f_{l}(x) \coloneqq \frac{1}{\sqrt{|B|}} \sum_{s=1}^{l} \int_{B} \langle (Uf)(\cdot, k), \psi_{s}(\cdot, k)\rangle_{L^{2}(\Omega)} \psi_{s}(x,k) dk \]
			and
			\begin{equation}
				u_{l} \coloneqq \frac{1}{\sqrt{|B|}} \sum_{s=1}^{l} \int_{B} \frac{1}{\lambda_{s}(k) - \lambda} \langle (Uf)(\cdot, k), \psi_{s}(\cdot, k)\rangle_{L^{2}(\Omega)} \psi_{s}(x, k) dk \label{ul}
			\end{equation} 

		Since $\lambda$ is chosen to be outside of the spectrum, the operator $(A_{k} - \lambda I)$ is invertible, and therefore the following equation has for every $f \in L^{2}(\R)$ and $k \in \overline{B}$ a unique solution $v \in \mathcal{D}(A_{k})$
		\begin{equation}
			(A_{k} - \lambda I) v(\cdot, k) = (Uf)(\cdot, k) \quad \text{ on } \Omega. \label{4.9}			
		\end{equation}
		Due to \eqref{4.9}, both $v(\cdot, k)$ and $\psi_{s}(\cdot, k)$ satisfy quasi-periodic boundary conditions. Hence, \eqref{eigv-problem}, \eqref{lambda-distance} and Parseval's identity (Theorem \ref{athm:parseval}) yield
		\begin{align*}
			\| (Uf)(\cdot, k)\|^{2}_{L^{2}(\Omega)} & = \sum_{s=1}^{\infty} |\langle (Uf)(\cdot, k), \psi_{s}(\cdot, k)\rangle_{L^{2}(\Omega)}|^{2} \\
			& = \sum_{s=1}^{\infty}|\langle (A_{k} - \lambda) v(\cdot, k), \psi_{s}(\cdot, k)\rangle_{L^{2}(\Omega)}|^{2} \\
			& = \sum_{s=1}^{\infty} |\lambda_{s}(k) - \lambda|^{2} |\langle v(\cdot, k), \psi_{s}(\cdot, k)\rangle_{L^{2}(\Omega)}|^{2} \\
			& \geq \delta^{2} \| v(\cdot, k)\|^{2}_{L^{2}(\Omega)}.
		\end{align*}
		By Theorem \ref{3.2:thm-UIsometricIsomorphism} we know that $f \in L^{2}(\Omega \times B)$. This implies $v \in L^{2}(\Omega \times B)$, and we can define $u \coloneqq U^{-1} v \in L^{2}(\R)$. Thus, \eqref{4.9} gives
			\begin{align*}
				\langle (Uf)(\cdot, k), \psi_{s}(\cdot, k) \rangle_{L^{2}(\Omega)} & = \langle (A_{k} - \lambda I)(Uu)(\cdot, k), \psi_{s}(\cdot, k) \rangle_{L^{2}(\Omega)} \\
					& = \langle (Uu)(\cdot,k), (A_{k} - \lambda I) \psi_{s}(\cdot, k) \rangle_{L^{2}(\Omega)} \\
					& = (\lambda_{s}(k) - \lambda) \langle Uu(\cdot, k), \psi_{s}(\cdot, k) \rangle_{L^{2}(\Omega)}.
			\end{align*}
		Now, we are able to apply Theorem \ref{3.3:thm-flConvergence} which yields for \eqref{ul} that
			\[ u_{l}(x) = \frac{1}{\sqrt{|B|}} \sum_{s=1}^{l} \int_{B} \langle (Uu)(\cdot, k), \psi_{s}(\cdot, k)\rangle_{L^{2}(\Omega)} \psi_{s}(x, k) dk, \]
		and whence Theorem \ref{3.3:thm-flConvergence} gives
			\begin{equation}
				u_{l} \rightarrow u, \quad f_{l} \rightarrow f \quad \text{ in } L^{2}(\R) \text{ as } l \rightarrow \infty. \label{ulflconvergence}
			\end{equation}
		We will now prove that, in a distributional sense, 
		\[ (A - \lambda I) u_{l} = f_{l}, \]
		for all $l \in \N$, which implies that
		\begin{equation}
				\langle u_{l}, (A - \lambda I) v \rangle = \langle f_{l}, v\rangle \text{ for all } l \in \N, v \in \mathcal{D}(A); \label{lefttoprove}
			\end{equation} 
		As $A$ is self-adjoint, by Theorem \ref{2.3:thm-ASelfAdjoint}, this implies $u_{l} \in \mathcal{D}(A)$, and $(A - \lambda I) u_{l} = f_{l}$ for all $l \in \N$. As furthermore every self-adjoint operator is also closed (Theorem \eqref{athm:propself}), \eqref{ulflconvergence} now implies
			\[ u \in \mathcal{D}(A) \text{ and } (A - \lambda I) u = f, \]
		which is the desired result. 
		~\\
				
		Eventually, we are left to prove \eqref{lefttoprove}. So, let $\varphi \in C_{0}^{\infty}(\R)$ be fixed, and let $K \subseteq \R$ denote an open ball containing $\supp(\varphi)$ in its interior. By Fubini's Theorem we know that 
		\begin{align*}
			r_{s}(x, k) & \coloneqq \frac{1}{\lambda_{s}(k) - \lambda} \langle (Uf)(\cdot, k), \psi_{s}(\cdot, k) \rangle_{L^{2}(\Omega)} \psi_{s}(x, k) \overline{(A - \lambda I) \varphi(x)}, 
			\intertext{and}
			t_{s}(x, k) & \coloneqq \langle (Uf)(\cdot, k), \psi_{s}(\cdot, k) \rangle_{L^{2}(\Omega)} \psi_{s}(x, k) \overline{\varphi(x)}
		\end{align*}
		are in $L^{2}(K \times B)$, since \eqref{lambda-distance}, $(A_{k} - \lambda I) \varphi \in L^{\infty}(K)$ and $\varphi \in L^{\infty}(K)$ imply 
		\[ \| r_{s} \|_{L^{2}(K \times B)} \leq c \| (Uf)(\cdot, k) \|^{2}_{L^{2}(\Omega)} \| \psi_{s}(\cdot, k) \|^{2}_{L^{2}(K)}\]
		and analogously for $t_{s}$. As $K$ is bounded there exists a finite number of copies of $\Omega$ such that they cover $K$, hence $\psi_{s}(\cdot, k)$ is in $L^{2}(K)$ as a function of $k$, and $(Uf)(\cdot, k)$ is in $L^{1}(B)$ by Theorem \ref{3.2:thm-UIsometricIsomorphism}. Since $B$ is equally bounded, $r$ and $t$ are also in $L^{1}(K \times B)$. Therefore, Fubini’s Theorem implies that the order of integration with respect to $x$ and $l$ may be exchanged for $r$ and $t$. Thus, by \eqref{ul}, the fact that $\varphi$ has compact support in the interior of $K$ and \eqref{eigv-problem} we conclude
			\begin{align*}
				\int_{K} u_{l}(x) \overline{(A - \lambda I) \varphi(x)} dx & = \frac{1}{\sqrt{|B|}} \sum_{s=1}^{l} \int_{K} \left( \int_{B} r_{s}(x, k) dk \right) dx \\
					& = \frac{1}{\sqrt{|B|}} \sum_{s=1}^{l} \int_{B} \langle (Uf)(\cdot, k), \psi_{s}(\cdot, k) \rangle_{L^{2}(\Omega)} \langle \psi_{s}(\cdot, k), \varphi \rangle_{L^{2}(K)} dk \\
					& = \int_{K} \left[ \frac{1}{\sqrt{|B|}} \sum_{s=1}^{l} \int_{B} \langle (Uf)(\cdot, k), \psi_{s}(\cdot, k) \rangle_{L^{2}(\Omega)} \psi_{s}(x, k) dk \right] \overline{\varphi(x)} dx \\
					& = \int_{K} f_{l}(x) \overline{\varphi(x)} dx,
			\end{align*}
			i.e. \eqref{lefttoprove}.
	\end{proof}
\end{theorem}