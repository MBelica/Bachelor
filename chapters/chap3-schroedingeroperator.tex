\chapter{The one-dimensional Schrödinger operator} \label{chap:3}

The mathematical representation of the problem stated above can be done by introducing a one-dimensional Schrödinger operator $A$ where the potential is given by a periodic delta-distribution. In this chapter we will examine properties of $A$ such as its domain and show its self-adjointness. Later, in Chapters \ref{chap:4} and \ref{chap:6}, we will need these results to deduce our main result, i.e. characteristics of the spectrum of $A$. ~\\ ~\\
Formally the operation of $A$ is defined by
\begin{equation}
	- \frac{d^{2}}{dx^{2}} + \rho \sum_{i \in \Z} \delta_{x_{i}} \label{the-operator-A-formally}
\end{equation}
on the whole of $\R$, where $\delta_{x_{i}}$ denotes the Delta-Distribution supported at the point $x_{i}$. $\Omega_{k}$ will hereafter identify the periodicity cell containing point $x_{k}$ and w.l.o.g. let $x_{0} = 0$ and $|\Omega_{i}| = 1$ for all $i \in \Z$. By this assumption, we then can rearrange all other points where the Delta-Distribution function is supported $(x_{i})_{i \in \Z \setminus \{0\}}$ such that $x_{i} = x_{0} + i$ for all $i \in \Z \setminus \{0\}$. 
~\\  

In general, one cannot say in which sense a solution to the formal problem 
	\begin{equation}
		Au = f \quad \text{for } f \in L^{2}(\R) \label{formal-problem}
	\end{equation}
	exists since the potential in $A$ consists of the Delta-Distribution. If we suppose for a moment that the problem is smooth, more specifically, that the potential is instead given by \eqref{eq:2.1-smooth_potential} for some $\epsilon > 0$, then formally multiplying it by a test function and integrating by parts yields the so-called weak-formulation to the problem whose solution requires less regularity. Motivated by this, by taking the limit of the potential in the weak-formulation, we henceforth consider the problem to find for $\mu \in \R$ a function $u \in H^{1}(\R)$ such that
\begin{equation}
	\int_{\R} u'(x) \overline{v'(x)} dx + \rho \sum_{i \in \Z} u(x_{i}) \overline{v(x_{i})} - \mu \int_{\R} u(x) \overline{v(x)} dx = \int_{\R} f(x) \overline{v(x)} dx \quad \forall v \in C_{0}^{\infty}(\R), \label{weak-formulation-of-A}
\end{equation}	
holds and call \eqref{weak-formulation-of-A} the weak-formulation of \eqref{formal-problem}. We should note that the left-hand side of problem \eqref{weak-formulation-of-A} is actually well-defined and finite, as for any $h \in (0, 1]$ we can estimate the norm with the help of some integration methods via
\begin{align}
	\sum_{i \in \Z} |u(x_{i})|^{2} & \leq \sum_{i \in \Z} \left( 2 |u( x_{i} + h )|^{2} +  2 h \int_{x_{i}}^{x_{i} + h} \left| u'(\tau) \right|^{2} d\tau \right) \notag \\
		 & \leq 2 \sum_{i \in \Z} \left( \frac{1}{h} \int_{\Omega_{i}} |u( x )|^{2} dx + h \int_{\Omega_{i}} \left| u'(\tau) \right|^{2} d\tau \right). \label{preestimation-for-potential}
\end{align}
The choice of $h = 1$ yields hence the bound
\begin{align} 
		\sum_{i \in \Z} |u(x_{i})|^{2} & \leq 2 \| u \|^{2}_{H^{1}(\R)}. \label{estimation-for-potential}
\end{align}

\begin{remark}
	Since $C_{0}^{\infty}(\R)$ is dense in $L^{2}(\R)$, \eqref{weak-formulation-of-A} holds also for all $v \in H^{1}(\R)$.
\end{remark}

\section{The resolvent-mapping of the one-dimensional Schrödinger operator} \label{sec:3.1}

As a first step in order to define the operator $A$ explicitly, we will show that for each $f \in L^{2}(\R)$ the equation \eqref{weak-formulation-of-A} has a unique solution $u \in H^{1}(\R)$. 
\begin{definition}
	Given $f \in L^{2}(\R)$, we define a functional $l_{f} \colon H^{1}(\R) \rightarrow \C$ by
	\[ l_{f}(v) \coloneqq \int_{\R} f(x) \overline{v}(x) dx \]
and a sesquilinear form $B_{\mu} \colon H^{1}(\R) \times H^{1}(\R) \rightarrow \C$ for $\mu \in \R$ by
	\[ B_{\mu}[u, v] \coloneqq \int_{\R} u'(x) \overline{v'(x)} dx + \rho \sum_{i \in \Z} u(x_{i}) \overline{v(x_{i})} - \mu \int_{\R} u(x) \overline{v(x)} dx. \]
\end{definition}
As a result, \eqref{weak-formulation-of-A} is equivalent to finding for $\mu \in \R$ a function $u \in H^{1}(\R)$ such that
	\begin{equation}
		B_{\mu}[u, v] =  l_{f}(v) \label{weak-formulation-of-A-for-LM}
	\end{equation}
holds for all $v \in H^{1}(\R)$. The existence of a unique $u \in H^{1}(\R)$ satisfying \eqref{weak-formulation-of-A-for-LM} now follows from Lax-Milgram's Theorem if the sesquilinear form $B_{\mu}$ is bounded and coercive and if $l_{f}$ is a bounded linear functional on $H^{1}(\R)$, which we will prove in the next two theorems.
\begin{theorem} \label{2.1:thm-LaxMilgram}
	The sesquilinear form $B_{\mu}$ is (for sufficiently small $\mu \in \R$)
	\begin{enumerate}[label=\alph*\upshape)]
		\item bounded, i.e. there exists a constant $\alpha > 0$ such that
			\[ \left| B_{\mu}[u,v] \right| \leq \alpha \|u\|_{H^{1}(\R)} \|v\|_{H^{1}(\R)} \]
			holds for all $u, v \in H^{1}(\R)$.
		\item coercive, i.e. there exists a constant $\beta > 0$ such that
			\[ \beta \|u\|_{H^{1}(\R)}^{2} \leq Re(B_{\mu}[u, u]) \]
			holds for all $u \in H^{1}(\R)$.
	\end{enumerate} 
	
	\begin{proof} ~\\
		$a)$ The boundedness follows from the Cauchy–Schwarz inequality and \eqref{estimation-for-potential} as for an arbitrary $\rho \in \R$
		\begin{align*}
			\left| B(u, v) \right|^{2} & \leq 3 \| u' \|_{L^{2}(\R)}^{2} \| v' \|_{L^{2}(\R)}^{2} + 3 |\rho |\left( \sum_{i \in \Z} |u(x_{i})|^{2} \right)\left( \sum_{i \in \Z} |v(x_{i})|^{2} \right) + 3 \left| \mu \right| \| u \|_{L^{2}(\R)}^{2} \| v \|_{L^{2}(\R)}^{2} \\
				& \leq 3 \| u' \|_{L^{2}(\R)}^{2} \| v' \|_{L^{2}(\R)}^{2} + 12 |\rho| \| u \|^{2}_{H^{1}(\R)} \| v \|^{2}_{H^{1}(\R)}  + 3 \left| \mu \right| \| u \|_{L^{2}(\R)}^{2} \| v \|_{L^{2}(\R)}^{2} \\
				& \leq \alpha \| u \|^{2}_{H^{1}(\R)} \| v \|^{2}_{H^{1}(\R)},
		\end{align*}
		holds for all $u, v \in H^{1}(\R)$ where $\alpha = \max \left\{ 12|\rho| + 3\left| \mu \right| , 12 |\rho| + 3 \right\}$. \\
		
		$b)$ Let $u \in H^{1}(\R)$. For the coercivity, we note first that for the given sesquilinear form $B[u, u] \in \R$ holds. Assuming $\rho \geq 0$ yields for $\mu < -1$ that
		\begin{align*}
			B[u, u] & = \langle u' , u' \rangle + \rho \sum_{i \in \Z} \left|u(x_{i})\right|^{2} - \mu \langle u , u \rangle \\
					& \geq \langle u' , u' \rangle  + \langle u , u \rangle \\
					& = \| u \|_{H^{1}(\R)}^{2}.
		\intertext{Analogously, for $\rho < 0$, using \eqref{preestimation-for-potential} we can choose $0 < h < \frac{1}{2 |\rho|}$ and with that if $\mu < - \frac{2|\rho|}{h}$}
			B[u, u] & = \langle u' , u' \rangle + \rho \sum_{i \in \Z} |u(x_{i})|^{2} - \mu 	\langle u , u \rangle \\
					& \geq \langle u' , u' \rangle + 2 \rho \sum_{i \in \Z} \left( \frac{1}{h} \int_{\Omega_{i}} |u( x )|^{2} dx + h \int_{\Omega_{i}} \left| u'(\tau) \right|^{2} d\tau \right) - \mu \langle u , u \rangle \\
					& = (2 \rho h + 1) \| u' \|_{L^{2}(\R)}^{2} + (2 \rho \frac{1}{h} - \mu) \| u \|_{L^{2}(\R)}^{2}  \\
					& \geq \beta \| u \|_{H^{1}(\R)}^{2},
		\end{align*}
		where $\beta = \min \left\{ 2 \rho h + 1u , 2 \rho \frac{1}{h} - \mu \right\}$.
	\end{proof}
\end{theorem}
\begin{theorem}
	Given $f \in L^{2}(\R)$ the functional $l_{f}$ is a bounded linear functional on $H^{1}(\R)$.
	
	\begin{proof}
		That $l_{f}$ is linear follows from the linearity of the integral. The Cauchy–Schwarz inequality yields for the boundedness
		\begin{equation*}
			| l_{f}(v) | \leq \| f \|_{L^{2}(\R)} \| v \|_{L^{2}(\R)} \leq \| f \|_{L^{2}(\R)} \| v \|_{H^{1}(\R)} \vspace{-1cm}
		\end{equation*}
	\end{proof}
\end{theorem}
Therefore, as used in Theorem \ref{2.1:thm-LaxMilgram}, we will subsequently assume that $\mu \in \R$ is small enough. In return, Lax-Migram's Theorem proves that for any fixed $f \in L^{2}(\R)$ a unique solution $u \in H^{1}(\R)$ to the problem \eqref{weak-formulation-of-A-for-LM} exists. This on the other hand allows us to proceed as follows.
\begin{definition}
	Let us define $R_{\mu} \colon L^{2}(\R) \rightarrow L^{2}(\R), f \mapsto u$ with $u$ being the solution of \eqref{weak-formulation-of-A-for-LM}.
\end{definition}
Again, due to the linearity of the integral and the uniqueness of the solution, $R_{\mu}$ is a linear operator. There are two more properties of $R_{\mu}$ for us left to show to explicitly define the operator $A$. 
\begin{theorem} \label{rmuinj}
	The mapping $R_{\mu}$ is bounded and injective.
	
	\begin{proof}
		By Theorem \ref{2.1:thm-LaxMilgram} there exists for $f \in L^{2}(\R)$ a function $u \in \mathcal{D}(A)$ as a solution of \eqref{weak-formulation-of-A-for-LM} and hence
		\[ \| R_{\mu} f \|_{L^{2}(\R)}^{2} = \| u \|_{L^{2}(\R)}^{2} \leq \| u \|_{H^{1}(\R)}^{2}. \] 
		Now, using \eqref{weak-formulation-of-A}, \eqref{estimation-for-potential} with a small enough $\mu \in \R$ yields with Cauchy–Schwarz's inequality
		\[ \| R_{\mu} f \|_{L^{2}(\R)}^{2} \leq \int_{\R} |u'(x)|^{2} dx + \rho \sum_{i \in \Z} |u(x_{i})|^{2} - \mu \int_{\R} |u(x)|^{2} dx \leq \| f \|_{L^{2}(\R)}^{2} \| u \|_{L^{2}(\R)}^{2}, \]	
		which shows the boundedness of the mapping $R_{\mu}$. Taking in mind that the range $\mathcal{R}(R_{\mu}) \subseteq H^{1}(\R)$, we know that for $f_{1}, f_{2} \in L^{2}(\R)$ there exist  $u_{1}, u_{2} \in \mathcal{R}(R_{\mu})$ with $u_{i} = R_{\mu} f_{i}$ for $i = 1, 2$. If now furthermore $R_{\mu} f_{1} = R_{\mu} f_{2}$ holds, \eqref{weak-formulation-of-A-for-LM} yields
		\begin{equation}
			0 = B_{\mu}[u_{1}, v] - B_{\mu}[u_{2}, v] = \int_{\R} (f_{1}(x) - f_{2}(x)) \overline{v(x)} dx \quad \forall v \in C_{0}^{\infty}(\R). \label{f1f2almosteverywhere}
		\end{equation} 
		As $C_{0}^{\infty}(\R)$ is dense in $L^{2}(\R)$, we know from the equation \eqref{f1f2almosteverywhere} that
		\[ 0 = \int_{\R} \left(f_{1}(x) - f_{2}(x)\right) \overline{v(x)} dx \quad \forall v \in L^{2}(\R), \]
		i.e. $f_{1} = f_{2}$ almost everywhere.
	\end{proof}
\end{theorem}

\section{The domain of the one-dimensional Schrödinger operator} \label{sec:3.2}

Resulting from Theorem \ref{rmuinj}, we know that $R_{\mu}$ is invertible. This allows us to define the aforementioned operator $A$ explicitly.
\begin{definition}
	Let $A \colon \mathcal{D}(A) \subseteq L^{2}(\R) \rightarrow L^{2}(\R)$ be the linear operator defined by
	\[ A \coloneqq R_{\mu}^{-1} + \mu I, \quad \mathcal{D}(A) = \mathcal{R}(R_{\mu}). \]
\end{definition}
Note that this definition is consistent with the formal definition in \eqref{the-operator-A-formally} and we will show that it is independent of the choice of $\mu \in \R$, still assuming $\mu$ is small enough as chosen in theorem \ref{2.1:thm-LaxMilgram}. 
\begin{remark}
	Note, $R_{\mu}$ is the resolvent of $A$.
\end{remark}
We will now use the fact that every element $u \in \mathcal{D}(A) = \mathcal{R}(R_{\mu})$ is a solution of \eqref{weak-formulation-of-A-for-LM} to find additional necessary  characteristics of $\mathcal{D}(A)$. However, we already know by Lax-Milram's Theorem that $u \in H^{1}(\R)$. Let us first for the sake of brevity define
\[ H^{2}\Big(\R \setminus \bigcup_{i \in \Z} x_{i} \Big) \coloneqq \Big\{ u\in L^{2}(\R) : u\big|_{(x_{i}, x_{i+1})} \in H^{2}(x_{i}, x_{i+1}) ~\forall i \in \Z, \sum_{i \in \Z} \|u\|^{2}_{H^{2}(x_{i}, x_{i+1})} < \infty \Big\}. \]
Then, considering in \eqref{weak-formulation-of-A} any fixed $k \in \Z$ and an arbitrary test function $v \in C^{\infty}(\R)$ with $\supp v \subseteq [x_{k}, x_{k+1}]$ we get 
	\begin{equation}
		\int_{x_{k}}^{x_{k + 1}} u'(x) \overline{v'(x)} dx = \int_{x_{k}}^{x_{k+1}} \left(Au\right)(x) \overline{v(x)} dx \iff \int_{x_{k}}^{x_{k+1}} - u(x) \overline{v''(x)} dx = \int_{x_{k}}^{x_{k+1}} \left(Au\right)(x) \overline{v(x)} dx, \label{temp-link}
	\end{equation} 
whence $u'' \in L^{2}(x_{k}, x_{k + 1})$ and $A u = - u''$ on $(x_{k}, x_{k + 1})$. Since we chose an arbitrary $k \in \Z$, we can note 
	$$ \mathcal{D}(A) \subseteq \Big\{ u \in H^{1}(\R) \colon u\big|_{(x_{i}, x_{i+1})} \in H^{2}(x_{i}, x_{i+1}) ~\forall i \in \Z \Big\}. $$
Using this, a test function $v \in C^{\infty}(\R)$ with the property $\supp v = \Omega_{k}$ yields in \eqref{weak-formulation-of-A} for any $k \in \Z$ through integration by parts on both sides of $x_{k}$ that
	\[ - \int_{x_{k}-\frac{1}{2}}^{x_{k}} u''(x) \overline{v(x)} dx - \int_{x_{k}}^{x_{k} + \frac{1}{2}} u''(x) \overline{v(x)} dx + \left( u'(x_{k}-0) \overline{v(x_{k})} - u'(x_{k} + 0) \overline{v(x_{k})} \right) \\ \]
	\[ +  \rho u(x_{k})\overline{v(x_{k})} = - \int_{x_{k} - \frac{1}{2}}^{x_{k}} u''(x) \overline{v(x)} dx - \int_{x_{k}}^{x_{k} + \frac{1}{2}} u''(x) \overline{v(x)} dx. \]
Now, choosing in addition $v$ to be non-zero in $x_{k}$ yields 
	\begin{equation}
		u'(x_{k}-0) - u'(x_{k}+0) + \rho u(x_{k}) = 0, \label{jump}
	\end{equation} 
and therefore
	\begin{equation}
		\mathcal{D}(A) \subseteq \Big\{ u \in H^{1}(\R) \colon u\big|_{(x_{i}, x_{i+1})} \in H^{2}(x_{i}, x_{i+1}), u'(x_{j} - 0) - u'(x_{j} + 0) + \rho u(x_{j}) = 0 ~\forall i, j \in \Z \Big\}
	\end{equation} 
	
Finally, choosing a function $v \in C_{0}^{\infty}(\R)$ with $\supp v = (x_{-n}, x_{n})$ in \eqref{weak-formulation-of-A} for some arbitrary $n \in \N$ yields with partial integration on every interval $(x_{i}, x_{i+1})$ by using \eqref{jump} that
\[ \sum_{i=-n}^{n-1} -\int_{x_{i}}^{x_{i+1}} u''(x) \overline{v(x)} dx + \sum_{i=-n}^{n-1} u' \overline{v} \big|_{x_{i}}^{x_{i+1}} + \rho \sum_{i=-n}^{n-1} u(x_{i}) \overline{v(x_{j})} - \mu \int_{x_{-n}}^{x_{n}} u(x) \overline{v(x)} dx = \int_{x_{-n}}^{x_{n}} f(x) \overline{v(x)} dx \]
\begin{equation}
	\iff \sum_{i=-n}^{n-1} \int_{x_{i}}^{x_{i+1}} u''(x) \overline{v(x)} dx = - \int_{x_{-n}}^{x_{n}} f(x) \overline{v(x)} dx - \mu \int_{x_{-n}}^{x_{n}} u(x) \overline{v(x)} dx. \label{refwa}
\end{equation} 
By defining $w_{n} \coloneqq \sum_{i=-n}^{n-1} u'' \mathds{1}_{[x_{i}, x_{i+1}]}$ we can estimate the left-hand side of \eqref{refwa} by
\begin{align}
	\left| \langle w_{n}, v \rangle \right| & \leq \left| \int_{x_{-n}}^{x_{n}} f(x) \overline{v(x)} dx \right| + \left| \mu \int_{x_{-n}}^{x_{n}} u(x) \overline{v(x)} dx \right|  \notag \\
		& \leq \|f\|_{L^{2}(x_{-n}, x_{n})} \|v\|_{L^{2}(x_{-n}, x_{n})} + |\mu| \|u\|_{L^{2}(x_{-n}, x_{n})} \|v\|_{L^{2}(x_{-n}, x_{n})} \notag \\
		& \leq c \|v\|_{L^{2}(x_{-n}, x_{n})}, \label{refwa2}
\end{align}
for some $c \in \R$, since $f \in L^{2}(\R)$ and $u \in H^{1}(\R)$. This constant is independent of $n$, from which, with \eqref{refwa2}, follows that $\sum_{i \in \Z} \|u''\|^{2}_{L^{2}(x_{i}, x_{i+1})} < \infty$. This yields the inclusion
	\begin{equation}
		\mathcal{D}(A) \subseteq \left\{ u \in H^{1}(\R): u \in H^{2}\Big(\R \setminus \bigcup_{i \in \Z} x_{i} \Big), u'(x_{j} - 0) - u'(x_{j} + 0) + \rho u(x_{j}) = 0 ~\forall j \in \Z \right\}. \label{firstdomaininclusion} 
	\end{equation}
Hence, for an arbitrary $u \in \mathcal{D}(A)$ we know from \eqref{temp-link} and \eqref{firstdomaininclusion} that
	\begin{equation}
		A u = \begin{cases}
					- u'' & \text{ on } (x_{k} - \frac{1}{2}, x_{k}) \\
					- u'' & \text{ on } (x_{k}, x_{k} + \frac{1}{2}),
			 \end{cases} \quad \forall k \in \Z. \label{Aaction}
	\end{equation} 
We are furthermore able to show in \eqref{firstdomaininclusion} the reverse inclusion by using the resolvent $R_{\mu}$. But first let us, again for brevity, denote with $B$ the right-hand side of \eqref{firstdomaininclusion}. Now, since $\mathcal{R}(R_{\mu}) = \mathcal{D}(A)$, we proceed by proving that each $u \in B$ is also in the range of $R_{\mu}$. More specifically, as $\mathcal{D}(R_{\mu}) = L^{2}(\R)$ let us define $f \coloneqq - u''$ on $(x_{k}, x_{k + 1})$ for all $i \in \Z$; as we already know that $u \in H^{2}\Big(\R \setminus \bigcup_{i \in \Z} x_{i} \Big)$ we can therefore ensure $f \in L^{2}(\R) = \mathcal{D}(R_{\mu})$. We want to show that $u = R_{\mu}(f - \mu u)$ or equivalently 
	\begin{align*}
		 \int_{\R} u'(x) \overline{v'(x)} dx + \rho \sum_{i \in \Z} u(x_{i}) \overline{v(x_{i})} - \mu \int_{\R} u(x) \overline{v(x)} dx = \int_{\R}(f(x)-\mu u(x)) \overline{v(x)} dx \\
		\iff \sum_{i \in \Z} \int_{\Omega_{i}} u'(x) \overline{v'(x)} + \rho \sum_{i \in \Z} u(x_{i}) \overline{v(x_{i})} = - \sum_{i \in \Z} \int_{x_{i}}^{x_{i+1}} u''(x) \overline{v(x)} dx.
	\end{align*}
For each $k \in \Z$ partial integration on both side of $x_{k}$ with a function $v \in C_{0}^{\infty}(\R)$ having $\supp v = \Omega_{k}$ and $v(x_{k}) \neq 0$ yields
	\begin{align*}
		\int_{\Omega_{k}} u'(x) \overline{v'(x)} dx + \rho u(x_{k}) \overline{v(x_{k})} & = \int_{x_{k} - \frac{1}{2}}^{x_{k}} u'(x) \overline{v'(x)} dx + \int_{x_{k}}^{x_{k} +\frac{1}{2}} u'(x) \overline{v'(x)} dx \\
		& ~\qquad ~\quad - u'(x_{k}-0) \overline{v(x_{k})}  + u'(x_{k}+0) \overline{v(x_{k})},
	\end{align*}
which is equivalent to
	\[ u'(x_{k}-0)\overline{v(x_{k})} - u'(x_{k}+0)\overline{v(x_{k})} + \rho u(x_{k})\overline{v(x_{k})} = 0. \]
Consequently, we conclude that
	\begin{equation}
		\mathcal{D}(A) = \Big\{ u \in H^{1}(\R): u \in H^{2}\Big(\R \setminus \bigcup_{i \in \Z} x_{i} \Big), u'(x_{j} - 0) - u'(x_{j} + 0) + \rho u(x_{j}) = 0 ~\forall j \in \Z \Big\}. \label{DomA}
	\end{equation}
\begin{remark}
	From \eqref{Aaction} and \eqref{DomA} follows that the definition of $A$ is independent of $\mu$.
\end{remark}

\section{The self-adjointness of the Schrödinger operator} \label{sec:3.3}

In Chapter \ref{chap:6} we will need the fact that the operator $A$ is self-adjoint. For this we first have to show that $R_{\mu}$ and $R_{\mu}^{-1}$ are symmetric operators.

\begin{theorem} \label{2.2:thm-RmuSymmetric}
	$R_{\mu}$ and $R_{\mu}^{-1}$ are symmetric operators.
	
	\begin{proof}
		We start with $R_{\mu}^{-1} = (A - \mu I)$. As for all $v \in \mathcal{D}(A)$ with \eqref{weak-formulation-of-A} follows:
			\begin{align*}
				\langle R_{\mu}^{-1} u, v \rangle & = \langle (A - \mu I) u, v \rangle \\
					& = \int_{\R} u'(x) \overline{v'(x)} dx -  \mu \int_{\R} u(x) \overline{v(x)} dx + \rho \sum_{i \in \Z} u(x_{i}) \overline{v(x_{i})} \\
					& = \langle u, (A - \mu I) v \rangle = \langle u,  R_{\mu}^{-1} v \rangle,
			\end{align*}
		thus, $R_{\mu}^{-1}$ is symmetric. Now, as $\mathcal{D}(R_{\mu}) = L^{2}(\R)$ and $\mathcal{R}(R_{\mu}) = \mathcal{D}(R_{\mu}^{-1})$ for each $f, g \in L^{2}(\R)$ it follows
		\[  \langle R_{\mu} f, g \rangle =  \langle R_{\mu} f, R_{\mu}^{-1} R_{\mu} g \rangle = \langle f, R_{\mu} g \rangle, \]
		thus, $R_{\mu}$ is also symmetric.
	\end{proof}
\end{theorem}

Using the fact that $R_{\mu}$ and $R_{\mu}^{-1}$ are symmetric operators we can now prove the main statement of this section. Since every symmetric operator has an entirely real spectrum, this theorem yields further our first result about the spectrum of $A$, for a proof see Theorem \ref{spectrul-sa-real}.

\begin{theorem} \label{2.3:thm-ASelfAdjoint}
	$A$ is a self-adjoint operator.
		
	\begin{proof}
		We proceed by proving first that $R_{\mu}^{-1}$ is self-adjoint. As we already know that $R_{\mu}^{-1}$ is symmetric, showing that $R_{\mu}^{-1}$ is self-adjoint is equivalent to showing that if $v \in \mathcal{D}({R_{\mu}^{-1}}^{*})$ and $v^{*} \in L^{2}(\R)$ are such that
		\begin{align}
			\langle R_{\mu}^{-1} u, v \rangle = \langle u, v^{*} \rangle \quad \forall u \in \mathcal{D}(R_{\mu}^{-1}), \label{*-condition}
		\end{align}
		then $v \in \mathcal{D}(R_{\mu}^{-1})$ and $R_{\mu}^{-1} v = v^{*}$.
		In \eqref{*-condition} for any $u \in \mathcal{D}(R_{\mu}^{-1})$ exists $f \in L^{2}(\R)$ such that $u = R_{\mu} f$; using use the fact that $R_{\mu}$ is symmetric and defined on the whole of $L^{2}(\R)$ yields
		\[  \langle f, v \rangle = \langle R_{\mu} f, v^{*} \rangle = \langle f, R_{\mu} v^{*} \rangle, \]
		
		which means that $v \in \mathcal{R}(R_{\mu}) = \mathcal{D}(R_{\mu}^{-1})$ and $R_{\mu} v^{*} = v$, i.e. $R_{\mu}^{-1}$ is self-adjoint. As the operator $A$ is simply $R_{\mu}^{-1}$ shifted by $\mu \in \R$, $A$ is hence self-adjoint as well.		
	\end{proof}
\end{theorem}