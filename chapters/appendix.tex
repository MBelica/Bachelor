\chapter*{Appendix} \addcontentsline{toc}{chapter}{Appendix} 

\begin{atheorem}[Alternative definition of the Delta-Distribution] \label{athem:delta}
	For the sequence of functionals $\delta_{\epsilon}$ for $\epsilon > 0$, $x_{0} \in \R$ and $f \in \mathfrak{D}(\R)$ defined through $\delta_{\epsilon}(f) \coloneqq \frac{1}{\sqrt{2 \pi} \epsilon} \int_{\R} e^{-\frac{(x - x_{0})^{2}}{2 \epsilon^{2}}} f(x) dx$ it holds that  
 		\[ \delta_{x_{0}}(f) = \lim_{\epsilon \rightarrow 0} \delta_{\epsilon}(f). \]
	
	\begin{proof}
		Given $\delta_{\epsilon}$ and $x_{0} \in \R$ we have
			\[ \delta_{\epsilon}(f) = \frac{1}{\sqrt{2 \pi} \epsilon} \int_{-\infty}^{\infty} f(x) e^{-\frac{(x-x_{0})^{2}}{2 \epsilon^{2}} dx} \]
		Substituting $z \coloneqq \frac{x - x_{0}}{\sqrt{2} \epsilon}$ yields
			\[ \frac{1}{\sqrt{2 \pi} \epsilon} \int_{-\infty}^{\infty} f(x) e^{-\frac{(x-x_{0})^{2}}{2 \epsilon^{2}} dx} = \frac{1}{\sqrt{2 \pi} \epsilon} \int_{-\infty}^{\infty} f(\sqrt{2} \epsilon z + x_{0}) e^{-z^{2}} dz \]
		Using the Taylor series around $x_{0}$ and the Gaussian integral we then get
			\[ \lim_{\epsilon \rightarrow 0} \frac{1}{\sqrt{\pi}} \int_{-\infty}^{\infty} e^{-z^{2}} \left( f(x_{0}) + \mathcal{O}(\epsilon) \right) = f(x_{0}) \frac{1}{\sqrt{\pi}} \int_{-\infty}^{\infty} e^{-z^{2}} = f(x_{0}). \]
	\end{proof}
\end{atheorem}

\begin{atheorem}[$L^{p}$-Approximation by test functions]
	For $U \subseteq \R^{n}$ open, $C_{0}^{\infty}(U)$ is dense in $L^{p}(U)$, if $1 \leq p < \infty$.
	
	\begin{proof}
		See \cite[p. 31]{adams2003sobolev}.
	\end{proof}
\end{atheorem}

\begin{atheorem}[$H^{k}$-Approximation by test functions] 
	Let $\Omega \subseteq \R^{n}$ be an open set. Let $u \in H^{k}(\Omega)$, then there exists a sequence of functions $u_{k} \in C^{\infty}(\Omega)$ such that $\| u_{k} - u \|_{H^{k}(\Omega)}$ as $n \rightarrow \infty$.
	
	\begin{proof}
		See \cite[p. 138]{adams2003sobolev}.
	\end{proof}	
\end{atheorem}


\begin{atheorem}[Bessel's inequality]
	Let $H$ be a Hilbert space, and suppose that $e_1, e_2, ...$ is an orthonormal sequence in $H$. Then, for any $x \in H$ one has
	\[ \sum_{k=1}^{\infty}\left\vert\left\langle x,e_k\right\rangle \right\vert^2 \le \left\Vert x\right\Vert^2 \]
	where $\langle \cdot,\cdot \rangle$ denotes the inner product in the Hilbert space $H$.

	\begin{proof}
		See \cite[p. 233]{werner2006funkana}.
	\end{proof}
\end{atheorem}

\begin{atheorem}[Cauchy–Schwarz inequality]
	Fr all vectors $u$ and $v$ of an inner product space it is true that
		\[ \left| \langle u,v \rangle \right|^{2} \leq \langle u,u \rangle \cdot \langle v,v \rangle, \]
	where $\langle \cdot ,\cdot \rangle$ is the inner product. 

	\begin{proof}
		See \cite[p. 20]{werner2006funkana}.
	\end{proof}
\end{atheorem}

\begin{atheorem}[Closed graph theorem]
	Let $X$ be a Banach space. Is $A$ a closed operator and $\mathcal{D}(A) = X$, then $A$ is continuous on $X$.

	\begin{proof}
		See \cite[p. 417]{evans1998partial}.
	\end{proof}
\end{atheorem}

\begin{atheorem}[Closeness of $H^{1}_{k}$ in $H^{1}(\Omega)$] \label{h1kclosed}
	$H^{1}_{k}$ is a closed subset1 of $H^{1}(\Omega)$, and therefore a Hilbert space with respect to the norm of $H^{1}(\Omega)$.
	
	\begin{proof} 
		Let $(f_{n})_{n \in \N}$ be a sequence in $H^{1}_{k}$ converging to $f \in H^{1}(\Omega)$. We already know that convergence with respect to the $H^{1}$-Norm implies convergence of the function and its derivative almost everywhere. Let us therefore define $g \coloneqq f - f_{n}$ then
		\begin{align*}
			\left| g \left(- \frac{1}{2} \right) \right|^{2} & \leq 2 |g(x)|^{2} + 2 \left( \int_{-\frac{1}{2}}^{x} |g'(\tau)| d\tau \right)^{2} \\
			& \leq 2 |g(x)|^{2} + 2 \int_{-\frac{1}{2}}^{\frac{1}{2}} |g'(\tau)|^{2} d\tau \\
			& \leq 2 \int_{-\frac{1}{2}}^{\frac{1}{2}} |g(\tau)|^{2} d\tau + 2 \int_{-\frac{1}{2}}^{\frac{1}{2}} |g'(\tau)|^{2} d\tau \\
			& = 2 \| g \|_{H^{1}(-\frac{1}{2}, \frac{1}{2})}^{2} \longrightarrow 0
		\end{align*}
		for $j \rightarrow \infty$, and analogously on the other boundary.
	\end{proof}
\end{atheorem} % todo proof

\begin{atheorem}[Compact Embedding Theorem for Sobolev spaces] \label{compact-embedding-theorem} ~\
	\begin{enumerate}[label=\alph*\upshape)]
		\item Let $U \subseteq \R^{n}$ be a bounded open set of class $C^{1}$. Then the following compact embeddings hold:
			\begin{itemize}
				\item $H	^{1}(U) \subseteq L^{q}(U)$ for every $q \in [1, p^{*})$, where $n \geq 3$ and $p^{*} = \frac{2n}{n - 2}$.
				\item $H^{1}(U) \subseteq L^{q}(U)$ for every $q \in [1, \infty)$, if $n = 2$.
			\end{itemize}
		
			\begin{proof}
				Follows from Rellich-Kondrachov Compact Embedding Theorem, see \cite[p. 163]{precup2013linear} and \cite[p. 272]{evans1998partial}.
			\end{proof}
		\item Let $U \subseteq \R$ be a bounded, connected and open set. Then the embedding $H^{1}(U) \subseteq L^{2}(U)$ is compact.
			\begin{proof} 
				As $H^{1}(U) \subseteq C^{\frac{1}{2}}(U)$ we can estimate
					\[ |f(x) - f(y)| \leq c |x - y|^{\frac{1}{2}} \]
				for some $c > 0$ and for all $x, y \in U$. Let $B_{H^{1}_{k}} \coloneqq \{ f \in H^{1}_{k}(U) : \|f\|_{H^{1}(U)} \leq 1 \}$, then for $f \in B_{H^{1}_{k}}$ it holds that
				\begin{equation}
					|f(x)|^{2} \leq 2 \| f\|^{2}_{L^{2}(U)} + 2 \leq 4 \quad \forall x \in U. \label{eqbounded}
				\end{equation} 
				For an arbitrary $\epsilon > 0$ we now partition $U$ into $n_{\epsilon}$ equidistant, disjoint intervals $I_{k}$, i.e. $U = \bigcup_{k = 1}^{n_{\epsilon}} I_{k}$. Since all $f \in B_{H^{1}_{k}}$ are uniformly bounded on $U$ by \eqref{eqbounded}, there exist for each subinterval $I_{k}$ a finite number of constants $c_{1, k}, \dotsc, c_{\nu_{\epsilon}, k}$ such that
					\[ \forall f \in B_{H^{1}_{k}} ~\exists f \in \{1, \dotsc, \nu_{\epsilon} : \left| f\left(\frac{k}{n_{\epsilon}}\right) - c_{j, k} \right| < \epsilon ~\forall k \in \{ 1, \dotsc, n_{\epsilon} \}. \]
				Hence, there are finitely many step functions such that for any $f \in L^{2}(U)$ there exists one of those step functions $g \in L^{2}(U)$, with function value $c_{k}$ on subinterval $I_{k}$ for each $k \in \{1, \dotsc, n_{\epsilon}\}$, such that
				\begin{align*}
					\| f - g\|^{2}_{L^{2}(U)} & = \sum_{k=0}^{n-1} \int_{\frac{k}{n}}^{\frac{k+1}{n}} |f(x) - c_{k+1}|^{2} dx \\
					& \leq 2 \sum_{k=0}^{n-1} \int_{\frac{k}{n}}^{\frac{k+1}{n}} \left|f(x) - f\left(\frac{k}{n}\right)\right|^{2} dx +   \sum_{k=0}^{n-1} \int_{\frac{k}{n}}^{\frac{k+1}{n}} 2 \left| f\left(\frac{k}{n}\right) - c_{k+1} \right|^{2} dx \\ 
					& \leq 2 \sum_{n = 0}^{n-1} \frac{c}{n^{2}} + 2 \sum_{n=0}^{n-1} \frac{1}{n^{3}} = \frac{2}{n^{2}} \left( c + \frac{1}{n} \right) < \epsilon^{2}
				\end{align*}
				for $n$ large enough. This means, in conclusion, that $B_{H^{1}_{k}}$ is totally bounded in $L^{2}(U)$ and in return $H^{1}_{k}$ can be compactly embedded in $L^{2}(U)$.
			\end{proof}
	\end{enumerate}
\end{atheorem}

\begin{atheorem}[Dominated Convergence Theorem]
	Let ${f_n}$ be a sequence of real-valued measurable functions on a measure space $(S, \Sigma, \mu)$. Suppose that the sequence converges pointwise to a function $f$ and is dominated by some integrable function $g$ in the sense that
		\[ |f_n(x)| \le g(x) \]
	for all numbers n in the index set of the sequence and all points $x \in S$. Then f is integrable and
		\[ \lim_{n\to\infty} \int_S |f_n-f|\,d\mu = 0 \]
	which also implies $\lim_{n\to\infty} \int_S f_n\,d\mu = \int_S f\,d\mu$.

	\begin{proof}
		See \cite[p. 516]{werner2006funkana}.
	\end{proof}
\end{atheorem}

\begin{atheorem}[Eigenvectors of a compact, symmetric operator]
	Let $H$ be a separable Hilbert space, and suppose $S \colon H \rightarrow H$ is a compact and symmetric operator. Then there exists a countable orthonormal basis of $H$ consisting of eigenvectors of $S$.
	
	\begin{proof}
		See \cite[p. 645]{evans1998partial}.
	\end{proof}
\end{atheorem}

\begin{atheorem}[Embedding of $H^{1}$ in $C^{\frac{1}{2}}$]
	Let $[a, b]$ be a compact interval in $\R$. Then, $H^{1}([a, b])$ is embedded in $C^{\frac{1}{2}}([a, b])$
	
	\begin{proof}
		See \cite[p. 269]{evans1998partial}.
	\end{proof}
\end{atheorem}

\begin{atheorem}[Equivalent definitions of closed operators]
	Let X, Y be two Banach spaces. A linear operator $A \colon X \supset \mathcal{D}(A)  \rightarrow Y$ is closed if for every sequence $(x_n)_{n \in \N}$ in $\mathcal{D}(A)$ from
	\[ x_{n} \rightarrow x \in X \text{ and } Tx_{n} \rightarrow y \in Y \]
	follows that $x \in D$ and $Tx = y$.

	\begin{proof}
		 See \cite[p. 156]{werner2006funkana}.
	\end{proof}
\end{atheorem}	

\begin{atheorem}[Fubini's theorem for integrable functions]
	Suppose $X$ and $Y$ are $\sigma$-finite measure spaces, and suppose that $X \times Y$ is given the product measure (which is unique as $X$ and $Y$ are $\sigma$-finite). Fubini's theorem states that if $f(x,y)$ is $X \times Y$ integrable, meaning that it is measurable and
		\[  \int_{X\times Y} |f(x,y)|\,\text{d}(x,y)<\infty, \]
	then
		\[ \int_X\left(\int_Y f(x,y)\,\text{d}y\right)\,\text{d}x=\int_Y\left(\int_X f(x,y)\,\text{d}x\right)\,\text{d}y=\int_{X\times Y} f(x,y)\,\text{d}(x,y). \]
	The first two integrals are iterated integrals with respect to two measures, respectively, and the third is an integral with respect to the product measure

	\begin{proof}
		See \cite[p. 514]{werner2006funkana}.
	\end{proof}
\end{atheorem}

\begin{atheorem}[Lax-Milgram]
	Let $H$ be a Hilbert space where $\| \cdot \|$ denotes the norm on $H$, and let $B \colon H \times H \rightarrow \C$ be a sesquilinear form. If there exist constants $\alpha, \beta > 0$ such that
	\begin{enumerate}[label=\alph*\upshape)]
		\item $\left| B[u, v] \right| \leq \alpha \| u \| \|v \| \quad (u, v \in H)$ and
		\item $Re(B[u,u]) \geq \beta \|u\|^{2} \quad (u \in H)$,
	\end{enumerate}
	then there exists to each $l \in H^{*}$ a unique $w \in H$ such that
		\[ B[v, w] = l(v) \]
	hold for all $v \in H$.
		
	\begin{proof}
		See \cite[Amd to problem 51]{plum2015dglhr}.
	\end{proof}
\end{atheorem}

\begin{atheorem}[Monotone Convergence Theorem]
	Let $(X, \Sigma, \mu)$ be a measure space. Let $f_1, f_2, \ldots$  be a pointwise non-decreasing sequence of $[0, \infty]$-valued $\Sigma$–measurable functions, i.e. for every $k \geq 1$ and every $x$ in $X$,
		\[ 0 \leq f_k(x) \leq f_{k+1}(x). \] 
	Next, set the pointwise limit of the sequence $(f_{n})$ to be $f$. That is, for every $x$ in $X$,
		\[ f(x):= \lim_{k\to\infty} f_k(x). \]
	Then $f$ is $\Sigma$–measurable and
		\[ \lim_{k\to\infty} \int f_k \, \mathrm{d}\mu = \int f \, \mathrm{d}\mu. \]

	\begin{proof}
		See \cite[p. 516]{werner2006funkana}.
	\end{proof}
\end{atheorem}

\begin{atheorem}[Orthonormality of $\vartheta$]
	The sequence
		\[ \vartheta_{n}(k) \coloneqq \frac{1}{\sqrt{|B|}} e^{ikn} \]
	forms an orthonormal basis of $L^{2}(B)$.

	\begin{proof}
		 For $m, n \in \N$ we see that
		 \[ \langle \vartheta_{n}, \vartheta_{m} \rangle_{L^{2}(B)} = \frac{1}{|B|} \int_{B} e^{ikn} \overline{e^{ikm}} dk = \frac{1}{|B|} \int_{B} e^{ik(n-m)} dk = \begin{cases} 0 & \text{ for } n \neq m \\ 1 & \text{ for } n = m, \end{cases} \]
		 hence the asserted follows.
	\end{proof}
\end{atheorem}

\begin{atheorem}[Parseval's identity]
	Suppose that $H$ is a Hilbert space with inner product $\langle \cdot,\cdot \rangle$. Let $(e_{n})$ be an orthonormal basis of $H$; i.e., the linear span of the $e_n$ is dense in $H$, and the $e_n$ are mutually orthonormal:
		\[ \langle e_{m},e_{n}\rangle ={\begin{cases}1&{\mbox{if}}\ m=n\\0&{\mbox{if}}\ m\not =n.\end{cases}} \]
	Then Parseval's identity asserts that for every $x \in H$,
		\[ \sum _{n}|\langle x,e_{n}\rangle |^{2}=\|x\|^{2}.\]

	\begin{proof}
		See \cite[p. 236]{werner2006funkana}.
	\end{proof}
\end{atheorem}

\begin{atheorem}[Poincare's min-max principle for eigenvalues]
	Let $X$ be a seperable Hilbert space and $\langle \cdot, \cdot \rangle_{X}$ denote the scalar product on $X$. Let $A \colon \mathcal{D}(A) \rightarrow X$ be a self-adjoint operator where $\mathcal{D}(A) \subseteq X$. If the set of eigenvalues $\lambda_{s}$ is at most countable, then
	\begin{equation}
			\lambda_{s} = \underset{\dim U = s}{\min_{U \subseteq \mathcal{D}(A)}} \max_{v \in U \setminus \{ 0 \} } \frac{\langle A v, v \rangle_{X}}{\langle v, v \rangle_{X}}.  \label{poincare} 
	\end{equation} 

	\begin{proof}
		See \cite[p. 119]{teschl2014mathematical}.
	\end{proof}
\end{atheorem}

\begin{atheorem}[Properties of self-adjoint operators] ~\
	\begin{enumerate}[label=\alph*\upshape)]
		\item Every self-adjoint is symmetric and closed.
			\begin{proof}
				 Follows directly from the definitions for self-adjoint and closed operators.
			\end{proof}
		\item For $A$ being a self-adjoint operator, $\lambda \in \rho(A)$, $(A - \lambda I)^{-1}$ is bounded.
			\begin{proof}
				Since every self-adjoint is closed, $(A - \lambda I)$ is as the shift with $\lambda \in \R$ also closed. Furthermore, the graph of $(A - \lambda I)^{-1}$ is simply the graph of $(A - \lambda I)$ rotated and hence $(A - \lambda I)^{-1}$ is closed as well. The closed Graph Theorem now yields the desired result.
			\end{proof}
	\end{enumerate}
\end{atheorem}

\begin{atheorem}[Properties of the set of resolvent values]
	The resolvent set $\rho(A) \subseteq \mathbb{C}$ of a bounded linear operator $A$ is an open set.
	
	\begin{proof}
		See \cite[p. 259]{werner2006funkana}.
	\end{proof}
\end{atheorem}

\begin{atheorem}[Riesz' representation theorem]
	Let $H$ be a Hilbert space, and let $H^{*}$ denote its dual space, consisting of all continuous linear functionals from $H$ into $\R$ or $\C$. If $x$ is an element of $H$, then the function $\varphi_{x}$, for all $y$ in $H$ defined by
	\[ \varphi_{x}(y) = \left\langle y,x\right\rangle_{H}, \]
	where $\langle \cdot ,\cdot \rangle_{H}$ denotes the inner product of the Hilbert space, is an element of $H^{*}$. Hence, every element of $H^{*}$ can be written uniquely in this form.
	
	\begin{proof}
		See \cite[p. 284]{evans1998partial}
	\end{proof}
\end{atheorem}

\begin{atheorem}[The spectrum of self-adjoint operators] \label{spectrul-sa-real}
	The spectrum of a self-adjoint operator $A$ is real. 
	
	\begin{proof}
		Let $\lambda$ be an eigenvalue of $A$, i.e. there exists $x \in X$ such that $A x = \lambda x$. From this it follows that $\langle A x, x \rangle = \langle \lambda x , x \rangle$. Using then the fact that $A$ is self-adjoint we can further deduce
		\[ \lambda \langle x , x \rangle = \langle \lambda x , x \rangle = \langle A x, x \rangle = \langle x, A x \rangle = \langle x , \lambda x \rangle = \overline{\lambda} \langle  x , x \rangle \]
		Hence, $\lambda = \overline{\lambda}$, which shows the desired result.
	\end{proof}
\end{atheorem}

\begin{atheorem}[Trace Theorem]
	Assume $U$ is bounded and $\partial U$ is $C^{1}$. Then there exists a bounded linear operator
		\[ T \colon H^{1}(U) \rightarrow L^{2}(\partial U) \]
	such that
	\begin{enumerate}[label=\alph*\upshape)]
		\item $Tu = u\big|_{\partial U}$ if $u \in H^{1}(U) \cap C(\overline{U})$
		\item $\|Tu\|_{L^{2}(\partial U)} \leq C \|u\|_{H^{1}(U)}$
	\end{enumerate}
	for each $u \in H^{1}(U)$, with the constant $C$ depending only on $U$.
	
	\begin{proof}
		See \cite[p. 258]{evans1998partial}.
	\end{proof}
\end{atheorem}

\begin{atheorem}[Uniqueness of weak derivatives] \label{athem:uniqueness_of_weak_deriv}
	Let $\Omega \subseteq \R$ be open, if it exists, the $\alpha$-th weak derivative of $u$ is uniquely determined up to a set of measure zero.
	
	\begin{proof}
		Assume that $g, \tilde{g} \in L_{loc}^{1}(\Omega)$ satisfy for all $\varphi \in C_{0}^{\infty}(\Omega)$
		\[ (-1)^{\alpha} \int_{\Omega} f \varphi' = \int_{\Omega} g \varphi  = \int_{\Omega} \tilde{g} \varphi. \]
		Then $\int_{\Omega} \left( g - \tilde{g} \right) \varphi = 0$ for all $\varphi \in C_{0}^{\infty}(\Omega)$, whence $g - \tilde{g} = 0$ almost everywhere.	
	\end{proof}
\end{atheorem}  