\chapter*{Appendix} \addcontentsline{toc}{chapter}{Appendix} 

\begin{atheorem}[Alternative definition of the Delta-Distribution] \label{athem:delta}
	
	\begin{proof}
		This limit exists and it holds that $\delta_{x_{0}} \in \mathfrak{D}'(\R)$ and $\delta_{x_{0}}(f) = \lim_{\epsilon \rightarrow 0} \delta_{\epsilon}(f) = f(x_{0})$, for a proof see  % todo proof
	\end{proof}
\end{atheorem}

\begin{atheorem}[$L^{p}$-Approximation by test functions]
	For $U \subseteq \R^{n}$ open, $C_{0}^{\infty}(U)$ is dense in $L^{p}(U)$, if $1 \leq p < \infty$.
	
	\begin{proof}
		See \cite[p. 31]{adams2003sobolev}.
	\end{proof}
\end{atheorem}

\begin{atheorem}[$H^{k}$-Approximation by test functions] 
	Let $\Omega \subseteq \R^{n}$ be an open set. Let $u \in H^{k}(\Omega)$, then there exists a sequence of functions $u_{k} \in C^{\infty}(\Omega)$ such that $\| u_{k} - u \|_{H^{k}(\Omega)}$ as $n \rightarrow \infty$.
	
	\begin{proof}
		See \cite[p. 138]{adams2003sobolev}.
	\end{proof}	
\end{atheorem}


\begin{atheorem}[Bessel's inequality]
	Let $H$ be a Hilbert space, and suppose that $e_1, e_2, ...$ is an orthonormal sequence in $H$. Then, for any $x \in H$ one has
	\[ \sum_{k=1}^{\infty}\left\vert\left\langle x,e_k\right\rangle \right\vert^2 \le \left\Vert x\right\Vert^2 \]
	where $\langle \cdot,\cdot \rangle$ denotes the inner product in the Hilbert space $H$.

	\begin{proof}
		See \cite[p. 233]{werner2006funkana}.
	\end{proof}
\end{atheorem}

\begin{atheorem}[Cauchy–Schwarz inequality]
	Fr all vectors $u$ and $v$ of an inner product space it is true that
		\[ \left| \langle u,v \rangle \right|^{2} \leq \langle u,u \rangle \cdot \langle v,v \rangle, \]
	where $\langle \cdot ,\cdot \rangle$ is the inner product. 

	\begin{proof}
		See \cite[p. 20]{werner2006funkana}.
	\end{proof}
\end{atheorem}

\begin{atheorem}[Closed graph theorem]
	Let $X$ be a Banach space. Is $A$ a closed operator and $\mathcal{D}(A) = X$, then $A$ is continuous on $X$.

	\begin{proof}
		See \cite[p. 417]{evans1998partial}.
	\end{proof}
\end{atheorem}

\begin{atheorem}[Closeness of $H^{1}_{k}$ in $H^{1}(\Omega)$] \label{h1kclosed}
	$H^{1}_{k}$ is a closed subspace of $H^{1}(\Omega)$ and hence a Hilbert space.

	$H^{1}_{k}$ is a closed subspace of $H^{1}(\Omega)$, and therefore a  Hilbert space with respect to the norm of $H^{1}(\Omega)$.
	
	\begin{proof} % x martin do the proof
		coming
%		: Let (fn)n∈N be a sequence of elements in Hφ1(Ω) converging to
%f ∈ H1(Ω). Due to the Sobolev embedding of H1-functions into the space of continuous 1 ̃ ̃
%functions, we can identify the H -functions fn and f with continuous functions f,fn. Since convergence with respect to the H1-norm implies pointwise convergence almost
% ̃ ̃
%everywhere, we can conclude that f(x) converges everywhere to f(x). In the following
% ̃
%let us denote f again instead of f. Thus, for k = 1,2:
%f(ξk+) = lim fn(ξk+) = lim eiφk fn(ξk−) = eiφk lim fn(ξk−) = eiφk f(ξk−), n→∞ n→∞ n→∞
%which shows the closer of Hφ1(Ω).
%Analogously as in Chapter (2.1) let us regard the equation
% Aφu + μu = f
%with f being an element of L2(Ω) and μ having the same properties as before. Then we
%are able to formulate a weak formulation what we will write down in the following
	\end{proof}

	\begin{proof} % todo proof
		due to the fact that convergence in H1k impliese convergence of traces at -1/2 and 1/2 h1k is a closed subspace of H1
	\end{proof}
\end{atheorem}

\begin{atheorem}[Compact Embedding Theorem for Sobolev spaces] \label{compact-embedding-theorem} % todo proof 
	Assume $U$ is a bounded open subset of $\R^{n}$, and $\partial U$ is $C^{1}$. Define $p^{*} \coloneqq \frac{2 n}{n - 2}$.
	\begin{enumerate}[label=\alph*\upshape)]
		\item Suppose $n > 2$. Then $H^{1}(U) \subset\subset L^{q}(U)$ for each $1 \leq q \leq p^{*}$.
		
			\begin{proof}
				Follows from Rellich-Kondrachov Compactness Theorem, see for example \cite[p. 272]{evans1998partial}.
			\end{proof}
		\item Suppose $n \in \{1, 2\}$. Then $H^{1}(U) \subset\subset L^{2}(U)$.
			\begin{proof} To proof the embedding for $p=2$ note that from Rellich-Kondrachov Compactness Theorem it follows that $p^{*} \rightarrow \infty$ if $p \rightarrow n$. It is easily seen that if $(u_{n})_{n \in \N}$ is a $H^{1}(U)$-bounded sequence, then so it is bounded in $W^{1, n-\epsilon}(U)$ for some $\epsilon > 0$. Choosing $\epsilon$ such that $(n - \epsilon)^{*} > n$ hence allows using again via part a) the Rellich-Kondrachov Compactness Theorem and  this provides he existence of a $L^{2}(U)$ convergent subsequence. ~\\
			For $n = 1$ this follows from Morrey's inequality and the Arzela Ascoli compactness criterion, see for example \cite[p. 274]{evans1998partial}.
			\end{proof}
	\end{enumerate}
\end{atheorem}

\begin{atheorem}[Dominated Convergence Theorem]
	Let ${f_n}$ be a sequence of real-valued measurable functions on a measure space $(S, \Sigma, \mu)$. Suppose that the sequence converges pointwise to a function $f$ and is dominated by some integrable function $g$ in the sense that
		\[ |f_n(x)| \le g(x) \]
	for all numbers n in the index set of the sequence and all points $x \in S$. Then f is integrable and
		\[ \lim_{n\to\infty} \int_S |f_n-f|\,d\mu = 0 \]
	which also implies $\lim_{n\to\infty} \int_S f_n\,d\mu = \int_S f\,d\mu$.

	\begin{proof}
		See \cite[p. 516]{werner2006funkana}.
	\end{proof}
\end{atheorem}

\begin{atheorem}[Eigenvectors of a compact, symmetric operator]
	Let $H$ be a separable Hilbert space, and suppose $S \colon H \rightarrow H$ is a compact and symmetric operator. Then there exists a countable orthonormal basis of $H$ consisting of eigenvectors of $S$.
	
	\begin{proof}
		See \cite[p. 645]{evans1998partial}.
	\end{proof}
\end{atheorem}

\begin{atheorem}[Embedding of $H^{1}$ in $C$]
	Let $[a, b]$ be a compact interval in $\R$. Then, $H^{1}([a, b])$ is embedded in $C([a, b])$
	
	\begin{proof}
		Let $f$ be a smooth function and $x, y \in [a, b]$ such that $x \leq y$, then:
		~\\ ~\\ % todo proof
		Thus, the supremum norm is dominated by the $H^{1}$-norm, which implies that, which means that this estimation holds for the completion H1([a, b]) as well.
	\end{proof}
\end{atheorem}

\begin{atheorem}[Equivalent definitions of closed operators]
	Equivalent Definition von Closed und Zeigen dass und 

	\begin{proof}
		 % todo proof
	\end{proof}
\end{atheorem}	

\begin{atheorem}[Fubini's theorem for integrable functions]
	Suppose $X$ and $Y$ are $\sigma$-finite measure spaces, and suppose that $X \times Y$ is given the product measure (which is unique as $X$ and $Y$ are $\sigma$-finite). Fubini's theorem states that if $f(x,y)$ is $X \times Y$ integrable, meaning that it is measurable and
		\[  \int_{X\times Y} |f(x,y)|\,\text{d}(x,y)<\infty, \]
	then
		\[ \int_X\left(\int_Y f(x,y)\,\text{d}y\right)\,\text{d}x=\int_Y\left(\int_X f(x,y)\,\text{d}x\right)\,\text{d}y=\int_{X\times Y} f(x,y)\,\text{d}(x,y). \]
	The first two integrals are iterated integrals with respect to two measures, respectively, and the third is an integral with respect to the product measure

	\begin{proof}
		See \cite[p. 514]{werner2006funkana}.
	\end{proof}
\end{atheorem}

\begin{atheorem}[Lax-Milgram]
	Let $H$ be a Hilbert space where $\| \cdot \|$ denotes the norm on $H$, and let $B \colon H \times H \rightarrow \C$ be a sesquilinear form. If there exist constants $\alpha, \beta > 0$ such that
	\begin{itemize}
		\item $\left| B[u, v] \right| \leq \alpha \| u \| \|v \| \quad (u, v \in H)$ and
		\item $Re(B[u,u]) \geq \beta \|u\|^{2} \quad (u \in H)$,
	\end{itemize}
	then there exists to each $l \in H^{*}$ a unique $w \in H$ such that
		\[ B[v, w] = l(v) \]
	hold for all $v \in H$.
		
	\begin{proof}
		See \cite[Amd to problem 51]{plum2015dglhr}.
	\end{proof}
\end{atheorem}

\begin{atheorem}[Monotone Convergence Theorem]
	Let $(X, \Sigma, \mu)$ be a measure space. Let $f_1, f_2, \ldots$  be a pointwise non-decreasing sequence of $[0, \infty]$-valued $\Sigma$–measurable functions, i.e. for every $k \geq 1$ and every $x$ in $X$,
		\[ 0 \leq f_k(x) \leq f_{k+1}(x). \] 
	Next, set the pointwise limit of the sequence $(f_{n})$ to be $f$. That is, for every $x$ in $X$,
		\[ f(x):= \lim_{k\to\infty} f_k(x). \]
	Then $f$ is $\Sigma$–measurable and
		\[ \lim_{k\to\infty} \int f_k \, \mathrm{d}\mu = \int f \, \mathrm{d}\mu. \]

	\begin{proof}
		See \cite[p. 516]{werner2006funkana}.
	\end{proof}
\end{atheorem}

\begin{atheorem}[Orthonormality of $\vartheta$]
	\[ \vartheta_{n}(k) \coloneqq \frac{1}{\sqrt{|B|}} e^{ikn} \]

	\begin{proof}
		 % todo proof
	\end{proof}
\end{atheorem}

\begin{atheorem}[Parseval's identity]
	Suppose that $H$ is a Hilbert space with inner product $\langle \cdot,\cdot \rangle$. Let $(e_{n})$ be an orthonormal basis of $H$; i.e., the linear span of the en is dense in $H$, and the $e_n$ are mutually orthonormal:
		\[ \langle e_{m},e_{n}\rangle ={\begin{cases}1&{\mbox{if}}\ m=n\\0&{\mbox{if}}\ m\not =n.\end{cases}} \]
	Then Parseval's identity asserts that for every $x \in H$,
		\[ \sum _{n}|\langle x,e_{n}\rangle |^{2}=\|x\|^{2}.\]

	\begin{proof}
		See \cite[p. 236]{werner2006funkana}.
	\end{proof}
\end{atheorem}

\begin{atheorem}[Poincare's min-max principle for eigenvalues]
	a % todo todo

	\begin{proof}
		a
	\end{proof}
\end{atheorem}

\begin{atheorem}[Properties of compact operators]
	aus kompakt folgt bounded,	

	\begin{proof}
		 % todo proof
	\end{proof}
\end{atheorem}

\begin{atheorem}[Properties of self-adjoint operators]
	\begin{enumerate}
		\item Self-Adjoint $\Rightarrow$ Symmetric und closed
			\begin{proof}
				 % todo proof
			\end{proof}
		\item For $A$ being a self-adjoint operator, $\lambda \in \rho(A)$, $(A - \lambda I)^{-1}$ is bounded.
			\begin{proof}
				Since every self-adjoint is closed, $(A - \lambda I)$ is as the shift also closed. Furthermore, the graph of $(A - \lambda I)^{-1}$ is simply the graph of $(A - \lambda I)$ rotated and hence $(A - \lambda I)^{-1}$ is closed as well. The closed Graph Theorem now yields the desired result.
			\end{proof}
	\end{enumerate}
\end{atheorem}

\begin{atheorem}[Properties of the set of resolvent values]
	The resolvent set $\rho(A) \subseteq \mathbb{C}$ of a bounded linear operator $A$ is an open set.
	
	\begin{proof}
		See \cite[p. 259]{werner2006funkana}.
	\end{proof}
\end{atheorem}

\begin{atheorem}[Riesz' representation theorem]
	Let $H$ be a Hilbert space, and let $H^{*}$ denote its dual space, consisting of all continuous linear functionals from $H$ into $\R$ or $\C$. If $x$ is an element of $H$, then the function $\varphi_{x}$, for all $y$ in $H$ defined by
	\[ \varphi_{x}(y) = \left\langle y,x\right\rangle_{H}, \]
	where $\langle \cdot ,\cdot \rangle_{H}$ denotes the inner product of the Hilbert space, is an element of $H^{*}$. Hence, every element of $H^{*}$ can be written uniquely in this form.
	
	\begin{proof}
		See \cite[p. 284]{evans1998partial}
	\end{proof}
\end{atheorem}

\begin{atheorem}[The spectrum of self-adjoint operators] \label{spectrul-sa-real}
	The spectrum of a self-adjoint operator $A$ is real. 
	
	\begin{proof}
		Let $\lambda$ be an eigenvalue of $A$, i.e. there exists $x \in X$ such that $A x = \lambda x$. From this it follows that $\langle A x, x \rangle = \langle \lambda x , x \rangle$. Using then the fact that $A$ is self-adjoint we can further deduce
		\[ \lambda \langle x , x \rangle = \langle \lambda x , x \rangle = \langle A x, x \rangle = \langle x, A x \rangle = \langle x , \lambda x \rangle = \overline{\lambda} \langle  x , x \rangle \]
		Hence, $\lambda = \overline{\lambda}$, which shows the desired result.
	\end{proof}
\end{atheorem}

\begin{atheorem}[Trace Theorem]
	b % todo todo

	\begin{proof}
		a
	\end{proof}
\end{atheorem}

\begin{atheorem}[Uniqueness of weak derivatives] \label{athem:uniqueness_of_weak_deriv}
	Let $\Omega \subseteq \R$ be open, if it exists, the $\alpha$-th weak derivative of $u$ is uniquely determined up to a set of measure zero.
	
	\begin{proof}
		Assume that $g, \tilde{g} \in L_{loc}^{1}(\Omega)$ satisfy
		\[ (-1)^{\alpha} \int_{\Omega} f \varphi' = \int_{\Omega} g \varphi  = \int_{\Omega} \tilde{g} \varphi  \]
		for all $\varphi \in C_{0}^{\infty}(\Omega)$. Then
		\[ \int_{\Omega} \left( g - \tilde{g} \right) \varphi = 0 \]	
		for all $\varphi \in C_{0}^{\infty}(\Omega)$, whence $g - \tilde{g} = 0$ almost everywhere.	
	\end{proof}
\end{atheorem}  