\chapter{Fundamental domain of periodicity and the Brillouin zone}  \label{chap:4}

In this chapter we will restrict the Kronig-Penney model to one periodicity cell and examine the spectrum of the resulting operator. Solving the eigenvalue problem on the period cell while varying specific boundary conditions for the solution functions, with the help of tools we introduce in chapter \ref{chap:5}, can be used to determine the eigenvalues of the unrestricted problem. This is precisely the approach we will choose in Chapter \ref{chap:6}.
~\\ 

Let $\Omega$ be the fundamental domain of periodicity associated with \eqref{the-operator-A-formally}, for simplicity let $\Omega \coloneqq \Omega_{0}$ and thus $x_{0} = 0$ being contained in $\Omega$. As commonly used in literature the reciprocal lattice for $\Omega$ is $[-\pi, \pi]$, the so-called one-dimensional Brillouin zone $B$, see for example \cite[chapter 3]{dorfler2011photonic}. For brevity let us introduce the following:
\begin{definition} 
	We define for every $k \in B$ the set  
	\begin{equation} 
		H^{1}_{k} \coloneqq \left\{ \psi \in H^{1}(\Omega): ~ \psi\left(\frac{1}{2}\right) = e^{ik} \psi\left(-\frac{1}{2}\right) \right\}. \label{quasi-periodic-condition}
	\end{equation}
	Hereafter, we will refer to the boundary conditions in \eqref{quasi-periodic-condition} as quasi-periodic boundary conditions.
\end{definition}
In this chapter, we will consider the operator $A_{k}$ on $\Omega$ formally defined by the operation 
	\[ -\frac{d^{2}}{dx^{2}} + \rho \delta_{x_{0}}, \]
subject to the space $H^{1}_{k}$ for fixed $k \in B$.
	
\begin{remark}
	$H^{1}_{k}$ is a closed subspace of $H^{1}(\Omega)$ and hence a Hilbert space.
\end{remark}

\begin{proof}
	See Theorem \ref{h1kclosed}.
\end{proof}

Analogously to Section \ref{sec:3.1}, we now define $A_{k}$ by considering the problem to find for $f \in L^{2}(\Omega)$ a function $u \in H^{1}_{k}$ such that the equation
	\begin{equation}
		\int_{\Omega} u'(x) \overline{v'(x)} dx + \rho u(x_{0}) \overline{v(x_{0})} - \mu \int_{\Omega} u(x) \overline{v(x)} dx = \int_{\Omega} f(x) \overline{v(x)} dx \label{weak-formulation-to-the-restricted}
	\end{equation}

holds for all $v \in H^{1}_{k}$. Using the last remark, i.e that $H^{1}_{k}$ is a Hilbert space, we can use similar arguments as in Section \ref{sec:3.1} to prove that 
	\[ R_{\mu, k} \colon L^{2}(\Omega) \rightarrow H^{1}_{k},  f \mapsto u \]
is well-defined and injective. Consequently, we are able to define
	\[ A_{k} \coloneqq R_{\mu, k}^{-1} + \mu I, \quad \mathcal{D}(A_{k}) = \mathcal{R}(R_{\mu, k}^{-1}). \] 
	
\begin{remark}
	Note, $R_{\mu, k}$ is the resolvent of $A_{k}$.
\end{remark}

\section{The domain of the restricted Schrödinger operator}

We already know that $\mathcal{D}(A_{k}) \subseteq H^{1}_{k}$. Therefore, proceeding as in Section \ref{sec:3.2}, by choosing the similar test functions $v \in H^{1}_{k}$ in \eqref{weak-formulation-to-the-restricted}, we are able to show that
\begin{align*}
	\mathcal{D}(A_{k}) & = \Big\{ \psi \in H^{1}(\Omega) \colon ~\psi\left(\frac{1}{2}\right) = e^{ik} \psi\left(-\frac{1}{2}\right), ~ \psi'\left(\frac{1}{2}\right) = e^{ik} \psi'\left(-\frac{1}{2}\right), ~ u\big|_{\left(-\frac{1}{2}, 0\right)} \in H^{2}\left(\left(-\frac{1}{2}, 0\right)\right) \\	
	 & ~\qquad ~\qquad ~\qquad ~\qquad ~ u\big|_{\left(0, \frac{1}{2}\right)} \in H^{2}\left(\left(0, \frac{1}{2}\right)\right), ~ u'(x_{0}-0) - u(x_{0} + 0) + \rho u(x_{0}) = 0 \Big\}. 
\end{align*}

The last three properties follow directly through the same approach as in Section \ref{sec:3.2}. The quasi-periodic condition for $\psi$ comes from the boundary condition on $H^{1}_{k}$. Therefore, new is only the quasi-condition condition for $\psi'$ which we can see through the following. Choosing in \eqref{weak-formulation-to-the-restricted} a function $v \in C_{0}^{\infty} \cap \mathcal{D}(A_{k})$ and partial integration while $\supp v \supseteq \Omega$ and $v\left(\frac{1}{2}\right) \neq 0$ yields
\[ - \int_{\Omega} u''(x) \overline{v(x)} dx +  u'(x) \overline{v(x)} \big|_{-\frac{1}{2}}^{\frac{1}{2}} + \rho u(x_{0}) \overline{v(x_{0})} - \mu \int_{\Omega} u(x) \overline{v(x)} dx = \int_{\Omega} f(x) \overline{v(x)} dx \] 
\[ \iff u'\left(\frac{1}{2}\right) e^{-ik} = u'\left(-\frac{1}{2}\right). \] % todo Andrii noch einmal fragen, ob das passt

In the remainder of this chapter we will further investigate the operator $A_{k}$. For this purpose, we need to show that $R_{\mu, k}$ is compact from which we deduce that the eigenfunctions of $A_{k}$ form a complete and orthonormal system in $H^{1}_{k}$.

\section{The compactness of the restricted resolvent} 

\begin{theorem} \label{3.1:thm-Rmuk.isCompact}
	The operator $R_{\mu, k}$ is compact.

	\begin{proof}
	Let $(f_{j})_{j \in \N} \in L^{2}(\Omega)$ be a  bounded sequence. We will show that 
		\[ u_{j} \coloneqq R_{\mu, k} f_{j} \quad \text{ for all } j \geq 1 \]
	is a bounded sequence with respect to the $H^{1}$-Norm as well. Each such $u_{j}$ is by definition in $H^{1}_{k}$ and has to satisfy
		\begin{equation}
			\int_{\Omega} u_{j}'(x) \overline{v'(x)} dx + \rho u_{j}(x_{0}) \overline{v(x_{0})} - \mu \int_{\Omega} u_{j}(x) \overline{v(x)} dx = \int_{\Omega} f_{j}(x) \overline{v(x)} dx \quad \forall v \in H^{1}_{k}. \label{ujsatisfy}
		\end{equation} 
	Now, the particular choice of $v = u_{j}$ in \eqref{ujsatisfy} yields with \eqref{estimation-for-potential} for small enough $\mu$
		\[  \| u_{j} \|_{H^{1}(\Omega)} \leq \| f_{j} \|_{L^{2}(\Omega)} \| u_{j} \|_{L^{2}(\Omega)} \leq c \sqrt{vol(\Omega)}. \]
	Thus, $\| u_{j} \|_{H^{1}(\Omega)} \leq C$ for all $j$. The assertion follows from the Compact Embedding Theorem for Sobolev spaces, see Theorem \ref{compact-embedding-theorem}.
	\end{proof}	
\end{theorem}		

\section{The spectrum of the restricted Schrödinger operator}
Using the compactness of $R_{\mu, k}$, we know on the one hand that every non-zero $\lambda \in \sigma(R_{\mu, k})$ is an eigenvalue of $R_{\mu, k}$ and on the other hand that the at most countable sequence of eigenvalues can only accumulate at $0$, for proofs see \cite[page 74 - 76]{weis2015funkana}. We will from now consider the eigenvalue problem to find $\psi \in \mathcal{D}(A_{k}) \subseteq H_{k}^{1}$ such that
	\begin{equation}
		A_{k} \psi = \lambda \psi \text{ on } \Omega. \label{eigv-problem}
	\end{equation}
The eigenvalue $\lambda$ depends on the boundary condition we set on the domain, more specifically it is a function of $k$. We understand $\psi$ extended by the boundary condition on $\partial \Omega$ in \eqref{quasi-periodic-condition} to the whole of $\R$ and call them Bloch waves. By considering any eigenfunction $w$ of $R_{\mu, k}$ with the corresponding eigenvalue $\lambda(k)$ we can see that
	\[ A_{k} w = R_{\mu, k}^{-1} w + \mu w = \left(\frac{1}{\lambda(k)} + \mu\right) w, \]
	i.e. $A_{k}$ has the same sequence of eigenfunctions as $R_{\mu, k}$, and then respectively
	\begin{equation}
		\tilde{\lambda}(k) \coloneqq \frac{1}{\lambda(k)} - \mu \label{ewresp}
	\end{equation} 
is an eigenvalue for the eigenfunction $w$ except that now of the operator $A_{k}$. Using the compactness of $R_{\mu, k}$ and \eqref{ewresp}, we see that $A_{k}$ has a purely discrete spectrum satisfying
	\begin{equation}
		\lambda_{1}(k) \leq \lambda_{2}(k) \leq \dotsc \leq \lambda_{s}(k) \rightarrow \infty \text{ as } s \rightarrow \infty. \label{comment-after}
	\end{equation} 
and the corresponding eigenfunctions form a $\langle \cdot , \cdot \rangle$-orthonormal and complete system $(\psi_{s}(\cdot, k))_{s \in \N}$ of eigenfunctions for \eqref{quasi-periodic-condition}, for proof see \cite[page 643 - 645]{evans1998partial}.
~\\

At the end of this chapter, we transform the eigenvalue problem \eqref{eigv-problem} such that the boundary condition is independent of $k$. This step is needed in Chapter \ref{chap:6} to show that the eigenvalues in \ref{eigv-problem} depend continuously on $k$. For this, we define first
	\[ \varphi_{s}(x, k) \coloneqq e^{-ikx} \psi_{s}(x, k). \]
This yields
	\begin{align}
		A_{k} \psi_{s}(x, k) & = \frac{d^{2}}{dx^{2}} \psi_{s}(x, k)\big|_{(x_{0} - \frac{1}{2}, x_{0})} \cdot \mathds{1}_{(x_{0} - \frac{1}{2}, x_{0})} + \frac{d^{2}}{dx^{2}} \psi_{s}(x, k)\big|_{(x_{0}, x_{0}  + \frac{1}{2})} \cdot \mathds{1}_{(x_{0}, x_{0} + \frac{1}{2})} \notag \\
			& = e^{ikx} \left( \frac{d}{dx} + ik \right)^{2} \varphi_{s}(x, k)\big|_{(x_{0} - \frac{1}{2}, x_{0})} \cdot \mathds{1}_{(x_{0} - \frac{1}{2}, x_{0})} \notag \\
			& ~\qquad + e^{ikx} \left( \frac{d}{dx} + ik \right)^{2} \varphi_{s}(x, k)\big|_{(x_{0}, x_{0}  + \frac{1}{2})} \cdot \mathds{1}_{(x_{0}, x_{0} + \frac{1}{2})}. \label{transformed}
	\end{align}
Therefore, we define the operator $\tilde{A_{k}} \colon \mathcal{D}(A_{k}) \rightarrow L^{2}(\R)$ through  % todo Passt der Definitionsbereich?
	\[ \tilde{A}_{k} \varphi_{s}(x, k) \coloneqq \begin{cases}
		\left( \frac{d}{dx} + ik \right)^{2} \varphi_{s}(x, k)|_{(x_{0} - \frac{1}{2}, x_{0})} & \text{for } x \in (x_{0} - \frac{1}{2}, x_{0}) \\ \left( \frac{d}{dx} + ik \right)^{2} \varphi_{s}(x, k)|_{(x_{0}, x_{0}  + \frac{1}{2})} & \text{for } x \in (x_{0}, x_{0} + \frac{1}{2}).
	\end{cases} \] 
Using \eqref{eigv-problem} and \eqref{quasi-periodic-condition}, yields
	\[ \varphi_{s}\left(x - \frac{1}{2}, k\right) = e^{-ik(x - \frac{1}{2})} \psi_{s}\left(x - \frac{1}{2}, k\right) = e^{-ik(x + \frac{1}{2})} \psi_{s}\left(x + \frac{1}{2}, k\right) = \varphi_{s}\left(x + \frac{1}{2}, k\right). \]
From this, \eqref{comment-after} and from Theorem \ref{3.1:thm-Rmuk.isCompact} follows that $(\varphi_{s}(\cdot, k))_{s \in \N}$ is an orthonormal and complete system of eigenfunctions in $L^{2}(\R)$ to the periodic eigenvalue problem
	\begin{align}
		\tilde{A}_{k} \varphi = \lambda_{s}(k) \varphi \text{ on } \Omega, \label{mod-eigv-problem} \\
		\varphi\left(x - \frac{1}{2}\right) = \varphi\left(x + \frac{1}{2}\right). \label{periodic-condition}
	\end{align}
with the identical eigenvalue sequence $(\lambda_{s}(s))_{s \in \N}$ as in \eqref{eigv-problem} by \eqref{transformed}.