\chapter{\texorpdfstring{The Schrödinger operator $A$}{The Schrödinger operator A}}
%% todo COMPLETE ANFANG
The mathematical representation of the above introduced problem is an one-dimensional Schrödinger operator where the potential is given by a periodic delta-distribution, formally this means that the operation of $A$ is defined by % todo Andrii do I have to explain why? 
% todo Markus: is that better?
\begin{equation}
	- \frac{d^{2}}{dx^{2}} + \rho \sum_{i \in \Z} \delta_{x_{i}} \label{the-operator-A-formally}
\end{equation}
on the whole of $\R$, where $\delta_{x_{i}}$ denotes the Dirac delta distribution in $x_{i}$. $\Omega_{k}$ will hereafter identify the periodicity cell containing delta point $x_{k}$ and w.o.l.g. let $x_{0} = 0$ and $|\Omega_{i}| = 1$ for all $i \in \Z$.
~\\ ~\\ % todo Martin: lesen verbessern...das hier!
In general, one cannot expect that given a right-hand side $f \in L^{2}$  a classical solution exists in \eqref{the-operator-A-formally}. For the existence of a classical solution, all the problem has to be sufficiently regular, which for a distributional potential is never the case. Nevertheless, for $\mu \in \R$ the problem % todo Markus: I don't like this 
\begin{equation}
	\int_{\R} u' \overline{v'} + \rho \sum_{i \in \Z} u(x_{i}) \overline{v(x_{i})} - \mu \int u \overline{v} = \int f \overline{v} \quad \forall v \in H^{1}(\R), \label{weak-formulation-of-A}
\end{equation}	
where $u \in H^{1}(\R)$ and $f \in L^{2}$ requires much less regularity, as it is the weak-formulation of our operator $A$ shifted by the constant $\mu$. % todo Markus: neu formulieren
~\\ ~\\ %% todo COMPLETE ENDE
We should note that left-hand side of problem \eqref{weak-formulation-of-A} is actually well-defined and finite, as for arbitrary $\tilde{x}_{i} \in \Omega_{i}$
\begin{eqnarray}
	\sum_{i \in \Z} |u(x_{i})|^{2} & \leq & \sum_{i \in \Z} \left( \big| u(\tilde{x}_{i}) + \int_{\tilde{x}_{i}}^{x_{i}} u'( \tau ) d\tau \big| \right)^{2} \notag \\
		 & \leq & 2 \sum_{i \in \Z} \left( \int_{\Omega_{i}} |u( x )|^{2} dx +  \int_{\Omega_{i}} \left| u'(\tau) \right|^{2} d\tau \right) \notag \\
		 & \leq & 2 \cdot \| u \|^{2}_{H^{1}(\R)}. \label{estimation-for-potential}
\end{eqnarray}

We will now show that for each $f \in L^{2}(\R)$ the equation \eqref{weak-formulation-of-A} has a unique solution. Given $f \in L^{2}(\R)$, we define a functional $l \colon H^{1} \rightarrow \R$ through
	\[ l(v) \coloneqq \int_{\R} f v \]
and a bilinear form $B_{\mu} \colon H^{1}(\R) \times H^{1}(\R) \rightarrow \R$ for $\mu \in \R$ through
	\[ B_{\mu}[u, v] \coloneqq \int_{\R} u' \overline{v'} + \rho \sum_{i \in \Z} u(x_{i}) \overline{v(x_{i})} - \mu \int_{\R} u \overline{v}. \]
Such that \eqref{weak-formulation-of-A} is equivalent to finding $u \in H^{1}(\R)$ such that
	\begin{equation}
		B_{\mu}[u, v] =  l(v) \quad \forall v \in H^{1}(\R). \label{weak-formulation-of-A-for-LM}
	\end{equation}
The existence of a unique $u \in H^{1}(\R)$ satisfying \eqref{weak-formulation-of-A-for-LM} follows from Lax Milgram's Theorem asserts if the bilinear form $B$ is bounded and coercive, which we will prove in the next theorem. % todo Markus: Zitate, wie verweisen?

\begin{theorem} \label{2.1:thm-LaxMilgram}
	The bilinear form $B_{\mu}$
	\begin{enumerate}
		\item[i)] is bounded, i.e. there exists a constant $\alpha > 0$ such that
			\[ \left| B_{\mu}[u,v] \right| \leq \alpha \|u\| \|v\| \]
			holds for all $u, v \in H^{1}(\R)$.
		\item[ii)] is coercive, i.e. there exists a constant $\beta > 0$ such that
			\[ \beta \|u\|^{2} \leq B_{\mu}[u, u] \]
			for all $u \in H^{1}(\R)$.
	\end{enumerate} 

	\begin{proof} ~\\
		i) The boundedness follows from \eqref{estimation-for-potential} as for an arbitrary $\rho \in \R$ there exists $\alpha \in \R$ such that % todo Markus: is die Begründung jetzt besser?
		\begin{align*}
			|B(u, \varphi)|^{2} & \leq \| u' \| \cdot \| v' \| + 2 \rho \sum_{i \in \Z} |u(x_{i})|^{2} |v(x_{i})|^{2} - \mu \| u \| \cdot \| v \| \\
				& \leq \| u' \| \cdot \| v' \| + 8 \rho \cdot \| u \|^{2}_{H^{1}(\R)} \| v \|^{2}_{H^{1}(\R)}  - \mu \| u \| \cdot \| v \| \\
				& = (8\rho - \mu) \| u \| \cdot \| v \| + 8\rho \left( \| u \| \cdot \| v' \| + \| u' \| \cdot \| v \| \right) + (8\rho + 1) \| u' \| \cdot \| v'\| \\
				& \leq \alpha \cdot \| u \|_{H^{1}} \cdot \| \varphi \|_{H^{1}}
		\end{align*}
		ii)
		For the coercivity we first assume $\rho \geq 0$. Now, for $\mu < -1$ we get 
		\begin{align*}
			B(u, u) & = \langle u' , u' \rangle + \rho \sum_{i \in \Z} u(x_{i})^{2} - \mu \langle u , u \rangle \\
					& \geq \langle u' , u' \rangle - \mu \langle u , u \rangle \geq \langle u' , u' \rangle  + \langle u , u \rangle \\
					& = \| u \|_{H^{1}}^{2}.
		\intertext{Analogously for $\rho < 0$, if we choose $\mu < 2\rho$ there exists $\beta \in \R$ such that}
			B(u, u) & = \langle u' , u' \rangle + \rho \sum_{i \in \Z} |u(x_{i})|^{2} - \mu 	\langle u , u \rangle \\
					& = \langle u' , u' \rangle + \rho \sum_{i \in \Z} \big| u(\tilde{x}_{i}) + \int_{\tilde{x}_{i}}^{x_{i}} u(x) dx \big|^{2} - \mu \langle u , u \rangle \\
					& \geq \langle u' , u' \rangle + 2 \rho \left( \int_{\R} |u(x)|^{2} dx + \int_{\R} |u'(\tau)|^{2} d\tau \right) - \mu \langle u , u \rangle \\
					& = (2 \rho + 1) \| u' \|^{2} + (2\rho - \mu) \| u \|^{2}  \\
					& \geq \beta \| u \|_{H^{1}}^{2},
		\end{align*}
		Such that $B_{\mu}$ is coercive for $\mu$ small enough.
	\end{proof}
\end{theorem}
Thus, there exists a unique solution $u \in H^{1}(\R)$ to the problem \eqref{weak-formulation-of-A-for-LM} for fixed $f \in L^{2}(\R)$ and  the operator $R_{\mu} \colon L^{2}(\R) \rightarrow H^{1}(\R), f \mapsto u$ is well-defined for $\mu \in \R$ small enough; this mapping is one-to-one since for $u_{1} = u_{2}$
	\begin{equation}
		0 = B_{\mu}[u_{1}, v] - B_{\mu}[u_{2}, v]= \int (f_{1} - f_{2}) \overline{v} \quad \forall v \in H^{1}(\R). \label{f1f2almosteverywhere}
	\end{equation} 
As further $H^{1}(\R)$ is dense in $L^{2}(\R)$ this yields that the equation \eqref{f1f2almosteverywhere} holds also for all $v \in L^{2}(\R)$ and therefore $f_{1} = f_{2}$ almost everywhere. Accordingly $R_{\mu}$ is bijective and in return we can now define the Schrödinger operator as follows % todo Markus: Zitate, Verweis? cite befehl
		\[ A \coloneqq R_{\mu}^{-1} + \mu I \]
from which additionally follows that $R_{\mu}$ is the resolvent of $A$.

\section{\texorpdfstring{The Domain of $A$}{The Domain of A}}
If for every fixed $k \in \Z$ we consider in \eqref{weak-formulation-of-A} a test function $v \in C^{\infty}(\R)$ such that $\supp v = \Omega_{k}$ we get furthermore
	\[ \int_{x_{k}-\nicefrac{1}{2}}^{x_{k}} u'(x) \overline{v'(x)} dx = \int_{x_{k}-\nicefrac{1}{2}}^{x_{k}} A u \overline{v} \iff \int_{x_{k}-\nicefrac{1}{2}}^{x_{k}} u(x) \overline{v''(x)} dx = \int_{x_{k}-\nicefrac{1}{2}}^{x_{k}} - A u \overline{v}, \]
such that $A u = - u'' \in L^{2}$ on $(x_{k} -\nicefrac{1}{2}, x_{k})$ and analogous on $(x_{k}, x_{k} + \nicefrac{1}{2})$. % todo Martin/Markus war das nicht klar?
As $k \in \Z$ was arbitrary this means 
	$$ \mathcal{D}(A) \subset \Big\{ u \in \bigcap_{i \in \Z} \left( H^{2}(x_{i}-\nicefrac{1}{2}, x_{i}) \cap H^{2}(x_{i}, x_{i} + \nicefrac{1}{2}) \right) \Big\}. $$
	
Next, a test function $v \in C^{\infty}(\R)$ sucht that $\supp v = \Omega_{k}$ will yield for an arbitrary $k \in \Z$ from \eqref{weak-formulation-of-A}   through integration by parts on both sides of $x_{k}$ that
	\[ -\left( \int_{x_{k}-\nicefrac{1}{2}}^{x_{k}} + \int_{x_{k}}^{x_{k} + \nicefrac{1}{2}}\right) u'' \cdot \overline{v} + \left( u'(x_{k}-0) \overline{v(x_{k})} - u'(x_{k} + 0) \overline{v(x_{k})} \right) \\ \]
	\[ +  \rho u(x_{k})\overline{v(x_{k})} = - \int_{x_{k} - \nicefrac{1}{2}}^{x_{k}} u'' \overline{v} - \int_{x_{k}}^{x_{k} + \nicefrac{1}{2}} u'' \overline{v}. \]
But as we chose $v \in C^{\infty}(\R)$ this is equivalent to
	\[ u'(x_{k}-0) - u'(x_{k}+0) + \rho u(x_{k}) = 0, \]
such that
	\begin{equation}
		\mathcal{D}(A) \subset \Big\{ u \in \bigcap_{i \in \Z} H^{2}(x_{i}, x_{i + 1}) : u'(x_{i} - 0) - u'(x_{i} + 0) + \rho u(x_{i}) = 0 , ~\forall i \in \Z \Big\} \eqqcolon B \label{firstdomaininclusion}
	\end{equation} 
Hence, the operator defined by the action
	\[ A u = \begin{cases}
					- u'' & \text{ on } (x_{k} - \frac{1}{2}, x_{k}) \\
					- u'' & \text{ on } (x_{k}, x_{k} + \frac{1}{2}),
			 \end{cases} \quad \forall k \in \Z \] % todo Markus: on (...) or x \in (...)?
				
We can further prove the opposite inclusion for \eqref{firstdomaininclusion}. As $\mathcal{R}(R_{\mu}) = \mathcal{D}(A)$, we proceed by proving each $u \in B$ is also in the range of $R_{\mu}$. More specifically, as $\mathcal{D}(R_{\mu}) = L^{2}(\R)$ define $f \coloneqq A u$. To show $u = R_{\mu}(f - \mu u)$ consider\pdfcomment{I guess I have to show that $f - \mu$ u is onto though $\mu$ is fixed..}
	\[ \int_{\R} u' \overline{v'} + \rho \sum_{i \in \Z} u(x_{i}) \overline{v(x_{i})} - \mu \int_{\R} u \overline{v}= \int_{\R}(f-\mu u) \overline{v} \]
	\[ \iff \sum_{i \in \Z} \int_{\Omega_{i}} u' \overline{v'} + \rho u(x_{i}) \overline{v(x_{i})} = - \sum_{i \in \Z} \int_{x_{i} - \nicefrac{1}{2}}^{x_{i}} u'' \overline{v} + \int_{x_{i}}^{x_{i} + \nicefrac{1}{2}} u'' \overline{v}. \]
	For each $k \in \Z$ partial integration with a function $v$ having $\supp v = (x_{k} - \nicefrac{1}{2}, x_{k} + \nicefrac{1}{2})$ yields
	\[ \left( \int_{x_{k} - \nicefrac{1}{2}}^{x_{k}} + \int_{x_{k}}^{x_{k} +\nicefrac{1}{2}} \right) u' \overline{v'} - u'(x_{k}-0) \overline{v(x_{k})}  + u'(x_{k}+0) \overline{v(x_{k})}  = \int_{\Omega_{k}} u' \overline{v'} + \rho u(x_{k}) \overline{v(x_{k})} \]
	\[ \iff u'(x_{k}+0) - u'(x_{k}-0) - \rho u(x_{k}) = 0 \]
	such that we conclude
	\begin{align*}
		\mathcal{D}(A) & = \Big\{ u \in H^{1}(\R): u \in \bigcap_{j \in \Z} H^{2}(x_{j} , x_{j+1}), u'(x_{j} - 0) - u'(x_{j} + 0) + \rho \cdot u(x_{j}) = 0 ~\forall j \Big\}.
	\end{align*}

\section{The self-adjointness}

In chapter 4, we will further utilise the fact that the operator $A$ is self-adjoint. A self-adjoint operator is always closed, symmetric and has a completely real spectrum which narrows our analysis its spectrum down. 

\begin{theorem} \label{2.2:thm-RmuSymmetric}
	$R_{\mu}$ and $R_{\mu}^{-1}$ are both symmetric operator.
	
	\begin{proof}
		First, focus on $R_{\mu}^{-1} = (A - \mu I)$. As for all $v \in D(A)$:
			\begin{align*}
				\langle R_{\mu}^{-1} u, v \rangle & = \langle (A - \mu I) u, v \rangle \\
					& = \int u'\overline{v'} -  \mu \int u \overline{v} + \rho \sum_{i \in \Z} u(x_{i}) \overline{v(x_{i})} \\
					& = \langle u, (A - \mu I) v \rangle = \langle u,  R_{\mu}^{-1} v \rangle.
			\end{align*}

		$R_{\mu}^{-1}$ is symmetric. Now, as $\mathcal{D}(R_{\mu}) = L^{2}(\R)$ and $\mathcal{R}(R_{\mu}) = \mathcal{D}(R_{\mu}^{-1})$ for each $f, g \in L^{2}(\R)$ it follows
		
		\[  \langle R_{\mu} f, g \rangle =  \langle R_{\mu} f, R_{\mu}^{-1} R_{\mu} g \rangle = \langle f, R_{\mu} g \rangle \]
		
		such that $R_{\mu}$ is also symmetric.
	\end{proof}
\end{theorem}

Now, using both symmetries we can show that $A$ is self-adjoint:

\begin{theorem} \label{2.3:thm-ASelfAdjoint}
	$A$ is a self-adjoint operator. % todo Markus: genauer? Was meinst du mit genauer
		
	\begin{proof}
		As we already know that $R_{\mu}$ and $R_{\mu}^{-1}$ are symmetric, showing that $R_{\mu}^{-1}$ is self-adjoint is equivalent to showing that if $v \in \mathcal{D}({R_{\mu}^{-1}}^{*})$ and $v^{*} \in L^{2}(\R)$ are such that
		\[ \langle R_{\mu}^{-1} u, v \rangle = \langle u, v^{*} \rangle, \quad \forall u \in \mathcal{D}(R_{\mu}^{-1}) \tag*{(*)} \]
		then $v \in \mathcal{D}(R_{\mu}^{-1})$ and $R_{\mu}^{-1} v = v^{*}$.
		In $(*)$ we define $u \coloneqq R_{\mu} f$ for $f \in L^{2}$ and use that $R_{\mu}$ is symmetric and defined on the whole of $L^{2}(\R)$:
		\[  \langle f, v \rangle = \langle R_{\mu} f, v^{*} \rangle = \langle f, R_{\mu} v^{*} \rangle, \quad \forall u \in \mathcal{D}(R_{\mu}^{-1}) \]
		
		Which means that $v \in \mathcal{R}(R_{\mu}) = \mathcal{D}(R_{\mu}^{-1})$ and $R_{\mu}^{-1} v = v^{*}$, i.e. $R_{\mu}^{-1}$ is self-adjoint. As the operator $A$ is simply $R_{\mu}^{-1}$ shifted by $\mu \in \R$, $A$ is self-adjoint as well.		
	\end{proof}
\end{theorem}