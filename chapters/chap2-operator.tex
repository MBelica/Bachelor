\chapter{The one-dimensional Schrödinger operator $A$} \label{chap2}

The mathematical representation of the above introduced problem can be done by introducing a one-dimensional Schrödinger operator $A$ where the potential is given by a periodic delta-distribution. In this chapter we are going to examine properties of $A$ such as its domain and show that $A$ is self-adjoint. Later, in chapters \ref{chap3} and \ref{chap4}, we will need these results to deduce characteristics about the spectrum of $A$.
~\\ ~\\
Formally the operation of $A$ is defined by
\begin{equation}
	- \frac{d^{2}}{dx^{2}} + \rho \sum_{i \in \Z} \delta_{x_{i}} \label{the-operator-A-formally}
\end{equation}
on the whole of $\R$, where $\delta_{x_{i}}$ denotes the Dirac delta distribution supported at the point $x_{i}$. $\Omega_{k}$ will hereafter identify the periodicity cell containing point $x_{k}$ and w.l.o.g. let $x_{0} = 0$ and $|\Omega_{i}| = 1$ for all $i \in \Z$.

~\\  
In general, one cannot say, given $f \in L^{2}(\R)$, in which sense a solution to the formal problem 
	\begin{equation}
		Au = f \label{formal-problem}
	\end{equation}
	exists since the potential in $A$ is consists of a singular distribution. If we suppose for a moment that the problem is smooth, more specifically if the potential is given by \eqref{smooth-potential}, then formally multiplying it by a test function and integrating by parts yields the so called weak-formulation to the problem for whose solution less regularity is needed. Motivated by this, by taking the limit of the potential in the weak-formulation, we henceforth consider the problem to find for $\mu \in \R$ a function $u \in H^{1}(\R)$ such that
\begin{equation}
	\int_{\R} u' \overline{v'} + \rho \sum_{i \in \Z} u(x_{i}) \overline{v(x_{i})} - \mu \int_{\R} u \overline{v} = \int_{\R} f \overline{v} \quad \forall v \in C_{0}^{\infty}(\R), \label{weak-formulation-of-A}
\end{equation}	
holds and call it the weak-formulation of \eqref{formal-problem}.

\begin{remark}
	Since $C_{0}^{\infty}(\R)$ is dense in $H^{1}(\R)$ with respect to the norm in $H^{1}(\R)$, \eqref{weak-formulation-of-A} holds also for all $v \in H^{1}(\R)$. % todo INTRODUCTION: C_0înf dence in H^1
\end{remark}

We should first note that the left-hand side of problem \eqref{weak-formulation-of-A} is actually well-defined and finite, as for any $h \in (0, 1]$ we can estimate
\begin{align}
	\sum_{i \in \Z} |u(x_{i})|^{2} & \leq \sum_{i \in \Z} \left( 2 |u( x_{i} + h )|^{2} +  2 h \int_{x_{i}}^{x_{i} + h} \left| u'(\tau) \right|^{2} d\tau \right) \notag \\
		 & \leq 2 \sum_{i \in \Z} \left( \frac{1}{h} \int_{\Omega_{i}} |u( x )|^{2} dx + h \int_{\Omega_{i}} \left| u'(\tau) \right|^{2} d\tau \right). \label{preestimation-for-potential}
\end{align}
The particularly choice of $h = 1$ yields hence the estimation
\begin{align} 
		\sum_{i \in \Z} |u(x_{i})|^{2} & \leq 2 \| u \|^{2}_{H^{1}(\R)}. \label{estimation-for-potential}
\end{align}

\section{The resolvent-mapping of $A$}

To explicitly define our operator $A$ we will first show that for each $f \in L^{2}(\R)$ the equation \eqref{weak-formulation-of-A} has a unique solution $u \in H^{1}(\R)$. 

\begin{definition}
	Given $f \in L^{2}(\R)$, we define a functional $l_{f} \colon H^{1}(\R) \rightarrow \C$ by
	\[ l_{f}(v) \coloneqq \int_{\R} f v \]
and a sesquilinear form $B_{\mu} \colon H^{1}(\R) \times H^{1}(\R) \rightarrow \C$ for $\mu \in \R$ by
	\[ B_{\mu}[u, v] \coloneqq \int_{\R} u' \overline{v'} + \rho \sum_{i \in \Z} u(x_{i}) \overline{v(x_{i})} - \mu \int_{\R} u \overline{v}. \]
\end{definition}

As a result, \eqref{weak-formulation-of-A} is equivalent to finding for $\mu \in \R$ a function $u \in H^{1}(\R)$ such that
	\begin{equation}
		B_{\mu}[u, v] =  l_{f}(v) \quad \forall v \in H^{1}(\R). \label{weak-formulation-of-A-for-LM}
	\end{equation}
	
The existence of a unique $u \in H^{1}(\R)$ satisfying \eqref{weak-formulation-of-A-for-LM} now follows from Lax-Milgram's Theorem if the sesquilinear form $B_{\mu}$ is bounded and coercive and if $l_{f}$ is a bounded linear functional on $H^{1}(\R)$, which we will prove in the next two theorems, but above all note that $B[u, u] \in \R$.
% todo INTRODUCTION & IMPORTANT: lax-Milgram FIND VERSION WITH COMPLEX
\begin{theorem} \label{2.1:thm-LaxMilgram}
	The sesquilinear form $B_{\mu}$ is (for small enough $\mu \in \R$)
	\begin{enumerate}
		\item[i)] bounded, i.e. there exists a constant $\alpha > 0$ such that
			\[ \left| B_{\mu}[u,v] \right| \leq \alpha \|u\| \|v\| \]
			holds for all $u, v \in H^{1}(\R)$.
		\item[ii)] coercive, i.e. there exists a constant $\beta > 0$ such that
			\[ \beta \|u\|^{2} \leq B_{\mu}[u, u] \]
			for all $u \in H^{1}(\R)$.
	\end{enumerate} 

	\begin{proof} ~\\
		i) The boundedness follows from \eqref{estimation-for-potential} as for an arbitrary $\rho \in \R$ there exists $\alpha \in \R$ such that
		\begin{align*}
			|B(u, \varphi)|^{2} & \leq \| u' \| \| v' \| + 2 |\rho |\sum_{i \in \Z} |u(x_{i})|^{2} |v(x_{i})|^{2} - \mu \| u \| \| v \| \\
				& \leq \| u' \| \| v' \| + 8 |\rho| \| u \|^{2}_{H^{1}(\R)} \| v \|^{2}_{H^{1}(\R)}  - \mu \| u \| \| v \| \\
				& = (8|\rho| - \mu) \| u \| \| v \| + 8 |\rho| \left( \| u \| \| v' \| + \| u' \| \| v \| \right) + (8 |\rho| + 1) \| u' \| \| v'\| \\
				& \leq \alpha \| u \|_{H^{1}(\R)} \| \varphi \|_{H^{1}(\R)}
		\end{align*}
		where $\alpha = \max \left\{ 8|\rho| - \mu , 8 |\rho| + 1 \right\}$. \\
		ii) For the coercivity, we first assume $\rho \geq 0$. Now, if $\mu < -1$ we get 
		\begin{align*}
			B[u, u] & = \langle u' , u' \rangle + \rho \sum_{i \in \Z} \left|u(x_{i})\right|^{2} - \mu \langle u , u \rangle \\
					& \geq \langle u' , u' \rangle  + \langle u , u \rangle \\
					& = \| u \|_{H^{1}(\R)}^{2}.
		\intertext{Analogously for $\rho < 0$, using \eqref{preestimation-for-potential} we can choose $h < \frac{1}{2 |\rho|}$ and with that if $\mu < - \frac{2|\rho|}{h}$ there exists $\beta \in \R$ such that}
			B[u, u] & = \langle u' , u' \rangle + \rho \sum_{i \in \Z} |u(x_{i})|^{2} - \mu 	\langle u , u \rangle \\
					& \geq \langle u' , u' \rangle + 2 \rho \sum_{i \in \Z} \left( \frac{1}{h} \int_{\Omega_{i}} |u( x )|^{2} dx + h \int_{\Omega_{i}} \left| u'(\tau) \right|^{2} d\tau \right) - \mu \langle u , u \rangle \\
					& = (2 \rho h + 1) \| u' \|^{2} + (2 \rho \frac{1}{h} - \mu) \| u \|^{2}  \\
					& \geq \beta \| u \|_{H^{1}(\R)}^{2},
		\end{align*}
		where $\beta = \min \left\{ 2 \rho h + 1u , 2 \rho \frac{1}{h} - \mu \right\}$.
	\end{proof}
\end{theorem}

\begin{theorem}
	Given $f \in L^{2}(\R)$ the functional $l_{f}$ is a bounded linear functional on $H^{1}(\R)$.
	% todo INTRODUCTION Cauchy-Schwarz
	\begin{proof}
		It is easily seen that $l_{f}$ is linear, for the boundedness the Cauchy–Schwarz inequality yields
		\[ | l_{f}(v) | \leq \| f \|_{L^{2}(\R)} \| v \|_{H^{1}(\R)} \]
	\end{proof}
\end{theorem}

Therefore, as in theorem \ref{2.1:thm-LaxMilgram} used we will subsequent assume that $\mu \in \R$ is small enough. In return Lax-Migram's Theorem shows that for any fixed $f \in L^{2}(\R)$ a unique solution $u \in H^{1}(\R)$ to the problem \eqref{weak-formulation-of-A-for-LM} exists. This on the other hand allows us to proceed as follows.

\begin{definition}
	Let us define $R_{\mu} \colon L^{2}(\R) \rightarrow L^{2}(\R), f \mapsto u$ with $u$ being the solution of \eqref{weak-formulation-of-A-for-LM}.
\end{definition}
 
Taking in account that $R_{\mu}$ is a linear operator, which is easy to see, there are two more properties of $R_{\mu}$ for us left to show. 

% todo INTRODUCTION H^1 dense in L^2
\begin{theorem} \label{rmuinj}
	The mapping $R_{\mu}$ is bounded and injective.
	
	\begin{proof}
		For $f \in L^{2}(\R)$ there exists $u \in \mathcal{D}(A)$ such that
		\[ \| R_{\mu} f \|_{L^{2}(\R)}^{2} \leq \| u \|_{H^{1}(\R)}^{2}. \] % todo Martin/andrii check
		Now, using \eqref{estimation-for-potential} with a small enough $\mu \in \R$ yields with Cauchy–Schwarz's inequality
		\[ \| R_{\mu} f \|_{L^{2}(\R)}^{2} \leq \left| \int_{\R} |u'|^{2} + \rho \sum_{i \in \Z} |u(x_{i})|^{2} - \mu \int_{\R} |u|^{2} \right|^{2}  \leq \| f \|_{L^{2}(\R)}^{2} \| u \|_{L^{2}(\R)}^{2} \]	
		Taking in mind that $\mathcal{R}(R_{\mu}) \subseteq H^{1}(\R)$, we know that for $u_{1} = u_{2}$
		\begin{equation}
			0 = B_{\mu}[u_{1}, v] - B_{\mu}[u_{2}, v]= \int_{\R} (f_{1} - f_{2}) \overline{v} \quad \forall v \in H^{1}(\R). \label{f1f2almosteverywhere}
		\end{equation} 
		As $H^{1}(\R)$ is dense in $L^{2}(\R)$ this yields that the equality \eqref{f1f2almosteverywhere} holds also for all $v \in L^{2}(\R)$, hence $f_{1} = f_{2}$ almost everywhere. 
	\end{proof}
\end{theorem}



\section{The domain of $A$}

Resulting from theorem \ref{rmuinj}, we know that $R_{\mu}$ is invertible. This allows us to define the aforementioned operator $A$ explicitly.

\begin{definition}
	Let $A \colon \mathcal{D}(A) \subseteq L^{2}(\R) \rightarrow L^{2}(\R)$ be the linear operator defined by
	\[ A \coloneqq R_{\mu}^{-1} + \mu I, \quad \mathcal{D}(A) = \mathcal{R}(R_{\mu}). \]
\end{definition}

Note that this definition makes sense regarding formal definition in \eqref{the-operator-A-formally} and as show below is independent of $\mu$. % todo Andrii: is this enough for 

\begin{remark}
	This allows us to draw the conclusions that $R_{\mu}$ is the resolvent of $A$.
\end{remark}

We will now use the fact that every element $u \in \mathcal{D}(A) = \mathcal{R}(R_{\mu})$ is a solution of \eqref{weak-formulation-of-A-for-LM} to find additional, necessary  characteristics of $u$ or rather $\mathcal{D}(A)$. However, we already know by Lax-Milram's Theorem that $u \in H^{1}(\R)$.
~\\ ~\\
First, let us for the sake of brevity define
\[ H^{2}\Big(\R \setminus \bigcup_{i \in \Z} x_{i} \Big) \coloneqq \Big\{ u \in L^{2}(\R) : u \in \bigcap_{i \in \Z} H^{2}(x_{i}, x_{i+1}), \sum_{i \in \Z} \|u\|^{2}_{H^{2}(x_{i}, x_{i+1})} < \infty \Big\} \]

Then, considering in \eqref{weak-formulation-of-A} any fixed $k \in \Z$ and an arbitrary test function $v \in C^{\infty}(\R)$ with $\supp v \subseteq [x_{k}, x_{k+1}]$ we get 
	\[ \int_{x_{k}}^{x_{k + 1}} u'(x) \overline{v'(x)} dx = \int_{x_{k}}^{x_{k+1}} A u  \overline{v} \iff \int_{x_{k}}^{x_{k+1}} - u(x) \overline{v''(x)} dx = \int_{x_{k}}^{x_{k+1}} A u \overline{v}, \]
whence $- u'' \in L^{2}(x_{k}, x_{k + 1})$ and $A u = - u''$. Since we chose $k \in \Z$  arbitrary we obtain
	$$ \mathcal{D}(A) \subseteq \Big\{ u \in \bigcap_{i \in \Z} H^{2}(x_{i}, x_{i+1}) \Big\}. $$
Next, a test function $v \in C^{\infty}(\R)$ with the property $\supp v = \Omega_{k}$ yields in \eqref{weak-formulation-of-A} for any $k \in \Z$ through integration by parts on both sides of $x_{k}$ that
	\[ -\left( \int_{x_{k}-\frac{1}{2}}^{x_{k}} + \int_{x_{k}}^{x_{k} + \frac{1}{2}}\right) u'' \overline{v} + \left( u'(x_{k}-0) \overline{v(x_{k})} - u'(x_{k} + 0) \overline{v(x_{k})} \right) \\ \]
	\[ +  \rho u(x_{k})\overline{v(x_{k})} = - \int_{x_{k} - \frac{1}{2}}^{x_{k}} u'' \overline{v} - \int_{x_{k}}^{x_{k} + \frac{1}{2}} u'' \overline{v}. \]
Now then choosing in addition $v$ to be non-zero in $x_{k}$ yields 
	\[ u'(x_{k}-0) - u'(x_{k}+0) + \rho u(x_{k}) = 0, \]
and therefore
	\begin{equation}
		\mathcal{D}(A) \subseteq \Big\{ u \in \bigcap_{i \in \Z} H^{2}(x_{i}, x_{i+1}) : u'(x_{i} - 0) - u'(x_{i} + 0) + \rho u(x_{i}) = 0 ~\forall i \in \Z \Big\}.
	\end{equation} 
Last but not least we need one properties. Choosing a function $v \in C_{0}^{\infty}(\R)$ with $\supp v = (x_{-n}, x_{n+1})$ in \eqref{weak-formulation-of-A} yields with partial integration on every interval $(x_{i}, x_{i+1})$ with that
\[ \sum_{i=-n}^{n-1} -\int_{x_{i}}^{x_{i+1}} u'' \overline{v} + \sum_{i=-n}^{n-1} u' v \Big|_{x_{i}}^{x_{i+1}} + \rho \sum_{i=-n}^{n-1} u(x_{i}) \overline{v(x_{j})} - \mu \int_{x_{-n}}^{x_{n}} u \overline{v} = \int_{x_{-n}}^{x_{n}} f \overline{v} \]
\begin{equation} 
	\iff \sum_{i=-n}^{n-1} \int_{x_{i}}^{x_{i+1}} u'' \overline{v} = - \int_{x_{-n}}^{x_{n}} f \overline{v} - \mu \int_{x_{-n}}^{x_{n}} u \overline{v} \label{refwa}
\end{equation} 
By defining $w_{n} \coloneqq \sum_{i=-n}^{n-1} u'' \mathds{1}_{[x_{i}, x_{i+1}]}$ we can estimate the left-hand side of \eqref{refwa} by
\begin{align}
	\left| \langle w_{n}, v \rangle \right| & \leq \left| \mu \int_{x_{-n}}^{x_{n}} u \overline{v} \right| + \left| \int_{x_{-n}}^{x_{n}} f v \right| \notag \\
		& \leq |\mu| \|u\|_{L^{2}(x_{-n}, x_{n})} \|v\|_{L^{2}(x_{-n}, x_{n})} + \|f\|_{L^{2}(x_{-n}, x_{n})} \|v\|_{L^{2}(x_{-n}, x_{n})} \notag \\
		& \leq c \|v\|_{L^{2}(x_{-n}, x_{n})}, \label{refwa2}
\end{align}
for some $c \in \R$. Since $c$ is independent of $n$ we hence know by \eqref{refwa2} that
	\[ \sum_{i \in \Z} \|u''\|^{2}_{L^{2}(x_{i}, x_{i+1})} < \infty. \]
This yields
	\begin{equation}
		\mathcal{D}(A) \subseteq \left\{ u \in H^{1}(\R): u \in H^{2}\Big(\R \setminus \bigcup_{i \in \Z} x_{i} \Big), u'(x_{j} - 0) - u'(x_{j} + 0) - \rho u(x_{j}) = 0 ~\forall j \right\}. \label{firstdomaininclusion} 
	\end{equation}
For an arbitrary $u \in \mathcal{D}(A)$ we know hence from \eqref{firstdomaininclusion} that
	\[ A u = \begin{cases}
					- u'' & \text{ on } (x_{k} - \frac{1}{2}, x_{k}) \\
					- u'' & \text{ on } (x_{k}, x_{k} + \frac{1}{2}),
			 \end{cases} \quad \forall k \in \Z. \]

For the reverse inclusion of \eqref{firstdomaininclusion} we use the operator $R_{\mu}$ but first let us, again for brevity, denote with $B$ the right-hand side of \eqref{firstdomaininclusion}. Now, since $\mathcal{R}(R_{\mu}) = \mathcal{D}(A)$, we proceed by proving each $u \in B$ is also in the range of $R_{\mu}$. More specifically, as $\mathcal{D}(R_{\mu}) = L^{2}(\R)$ define $f \coloneqq - u''$ on $(x_{k}, x_{k + 1})$ for all $i \in \Z$; as we know that $u \in H^{2}\Big(\R \setminus \bigcup_{i \in \Z} x_{i} \Big)$ we ensure $f \in L^{2}$. 
Hence, we have to show $u = R_{\mu}(f - \mu u)$ or equivalently
	\begin{align*}
		 \int_{\R} u' \overline{v'} + \rho \sum_{i \in \Z} u(x_{i}) \overline{v(x_{i})} - \mu \int_{\R} u \overline{v}= \int_{\R}(f-\mu u) \overline{v} \\
		\iff \sum_{i \in \Z} \int_{\Omega_{i}} u' \overline{v'} + \rho u(x_{i}) \overline{v(x_{i})} = - \sum_{i \in \Z} \int_{x_{i}}^{x_{i+1}} u'' \overline{v}.
	\end{align*}


	For each $k \in \Z$ partial integration with a function $v \in C_{c}^{\infty}(\R)$ having $\supp v = (x_{k} - \frac{1}{2}, x_{k} + \frac{1}{2})$ yields
	\[ \int_{\Omega_{k}} u' \overline{v'} + \rho u(x_{k}) \overline{v(x_{k})} =\left( \int_{x_{k} + \frac{1}{2}}^{x_{k}} - \int_{x_{k}}^{x_{k} +\frac{1}{2}} \right) u' \overline{v'} - u'(x_{k}-0) \overline{v(x_{k})}  + u'(x_{k}+0) \overline{v(x_{k})}  \]
	\[ \iff u'(x_{k}-0) - u'(x_{k}+0) - \rho u(x_{k}) = 0. \]
	Such that we conclude
	\begin{align*}
		\mathcal{D}(A) & = \Big\{ u \in H^{1}(\R): u \in H^{2}\Big(\R \setminus \bigcup_{i \in \Z} x_{i} \Big), u'(x_{j} - 0) - u'(x_{j} + 0) - \rho u(x_{j}) = 0 ~\forall j \Big\}.
	\end{align*}


\begin{remark}
	The definition of $A$ is independent of $\mu$ since as seen above the domain is independent of $\mu$ and $\mu$-dependent terms cancel each other out.
\end{remark}
 

\section{The operator $A$ is self-adjoint}

In chapter \ref{chap4}, we will use the fact that the operator $A$ self-adjoint, from with follows that $A$ is a closed and symmetric operator. For this purpose we start by showing that $R_{\mu}$ and $R_{\mu}^{-1}$ are symmetric operators.
% INTRODUCTION CLOSE AND SYMMETRIC SELF ADJOINT 
\begin{theorem} \label{2.2:thm-RmuSymmetric}
	$R_{\mu}$ and $R_{\mu}^{-1}$ are symmetric operator.
	
	\begin{proof}
		First, focus on $R_{\mu}^{-1} = (A - \mu I)$. As for all $v \in \mathcal{D}(A)$:
			\begin{align*}
				\langle R_{\mu}^{-1} u, v \rangle & = \langle (A - \mu I) u, v \rangle \\
					& = \int u'\overline{v'} -  \mu \int u \overline{v} + \rho \sum_{i \in \Z} u(x_{i}) \overline{v(x_{i})} \\
					& = \langle u, (A - \mu I) v \rangle = \langle u,  R_{\mu}^{-1} v \rangle,
			\end{align*}

		thus $R_{\mu}^{-1}$ is symmetric. Now, as $\mathcal{D}(R_{\mu}) = L^{2}(\R)$ and $\mathcal{R}(R_{\mu}) = \mathcal{D}(R_{\mu}^{-1})$ for each $f, g \in L^{2}(\R)$ it follows
		
		\[  \langle R_{\mu} f, g \rangle =  \langle R_{\mu} f, R_{\mu}^{-1} R_{\mu} g \rangle = \langle f, R_{\mu} g \rangle \]
		
		such that $R_{\mu}$ is also symmetric.
	\end{proof}
\end{theorem}

Now, using the fact that $R_{\mu}$ and $R_{\mu}^{-1}$ are both symmetric we can  prove that $A$ is self-adjoint. 

\begin{theorem} \label{2.3:thm-ASelfAdjoint}
	$A$ is a self-adjoint operator.
		
	\begin{proof}
		As we already know that $R_{\mu}$ and $R_{\mu}^{-1}$ are symmetric, showing that $R_{\mu}^{-1}$ is self-adjoint is equivalent to showing that if $v \in \mathcal{D}({R_{\mu}^{-1}}^{*})$ and $v^{*} \in L^{2}(\R)$ are such that
		\begin{align}
			\langle R_{\mu}^{-1} u, v \rangle = \langle u, v^{*} \rangle, \quad \forall u \in \mathcal{D}(R_{\mu}^{-1}) \label{*-condition}
		\end{align}
		then $v \in \mathcal{D}(R_{\mu}^{-1})$ and $R_{\mu}^{-1} v = v^{*}$.
		In \eqref{*-condition} we define $u \coloneqq R_{\mu} f$ for any $f \in L^{2}(\R)$ and use the fact that $R_{\mu}$ is symmetric and defined on the whole of $L^{2}(\R)$:
		\[  \langle f, v \rangle = \langle R_{\mu} f, v^{*} \rangle = \langle f, R_{\mu} v^{*} \rangle, \]
		
		which means that $v \in \mathcal{R}(R_{\mu}) = \mathcal{D}(R_{\mu}^{-1})$ and $R_{\mu}^{-1} v = v^{*}$, i.e. $R_{\mu}^{-1}$ is self-adjoint. As the operator $A$ is simply $R_{\mu}^{-1}$ shifted by $\mu \in \R$, $A$ is self-adjoint as well.		
	\end{proof}
\end{theorem}
% todo INTRODUCTION REAL SPECTRUM
Every symmetric operator has an entirely real spectrum, hence theorem \ref{2.3:thm-ASelfAdjoint} yields our first result about the spectrum of $A$.