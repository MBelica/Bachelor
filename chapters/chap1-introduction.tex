\chapter{Introduction}

%In 1962 Erwin Schrödinger published his theory based on de Broglie's idea of understanding systems with discrete energies as standing waves. A wave function in classical mechanics and electrodynamics is a periodic function with the property that
%	\begin{equation}
%		\psi(x, t) = A \cdot \sin( \omega t - kx) \label{wavefunction}
%	\end{equation} 
%where $x$ denotes the space, $t$ the time, $\omega$ the circular frequency, $k$ the wave number and $A$ the amplitude. Differentiating the equation \eqref{wavefunction} twice with respect to $x$ we get
%	\[ \frac{\partial^{2}}{\partial x^{2}} \psi(x, t) = -k^{2} A \sin( \omega t - kx) = -k^{2} \psi(x, t) \]
%and differentiating twice with respect to $t$ yields
%	\[ \frac{\partial^{2}}{\partial t^{2}} \psi(x, t)  = -\omega^{2} A \sin( \omega t - kx) = -\omega^{2} \psi(x, t). \]
%Therefore $\psi$ satisfies the following differential equations
%	\begin{equation}
%		 \frac{\partial^{2}}{\partial x^{2}} \psi(x, t) = \frac{k^{2}}{\omega^{2}} \frac{\partial^{2}}{\delta t^{2}} \psi(x, t) \label{11.6}
%	\end{equation}
%Now, we are looking for a wave function for mass tainted particles, like electrons. For that use the de-Broglie relation $\lambda = \frac{h}{p}$ in the quotient of \eqref{11.6}:
%	\begin{equation}
%		\frac{k^{2}}{\omega^{2}} = \frac{p^{2}}{\nu^{2} h^{2}}. \label{11.7}
%	\end{equation} 
%The kinetic energy of such a particle is his total energy $E$ without the potential energy $V$:
%	\[ E_{kin} = \frac{p^{2}}{2m} = E - V \]
%Solving this equation to the impulse squared $p^{2}$ yields:
%	\[ p^{2} = 2m \cdot ( E - V), \]
%and using this expression in \eqref{11.7} we conclude with the new differential equation:
%	\begin{equation} 
%		 \frac{\partial^{2}}{\partial x^{2}} \psi(x, t) = \frac{2m (E - V)}{h^{2} \nu^{2}} \frac{\partial^{2}}{\partial t^{2}} \psi(x, t). \label{differentialeq}
%	\end{equation}
%For simplicity, let us consinder only time-independent solutions of the differential eqation \eqref{differentialeq} and focus on systems for which the total energy stays constant. Because of $E = h \nu$, the frequency will stay constant too; hence the  quotient $\frac{2m/(E-V)}{h^{2}v^{2}}$ on the right-hand side of \eqref{differentialeq} is a constant as well. Solutions for such a differential equation can be split into one only space-dependent and one only time-dependent component:
%		\[ \psi(x, t) = f(t) \cdot \psi(x). \]
%	Using the approach $f(t) = \sin(\omega t)$ with $\frac{\partial^{2}}{\partial x^{2}} \psi(x, t) = f(t) \psi''(x)$ and $\frac{\partial^{2}}{\partial t^{2}} \psi(x, t) = \psi(x) f''(t)$ shows that \eqref{differentialeq} is equivalent to
%	\begin{align*}
%		f(t) \psi''(x) & = \frac{2m (E - V)}{h^{2} \nu^{2}} \psi(x) f''(t) \\
%		\iff - \psi''(x) + & \frac{2m\omega^{2}}{h^{2} \nu^{2}} V \psi(x) = \frac{2m \omega^{2}E}{h^{2} \nu^{2}} \psi(x) 
%	\end{align*}
Here comes an introduction the the Schrödinger operator... \\
Based on this, we are interested in the one-dimensional Schrödinger operator with describes the simplified movement of a non-relativistic movement of a quantum particle in a potential, especially in the spectral problem of such an operator; note that in this thesis we will a potential given by a delta-distribution, thereby this operator is formally defined through the operation
\begin{equation}
	- \frac{d^{2}}{dx^{2}} + \rho \sum_{i \in \Z} \delta_{x_{i}} \label{the-operator-A-formally}
\end{equation}
on the whole of $\R$, where $\delta$ denotes the Dirac delta distribution and $x_{i}$ are periodically distributed points on $\R$. $\Omega_{k}$ will hereafter identify the periodicity cell containing delta point $x_{k}$ and let w.o.l.g. $x_{0} = 0$ and $|\Omega_{i}| = 1$ for all $i \in \Z$. \\

In general, one cannot expect that \eqref{the-operator-A-formally} has a classical
solution. For the existence of a classical solution, all parameters have to be in a sense sufficiently smooth, which, for our distributional potential, is never the case. Nevertheless a solution of the so called weak formulation which requires less regularity can still exist, consider for this for $\mu \in \R$ the problem
\begin{equation}
	\int u' \overline{v'} + \rho \sum_{i \in \Z} u(x_{i}) \overline{v(x_{i})} - \mu \int u \overline{v} = \int f \overline{v} \quad \forall v \in H^{1}(\R), \label{weak-formulation-of-A}
\end{equation}	
where $u \in H^{1}(\R)$ and $f \in L^{2}(\R)$. This is obtained by multiplying
	\[ Au + \mu u = f \]
with some $v \in C_{c}^{\infty}(\R)$, partial integration and using the fact that $C_{c}^{\infty}(\R)$ is dense in $H^{1}(\R)$. \\
	
One should note that left-hand side of problem \eqref{weak-formulation-of-A} is actually convergent, as for arbitrary $\tilde{x}_{i} \in \Omega_{i}$
\begin{eqnarray}
	\sum_{i \in \Z} |u(x_{i})|^{2} & \leq & \sum_{i \in \Z} \left( \big| u(\tilde{x}_{i}) + \int_{\tilde{x}_{i}}^{x_{i}} u'( \tau ) d\tau \big| \right)^{2} \notag \\
		 & \leq & 2 \sum_{i \in \Z} \left( \int_{\Omega_{i}} |u( x )|^{2} dx +  \int_{\Omega_{i}} \left| u'(\tau) \right|^{2} d\tau \right) \notag \\
		 & \leq & 2 \cdot \| u \|^{2}_{H^{1}(\R)}. \label{estimation-for-potential}
\end{eqnarray}