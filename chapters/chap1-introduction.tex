\chapter{Introduction} \label{chap:1}

The problem considered in this thesis arises from the Kronig-Penney model, see for example \cite[Chapter 3]{heering2002elektrophysik}. We are interested in an idealised quantum-mechanical system that models a quantum particle behaving as a matter wave moving in one dimension through an infinite periodic array of rectangular potential barriers, i.e. through a space area in which a potential attains a local maximum. Such an array commonly occurs in models of periodic crystal lattices where the potential is caused by ions in the crystal structure. Those charged atoms or molecules create an electromagnetic field around themselves. Hence, any particle moving through such a crystal would be subject to a recurrent electromagnetic potential. Although a solid particle, simplified as a point mass, would be reflected at such a barrier, there is a possibility that the quantum particle, as it behaves like a wave, penetrates the barrier and continues its movement beyond. 
~\\

Assuming the spacing between all ions is equidistant the potential function $V(x)$ in the lattice can be approximated by a rectangular potential as depicted in Figure \ref{figure-kpm}, where $b$ is the width of the support and $\rho$ the magnitude of the potentials.
	\vspace{0.25cm}
	\begin{figure}[!ht] \centering 
	  \resizebox{.65\linewidth}{!}{
		\definecolor{lightgray}{rgb}{0.3765,0.3765,0.3765}
		\begin{tikzpicture}[line cap=round,line join=round,>=triangle 45,x=1.0cm,y=1.0cm]
			\draw[->,color=lightgray] (-5.5,0.) -- (5.5,0.);
			\foreach \x in {-5.,-4.,-3.,-2.,-1.,1.,2.,3.,4.,5.}
			 \draw[shift={(\x,0)},color=lightgray] (0pt,-2pt);
			\draw[->,color=lightgray] (0.,-1.1) -- (0.,4.25);
			\clip(-5.5,-1.1) rectangle (5.5,4.25);
			\draw (-0.3,3.)-- (0.3,3.);
			\draw (0.3,0.)-- (0.3,3.);
			\draw (-0.3,3.)-- (-0.3,0.);
			\draw (-1.7,3.)-- (-1.7,0.);
			\draw (-2.3,3.)-- (-1.7,3.);
			\draw (-2.3,3.)-- (-2.3,0.);
			\draw (-1.7,3.)-- (-1.7,0.);
			\draw (-3.7,0.)-- (-3.7,3.);
			\draw (-3.7,3.)-- (-4.3,3.);
			\draw (-4.3,3.)-- (-4.3,0.);
			\draw (-3.7,3.)-- (-3.7,0.);
			\draw (-3.7,3.)-- (-4.3,3.);
			\draw (1.7,3.)-- (1.7,0.);
			\draw (1.7,3.)-- (2.3,3.);
			\draw (2.3,3.)-- (2.3,0.);
			\draw (3.7,0.)-- (3.7,3.);
			\draw (3.7,3.)-- (4.3,3.);
			\draw (4.3,3.)-- (4.3,0.);
			\draw (-3.7,0.)-- (-2.3,0.);
			\draw[pattern=north west lines, pattern color=black!30] (2.3,0) rectangle (1.7,3);
			\draw (1.725,1.9) node[anchor=north west, circle, minimum size=0.5em] {$V_{0}$};
			\draw (-1.7,0.)-- (-0.3,0.);
			\draw (0.3,0.)-- (1.7,0.);
			\draw (2.3,0.)-- (3.7,0.);
			\draw (4.3,0.)-- (5.7,0.);
			\draw (-4.3,0.)-- (-5.7,0.);
			\draw [->] (0.65,1.5) -- (0.65,3.);
			\draw [->] (0.65,1.5) -- (0.65,0.);
			\draw [->] (0.65,1.5) -- (0.65,3.);
			\draw [dotted] (0.3,3.)-- (0.65,3.);
			\draw [dotted] (1.,3.)-- (0.65,3.);
			\draw [->] (2.75,-0.5) -- (2.3,-0.5);
			\draw [->,line width=0.4pt] (1.25,-0.5) -- (1.7,-0.5);
			\draw [dotted] (1.7,0.)-- (1.7,-1.);
			\draw [dotted] (2.3,0.)-- (2.3,-0.5);
			\draw [dotted] (2.3,-0.5)-- (2.3,-1.);
			\draw (1.8,-0.05) node[anchor=north west] {$b$};
			\draw (0.75,1.9) node[anchor=north west] {$\rho$};
			\draw (0.1,4.25) node[anchor=north west,color=lightgray] {$V(x)$};
			\draw (5,-0.1) node[anchor=north west,color=lightgray] {$x$};
		  \end{tikzpicture}
	    }
	  \caption{Potential $V(x)$ of the Kronig-Penney model} \label{figure-kpm} 
	\end{figure}
~\\

In this thesis, we are interested in the spectrum of the operator describing the situation in the Kronig-Penney model when the particle moves through periodically distributed, singular potentials. With respect to the above assumptions	 this means taking the limit $b \rightarrow 0$ while $V_{0}$ remains of order $\rho b^{-1}$. On this account, we will build on the research presented in \cite{dorfler2011photonic} where the spectrum of periodic differential operators with smooth coefficients has been analysed. We will show that an operator modelling the aforementioned situation has, as in the case of smooth coefficients, a spectrum that consists of a union of compact intervals in $\R$ which form a so-called spectral band. 
~\\

In a physical sense, spectral bands represent energy levels. Only an electron with an energy level within the spectral band can exist inside the crystal. In other words, any gaps between the spectral bands, if they exist, represent a forbidden energy range. According to Bragg's law, standing waves form at the boundary between the spectral band and the forbidden energy levels. The closer the electrons orbit the nucleus of their atom the more energetically favourable are the standing waves which is a preferable state by the electrons, for a detailed explanation see \cite[Section 3.2]{heering2002elektrophysik}. Therefore, Bragg's law explains the forbidden energy levels. Because of this principle, for knowing about possible energetic levels within periodic crystal lattices knowledge about the corresponding spectral properties is needed.
~\\

The remainder of this thesis is structured as follows. We begin with a few preliminaries in Chapter 2 to recall some key concepts in functional analysis and spectral theory. In Chapter 3, we introduce a Schrödinger operator with a periodic potential, describing the motion of an electron under study, and, as the first step into the analysis of its spectrum, we show that the operator is self-adjoint. As principal mathematical tool for analysing the spectrum of the operator we use the Floquet-Bloch theory. We then transfer the spectral problem of the operator on the whole of $\R$ to a family of eigenvalue problems on the periodicity cell. Hence, we proceed in Chapter 4 by restricting the problem to its fundamental domain of periodicity. We analyse the spectrum of the restricted operator by showing the compactness of its resolvent while varying quasi-periodic boundary conditions. In order to be able to extend the restricted-case results to the more general periodic case, in Chapter 5 we introduce two more concepts, the Floquet transformation and the Bloch waves. Based on these methods, in Chapter 6 we show our main result for the one-dimensional case, i.e. that such an operator has a spectrum consisting of a union of compact intervals in $\R$, and extrapolate this result to the multi-dimensional case in Chapter 7. Finally, we conclude this thesis in Chapter 8 with a discussion of possible gaps between the compact intervals and a brief overview of some recent research. % todo discussion ist ein wenig viel...