\chapter{Introduction} \label{chap:1}

The problem considered in this thesis arises from the Kronig-Penney model, see for example \cite[chapter 3]{heering2002elektrophysik}, which describes an idealised quantum-mechanical system that models a quantum particle behaving as a matter wave moving in one-dimension through an infinite periodic array of rectangular potential barriers, i.e. through a space area in which a potential attains a local maximum. Such an array commonly occurs in models of periodic crystal lattices where the potential is caused by ions in the crystal structure. Those charged molecules create an electromagnetic field around themselves. Hence, any particle moving through such a crystal would be subject to a recurrent electromagnetic potential. Although a solid particle, simplified as a point mass, would be reflected at such a barrier, there is a possibility that the quantum particle, as it behaves like a wave, penetrates the barrier and continues its movement beyond. 
~\\

Assuming the spacing between all ions is equidistant the potential function $V(x)$ in the lattice can be approximated by a rectangular potential as depicted in Figure \ref{figure-kpm}, where $b$ is the width of the support and $\rho$ the magnitude of the potentials.
	\vspace{0.25cm}
	\begin{figure}[!ht] \centering 
	  \resizebox{.65\linewidth}{!}{
		\definecolor{lightgray}{rgb}{0.3765,0.3765,0.3765}
		\begin{tikzpicture}[line cap=round,line join=round,>=triangle 45,x=1.0cm,y=1.0cm]
			\draw[->,color=lightgray] (-5.5,0.) -- (5.5,0.);
			\foreach \x in {-5.,-4.,-3.,-2.,-1.,1.,2.,3.,4.,5.}
			 \draw[shift={(\x,0)},color=lightgray] (0pt,-2pt);
			\draw[->,color=lightgray] (0.,-1.1) -- (0.,4.25);
			\clip(-5.5,-1.1) rectangle (5.5,4.25);
			\draw (-0.3,3.)-- (0.3,3.);
			\draw (0.3,0.)-- (0.3,3.);
			\draw (-0.3,3.)-- (-0.3,0.);
			\draw (-1.7,3.)-- (-1.7,0.);
			\draw (-2.3,3.)-- (-1.7,3.);
			\draw (-2.3,3.)-- (-2.3,0.);
			\draw (-1.7,3.)-- (-1.7,0.);
			\draw (-3.7,0.)-- (-3.7,3.);
			\draw (-3.7,3.)-- (-4.3,3.);
			\draw (-4.3,3.)-- (-4.3,0.);
			\draw (-3.7,3.)-- (-3.7,0.);
			\draw (-3.7,3.)-- (-4.3,3.);
			\draw (1.7,3.)-- (1.7,0.);
			\draw (1.7,3.)-- (2.3,3.);
			\draw (2.3,3.)-- (2.3,0.);
			\draw (3.7,0.)-- (3.7,3.);
			\draw (3.7,3.)-- (4.3,3.);
			\draw (4.3,3.)-- (4.3,0.);
			\draw (-3.7,0.)-- (-2.3,0.);
			\draw (-1.7,0.)-- (-0.3,0.);
			\draw (0.3,0.)-- (1.7,0.);
			\draw (2.3,0.)-- (3.7,0.);
			\draw (4.3,0.)-- (5.7,0.);
			\draw (-4.3,0.)-- (-5.7,0.);
			\draw [->] (0.65,1.5) -- (0.65,3.);
			\draw [->] (0.65,1.5) -- (0.65,0.);
			\draw [->] (0.65,1.5) -- (0.65,3.);
			\draw [dotted] (0.3,3.)-- (0.65,3.);
			\draw [dotted] (1.,3.)-- (0.65,3.);
			\draw [->] (2.75,-0.5) -- (2.3,-0.5);
			\draw [->,line width=0.4pt] (1.25,-0.5) -- (1.7,-0.5);
			\draw [dotted] (1.7,0.)-- (1.7,-1.);
			\draw [dotted] (2.3,0.)-- (2.3,-0.5);
			\draw [dotted] (2.3,-0.5)-- (2.3,-1.);
			\draw (1.8,-0.05) node[anchor=north west] {$b$};
			\draw (0.76,1.9) node[anchor=north west] {$V_{0}$};
			\draw (0.75,1.92) node[anchor=north west] {$V_{0}$};
			\draw (0.1,4.25) node[anchor=north west,color=lightgray] {$V(x)$};
			\draw (5,-0.1) node[anchor=north west,color=lightgray] {$x$};
		  \end{tikzpicture}
	    }
	  \caption{Potential $V(x)$ of the Kronig-Penney model} \label{figure-kpm} 
	\end{figure}
~\\

In this thesis, we are interested in the spectrum of the operator describing the situation in the Kronig-Penney model when the particle moves through periodically distributed, singular potentials. With respect to the above this means taking the limit $b \rightarrow 0$ while $V_{0}$ remains of order $\rho b^{-1}$. Therefore, we will extend the research in \cite{dorfler2011photonic} where the spectrum of periodic differential operators with smooth coefficients was analysed. We will show that an operator modelling the aforementioned situation has, as in the case of smooth coefficients, a spectrum that consists of a union of compact intervals in $\R$ which form a so-called spectral band. 
~\\

In a physical sense, the spectral bands represent energy levels. Only electron with a energy level within the spectral band can exists inside the crystal. Hence, the possible areas between the intervals, if they exist, represent a forbidden energy range. As a result of Bragg's law, standing waves form on the boundary between the spectral band and the forbidden energy levels. The closer now the electrons accumulate to the nucleus of the relative ion the energetically more favourable are the standing waves which is a desired state by the electrons, see for a detailed explanation \cite[section 3.2]{heering2002elektrophysik}. Therefore, the Bragg's law can be considered as cause of the forbidden energy levels. Because of this causality, for the knowing possible energetically levels within periodic crystal lattices we need knowledge about the corresponding spectral properties.
~\\

The remainder of this thesis is structured as follows. We begin with some preliminaries in Chapter 2 to review some concepts from functional analyses and spectral theory. In Chapter 3, we will examine the operator and show that its self-adjointness, which is our first step in analysing its spectrum. The main mathematical tool for analysing the spectrum of such an operator is the so-called Floquet–Bloch theory. We will transfer the spectral problem the operator on the whole of $\R$ to a family of eigenvalue problems on the periodicity cell. Hence, we proceed in Chapter 4 by restricting the problem to its fundamental domain of periodicity. There we are able to analyse the spectrum of the restricted operator by showing the compactness of its resolvent while varying quasi-periodic boundary conditions. In Chapter 5 we introduce two concepts to transfer the results from the restricted case into the unrestricted, namely the above mentioned Floquet transformation and the Bloch waves. Based on this methods, in Chapter 6 we are able to show the main result for the one-dimensional case, i.e. such an operator has a spectrum consisting of a union of compact intervals in $\R$, and extrapolate this result to the multi-dimensional case in Chapter 7. We close this thesis in Chapter 8, where we discuss possible gaps between the compact intervals and list current researches.