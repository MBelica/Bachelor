\chapter{Introduction} \label{chap1}

The problem considered in this thesis arises from the Kronig-Penney model, see for example \cite[Chap. 3]{HeeringEP}, which describes an idealised quantum-mechanical system that models a quantum particle behaving as a matter wave moving in one-dimension through an infinite periodic array of rectangular potential barriers, i.e. through a space area in which a potential attains a local maximum. Such an array commonly occurs in models of periodic crystal lattices where the potential is caused by ions in the crystal structure. Those charged molecules create an electromagnetic field around themselves. Hence, any particle moving through such a crystal would be subject to a recurrent electromagnetic potential. Although a solid particle, simplified as a point mass, would be reflected at such a barrier, there is a possibility that the quantum particle, as it behaves like a wave, penetrates the barrier and continues its movement beyond. Assuming the spacing between all ions is equidistant the potential function $V(x)$ in the lattice can be approximated by a rectangular potential like this:

\begin{figure}[!ht] \centering
	\resizebox{.55\linewidth}{!}{
		\definecolor{wqwqwq}{rgb}{0.3764705882352941,0.3764705882352941,0.3764705882352941}
		\begin{tikzpicture}[line cap=round,line join=round,>=triangle 45,x=1.0cm,y=1.0cm]
			\draw[->,color=wqwqwq] (-5.5,0.) -- (5.5,0.);
			\foreach \x in {-5.,-4.,-3.,-2.,-1.,1.,2.,3.,4.,5.}
				\draw[shift={(\x,0)},color=wqwqwq] (0pt,-2pt);
			\draw[->,color=wqwqwq] (0.,-1.1) -- (0.,4.25);
			\clip(-5.5,-1.1) rectangle (5.5,4.25);
			\draw (-0.3,3.)-- (0.3,3.);
			\draw (0.3,0.)-- (0.3,3.);
\draw (-0.3,3.)-- (-0.3,0.);
\draw (-1.7,3.)-- (-1.7,0.);
\draw (-2.3,3.)-- (-1.7,3.);
\draw (-2.3,3.)-- (-2.3,0.);
\draw (-1.7,3.)-- (-1.7,0.);
\draw (-3.7,0.)-- (-3.7,3.);
\draw (-3.7,3.)-- (-4.3,3.);
\draw (-4.3,3.)-- (-4.3,0.);
\draw (-3.7,3.)-- (-3.7,0.);
\draw (-3.7,3.)-- (-4.3,3.);
\draw (1.7,3.)-- (1.7,0.);
\draw (1.7,3.)-- (2.3,3.);
\draw (2.3,3.)-- (2.3,0.);
\draw (3.7,0.)-- (3.7,3.);
\draw (3.7,3.)-- (4.3,3.);
\draw (4.3,3.)-- (4.3,0.);
\draw (-3.7,0.)-- (-2.3,0.);
\draw (-1.7,0.)-- (-0.3,0.);
\draw (0.3,0.)-- (1.7,0.);
\draw (2.3,0.)-- (3.7,0.);
\draw (4.3,0.)-- (5.7,0.);
\draw (-4.3,0.)-- (-5.7,0.);
\draw [->] (0.65,1.5) -- (0.65,3.);
\draw [->] (0.65,1.5) -- (0.65,0.);
\draw [->] (0.65,1.5) -- (0.65,3.);
\draw [dotted] (0.3,3.)-- (0.65,3.);
\draw [dotted] (1.,3.)-- (0.65,3.);
\draw [->] (2.75,-0.5) -- (2.3,-0.5);
\draw [->,line width=0.4pt] (1.25,-0.5) -- (1.7,-0.5);
\draw [dotted] (1.7,0.)-- (1.7,-1.);
\draw [dotted] (2.3,0.)-- (2.3,-0.5);
\draw [dotted] (2.3,-0.5)-- (2.3,-1.);
\draw (1.8,-0.05) node[anchor=north west] {$b$};
<<<<<<< HEAD
\draw (0.76,1.9) node[anchor=north west] {$V_{0}$};
=======
\draw (0.75,1.92) node[anchor=north west] {$V_{0}$};
>>>>>>> origin/master
\draw (0.1,4.25) node[anchor=north west,color=wqwqwq] {$V(x)$};
\draw (5,-0.1) node[anchor=north west,color=wqwqwq] {$x$};
\end{tikzpicture}
	}
	\caption{Potential $V(x)$ of the Kronig-Penney model}
\end{figure}

where $b$ is the \enquote{support} and $\rho$ the magnitude of the potential. We are interested in the spectrum of the operator describing the situation of the Kronig-Penney model when the particle moves through periodically distributed, singular potentials. With respect to the above this means taking the limit $b \rightarrow 0$ while $V_{0}$ remains of order $\rho b^{-1}$. 

% Missing: Research questions; Missing: Andeuten der Main results; Missing: The remainder of this thesis is structerud as follows. 

\chapter{Preliminaries}

For the upcoming analysis some basic concepts from functional analysis and spectral theory are here briefly reviewed:
~\newline ~\newline
Let $C_{0}^{\infty}$ denote the linear space containing all smooth function $f \colon \R \rightarrow \R$ with compact support, i.e. for $f \in C_{0}^{\infty}$ there exists a compact interval $I \subseteq \R$ such that $f(x) = 0$ for all $x \notin I$. And hereafter $\langle x, x \rangle$ will denote the scalar product in $L^{2}(\R)$.
\begin{definition}[Weak derivative]
Let $\Omega \subseteq \R$ be open, and $f \in L^{1}_{loc}(\Omega)$. The function $f$ is said to have the weak derivative $g \in L^{1}_{loc}(\Omega)$ in $\Omega$ if
  \[ - \int_{\Omega} f \varphi' = \int_{\Omega} g \varphi \]
holds for all $\varphi \in C_{0}^{\infty}(\Omega)$.
\end{definition}
Let $\alpha \in \N$, we denote with $D^{\alpha} u$ the $\alpha$-th weak derivate of $u$. Therewith, if two functions are weak derivatives of the same function they are equal except on a set with Lebesgue measure zero, i.e. they are equal almost everywhere. A central point in this study will be a special Hilbert space the Sobolev space $H^{k}(\Omega)$. % todo proof uniqueness evans 243
\begin{definition} Let us define the space
\[ H^{k}(\Omega) \coloneqq \left\{ u \in L^{2}(\Omega) : D^{\alpha} u \text{ exists and } D^{\alpha} u \in L^{2}(\Omega) \text{ for } 0 \leq \alpha \leq k \right\} \]
and equipped with the norm $\| \cdot \|_{H^{k}(\Omega)} \coloneqq \left( \sum_{0 \leq \alpha \leq k} \| D^{\alpha} \cdot \|_{L^{2}(\Omega)}^{2} \right)^{\frac{1}{2}}$ we call it Sobolev space.
\end{definition}
By admitting the inner product in terms of the $L^{2}(\Omega)$ inner product for all derivatives up to order $k$: 
	\[ \langle u , v \rangle_{H^{k}(\Omega)} = \sum_{\alpha=0}^{k} \left\langle D^{\alpha}u , D^{\alpha} v \right\rangle_{L^{2}(\Omega)}. \] 	

\begin{definition}[Distributions]
	On $C_{0}^{\infty}$ a sequence $(f_{n})$ converges to $f \in C_{0}^{\infty}$ if the support of all members of the sequence is in a compact interval $I \subset \R$, i.e.
	$$ \supp (f_{n}) \subseteq I \quad \forall n \in \N, $$
	and on this interval $f_{n}$ and all its derivatives converge uniformly to $f$, i.e.
	\[ \| f_{n}^{(i)} - f^{(i)} \|_{\infty} \rightarrow 0 \quad \text{ for } n \rightarrow \infty \]
	for all $i \in \N_{0}$. One can proof that this concept of convergence generates a topology on $C_{0}^{\infty}$ and one usually denoted with $\mathfrak{D}(\R)$ the space $C_{0}^{\infty}$ equipped with this topology. % todo Peter muss ich die topology wirklich erklären?
\end{definition}

From now on in the remainder of this thesis, we denote with $\mathfrak{D}'(\R)$ the space of all linear functionals on $C_{0}^{\infty}$ that are continuous with respect to this topology and call those functionals distributions. 
~\\

An important example for a distribution is the Dirac delta function $\delta_{x_{0}}$ where $x_{0} \in \R$. It is defined as the weak limit of a weakly converging sequence of functionals over normed symmetric around $x_{0}$ cumulative distribution functions $\delta_{\epsilon}$, where the support of those cumulative distributions converges to zero. It holds $\delta_{x_{0}} = \lim_{\epsilon \rightarrow 0} \delta_{\epsilon}$ in $\mathfrak{D}'(\R)$. An example for such a sequence is % todo absatz blöd

	\begin{equation}
		\delta_{\epsilon}(x) = \frac{1}{\sqrt{2 \pi} \epsilon} e^{-\frac{x^{2}}{2 \epsilon^{2}}}. \label{smooth-potential}
	\end{equation}
	 
Which implies the definition

	\[ \delta_{x_{0}}(f) \coloneqq \int_{\R} \delta_{x_{0}} f(x) dx \coloneqq \lim_{\epsilon \rightarrow 0} \int_{\R} \delta_{\epsilon}(x - x_{0}) f(x) dx. \]
	
Moreover, we can check that $\delta_{x_{0}}(f) = \lim_{\epsilon \rightarrow 0} \delta_{\epsilon}(f) = f(x_{0})$, for a proof see \cite[Chap. 1.4]{WeisST}.

\begin{definition} % todo text zwischen theoremen
Let $X, Y$ be Banach spaces and let $A \colon \mathcal{D}(A) \rightarrow Y$ be a linear operator with domain $\mathcal{D}(A) \subseteq X$. 
	\begin{enumerate}[label=\alph*\upshape)]
		\item We call $A$ closed if $graph(A) \coloneqq \{ (x, Ax) : x \in \mathcal{D}(A) \} \subseteq X \times Y$ is a closed set with respect to the product topology.
		\item The operator $A^{*} : \mathcal{D}(A^{*}) \rightarrow H $ is called the adjoint of $A$ and is defined by
		\[ \mathcal{D}(A^{*}) \coloneqq \{ u \in H : \exists u^{*} \in H ~\forall v \in \mathcal{D}(A) \langle u, A v \rangle = \langle u^{*} , v \rangle \} \]
		and $A^{*} u \coloneqq u^{*}$ for $u \in \mathcal{D}(A^{*})$. Note that for $u \in \mathcal{D}(A^{*})$, $u^{*}$ is uniquely determined. 
	\end{enumerate}
\end{definition}

\begin{definition}
Let $X$ be a Hilbert space, $\langle \cdot, \cdot \rangle$ denotes the scalar product on $X$ and $A$ a bounded operator. We call
	\begin{enumerate}[label=\alph*\upshape)]
		\item $A$ symmetric, if $\langle Tx,y \rangle = \langle x ,Ty \rangle$ for all $x,y \in \mathcal{D}(A)$, and
		\item $A$ self-adjoint, if $A$ is densely defined on $X$ and coincides with its adjoint.
	\end{enumerate}
	
\end{definition}

\begin{definition}
Furthermore, let $I$ denote the identity operator on $X$ and $A$ be a linear, bounded and closed operator.
	\begin{enumerate}[label=\alph*\upshape)]
		\item $\lambda \in \C$ belongs in the resolvent set of $A$, $\lambda \in \rho(A)$, if
			\[  A  - \lambda I \colon \mathcal{D}(A) \rightarrow X \text{ bijective, i.e. } (A - \lambda I)^{-1} \colon X \rightarrow \mathcal{D}(A) \text{ is a bounded linear operator,} \]
		\item $\sigma(A) = \C \setminus \rho(A)$ is called the spectrum of $A$, and
		\item $\lambda \in \rho(A) \rightarrow R(\lambda, A) = (A - \lambda I)^{-1}$ is the resolvent function of $A$.
	\end{enumerate}		
\end{definition}

Finally, in chapters \ref{chap3} and \ref{chap5} we will examine so called compact operators and some of their properties.

\begin{definition}
	Let $X$ be a normed space and $Y$ a Banach space. A linear operator $A \colon X \rightarrow Y$ is called compact, if $T(U_{X})$ is relativ compact in $Y$.
\end{definition}

Additionally, throughout this thesis we will need some theorems and lemmata from functional analysis and spectral theory, which will be listed in the appendix.