\chapter{Introduction}

The problem considered in this thesis arises from the Kronig-Penney model which describes an idealised quantum-mechanical system, that demonstrates a particle behaving as a matter wave moving in one-dimension trough an infinite periodic array of rectangular potential barriers, i.e. through a space area in which a potential attains a local maximum. Such an array commonly occurs in models of periodic crystal lattices where the potential is caused by ions in the crystal. They create an electromagnetic field around themselves and hence any particle moving through such a crystal would be subject to a periodic electromagnetic potential. Although a solid particle, simplified as a point mass, would be reflected at such a barrier, there is a possibility that the quantum particle, as it behaves like a wave, penetrates the barrier and continues its movement beyond. Assuming the spacing between all ions is $a$, the potential function $V(x)$ in the lattice can be approximated by a rectangular potential like this:

\begin{figure*}[h!] \centering
	  \includegraphics[width=0.75\textwidth]{Periodic_square_potential_130707} 
\end{figure*}

where $b$ is the \enquote{support} and $\rho$ the magnitude of the potential.
~\newline ~\newline
This thesis will examine the spectrum of an operator describing a special case of the Kronig-Penney model, namely by taking the limit $b \rightarrow 0$ while $V_{0}$ remains in order $\rho b$, which represents the ion creating a finite singular potential.

\section*{Mathematical Basics}

For the upcoming analysis we need some basic concepts from functional analysis and spectral theory I want to briefly recapitulate:
~\newline ~\newline
Let $C_{c}^{\infty}$ denote the linear space containing all smooth function $f \colon \R \rightarrow \R$ with compact support, i.e. for $f \in C_{0}^{\infty}$ there exists a compact interval $I \subseteq \R$ such that $f(x) = 0$ for all $x \notin I$. From now on we will denote with $\langle x, x \rangle$ the scalar product in $L^{2}(\R)$.
\subsection*{The Sobolev space $H^{k}(\Omega)$}
Let $\Omega \subseteq \R$ be an open set and let $u$ be a function in the Lebesgue space $L^{1}(\Omega)$. Then $v$ in $L^{1}(\Omega)$ is a weak derivative of $u$ if, 
	\[ \int_{\Omega} u(t)\varphi'(t) dt = -\int_{\Omega} v(t) \varphi(t) dt \]
for all $\varphi \in C_{0}^{\infty}(\Omega)$. Now, an important example for a Hilbert space is the Sobolev space $H^{k}(\Omega)$, which is defined to be the set of functions $f$ in $L^{2}(\Omega)$ such that the function $f$ and its weak derivatives up to the order $k$ have a finite $L^{2}(\Omega)$ norm, by admitting the inner product in terms of the $L^{2}(\Omega)$ inner product for all derivatives up to order $k$: 
	\[ \langle u , v \rangle_{H^{k}(\Omega)} = \sum_{i=0}^{k} \left\langle D^{i}u , D^{i} v \right\rangle_{L^{2}(\Omega)}. \] 	
\subsection*{Distributions}
	On $C_{0}^{\infty}$ a sequence $(f_{n})$ converges against $f \in C_{0}^{\infty}$ if the support of all members of the sequence is in a compact interval $I \subset \R$, i.e.
	$$ \supp (f_{n}) \subseteq I \quad \forall n \in \N, $$
	and on this interval $f_{n}$ and all its derivatives converge uniformly against $f$, i.e.
	\[ \| f_{n}^{(i)} - f^{(i)} \|_{\infty} \rightarrow 0 \quad \text{ for } n \rightarrow \infty \]
	for all $i \in \N_{0}$. One can proof that this concept of convergence generates a topology on $C_{0}^{\infty}$ and one usually denoted with $\mathcal{D}(\R)$ the space $C_{0}^{\infty}$ equipped with this topology. From now on, we denote with $D'(\R)$ the space of all linear functionals on $C_{0}^{\infty}$ that are continuous with respect to this topology and call those functionals distributions.
~\newline ~\newline
An important example for a distribution is the Dirac delta function $\delta_{x_{0}}$ where $x_{0} \in \R$. It is defined as the limit of a weakly converging sequence of functionals over normed symmetric around $x_{0}$ cumulative distribution functions $\delta_{\epsilon}$, whereas the support of those cumulative distributions converges to zero. It holds $\delta_{x_{0}} = \lim_{\epsilon \rightarrow 0} \delta_{\epsilon}$ in $D'(\R)$. An example for such a sequence is be
	\[ \delta_{\epsilon}(x) = \frac{1}{\sqrt{2 \pi} \epsilon} e^{-\frac{x^{2}}{2 \epsilon^{2}}}. \]
Which implies the definition
	\[ \delta_{x_{0}}(f) \coloneqq \int_{-\infty}^{\infty} \delta_{x_{0}} f(x) dx \coloneqq \lim_{\epsilon \rightarrow 0} \int_{-\infty}^{\infty} \delta_{\epsilon}(x - x_{0}) f(x) dx. \]
Moreover, is easily seen that $\delta_{x_{0}}(f) = \lim_{\epsilon \rightarrow 0} \delta_{\epsilon}(f) = f(x_{0})$.
\subsection*{Spectrum and resolvent of an operator}
Let $X, Y$ be Banach spaces and let $A \colon \mathcal{D}(A) \rightarrow Y$ be a linear operator with domain $\mathcal{D}(A) \supset X$. On the set of all linear operators we define the operator norm 
	\[ \| A \| \coloneqq \sup_{x \in \mathcal{D}(A)} \left\{ \| A x \|_{Y} \colon \| x \|_{X} \leq 1 \right\}, \]
and call an operator bounded if its operator norm is bounded. We call a bounded operator $A$ symmetric, if % todo I think this definition is wrong
	\[  \langle Tx,y \rangle = \langle x ,Ty \rangle \]
for all $x,y \in \mathcal{D}(A)$, and self-adjoint, if $T$ coincides with its adjoint, $T^{*} = T$, i.e. 
	\[ \langle Tx,y \rangle = \langle x ,Ty \rangle \]
for all $x,y \in X$. Furthermore, let $I$ denote the identity operator on $X$. Then we define for any $\lambda \in \C$
	\begin{enumerate}[label=\alph*\upshape)]
		\item $\lambda$ belongs in the resolvent set of $A$, $\lambda \in \rho(A)$, if and only if
			\[ \lambda I - A \colon \mathcal{D}(A) \rightarrow X \text{ bijective, i.e. } (\lambda I - A)^{-1} \colon X \rightarrow \mathcal{D}(A) \text{ is a bounded linear operator.} \]
		\item $\sigma(A) = \C \setminus \rho(A)$ is called spectrum of $A$.
		\item $\lambda \in \rho(A) \rightarrow R(\lambda, A) = (\lambda - A)^{-1}$ is the resolvent function of $A$.
	\end{enumerate}		

\begin{theorem}
	The resolvent set $\rho(A) \subseteq \mathbb{C}$ of a bounded linear operator $A$ is an open set.
	
	\begin{proof}
		First, we note that the resolvent set is bounded as for $|\lambda| > \|A\|$ then $\| \lambda^{-1} A \| < 1$ and the operator $A - \lambda I = -\lambda (I - \lambda^{-1} A)$ has by the Neumann series the inverse
		$$ R(\lambda, A) = (A - \lambda I)^{-1} = - \sum_{k=0}^{\infty} \lambda^{-k-1} A^{k}. $$
		Now, to show that $\rho(A)$ is open we have proceed by showing that for any $\lambda \in \rho(A)$ there exist $\epsilon > 0$ such that all $\mu$ with $|\lambda- \mu| < \epsilon$ are also in $\rho(A)$. For that consider
		\begin{align*}
			A - \mu I & = A - \lambda I + (\lambda - \mu) I \\
					  & = (A - \lambda I)\left(I + (\lambda - \mu) (A - \lambda I)^{-1} \right).
		\end{align*}
		The last expression is an invertible operator because $T - \lambda I$ is invertible by the assumption and $I + (\lambda - \mu)(T - \lambda I)^{-1}$ is invertible again by the Neumann series, since $\|(\lambda - \mu)(A - \lambda I)^{-1}\| < 1$ if $\epsilon < \|(A - \lambda I)^{-1}\|$.
	\end{proof}
\end{theorem}

