\chapter{The multi-dimensional case} \label{chap5}

In this last chapter, we are going to examine a more complex situation. We now want to model the movement of a particle in $\R^{n}$ with periodically distributed, smooth, $(n-1)$-dimensional surfaces supporting a potential. To show that the basic concepts presented in the previous chapters hold also in the new setting, we will merely give a formal justification of applicability of the one-dimensional proofs to the multi-dimensional case in a series of theorems, as below.
~\newline ~\newline
To start with, let $Y$ denote a periodicity cell in $\R^{n}$ and $B^{n}$ the corresponding Brillouin zone, for simplicity assume $Y$ being the unit cube $Y = [0, 1]^{n}$ and hence $B = [-\pi, \pi]^{n}$. Contained in $Y$ let $S$ be a smooth surface without a boundary with the conditions $\dim S = n - 1$ and $S \subseteq \overset{\circ}{Y}$. Furthermore, let $B \subseteq Y$ denote the set enclosed by $S$, such that $S = \partial B$. We will denote for any $j \in \Z^{n}$ with $Y_{j} = Y + j$ the $j$th copy of $Y$, which results through translation of the periodicity cell by $j$, and analogously for $S_{j} = S + j$ and $B_{j} = B + j$. Furthermore, we denote with $S_{i}^{+}$ and $S_{i}^{-}$ the opposing edges of $Y$ for $i = 1, 2$.

\begin{figure}[!ht] \centering
	\resizebox{.35\linewidth}{!}{
	  \begin{tikzpicture}[line cap=round,line join=round,>=triangle 45,x=4.285714285714286cm,y=4.285714285714286cm]
	  \clip(-1.2,-1.2) rectangle (1.2,1.3);
	  \fill[transparent,line width=0.4pt] (-1.,1.) -- (-1.,-1.) -- (1.,-1.) -- (1.,1.) -- cycle;
	  \draw [line width=0.4pt] (-1.,1.)-- (-1.,-1.);
	  \draw [line width=0.4pt] (-1.,-1.)-- (1.,-1.);
	  \draw [line width=0.4pt] (1.,-1.)-- (1.,1.);
	  \draw [line width=0.4pt] (1.,1.)-- (-1.,1.);
	  \draw [rotate around={-166.60970691015552:(-0.017884043865981074,0.012185928443575623)}] (-0.017884043865981074,0.012185928443575623) ellipse (2.915306454226339cm and 1.8911924645535707cm);
	  \draw ( 0.18, 0.35) node[anchor=north west] {$S$};
	  \draw (-0.85,-0.56) node[anchor=north west] {$Y$};
	  \draw (-1.2,0.125) node[anchor=north west] {$S_1^-$};
	  \draw (1.05,0.125) node[anchor=north west] {$S_1^+$};
	  \draw (-0.125,-1.05) node[anchor=north west] {$S_2^-$};
	  \draw (-0.125, 1.25) node[anchor=north west] {$S_2^+$};
	  \end{tikzpicture}
	}
	\caption{Periodicity cell for the multi-dimensional potential}
\end{figure}

The mathematical representation of the above is a multi-dimensional Schrödinger operator $A^{n}$ whose operation is formally defined by
\begin{equation}
	- \Delta + \rho \sum_{i \in \Z} \delta_{S_{i}} \label{md-the-operator-A-formally}
\end{equation}
on the whole of $\R^{n}$, where $\delta_{S_{i}}$ denotes the Dirac delta distribution on hypersurface $S_{i}$. that integrates any function $u \in L^{2}(\R)$ over the compact set $S_{i}$.  % todo rewrite this, maybe at the very start with direct delta functions get the definitions
~\\ ~\\ 
Again motivated by the weak-formulation, given a right-hand side $f \in L^{2}(\R^{n})$ we consider for some $\mu \in \R$ the problem to find $u \in H^{1}(\R^{n})$ such that
	\begin{equation}
		\int_{\R^{n}} \nabla u \overline{\nabla v} + \rho \sum_{i \in \Z^{n}} \int_{S_{j}} u \overline{v} ds - \mu \int_{\R^{n}} u \overline{v} = \int_{\R^{n}} f \overline{v} \label{md-weak-formulation}
	\end{equation} 
holds for all $v \in H^{1}(\R^{n})$, where $s$ is the hypersurface measure associated to all $S_{j}$. 

\begin{remark}
	The term originating from the potential is finite as
	\[ \left| \sum_{j \in \Z^{n}} \int_{S_{j}} u \overline{v} \right|^{2} \leq \left( \sum_{j \in \Z^{n}} \| u \|_{L^{2}(S_{j})}^{2} \right) \left( \sum_{j \in \Z^{n}} \| v \|_{L^{2}(S_{j})}^{2} \right). \] % TODO CHECK THIS
	Both terms on the right-hand side can be finitely estimated by the Trace theorem \cite[Chap. 5]{Evans98} as
	\[ \| u \|_{L^{2}(S_{j})}^{2} \leq 2 \left( \frac{1}{h} \|u\|_{L^{2}(B_{j})}^{2} + h \| \nabla u \|_{L^{2}(B_{j})}^{2} \right) \]
	for some $h > 0$. % todo more explicit and check thiS
\end{remark}

We can once again exert Lax-Migram's Theorem to prove in \eqref{md-weak-formulation} the existence of a unique solution $u \in H^{1}(\R^{n})$ in \eqref{md-weak-formulation} for any $f \in L^{2}(\R^{n})$ if $\mu$ is small enough, define the operator injective $R_{\mu}^{n} \colon f \mapsto u$ and define $A^{n}$ by means of $R_{\mu}^{n}$.

\begin{remark}
	The operator $A^{n}$ is self-adjoint.	
\end{remark}

% todo check this and rewrite theorem with proof
\begin{theorem}[Characterisation of $\mathcal{D}(A^{n})$] Let $\Omega \coloneqq \R^{n} \setminus \overline{\bigcup_{j \in \Z^{n}} B_{j}}$. By choosing similar to above different functions $v \in C^{\infty}(\R^{n})$ in \eqref{md-weak-formulation} we can further characterise such $u \in \mathcal{D}(A^{n})$, namely for all $j \in \Z^{n}$ it holds:
	\begin{enumerate}
		\item $\Delta u \in L^{2}(B_{j})$, $\Delta u \in L^{2}(\Omega)$ and $\sum_{j \in \Z^{n}} \|\Delta u \|_{L^{2}(B_{j})}^{2} < \infty$
		\item $u \big|_{S_{j} - 0} = u \big|_{S_{j} + 0}$
		\item $\frac{\partial u}{\partial \eta_{j}} \big|_{S_{j} - 0} - \frac{\partial u}{\partial \eta_{j}} \big|_{S_{j} + 0} - \rho u \big|_{S_{j}} = 0$ where $\eta_{j}$ denotes the normal on $S_{j}$
	\end{enumerate}
\end{theorem}

\begin{figure}[!ht] \centering
	\resizebox{.35\linewidth}{!}{
	  \begin{tikzpicture}[line cap=round,line join=round,>=triangle 45,x=5.0cm,y=5.0cm]
		\clip(-1,-1) rectangle (1.1,1.1);
		\fill[transparent,line width=0.4pt] (-1.,1.) -- (-1.,-1.) -- (1.,-1.) -- (1.,1.) -- cycle;
		\draw [line width=0.4pt] (-1.,1.)-- (-1.,-1.);
		\draw [line width=0.4pt] (-1.,-1.)-- (1.,-1.);
		\draw [line width=0.4pt] (1.,-1.)-- (1.,1.);
		\draw [line width=0.4pt] (1.,1.)-- (-1.,1.);
		\draw [rotate around={-166.60970691015552:(-0.017884043865981074,0.012185928443575623)}] (-0.017884043865981074,0.012185928443575623) ellipse (3.4011908632640617cm and 2.2063912086458326cm);
		\draw (0.17773598812780403,0.35749896759640476) node[anchor=north west] {$S$};
		\draw (-0.8520310518542072,-0.5460385642587798) node[anchor=north west] {$Y$};
		\draw [->] (0.14155897886779217,0.46905768120862523) -- (0.20343212114361697,0.7943332288652334);
		\draw (0.20431062141766237,0.7694057835892095) node[anchor=north west] {$\eta$};
		\draw (-0.014930103223669057,-0.24043028142540857) node[anchor=north west] {$-$};
		\draw (0.15116135483794568,-0.4928892976790631) node[anchor=north west] {$+$};
	  \end{tikzpicture}
	}
	\caption{Normal $\eta$ on the hypersurface $S$ in a periodicity cell}
\end{figure}

Now, restricting again this problem to a fundamental domain of periodicity, for example $Y$,
% todo todo
We consider, analogous to before, the problem to find 
	\[ u \in H^{1}_{k, n} \coloneqq \left\{ w \in H^{1}(Y) \colon w \big|_{S_{j}^{-}} = w \big|_{S_{j}^{+}} e^{i k_{j}} \text{ for } k \in [-\pi, \pi]^{2}, j = 1,2 \right\} \]
such that
	\begin{equation}
		\int_{Y} \nabla u \overline{\nabla v} + \rho \int_{S} u \overline{v} ds - \mu \int_{Y} u \overline{v} = \int_{Y} f \overline{v} \label{md-weak-formulation-res}
	\end{equation} 
for all $v \in H^{1}_{k, n}$. 
Again exerting Lax-Milgram's Theorem the existence of a unique solution $u \in H^{1}_{k, n}$ is ensured if $\mu$ is small enough, and the operator $R_{\mu, k} \colon f \mapsto u$ is in return well-defined and injective. This allows us to define 
	\[ A_{k}^{n} \coloneqq R_{\mu, k}^{n} + \mu I. \]
The semi-periodic boundary conditions on $H^{1}_{k,n}$ require a solution $u \in H^{1}_{k, n}$ to \eqref{md-weak-formulation-res} to satisfy furthermore
	\[ \frac{\partial u}{\partial x_{j}}\big|_{S_{j}^{-}} = e^{ik_{j}} \frac{\partial u}{\partial x_{j}}\big|_{S_{j}^{+}} \quad \text{for } j = 1, 2.  \] % todo check this
	
\begin{theorem}
	The operator $R_{\mu, k}^{n}$ is compact.	

	\begin{proof}
		As in chapter \ref{chap3} used, the compact embedding yields the desired result, for a multi-dimensional proof see \cite[Chap. 4]{Adams}.	
	\end{proof}
\end{theorem}

By similar transformations of the problem \eqref{md-weak-formulation-res} as in \eqref{mod-eigv-problem} and \eqref{periodic-condition} we are able to show that the eigenvalues of $A^{n}_{k}$ are continuous functions of $k \in \overline{B}$, and thus $I_{s} = \{ \lambda_{s}(k) : k \in \overline{B} \}$ is a compact real interval for each $s \in \N$. % todo explain Is more detailed
~\newline ~\newline
Ultimately, using the same arguments as in Chapter 4 based on Bloch waves, Floquet transform and a similar cut-off function $\eta$ as in \eqref{eta}, we are then able to prove the main result even for this multi-dimensional case, namely that the spectrum of a self-adjoint Schrödinger operator with periodic delta-potential on a hypersurface is the union of compact intervals, i.e.
	\[ \sigma(A^{n}) = \bigcup_{s \in \N} I_{s}. \]

\section{Outlook}	
	
We have to notice that we haven't determined the nature of the spectrum that is if there are possible gaps in the spectrum. However,... % todo todo schluss