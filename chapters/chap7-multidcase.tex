\chapter{The spectrum of the multi-dimensional Schrödinger operator}  \label{chap:7} 

In this last chapter, we want to model the movement of a particle in $\R^{n}$ with periodically distributed, smooth, $(n-1)$-dimensional surfaces supporting a potential. To show that the basic concepts presented in the previous chapters hold also in the new setting, we will give a formal justification of applicability of the one-dimensional proofs to the multi-dimensional case in a series of theorems.
~\\

To start with, let $\Omega$ denote the periodicity cell in $\R^{n}$ relating to the problem introduced above and $B^{n}$ the corresponding Brillouin zone, for simplicity assume $\Omega$ being the unit cube $[0, 1]^{n}$. Contained in $\Omega$ let $S$ be a smooth surface without a boundary subject to the conditions $\dim S = n - 1$ and $S \subseteq \overset{\circ}{\Omega}$. Furthermore, let $D \subseteq Y$ denote the set enclosed by $S$, such that $S = \partial D$. We will denote with $\Omega_{j} = Y + j$ the $j$th copy of $\Omega$ for any $j \in \Z^{n}$, which results through translation of the periodicity cell by $j$, and analogously for $S_{j} = S + j$ and $S_{j} = S + j$. Finally, we denote with $\Omega_{i}^{+}$ and $\Omega_{i}^{-}$ the opposing edges of $\Omega$ for $i = 1, 2$, as illustrated in Figure \ref{fig:7.1}.
~\\

The mathematical representation of the above is a multi-dimensional Schrödinger operator $A^{n}$ whose operation is formally defined by
\begin{equation}
	- \Delta + \rho \sum_{i \in \Z} \delta_{S_{i}} \label{eq:7.1-md_the_operator_A_formallyy}
\end{equation}
on the whole of $\R^{n}$, where $\delta_{S_{i}}$ denotes the Delta-Distribution on hypersurface $S_{i}$. Let us recall that on a hypersurface $S$ the Delta-Distribution acts on $\varphi \in C_{0}^{\infty}(\R^{n})$ by 
	\[ \delta_{S}\left(\varphi\right) \coloneqq \int_{S} \varphi(s) ds . \]
where $s$ is the hypersurface measure associated to $S_{j}$, for a more detailed explanation see \cite[Chapter 14]{forster2012analysis}. Since $C_{0}^{\infty}(\R^{n})$ is dense in $H^{1}(\R^{n})$ we find for every $u \in H^{1}(\R^{n})$ a sequence $(u_{n})_{n \in \N} \in C_{0}^{\infty}(\R^{n})$ such that $\lim_{n \rightarrow \infty} u_{n} = u$, and hence we define
	\[ \delta_{S}\left(u\right) \coloneqq \lim_{n \rightarrow \infty} \delta_{S}\left(u_{n}\right) . \]
\begin{remark}
	This definition independent of the chosen sequence in $C_{0}^{\infty}(\R^{n})$.
\end{remark}

\begin{proof}
	For $u \in H^{1}(\R^{n})$ let $(u_{n})_{n \in \N}, (v_{n})_{n \in \N} \in C_{0}^{\infty}(\R^{n})$ such that
	\[ \lim_{n \rightarrow \infty} \| u_{n} - u \|_{H^{1}(\R^{n})} = 0 \quad \text{ and } \lim_{n \rightarrow \infty} \| v_{n} - u\|_{H^{1}(\R^{n})} = 0.  \]
	As $(u_{n} - v_{n}) \in C_{0}^{\infty}(\R^{n})$, we then get with the help of Hölder's inequality, the triangle inequality and the trace theorem
	\[ \lim_{n \rightarrow \infty} \int_{S} \left| u_{n}(s) - v_{n}(s) \right| ds \leq |S| \lim_{n \rightarrow \infty} \| u_{n} - v_{n} \|_{H^{1}(\Omega)} = 0. \] 
\end{proof}

\begin{figure}[!ht] \centering
	\resizebox{.5\linewidth}{!}{
	  \begin{tikzpicture}[line cap=round,line join=round,>=triangle 45,x=4.286cm,y=4.286cm]
	  \clip(-1.2,-1.2) rectangle (1.2,1.3);
	  \fill[transparent,line width=0.4pt] (-1.,1.) -- (-1.,-1.) -- (1.,-1.) -- (1.,1.) -- cycle;
	  \draw [line width=0.4pt] (-1.,1.)-- (-1.,-1.);
	  \draw [line width=0.4pt] (-1.,-1.)-- (1.,-1.);
	  \draw [line width=0.4pt] (1.,-1.)-- (1.,1.);
	  \draw [line width=0.4pt] (1.,1.)-- (-1.,1.);
	  \draw [rotate around={-166.61:(-0.018,0.012)}] (-0.0179,0.0122) ellipse (2.9153cm and 1.891cm);
	  \draw ( 0.2, 0.4) node[anchor=north west] {\large $S$};
	  \draw (-0.8,-0.6) node[anchor=north west] {\large $\Omega$};
	  \draw (-1.2,0.125) node[anchor=north west] {\large $\Omega_1^-$};
	  \draw (1.05,0.125) node[anchor=north west] {\large $\Omega_1^+$};
	  \draw (-0.125,-1.05) node[anchor=north west] {\large $\Omega_2^-$};
	  \draw (-0.125, 1.2) node[anchor=north west] {\large $\Omega_2^+$};
	  \end{tikzpicture}
	}
	\caption{Periodicity cell for the multi-dimensional potential} \label{fig:md-cell}  \label{fig:7.1}
\end{figure}

Again motivated by the weak-formulation, given a right-hand side $f \in L^{2}(\R^{n})$ we consider for some	 $\mu \in \R$ the problem to find $u \in H^{1}(\R^{n})$ such that
	\begin{equation}
		\int_{\R^{n}} \nabla u(x) \overline{\nabla v(x)} dx + \rho \sum_{i \in \Z^{n}} \int_{S_{j}} u(s) \overline{v(s)} ds - \mu \int_{\R^{n}} u(x) \overline{v(x)} dx = \int_{\R^{n}} f(x) \overline{v(x)} dx \label{eq:7.2-md-weak-formulation}
	\end{equation} 
holds for all $v \in H^{1}(\R^{n})$. Note that in the second term we denote by $u, v$ the traces. As $u, v \in H^{1}(\R)$, the traces they are in $L^{2}(\R^{n})$ by \cite[p. 251, Theorem 5.1]{evans1998partial} and \cite[p. 164]{adams2003sobolev}, and thus, the second term on the left-hand side in \eqref{eq:7.2-md-weak-formulation} is well defined by the next remark.

\begin{remark} 
	The term in \eqref{eq:7.2-md-weak-formulation} originating from the potential is finite.
	
	\begin{proof}
	 First, the Cauchy–Schwarz inequality yields
	\[ \left| \sum_{j \in \Z^{n}} \int_{S_{j}} u(s) \overline{v(s)} ds \right|^{2} \leq \left( \sum_{j \in \Z^{n}} \| u \|_{L^{2}(S_{j})}^{2} \right) \left( \sum_{j \in \Z^{n}} \| v \|_{L^{2}(S_{j})}^{2} \right). \] 
	Both terms on the right-hand side can then be finitely estimated by the Trace Theorem \cite[p. 258]{evans1998partial} and Poincaré inequality for some $h > 0$ through
	\[ \| u \|_{L^{2}(S_{j})}^{2} \leq c \left( \frac{1}{h} \|u\|_{L^{2}(D_{j})}^{2} + h \| \nabla u \|_{L^{2}(D_{j})}^{2} \right), \]	
	for some $c \in \R$.
	\end{proof}
\end{remark}

Given $f \in L^{2}(\R^{n})$, following the proves in Section \ref{sec:3.1} we can show that for a $\mu \in \R$ small enough the sesquilinear form $B_{\mu} \colon H^{1}(\R^{n}) \times H^{1}(\R^{n}) \rightarrow \C$, defined by
\[ B_{\mu}[u, v] \coloneqq \int_{\R^{n}} \nabla u(x) \overline{\nabla v(x)} dx + \rho \sum_{i \in \Z^{n}} \int_{S_{j}} u(s) \overline{v(s)} ds - \mu \int_{\R^{n}} u(x) \overline{v(x)} dx, \]
 is  bounded and coercive. Furthermore, the functional $l_{f} \colon H^{1}(\R) \rightarrow \C$ defined by
	\[ l_{f}(v) \coloneqq \int_{\R^{n}} f(x) \overline{v}(x) dx \]
is a bounded, linear functional. Hence, Lax-Migram's Theorem proves the existence of a unique solution $u \in H^{1}(\R^{n})$ in \eqref{eq:7.2-md-weak-formulation} for any $f \in L^{2}(\R^{n})$, and in return, the operator $R_{\mu}^{n} \colon L^{2}(\R^{n}) \rightarrow H^{1}(\R^{n}), f \mapsto u$ where $u \in H^{1}(\R^{n})$ is the solution of \eqref{eq:7.2-md-weak-formulation} is well-defined. 

\begin{theorem} 
	$R_{\mu}^{n}$ is an injective, bounded linear operator. 
	
	\begin{proof}
		The proof follows equally Theorem \ref{rmuinj}.
	\end{proof}
\end{theorem}

This on the other hand enables us to explicitly define $A^{n}$ by means of $R_{\mu}^{n}$:
\[ A^{n} \coloneqq \left(R_{\mu}^{n}\right)^{-1} + \mu I, \quad \mathcal{D}(A^{n}) = \mathcal{R}(R_{\mu}^{n}), \]

such that we can again characterise $\mathcal{D}(A^{n})$ explicitly by transferring the methods used in Section \ref{sec:3.2}.

\begin{theorem}[Characterisation of $\mathcal{D}(A^{n})$] Let $Y \coloneqq \R^{n} \setminus \overline{\bigcup_{j \in \Z^{n}} D_{j}}$. We can further characterise the solution $u \in \mathcal{D}(A^{n})$ from \eqref{eq:7.2-md-weak-formulation}, on the fundamental domain of periodicity as depicted in  Figure \ref{fig:7.2}, namely for all $j \in \Z^{n}$ it holds:
	\begin{enumerate} % todo Martin gehört hier nicht das nablaweg im1.?
		\item $\Delta u \in L^{2}(D_{j})$, $\Delta u \in L^{2}(Y)$ and $\sum_{j \in \Z^{n}} \|\Delta u \|_{L^{2}(D_{j})}^{2} < \infty$
		\item $u \big|_{S_{j} - 0} = u \big|_{S_{j} + 0}$
		\item $\frac{\partial u}{\partial \eta_{j}} \big|_{S_{j} - 0} - \frac{\partial u}{\partial \eta_{j}} \big|_{S_{j} + 0} - \rho u \big|_{S_{j}} = 0$ where $\eta_{j}$ denotes the normal on $S_{j}$
	\end{enumerate}
\end{theorem}

\begin{figure}[!ht] \centering
	\resizebox{.45\linewidth}{!}{
	  \begin{tikzpicture}[line cap=round,line join=round,>=triangle 45,x=5.0cm,y=5.0cm]
		\clip(-1,-1) rectangle (1.1,1.1);
		\fill[transparent,line width=0.4pt] (-1.,1.) -- (-1.,-1.) -- (1.,-1.) -- (1.,1.) -- cycle;
		\draw [line width=0.4pt] (-1.,1.)-- (-1.,-1.);
		\draw [line width=0.4pt] (-1.,-1.)-- (1.,-1.);
		\draw [line width=0.4pt] (1.,-1.)-- (1.,1.);
		\draw [line width=0.4pt] (1.,1.)-- (-1.,1.);
		\draw [rotate around={-166.6097:(-0.0179,0.0122)}] (-0.0179,0.0122) ellipse (3.4012cm and 2.2064cm);
		\draw (0.2,0.4) node[anchor=north west] {\large $S$};
		\draw (-0.8,-0.6) node[anchor=north west] {\large $\Omega$};
		\draw [->] (0.1416,0.469) -- (0.2034,0.7943);
		\draw (0.18,0.65) node[anchor=north west] {\large $\eta$};
		\draw (-0.015,-0.240) node[anchor=north west] {\large $-$};
		\draw (0.151,-0.475) node[anchor=north west] {\large $+$};
	  \end{tikzpicture}
	}
	\caption{Normal $\eta$ on the hypersurface $S$ in a periodicity cell}  \label{fig:7.2}
\end{figure}
	
\begin{proof}
 	In Section \ref{sec:3.2} we used particular functions $v \in C^{\infty}(\R)$ to prove equivalent properties of $\mathcal{D}(A)$. By using the same approach with similar functions for the multi-dimensional case in \eqref{eq:7.2-md-weak-formulation} the asserted follows. % todo expliziter
\end{proof}

We are interested in the periodic spectral problem of the operator $A^{n}$. Therefore, we will again relate the spectrum of the operator $A^{n}$ on the whole of $\R^{n}$ via the Floquet transform to a family of eigenvalue problems on the periodicity cell. First, we need the self-adjointness of the operator $A^{n}$.

\begin{remark}
	The operator $A^{n}$ is self-adjoint.	
\end{remark}

\begin{proof}
	The operators $R_{\mu}^{n}$ and $\left(R_{\mu}^{n}\right)^{-1}$ are symmetric. The fact that for any $u \in \mathcal{D}\left(\left(R_{\mu}^{n}\right)^{-1}\right)$ exists $f \in L^{2}(\R^{n})$ such that $u = R_{\mu}^{n} f$ and that $R_{\mu}^{n}$ is defined on the whole of $L^{2}(\R^{n})$ then yields the self-adjointness of $A^{n}$ similarly to Section \ref{sec:3.3}.
\end{proof}

We restrict this multi-dimensional problem to a corresponding fundamental domain of periodicity, let this for simplicity be $Y$ and for $k \in \overline{B^{n}}$ define 
\[ H^{1}_{k, n} \coloneqq \left\{ w \in H^{1}(Y) \colon w \big|_{Y_{j}^{+}} = w \big|_{Y_{j}^{-}} e^{i k_{j}} \text{ for } k \in [-\pi, \pi]^{2}, j = 1,2 \right\}. \]
 Let us consider the problem to find $u \in H^{1}_{k, n}$ such that
	\begin{equation}
		\int_{Y} \nabla u(x) \overline{\nabla v(x)} dx + \rho \int_{S} u(s) \overline{v(s)} ds - \mu \int_{Y} u(x) \overline{v(x)} dx = \int_{Y} f(x) \overline{v(x)} dx \label{eq:7.3-md_weak_formulation_res}
	\end{equation} 
holds for all $v \in H^{1}_{k, n}$. 
~\\ ~\\
Again, Lax-Milgram's Theorem ensures the existence of a unique solution $u \in H^{1}_{k, n}$ if $\mu \in \R$ is small enough, and the operator $R_{\mu, k} \colon f \mapsto u$ is in return well-defined, and we can show that $R_{\mu, k}$ is injective. This allows the definition 
	\[ A_{k}^{n} \coloneqq \left(R_{\mu, k}^{n}\right)^{-1} + \mu I, \quad \mathcal{D}(A_{k}^{n}) = \mathcal{R}(R_{\mu, k}^{n}), \]
as the operator considering multi-dimensional case of our problem on the fundamental cell of periodicity.
% todo k in Bn uberall einfügen
\begin{theorem}[Characterisation of $\mathcal{D}(A^{n}_{k})$] Let $Y \coloneqq \R^{n} \setminus \overline{\bigcup_{j \in \Z^{n}} D_{j}}$. We can characterise the solution $u \in \mathcal{D}(A^{n})$ from \eqref{eq:7.2-md-weak-formulation}, namely for all $j \in \Z^{n}$ it holds:
	\begin{enumerate} % todo Martin gehört hier nicht das nablaweg im1.?
		\item $\Delta u \in L^{2}(D_{j})$, $\Delta u \in L^{2}(Y)$ and $\sum_{j \in \Z^{n}} \|\Delta u \|_{L^{2}(B_{j})}^{2} < \infty$
		\item $u \big|_{S_{j} - 0} = u \big|_{S_{j} + 0}$
		\item $\frac{\partial u}{\partial \eta_{j}} \big|_{S_{j} - 0} - \frac{\partial u}{\partial \eta_{j}} \big|_{S_{j} + 0} - \rho u \big|_{S_{j}} = 0$ where $\eta_{j}$ denotes the normal on $S_{j}$
	\end{enumerate}
\end{theorem}
	
\begin{proof} % todo X 
 	In Section \ref{sec:3.2} we used particular functions $v \in C^{\infty}(\R)$ to prove equivalent properties of $\mathcal{D}(A)$. By using the same approach with similar functions for the multi-dimensional case in \eqref{eq:7.2-md-weak-formulation} the asserted follows. 
	The quasi-periodic boundary conditions on $H^{1}_{k,n}$ require a solution $u \in H^{1}_{k, n}$ of \eqref{eq:7.3-md_weak_formulation_res} to satisfy furthermore
	\[ \frac{\partial u}{\partial x_{j}}\big|_{\Omega_{j}^{+}} = e^{ik_{j}} \frac{\partial u}{\partial x_{j}}\big|_{\Omega_{j}^{-}} \quad \text{for } j = 1, 2.  \]	
\end{proof} \vspace{-0.75cm}

\begin{theorem}
	The operator $R_{\mu, k}^{n}$ is compact.	

	\begin{proof}
		As in Chapter \ref{chap:4}, we can show that $R_{\mu, k}^{n}$ is a bounded operator. Using the Compact Embedding Theorem for Sobolev spaces yields the claim, for proof see Theorem \ref{compact-embedding-theorem}.
	\end{proof}
\end{theorem}

Now, we will consider the eigenvalue problem to find $\psi \in \mathcal{D}(A^{n}_{k})$ such that
	\begin{equation}
		A^{n}_{k} \psi_{s} = \lambda^{n}_{s}(k) \psi_{s} \text{ on } Y. \label{md-eigv-problem}
	\end{equation}
We understand $\psi_{s}$ extended by the boundary condition on $\Omega$ in $H^{1}_{k, n}$ to the whole of $\R^{n}$ and call them Bloch waves. Using the compactness of $R_{\mu, k}^{n}$, we know on the one hand that every non-zero $\lambda \in \sigma(A_{k}^{n})$ is an eigenvalue of $A_{k}^{n}$ and that the purely discrete spectrum satisfies
	\begin{equation}
		\lambda^{n}_{1}(k) \leq \lambda^{n}_{2}(k) \leq \dotsc \leq \lambda^{n}_{s}(k) \rightarrow \infty \text{ as } s \rightarrow \infty. \label{md-comment-after}
	\end{equation}
By \cite[page 643 - 645]{evans1998partial} we know that the to \eqref{md-comment-after} corresponding eigenfunctions form a $\langle \cdot , \cdot \rangle$-orthonormal and complete system $(\psi_{s}(\cdot, k))_{s \in \N}$ of eigenfunctions in $L^{2}(\R^{n})$. By transformations of the problem \eqref{md-eigv-problem}, similar to the one in \eqref{mod-eigv-problem} and \eqref{periodic-condition}, we are then able to show that the eigenvalues of $A^{n}_{k}$ are continuous functions of $k \in \overline{B^{n}}$, and thus $I^{n}_{s} = \{ \lambda^{n}_{s}(k) : k \in \overline{B^{n}} \}$ is again a compact real interval for each $s \in \N$.
~\\ 

Ultimately, the main result for this multi-dimensional case follows from using the same arguments as in Chapter \ref{**-condition} based on Bloch waves, Floquet transform and a similar cut-off function $\eta$ as in \eqref{eta}. We are namely able to show that the spectrum of a self-adjoint Schrödinger operator with periodic delta-potential on a hypersurface is the union of the compact intervals $I^{n}_{s}$, i.e.
	\[ \sigma(A^{n}) = \bigcup_{s \in \N} I^{n}_{s}. \]