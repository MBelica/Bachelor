\chapter{The spectrum of the multi-dimensional Schrödinger operator}  \label{chap:7} % todo 6 check where k cant be negative

In this last chapter, we want to model the movement of a particle in $\R^{n}$ with periodically distributed, smooth, $(n-1)$-dimensional surfaces supporting a potential. To show that the basic concepts presented in the previous chapters hold also in the new setting, we will give a formal justification of applicability of the one-dimensional proofs to the multi-dimensional case in a series of theorems.
~\newline ~\newline
To start with, let $Y$ denote a periodicity cell in $\R^{n}$ and $B^{n}$ the corresponding Brillouin zone, for simplicity assume $Y$ being the unit cube $Y = [0, 1]^{n}$. Contained in $Y$ let $S$ be a smooth surface without a boundary with the conditions $\dim S = n - 1$ and $S \subseteq \overset{\circ}{Y}$. Furthermore, let $B \subseteq Y$ denote the set enclosed by $S$, such that $S = \partial B$. We will denote for any $j \in \Z^{n}$ with $Y_{j} = Y + j$ the $j$th copy of $Y$, which results through translation of the periodicity cell by $j$, and analogously for $S_{j} = S + j$ and $B_{j} = B + j$. Finally, we denote with $S_{i}^{+}$ and $S_{i}^{-}$ the opposing edges of $Y$ for $i = 1, 2$.

\begin{figure}[!ht] \centering
	\resizebox{.33\linewidth}{!}{
	  \begin{tikzpicture}[line cap=round,line join=round,>=triangle 45,x=4.285714285714286cm,y=4.285714285714286cm]
	  \clip(-1.2,-1.2) rectangle (1.2,1.3);
	  \fill[transparent,line width=0.4pt] (-1.,1.) -- (-1.,-1.) -- (1.,-1.) -- (1.,1.) -- cycle;
	  \draw [line width=0.4pt] (-1.,1.)-- (-1.,-1.);
	  \draw [line width=0.4pt] (-1.,-1.)-- (1.,-1.);
	  \draw [line width=0.4pt] (1.,-1.)-- (1.,1.);
	  \draw [line width=0.4pt] (1.,1.)-- (-1.,1.);
	  \draw [rotate around={-166.60970691015552:(-0.017884043865981074,0.012185928443575623)}] (-0.017884043865981074,0.012185928443575623) ellipse (2.915306454226339cm and 1.8911924645535707cm);
	  \draw ( 0.18, 0.35) node[anchor=north west] {$S$};
	  \draw (-0.85,-0.56) node[anchor=north west] {$Y$};
	  \draw (-1.2,0.125) node[anchor=north west] {$S_1^-$};
	  \draw (1.05,0.125) node[anchor=north west] {$S_1^+$};
	  \draw (-0.125,-1.05) node[anchor=north west] {$S_2^-$};
	  \draw (-0.125, 1.25) node[anchor=north west] {$S_2^+$};
	  \end{tikzpicture}
	}
	\caption{Periodicity cell for the multi-dimensional potential}
\end{figure}

The mathematical representation of the above is a multi-dimensional Schrödinger operator $A^{n}$ whose operation is formally defined by
\begin{equation}
	- \Delta + \rho \sum_{i \in \Z} \delta_{S_{i}} \label{eq:7.1-md_the_operator_A_formallyy}
\end{equation}
on the whole of $\R^{n}$, where $\delta_{S_{i}}$ denotes the Delta-Distribution on hypersurface $S_{i}$. Let us recall that on a hypersurface $S$ the Delta-Distribution acts on $u \in C_{0}^{\infty}(\R^{n})$ by 
	\[ \int_{\R^{n}} \left( \delta_{S} u \right) (x) dx \coloneqq \int_{S} u(s) ds . \]
where $s$ is the hypersurface measure associated to $S_{j}$, for a more detailed explanation see \cite{federer1978geomeasure}.
~\\ ~\\ % todo Zitat finden find that citation or whole replacement!
Again motivated by the weak-formulation, given a right-hand side $f \in L^{2}(\R^{n})$ we consider for some $\mu \in \R$ the problem to find $u \in H^{1}(\R^{n})$ such that
	\begin{equation}
		\int_{\R^{n}} \nabla u(x) \overline{\nabla v(x)} dx + \rho \sum_{i \in \Z^{n}} \int_{S_{j}} u(s) \overline{v(s)} ds - \mu \int_{\R^{n}} u(x) \overline{v(x)} dx = \int_{\R^{n}} f(x) \overline{v(x)} dx \label{eq:7.2-md-weak-formulation}
	\end{equation} 
holds for all $v \in H^{1}(\R^{n})$. Note that in the second term we denote by $u v$ the traces. They are well defined as $u, v \in H^{1}(\R)$ and hence the trace in $L^{2}(\R)$. For a detailed explanation see \cite[page 164]{adams2003sobolev}.  

\begin{remark}
	The term in \eqref{eq:7.2-md-weak-formulation} originating from the potential is finite since the Cauchy–Schwarz inequality yields
	\[ \left| \sum_{j \in \Z^{n}} \int_{S_{j}} u(s) \overline{v(s)} ds \right|^{2} \leq \left( \sum_{j \in \Z^{n}} \| u \|_{L^{2}(S_{j})}^{2} \right) \left( \sum_{j \in \Z^{n}} \| v \|_{L^{2}(S_{j})}^{2} \right). \] 
	Both terms on the right-hand side can be finitely estimated by the trace theorem \cite[page 258]{evans1998partial} for some $h > 0$ through
	\[ \| u \|_{L^{2}(S_{j})}^{2} \leq 2 \left( \frac{1}{h} \|u\|_{L^{2}(B_{j})}^{2} + h \| \nabla u \|_{L^{2}(B_{j})}^{2} \right). \]
\end{remark}

Analogously to section \ref{sec:3.1}, we can once again chose $\mu \in \R$ small enough such that Lax-Migram's Theorem proves the existence of a unique solution $u \in H^{1}(\R^{n})$ in \eqref{eq:7.2-md-weak-formulation} for any $f \in L^{2}(\R^{n})$. Hence, we are able to define the operator $R_{\mu}^{n} \colon L^{2}(\R^{n}) \rightarrow H^{1}(\R^{n}), f \mapsto u$ where $u \in H^{1}(\R^{n})$ is the solution of \eqref{eq:7.2-md-weak-formulation}. Through the same approach as in theorem \ref{rmuinj}, we can see that $R_{\mu}^{n}$ is a injective, bounded linear operator. This on the other hand enables us to explicitly define $A^{n}$ by means of $R_{\mu}^{n}$:
\[ A^{n} \coloneqq \left(R_{\mu}^{n}\right)^{-1} + \mu I, \quad \mathcal{D}(A^{n}) = \mathcal{R}(R_{\mu}^{n}). \]

We are interested in the spectrum of the operator $A^{n}$. The basic idea is again to transfer the problem to its fundamental domain of periodicity, solving the eigenvalue problem on this bounded set, which then will allow us through the Floquet transform to solve the problem for $A^{n}$.

\begin{theorem}[Characterisation of $\mathcal{D}(A^{n})$] Let $\Omega \coloneqq \R^{n} \setminus \overline{\bigcup_{j \in \Z^{n}} B_{j}}$. We can further characterise the solution $u \in \mathcal{D}(A^{n})$ from \eqref{eq:7.2-md-weak-formulation}, namely for all $j \in \Z^{n}$ it holds:
	\begin{enumerate} % todo gehört hier nicht das nablaweg im1.?
		\item $\Delta u \in L^{2}(B_{j})$, $\Delta u \in L^{2}(\Omega)$ and $\sum_{j \in \Z^{n}} \|\Delta u \|_{L^{2}(B_{j})}^{2} < \infty$
		\item $u \big|_{S_{j} - 0} = u \big|_{S_{j} + 0}$
		\item $\frac{\partial u}{\partial \eta_{j}} \big|_{S_{j} - 0} - \frac{\partial u}{\partial \eta_{j}} \big|_{S_{j} + 0} - \rho u \big|_{S_{j}} = 0$ where $\eta_{j}$ denotes the normal on $S_{j}$
	\end{enumerate}
	
	\begin{proof}
 		In section \ref{sec:3.2} we used particular functions $v \in C^{\infty}(\R)$ to prove equivalent properties of $\mathcal{D}(A)$. Hence, choosing in \eqref{eq:7.2-md-weak-formulation} similar functions for the multi-dimensional case will yield the asserted. 
	\end{proof}
\end{theorem}

\begin{remark}
	The operator $A^{n}$ is self-adjoint.	
\end{remark}

\begin{proof}
	todo.
	%todo 6 proof
\end{proof}

\begin{figure}[!ht] \centering
	\resizebox{.33\linewidth}{!}{
	  \begin{tikzpicture}[line cap=round,line join=round,>=triangle 45,x=5.0cm,y=5.0cm]
		\clip(-1,-1) rectangle (1.1,1.1);
		\fill[transparent,line width=0.4pt] (-1.,1.) -- (-1.,-1.) -- (1.,-1.) -- (1.,1.) -- cycle;
		\draw [line width=0.4pt] (-1.,1.)-- (-1.,-1.);
		\draw [line width=0.4pt] (-1.,-1.)-- (1.,-1.);
		\draw [line width=0.4pt] (1.,-1.)-- (1.,1.);
		\draw [line width=0.4pt] (1.,1.)-- (-1.,1.);
		\draw [rotate around={-166.60970691015552:(-0.017884043865981074,0.012185928443575623)}] (-0.017884043865981074,0.012185928443575623) ellipse (3.4011908632640617cm and 2.2063912086458326cm);
		\draw (0.17773598812780403,0.35749896759640476) node[anchor=north west] {$S$};
		\draw (-0.8520310518542072,-0.5460385642587798) node[anchor=north west] {$Y$};
		\draw [->] (0.14155897886779217,0.46905768120862523) -- (0.20343212114361697,0.7943332288652334);
		\draw (0.20431062141766237,0.7694057835892095) node[anchor=north west] {$\eta$};
		\draw (-0.014930103223669057,-0.24043028142540857) node[anchor=north west] {$-$};
		\draw (0.15116135483794568,-0.4928892976790631) node[anchor=north west] {$+$};
	  \end{tikzpicture}
	}
	\caption{Normal $\eta$ on the hypersurface $S$ in a periodicity cell}
\end{figure}

By restricting this multi-dimensional problem to a corresponding fundamental domain of periodicity, let this for simplicity be $Y$, the problem to find 
	\[ u \in H^{1}_{k, n} \coloneqq \left\{ w \in H^{1}(Y) \colon w \big|_{S_{j}^{+}} = w \big|_{S_{j}^{-}} e^{i k_{j}} \text{ for } k \in [-\pi, \pi]^{2}, j = 1,2 \right\} \] % todo derivates?!
such that
	\begin{equation}
		\int_{Y} \nabla u(x) \overline{\nabla v(x)} dx + \rho \int_{S} u(s) \overline{v(s)} ds - \mu \int_{Y} u(x) \overline{v(x)} dx = \int_{Y} f(x) \overline{v(x)} dx \label{eq:7.3-md_weak_formulation_res}
	\end{equation} 
holds for all $v \in H^{1}_{k, n}$. 
~\\ ~\\
Again, Lax-Milgram's Theorem ensures the existence of a unique solution $u \in H^{1}_{k, n}$ if $\mu \in \R$ is small enough, and the operator $R_{\mu, k} \colon f \mapsto u$ is in return well-defined and injective. This allows the definintion 
	\[ A_{k}^{n} \coloneqq \left(R_{\mu, k}^{n}\right)^{-1} + \mu I, \quad \mathcal{D}(A_{k}^{n}) = \mathcal{R}(R_{\mu, k}^{n}), \]
as the operator considering our problem only on the fundamental cell of periodicity. The semi-periodic boundary conditions on $H^{1}_{k,n}$ require a solution $u \in H^{1}_{k, n}$ of \eqref{eq:7.3-md_weak_formulation_res} to satisfy furthermore
	\[ \frac{\partial u}{\partial x_{j}}\big|_{S_{j}^{+}} = e^{ik_{j}} \frac{\partial u}{\partial x_{j}}\big|_{S_{j}^{-}} \quad \text{for } j = 1, 2.  \]
	
\begin{theorem}
	The operator $R_{\mu, k}^{n}$ is compact.	

	\begin{proof}
		As in chapter \ref{chap:4}, using the compact embedding theorem for Sobolev spaces yields the desired. % todo 6 show R_{\mu, k} bounded, UND konkreter: Analogously to theorem... by an applicaion of Rellich.
	\end{proof}
\end{theorem}

By a transformations of the problem \eqref{eq:7.3-md_weak_formulation_res} as in \eqref{mod-eigv-problem} and \eqref{periodic-condition} we are then able to show that the eigenvalues of $A^{n}_{k}$ are continuous functions of $k \in \overline{B}$, and thus $I^{n}_{s} = \{ \lambda^{n}_{s}(k) : k \in \overline{B} \}$ is again a compact real interval for each $s \in \N$.  % todo zeige transformiertes problem und eigenwerte dazu
~\newline ~\newline
Ultimately, the main result for this multi-dimensional case follows from using the same arguments as in Chapter 4 based on Bloch waves, Floquet transform and a similar cut-off function $\eta$ as in \eqref{eta}. We are namely able to show that the spectrum of a self-adjoint Schrödinger operator with periodic delta-potential on a hypersurface is the union of the  compact intervals $I^{n}_{s}$, i.e.
	\[ \sigma(A^{n}) = \bigcup_{s \in \N} I^{k}_{s}. \]