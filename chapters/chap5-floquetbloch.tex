\chapter{The Floquet transformation and the Bloch waves} \label{chap:5}

In chapter \ref{chap:6} we will show that the spectrum of the operator $A$ can be constructed from the eigenvalue sequences $(\lambda_{s}(k))_{s \in \N}$ introduced above by varying $k$ over the Brillouin zone $B$. For this purpose we will need two results involving the Floquet transformation to be able to move the problem from $L^{2}(\R)$ to $L^{2}(\Omega \times B)$ whereas $\Omega \times B$ is compact by assumption. For the sake of completeness, we include here the proofs of both theorems, as given in \cite[Chap. 3.4, 3.5]{dorfler2011photonic}.

\section{Properties of the Floquet transformation}

\begin{theorem} \label{3.2:thm-UIsometricIsomorphism}
	The Floquet transformation $U \colon L^{2}(\R) \rightarrow L^{2}(\Omega \times B)$ 
	\begin{equation}
		(Uf)(x, k) \coloneqq \frac{1}{\sqrt{|B|}} \sum_{n \in \Z} f(x - n) e^{ikn} \quad (x \in \Omega, k \in B). \label{floquet-transformation}
	\end{equation}
	is an isometric isomorphism, with inverse given by
		\begin{equation}
			(U^{-1}g)(x - n) = \frac{1}{\sqrt{|B|}} \int_{B} g(x, k) e^{-ikn} dk \quad (x \in \Omega, n \in \Z). \label{3.8}
		\end{equation} 
	If $g(\cdot, k)$ is extended to the whole of $\R$ by the semi-periodicity condition \eqref{quasi-periodic-condition}, the inverse formular simplifies to
		\begin{equation}
			U^{-1} g = \frac{1}{\sqrt{|B|}} \int_{B} g(\cdot, k) dk. \label{3.9}
		\end{equation}
		
	\begin{proof}
		For $f \in L^{2}(\R)$,
		\begin{equation}
			\int_{\R} |f(x)|^{2} dx = \sum_{n \in \Z} \int_{\Omega} |f(x - n)|^{2} dx,\label{functionoverperiodicity}
		\end{equation} 
		where we used Beppo Levi's Theorem to exchange summation and integration. This shows that
		\[ \sum_{n \in \Z} |f(x - n)|^{2} < \infty \text{ for almost every } x \in \Omega.\]
		Thus, $(Uf)(x, k)$ is well-defined by \eqref{floquet-transformation} (as a Fourier series with variable $k$) for almost every $x \in \Omega$, and Parseval's equality gives for these $x$
		\[ \int_{B}|(Uf)(x,k)|^{2} dk = \sum_{n \in \Z} |f(x - n)|^{2}. \]
	 	This expression is in $L^{2}(\Omega)$ by \eqref{functionoverperiodicity}, and we have $\| Uf \|_{L^{2}(\Omega \times B)} = \|f\|_{L^{2}(\R)}$. It is for us still to show that the mapping $U$ is surjective, and that $U^{-1}$ is given by \eqref{3.8} or \eqref{3.9}. Let $g \in L^{2}(\Omega \times B)$, then define
		\begin{equation}
			f(x - n) \coloneqq \frac{1}{\sqrt{|B|}} \int_{B} g(x, k) e^{-ikn} dk \quad (x \in \Omega, n \in\Z).\label{3.11}
		\end{equation}
		Parseval's Theorem states for fixed $x \in \Omega$ that $\sum_{n \in \Z} |f(x - n)|^{2} = \int_{B} |g(x, k)|^{2} dk$. Integrating this equality over $\Omega$ then yields
		\begin{align*}
			\int_{\Omega \times B} |g(x, k)|^{2} dx dk & = \int_{\Omega} \sum_{n \in \Z} |f(x - n)|^{2} dx  = \sum_{n \in\Z} \int_{\Omega} |f(x-n)|^{2} dx = \int_{\R} |f(x)|^{2} dx,	
		\end{align*}
		which means $f \in L^{2}(\R)$. For almost every $x \in \Omega$ \eqref{floquet-transformation} gives
		\[ f(x - n) = \frac{1}{\sqrt{|B|}} \int_{B} (Uf)(x,k) e^{-ikn} dk \quad (n \in \Z), \]
		whence \eqref{3.11} implies $U f = g$ and \eqref{3.8}. Now \eqref{3.9} follows from \eqref{3.8} and exploiting $g(x + n, k) = e^{ikn} g(x, k)$.
	\end{proof}				
\end{theorem}

\section{Completeness of the Bloch waves}

Using the Floquet transformation $U$, we can now prove the property of completeness of the Bloch waves $\psi_{s}(\cdot, k)$ in $L^{2}(\Omega)$ when we vary $k$ over the Brillouin zone $B$.
	
\begin{theorem} \label{3.3:thm-flConvergence}
		For each $f \in L^{2}(\R)$ and $l \in \N$, define
			\begin{equation}
				f_{l}(x) \coloneqq \frac{1}{\sqrt{|B|}} \sum_{s=1}^{l} \int_{B} \langle (Uf)(\cdot, k), \psi_{s}(\cdot, k) \rangle_{L^{2}(\Omega)} \psi_{s}(x, k) dk \quad (x \in \R). \label{3.15}
			\end{equation}
		Then, $f_{l} \rightarrow f$ in $L^{2}(\R)$ as $l \rightarrow \infty$.

	\begin{proof}
		The last theorem tells us that $Uf \in L^{2}(\Omega \times B)$, which in return means that $(Uf)(\cdot, k) \in L^{2}(\Omega)$ for almost all $k \in B$ by Fubini's Theorem. Since $(\psi_{s}(\cdot, k))_{s \in \N}$ is an orthonormal and complete system of eigenfunctions in $L^{2}(\Omega)$ for each $k \in B$, we derive
			\[ \lim_{l \rightarrow \infty} \| (Uf)(\cdot, k) - g_{l}(\cdot, k) \|_{L^{2}(\Omega)} = 0 \text{ for almost every } k \in B \]
		where 
			\begin{equation}
				g_{l}(x, k) \coloneqq \sum_{s=1}^{l} \langle(Uf)(\cdot, k), \psi_{s}(\cdot,k)\rangle_{L^{2}(\Omega)} \psi_{s}(x,k). \label{3.16}
			\end{equation}
		Moreover, we get by Bessel's inequality
			\[ \| (Uf)(\cdot, k) - g_{l}(\cdot, k) \|^{2}_{L^{2}(\Omega)} \leq \| (Uf)(\cdot, k) \|^{2}_{L^{2}(\Omega)}  \]
		for all $l \in \N$ and almost every $k \in B$. Next, $\|(Uf)(\cdot, k)\|^{2}_{L^{2}(\Omega)} \in L^{1}(B)$ as a function of $k$ by theorem \ref{3.2:thm-UIsometricIsomorphism}, thus by Lebesgue's Dominated Convergence theorem
		\[ \lim_{l \rightarrow \infty} \int_{B} \| (Uf)(\cdot, k) - g_{l}(\cdot, k) \|^{2}_{L^{2}(\Omega)} dk  = \int_{B} \lim_{l \rightarrow \infty}  \| (Uf)(\cdot, k) - g_{l}(\cdot, k) \|^{2}_{L^{2}(\Omega)} dk = 0. \]
		  All in all, this means
			\begin{equation}
				\| U f - g_{l} \|_{L^{2}(\Omega \times B)} \rightarrow 0 \text{ as } l \rightarrow \infty \label{3.17}
			\end{equation} 
		 Using \eqref{3.15}, \eqref{3.16} and \eqref{3.9}, we find that $f_{l} = U^{-1}g_{l}$, whence \eqref{3.17} gives
			\[ \| U(f - f_{l}) \|_{L^{2}(\Omega \times B)} \rightarrow 0 \text{ as } l \rightarrow \infty,\]
		and the assertion follows since $U \colon L^{2}(\R) \rightarrow L^{2}(\Omega \times B)$ is isometric by theorem \ref{3.2:thm-UIsometricIsomorphism}.
	\end{proof}
\end{theorem}