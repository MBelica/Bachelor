\chapter{Outlook and conclusion}	 \label{chap:8}
	
Auf jeden Fall zunächst verbale summary of the main findings of this thesis.	
~\\
	
We have to notice that we haven't determined the nature of the spectrum that is if there are possible gaps in the spectrum. However,... 

Floquet–Bloch theory does not give an answer to the question if there are really gaps in the spectrum or if the bands actually overlap. An asymptotic answer (for sufficiently “high contrast” in the coefficients) has been given in 'A. Figotin and P. Kuchment. Band gap structure of spectra of periodic dielectric and acoustic media. II. Two-dimensional photonic crystals. SIAM J. Appl. Math., 56:1561–1620, 1996'. 

In a more concrete case, existence of a gap has been proved by computer-assisted means in 'V. Hoang, M. Plum, and C. Wieners. A computer-assisted proof for photonic band gaps. 
Zeitschrift für Angewandte Mathematik und Physik, 60:1–18, 2009'.

'S. Albeverio, F. Gesztesy, R. Hoegh-Krohn, H. Holden {\it Solvable Models in Quantum Mechanics}. Second edition, with an appendix by P. Exner xvi+488 p.; AMS Chelsea Publishing, volume 350, Providence, R.I., 2005'
~\\
 
Hier auf Literatur verwiesen, die aktuell daran ansetzt.

