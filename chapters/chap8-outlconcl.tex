\chapter{Conclusion} \label{chap:8}
	
This thesis examined the spectrum of an operator describing the situation in the Kronig-Penney model when a particle moves through periodically distributed, singular potentials. The conclusion resulted from considering the problem with the help of the Floquet-Bloch theory, that is that the spectrum of such an operator has a real, band-gap structured spectrum. Even though the main focus was on the one-dimensional case of the delta potentials being aligned on one axis, we gave a formal justification of applicability of the basic concepts presented in the one-dimensional case for the case when the particle moves in $\R^{n}$ with periodically distributed, smooth, $(n-1)$-dimensional surfaces supporting a potential, such that the result also holds for the muti-dimensional case. 
~\\

We examined the periodic, one-dimensional Schrödinger operator and showed that it is self-adjoint. In chapters 4 and 5, we related the spectrum of this periodic operator on the whole of $\R$ to a family of eigenvalue problems on the corresponding periodicity cell with the help of the Floquet transform. On the period cell the eigenfunctions, the Bloch waves, are subject to semi-periodic boundary conditions depending on an additional parameter which varies over the Brillouin zone. The result, we concluded in chapter 6 and eventually extrapolated in chapter 7 to the multi-dimensional case, is the band-gap structure of the spectrum of the whole-space operator.
~\\

It is however difficult to find a definitive answer to the question whether there are really gaps in the band-gap structure.


 Moreover, it is not easy to make statements about the nature of the spectrum, for example to decide if (i. e. no eigenvalues occur).

	However, even though we did not give an answer to the question if there are really gaps in the spectrum or not. An asymptotic answer (for sufficiently “high contrast” in the coefficients) has been given in [5]. In a more concrete case, existence of a gap has been proved by computer-assisted means in [8]. Floquet–Bloch theory does not give an answer to the question if there are really gaps in the spectrum or if the bands actually overlap. An asymptotic answer (for sufficiently “high contrast” in the coefficients) has been given in 'A. Figotin and P. Kuchment. Band gap structure of spectra of periodic dielectric and acoustic media. II. Two-dimensional photonic crystals. SIAM J. Appl. Math., 56:1561–1620, 1996'. In a more concrete case, existence of a gap has been proved by computer-assisted means in 'V. Hoang, M. Plum, and C. Wieners. A computer-assisted proof for photonic band gaps. Zeitschrift für Angewandte Mathematik und Physik, 60:1–18, 2009'.
~\\
'S. Albeverio, F. Gesztesy, R. Hoegh-Krohn, H. Holden Solvable Models in Quantum Mechanics. Second edition, with an appendix by P. Exner xvi+488 p.; AMS Chelsea Publishing, volume 350, Providence, R.I., 2005'
~\\
A more general exposition for non-smooth coefficients has been published in [2].
~\\

Peter Kuchment, An Overview of Periodic Elliptic Operators: The article surveys the main topics, techniques, and results of the theory of periodic operators arising in mathematical physics and other areas. Close attention is paid to studying analytic properties of Bloch and Fermi varieties, which significantly influence most spectral features of such operators. The approaches described are applicable not only to the standard model example of Schrodinger operator with periodic electric potential but to a wide variety of elliptic periodic equations and systems, equations on graphs, operator, and other operators on abelian coverings of compact bases. Important applications are mentioned. However, due to the size restrictions, they are not dealt with in detail.
~\\

Andrii Khrabustovskyi, Michael Plum, Spectral properties of elliptic operator with double-contrast coefficients near a hyperplane: In this paper we study the asymptotic behaviour as of the spectrum of the elliptic operator posed in a bounded domain subject to Dirichlet boundary conditions on. When both coefficients a become high contrast in a small neighbourhood of a hyperplane intersecting. We prove that the spectrum of converges to the spectrum of an operator acting in L and generated by the operation, the Dirichlet boundary conditions on certain interface conditions on containing the spectral parameter in a nonlinear manner. The eigenvalues of this operator may accumulate at a finite point. Then we study the same problem, when is an infinite straight strip (waveguide) and is parallel to its boundary. We show that has at least one gap in the spectrum when is small enough and describe the asymptotic behaviour of this gap as. The proofs are based on methods of homogenisation theory.