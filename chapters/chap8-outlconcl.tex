\chapter{Summary and conclusions} \label{chap:8}
	
This thesis examined the spectrum of an operator describing the situation in the Kronig-Penney model when a particle moves through periodically distributed, singular potentials. The main result obtained from solving the problem with the help of the Floquet-Bloch theory is that such an operator has a real, band-gap structured spectrum.
~\\

First, we examined the periodic, one-dimensional Schrödinger operator where the delta potentials are periodically aligned along only one axis and showed that the operator is self-adjoint. With the help of the Floquet transform, we related the spectrum of this periodic operator on the whole of $\R$ to a family of eigenvalue problems on the corresponding periodicity cell. On the periodicity cell, we chose the eigenfunctions, the so-called Bloch waves, to be subject to semi-periodic boundary conditions depending on a parameter which varied over the Brillouin zone. We then proved the band-gap structure of the spectrum of the operator under consideration in $\R$. Finally, we gave a formal justification of applicability of the approach in $\R$ also to the case when the particle moves in $\R^{n}$ with periodically distributed, smooth, $(n-1)$-dimensional surfaces supporting a potential, such that the result in $\R$ would also hold for the multi-dimensional case and we showed that the obtained results also holds in $\R^{n}$. 
~\\

It is however difficult to find a definitive answer to the question whether there are really gaps in the band-gap structure, since the Floquet–Bloch theory does not give an answer to the question if the bands actually overlap. A systematic analysis of such a quantum mechanical problem has been provided in \cite{albeverio2012solvable}. Furthermore, in a more concrete context with the specialisation to a two-dimensional situation and to polarised waves, existence of a gap has been computer-assisted analysed in \cite{hoang2009computer}. 
~\\

Finally, two current papers offer further insight into the analysis of elliptic operators. In \cite{kuchment2016overview} an overview of periodic elliptic operators is provided and the main techniques and results of the theory is surveyed. The approaches described are applicable not only to the standard model example of Schrödinger operator with periodic potential but to a wide variety of elliptic periodic equations and systems. Moreover, in \cite{khrabustovskyi2016spectral} asymptotic behaviour of the spectrum of elliptic operators  in a bounded domain subject to Dirichlet boundary conditions is studied. Furthermore, based on the homogenisation theory, the same problem is examined when the domain consists of an infinite straight strip (waveguide) and a hyperplane intersecting the domain is parallel to its boundary. In the thesis it is shown that the spectrum in this case, given that the coefficients become high contrast in a small neighbourhood of a hyperplane, has at least one gap in the spectrum.