\chapter{Conclusion} \label{chap:8}
	
This thesis examined the spectrum of an operator describing the situation in the Kronig-Penney model when a particle moves through periodically distributed, singular potentials. The conclusion resulted from considering the problem with the help of the Floquet-Bloch theory, is that such an operator has a real, band-gap structured spectrum. Even though the main focus was on the one-dimensional case, where the delta potentials are periodically aligned on one axis, we gave a formal justification of applicability of the concepts presented in the one-dimensional case for the case when the particle moves in $\R^{n}$ with periodically distributed, smooth, $(n-1)$-dimensional surfaces supporting a potential, such that the result also holds for the muti-dimensional case. 
~\\

First, we examined the periodic, one-dimensional Schrödinger operator and showed that it is self-adjoint. In chapters 4 and 5, we related the spectrum of this periodic operator on the whole of $\R$ to a family of eigenvalue problems on the corresponding periodicity cell with the help of the Floquet transform. We chose on the period cell the eigenfunctions, the Bloch waves, to be subject to semi-periodic boundary conditions depending on a parameter which varied over the Brillouin zone. The result, which we concluded in chapter 6 and eventually extrapolated in chapter 7 to the multi-dimensional case, is the band-gap structure of the spectrum of the whole-space operator.
~\\

It is however difficult to find a definitive answer to the question whether there are really gaps in the band-gap structure, since the Floquet–Bloch theory does not give an answer to the question if the bands actually overlap. A systematic analysis of such a quantum mechanical problem has been provided in \cite{albeverio2012solvable}. Furthermore, in a more concrete context with the specialisation to a two-dimensional situation and to polarized waves, existence of a gap has been computer-assisted analysed in \cite{hoang2009computer}. 
~\\

Finally, two current papers offer further insight into the analysis of elliptic operators. In \cite{kuchment2016overview} an overview of periodic elliptic operators is provided and the main techniques and results of the theory is surveyed. The approaches described are applicable not only to the standard model example of Schrödinger operator with periodic potential but to a wide variety of elliptic periodic equations and systems. Moreover, in \cite{khrabustovskyi2016spectral} asymptotic behaviour of the spectrum of elliptic operators  in a bounded domain subject to Dirichlet boundary conditions is studied. The authors prove that the spectrum of converges to the spectrum of an operator acting in L and generated by the operation, the Dirichlet boundary conditions on certain interface conditions on containing the spectral parameter in a nonlinear manner. Furthermore, based on the homogenisation theory, the same problem, when an infinite straight strip (waveguide)is added and is parallel to its boundary, is examined and it is proven that the spectrum has at least one gap in the spectrum when is small enough and describe the asymptotic behaviour of this gap as.