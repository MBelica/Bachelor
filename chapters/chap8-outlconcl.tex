\chapter{Summary and conclusions} \label{chap:8}
	
This thesis examined the spectrum of an operator describing the situation in the Kronig-Penney model when a particle moves through periodically distributed, singular potentials. The main result obtained from solving the problem with the help of the Floquet-Bloch theory is that such an operator has a real, band-gap structured spectrum.
~\\

First, we examined the periodic, one-dimensional Schrödinger operator where the delta potentials are periodically aligned along only one axis and showed that the operator is self-adjoint. With the help of the Floquet transformation, we related the spectrum of this periodic operator on the whole of $\R$ to a family of eigenvalue problems on the corresponding periodicity cell. On the periodicity cell, we chose the eigenfunctions, the so-called Bloch waves, to be subject to semi-periodic boundary conditions depending on a parameter which varied over the Brillouin zone. We then proved the band-gap structure of the spectrum of the operator under consideration in $\R$. Finally, we gave a formal justification of applicability of the approach in $\R$ also to the case when the particle moves in $\R^{n}$ with periodically distributed, smooth, $(n-1)$-dimensional surfaces supporting a potential, such that the result in $\R$ also holds in $\R^{n}$. 
~\\

However, it is still not straightforward to conclude that there really are any \textit{gaps} in the band-gap structure since the Floquet–Bloch theory allows for gaps of any width, i. e. it also admits bands of zero width. A systematic analysis of such a quantum mechanical problem has been provided in \cite{albeverio2012solvable}. Furthermore, in a more concrete context of polarised waves in two dimensions, the existence of band gaps has been computer-assisted analysed in \cite{hoang2009computer}. 
~\\

Two recent papers offer further insight into the analysis of elliptic operators, a generalisation of the Laplace operators. In \cite{kuchment2016overview} an overview of periodic elliptic operators is provided and the main techniques and results of the theory are surveyed. The suggested approach is applicable not only to the standard model example of Schrödinger operator with periodic potential but also to a wide variety of periodic elliptic equations and systems. In \cite{khrabustovskyi2016spectral} methods of homogenisation theory are used to study asymptotic behaviour of the spectrum of the elliptic operator $A_{\epsilon} = - \frac{1}{b^{\epsilon}} \operatorname{div}(e^{\epsilon} \nabla)$ for $\epsilon \rightarrow 0$ in a bounded domain subject to Dirichlet conditions on the boundary. For $\epsilon$ tending to $0$ both coefficients $a$ and $b$ become high contrast in a small neighbourhood of a hyperplane intersecting the domain. The same problem is examined under the assumption that the domain consists of an infinite straight strip (waveguide) and a hyperplane intersecting the domain is parallel to its boundary. It is shown that the spectrum in this case, given that the coefficients become high contrast in a small neighbourhood of a hyperplane, has at least one gap.