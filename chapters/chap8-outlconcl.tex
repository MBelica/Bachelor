\chapter{Conclusion} \label{chap:8}
	
This thesis examines the spectrum of an operator describing the situation in the Kronig-Penney model when a particle moves through periodically distributed, singular potentials. The conclusion results from considering the problem with the help of the Floquet-Bloch theory, that is that the spectrum of such an operator has a real, band-gap structured spectrum. Even though the main focus is on the one-dimensional case of potentials being aligned on one axis, we give a formal justification of applicability of the basic concepts presented in the one-dimensional case for the case when the particle moves in $\R^{n}$ with periodically distributed, smooth, $(n-1)$-dimensional surfaces supporting a potential, such that the result also holds for the muti-dimensional case.
~\\

In Chapter 3, we examined the one-dimensional Schrödinger operator, described its domain and showed that the operator is self-adjoint. In Chapter 4, restricting the problem to its fundamental domain of periodicity and introducing the Brillouin zone.. The domain of the restricted Schrödinger operator...The compactness of the restricted resolvent...The spectrum of the restricted Schrödinger operator. In chapter 5, ... The Floquet transformation and the Bloch waves...Properties of the Floquet transformation...Completeness of the Bloch waves... In the last two chapters,... The spectrum of the one-dimensional Schrödinger operator...The spectrum of the multi-dimensional Schrödinger operator 
~\\

This might be generalised to other forms of pricing which are typically considered impractical. A strategy for minimising the memory traffic during the operation of such a solver was also presented. The GPU code in this chapter is the first to demonstrate the ability to solve real, numerically difficult problems. 

In Chapter 3, new techniques for working with permutation matrices, for updating a Schur complement, and for maintaining a structured basis, were described. The practicality of Kaul’s method was demonstrated, with new observations on its sparsity and numerical performance, and results were provided from the first known implementation showing that it is capable of reliably solving real problems. This work also contains the first description of a means to parse standard multi-commodity flow problem files without recourse to additional information.

In Chapter 4, a new representation for sparse matrices, transposed ELLPACK, and improved kernels for sparse matrix-vector multiplication with the constraint matrices of two problem classes were described. This work also demonstrated the use of an oracle for a constraint matrix in a matrix-free interior point method. In the appendices, a new practical dual phase one method was described which simplifies recovery from infeasibility. A preliminary description of a bound-flipping ratio test for the primal simplex method was provided, including considerations of post-solve. The appendices also include computational studies of basis updates and pricing which are more comprehensive in their range than those which have been previously published.
~\\
	
It relates the spectrum of a self-adjoint operator realising a periodic spectral problem on the whole of Rn to a family of eigenvalue problems on the periodicity cell. Here, the eigenfunctions (Bloch waves) are subject to semi-periodic boundary conditions depending on an additional parameter which varies over the so-called Brillouin zone. The result is the band-gap structure of the spectrum of the whole-space operator. Floquet–Bloch theory applies, in particular, to periodic Schrodinger equations and — what is most important within the scope of this book — to periodic Maxwell eigenvalue problems, i. e. to photonic crystals.
~\\

It is however difficult to decide, whether there are really gaps in the union (3.23). Moreover, it is not easy to make statements about the nature of the spectrum, for example to decide if (i. e. no eigenvalues occur). The only easy result is.	 Even though we did not give an answer to the question if there are really gaps in the spectrum or not. An asymptotic answer (for sufficiently “high contrast” in the coefficients) has been given in [5]. In a more concrete case, existence of a gap has been proved by computer-assisted means in [8]. Floquet–Bloch theory does not give an answer to the question if there are really gaps in the spectrum or if the bands actually overlap. An asymptotic answer (for sufficiently “high contrast” in the coefficients) has been given in 'A. Figotin and P. Kuchment. Band gap structure of spectra of periodic dielectric and acoustic media. II. Two-dimensional photonic crystals. SIAM J. Appl. Math., 56:1561–1620, 1996'. In a more concrete case, existence of a gap has been proved by computer-assisted means in 'V. Hoang, M. Plum, and C. Wieners. A computer-assisted proof for photonic band gaps. Zeitschrift für Angewandte Mathematik und Physik, 60:1–18, 2009'.
~\\
A more general exposition for non-smooth coefficients has been published in [2]. Important contributions to Floquet–Bloch theory have been made e. g. in [9, 10, 11].	
~\\
'S. Albeverio, F. Gesztesy, R. Hoegh-Krohn, H. Holden Solvable Models in Quantum Mechanics. Second edition, with an appendix by P. Exner xvi+488 p.; AMS Chelsea Publishing, volume 350, Providence, R.I., 2005'
~\\
Peter Kuchment, An Overview of Periodic Elliptic Operators: The article surveys the main topics, techniques, and results of the theory of periodic operators arising in mathematical physics and other areas. Close attention is paid to studying analytic properties of Bloch and Fermi varieties, which significantly influence most spectral features of such operators. The approaches described are applicable not only to the standard model example of Schrodinger operator with periodic electric potential but to a wide variety of elliptic periodic equations and systems, equations on graphs, operator, and other operators on abelian coverings of compact bases. Important applications are mentioned. However, due to the size restrictions, they are not dealt with in detail.
~\\
Andrii Khrabustovskyi, Michael Plum, Spectral properties of elliptic operator with double-contrast coefficients near a hyperplane: In this paper we study the asymptotic behaviour as of the spectrum of the elliptic operator posed in a bounded domain subject to Dirichlet boundary conditions on. When both coefficients a become high contrast in a small neighbourhood of a hyperplane intersecting. We prove that the spectrum of converges to the spectrum of an operator acting in L and generated by the operation, the Dirichlet boundary conditions on certain interface conditions on containing the spectral parameter in a nonlinear manner. The eigenvalues of this operator may accumulate at a finite point. Then we study the same problem, when is an infinite straight strip (waveguide) and is parallel to its boundary. We show that has at least one gap in the spectrum when is small enough and describe the asymptotic behaviour of this gap as. The proofs are based on methods of homogenisation theory.