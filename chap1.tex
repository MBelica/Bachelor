\chapter{The Operator}


An important problem in mathematical physics is the solution of the one-dimensional Schrödinger equation with distributional potential, which is formally defined by the operation

\begin{equation}
	- \frac{d^{2}}{dx^{2}} + \rho \sum_{i \in \Z} \delta_{x_{i}} \label{the-operator}
\end{equation}
	
on the whole of $\R$ where $f$ is a function modelling an external force and $x_{i}$ are periodically distributed. $\Omega_{k}$ will denote the periodicity cell containing delta point $x_{k}$ and let w.o.l.g. $x_{0} = 0$ and $|\Omega_{i}| = 1 ~\forall i \in \Z$.  Henceforth, consider	for a $\mu \in \R$ small enough the problem 
	
\begin{equation}
	\int u' \overline{v'} + \rho \sum_{i \in \Z} u(x_{i}) \overline{v(x_{i})} - \mu \int u \overline{v} = \int f \overline{v}, \quad \forall v \in H^{1}(\R) \label{weak-formulation}
\end{equation}	
where $f \in L^{2}(\R)$ and $u \in H^{1}(\R)$.
	
This expression actually converges as for arbitrary $\tilde{x}_{i} \in \Omega_{i}$
\begin{eqnarray}
	\sum_{i \in \Z} |u(x_{i})|^{2} & \leq & \sum_{i \in \Z} \left( \big| u(\tilde{x}_{i}) + \int_{\tilde{x}_{i}}^{x_{i}} u'( \tau ) d\tau \big| \right)^{2} \notag \\
		 & \leq & 2 \sum_{i \in \Z} \left( \int_{\Omega_{i}} |u( x )|^{2} dx +  \int_{\Omega_{i}} \left| u'(\tau) \right|^{2} d\tau \right) \notag \\
		 & \leq & 2 \cdot \| u \|^{2}_{H^{1}(\R)} \label{eq:potentialestim}
\end{eqnarray}

Now, as we can interpret the lefthand side of \eqref{weak-formulation} as a bounded bilinear mapping $B \colon H^{1}(\R) \times H^{1}(\R) \rightarrow \R$, Lax Milgram's Theorem asserts the existence of a unique element $u \in H^{1}$ satisfying

\begin{equation*}
	B[u, v] = \langle f, v \rangle
\end{equation*}

if there exist constants $\alpha, \beta > ß$ such that

\leqnomode
\begin{align*}
	\tag{i} \left| B[u,v] \right| \leq \alpha \|u\| \|v\| \quad (u, v \in H^{1}(\R))
\end{align*}
and
\begin{align*}
	\tag{ii} \beta \|u\|^{2} \leq B[u, u] \quad (u \in H^{1}(\R))
\end{align*}
\reqnomode

Taking these two condition under examination, \eqref{eq:potentialestim} yields for the norm of $B[u, v]$ both:

\begin{theorem} \label{1.1}
	The bilinear form $B[u, v]$ is bounded.
	\begin{proof}
		\begin{align*}
			| B(u, \varphi)|^{2} & \leq \| u' \| \cdot \| v' \| + 2 \rho \sum_{i \in \Z} |u(x_{i})|^{2} |v(x_{i})|^{2} - \mu \| u \| \cdot \| v \| \\
				& \leq \| u' \| \cdot \| v' \| + 8 \rho \cdot \| u \|^{2}_{H^{1}(\R)} \| v \|^{2}_{H^{1}(\R)}  - \mu \| u \| \cdot \| v \| \\
				& = (8\rho - \mu) \| u \| \cdot \| v \| + 8\rho \left( \| u \| \cdot \| v' \| + \| u' \| \cdot \| v \| \right) + (8\rho + 1) \| u' \| \cdot \| v'\| \\
				& \leq \alpha \cdot \| u \|_{H^{1}} \cdot \| \varphi \|_{H^{1}}
		\end{align*}
	\end{proof}
\end{theorem}

\begin{theorem} \label{1.2}
	$B[u, u]$ is coercive.
	\begin{proof}
		Lets first assume $\rho \geq 0$ then for $\mu < -1$:
		\begin{align*}
			B(u, u) & = \langle u' , u' \rangle + \rho \sum_{i \in \Z} u(x_{i})^{2} - \mu \langle u , u \rangle \\
					& \geq \langle u' , u' \rangle - \mu \langle u , u \rangle \geq \langle u' , u' \rangle  + \langle u , u \rangle \\
					& = \| u \|_{H^{1}}^{2}
		\end{align*}
		and for $\rho < 0$:
		\begin{align*}
			B(u, u) & = \langle u' , u' \rangle + \rho \sum_{i \in \Z} |u(x_{i})|^{2} - \mu 	\langle u , u \rangle \\
					& = \langle u' , u' \rangle + \rho \sum_{i \in \Z} \big| u(\tilde{x}_{i}) + \int_{\tilde{x}_{i}}^{x_{i}} u(x) dx \big|^{2} - \mu \langle u , u \rangle \\
					& \geq \langle u' , u' \rangle + 2 \rho \left( \int_{\R} |u(x)|^{2} dx + \int_{\R} |u'(\tau)|^{2} d\tau \right) - \mu \langle u , u \rangle \\
					& = (2 \rho + 1) \| u' \|^{2} + (2\rho - \mu) \| u \|^{2}  \\
					& \geq \beta \| u \|_{H^{1}}^{2} 
		\end{align*}
	\end{proof}
\end{theorem}

Such that that the problem \eqref{weak-formulation} has the unique element $u \in H$ and with that the resolvent mapping $R_{\mu} \colon L^{2}(\R) \rightarrow H^{1}(\R), f \mapsto u$ is well-defined; obviously the mapping is one-to-one since for $u_{1} = u_{2}$

	\[ 0 = B[u_{1}, v] - B[u_{2}, v]= \int (f_{1} - f_{2}) \overline{v}, \quad \forall v \in H^{1}(\R) \]
		
and as $H^{1}$ is dense in $L^{2}$ this means that this equation holds also for all $v \in L^{2}(\R)$ and therefore $f_{1} = f_{2}$ almost everywhere. Accordingly $R_{\mu}$ is bijective and in turn we can define 
		\[ A \coloneqq R_{\mu}^{-1} + \mu I \text{ and with that } \mathcal{D}(A) = \mathcal{R}(R_{\mu}) \]